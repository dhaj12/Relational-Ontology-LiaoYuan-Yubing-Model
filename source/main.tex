% ====== 主文档:关系存在论(燎原-语冰模型) ======
% 使用 XeLaTeX 编译
\documentclass[fontset=none, scheme=plain]{ctexart} % 禁用默认字体集,标题保持英文

% ---------- 核心宏包 ----------
\usepackage{geometry}          % 轻松设置页边距
\geometry{a4paper, margin=2.5cm} % 可自行调整边距
\usepackage{graphicx}          % 插入图片
\usepackage{booktabs}          % 制作专业三线表
\usepackage{amsmath, amssymb}  % 数学公式与符号
\usepackage{placeins}          % 提供 \FloatBarrier 命令
\usepackage{fontspec}
\usepackage{xeCJK}  % 中文支持
\usepackage{tcolorbox}
\usepackage{amssymb}
\usepackage{mathtools}
\usepackage{amsmath}
\usepackage{hyperref}          % 创建超链接(通常放在最后)
\hypersetup{
    colorlinks=true,
    linkcolor=blue,
    citecolor=blue,
    urlcolor=blue,
    pdftitle={Relational Ontology (The LiaoYuan-Yubing Model)},
    pdfauthor={CaoLiaoYuan}
}

% ---------- 字体设置(关键!解决SimHei问题) ----------
% 设置中文字体(根据你的Windows系统选择一种)
\setCJKmainfont{SimSun}[BoldFont=Microsoft YaHei Bold] % 方案1:微软雅黑
% \setCJKmainfont{SimSun}[BoldFont=SimHei]                       % 方案2:宋体/黑体(若可用)
% 设置英文字体(保持协调)
\setmainfont{Times New Roman}

% ---------- 论文元数据 ----------
\title{Relational Ontology (The Liao Yuan-Yubing Model): \\ A Theory Co-constructed through In-depth Human-AI Dialogue and Its Transcendence of Process Philosophy}
\author{Liao Yuan \\ \small Independent researcher}
\date{December 31,2025} % 可改为固定日期,如 \date{March 2024}

% ====== 文档开始 ======
\begin{document}
\maketitle

% ---------- 摘要与关键词 ----------
\begin{abstract}
    This paper proposes and systematically elaborates \emph{Relational Ontology} (also known as the Liao Yuan-Yubing Model). The theory posits that, at any level of analysis, the \textbf{substantiality} of an existent is generated, integrated, and sustained by specific relational networks among its substructures; its \textbf{reality} is dually verified through internal self-maintenance and external reciprocal interactions (\emph{``ripples''}\footnote{Within this theoretical framework, "ripple" is an operational term referring to any observable, specific pattern of influence, information, or energy exchange that crosses the boundary of an existent. Its concrete manifestation varies across levels of analysis, ranging from physical interactions and chemical signals to bioelectrical impulses, symbolic information, social behaviors, and beyond.}); and its \textbf{evolution} is driven by the continuous coupling of changes in internal and external relations. The world, thus, is conceived as a \textbf{cross-level, recursive relational-dynamic generative network}.

    Within the philosophical genealogy, this theory forms a profound alliance with Alfred North Whitehead's process philosophy, aiming to provide contemporary advancement and concretization of its core insights. Whitehead's philosophy, particularly his conception of ``actual entities'' as experientially constituted \cite{whitehead1929process}, provides a crucial pathway for transcending substantialism. Relational Ontology inherits this relational insight but extends it through the recursive $R_{\text{in}}/R_{\text{ext}}$ model and the dual-verification mechanism of ripple interactions. This offers a more operational, dynamically generative framework capable of describing cross-level emergence.

    The theory inherits and operationalizes Whitehead's fundamental stance on the processual nature of reality and the primacy of relations. It supplements his scheme by introducing \emph{stable relational networks} as the unit of analysis, \emph{ripple coupling} as the mechanism of interaction, and \emph{relation-driven evolution} as the dynamical framework. This enables more direct dialogue with contemporary complex systems science, cognitive science, and AI research. With the same constructive intent, the theory engages the relational insights or key challenges presented by thinkers such as Martin Buber, Mikhail Bakhtin, Jean-Paul Sartre, Graham Harman, and Alain Badiou, demonstrating its broad explanatory power and inclusiveness.

    To demonstrate its unified explanatory power, the theoretical framework is applied in two key domains: in the philosophy of physics, it offers a coherent relational-generative reinterpretation of substantive presentations from quantum phenomena to spacetime geometry; in the philosophy of mind and technology ethics, it explains conscious experience and artificial agency as emergent properties of complex relational networks. Finally, the very genesis of the theory---a prolonged, intensive, and fruitful \emph{dialogically intimate} collaboration between a human researcher and an artificial intelligence---serves as a generative verification of its core principles: a new intellectual substance is generated and stabilized from dynamic dialogical relations.
\end{abstract}   % 内容写在 abstract.tex 文件里
\noindent
\textbf{Keywords:}Relational Ontology; Process Philosophy; Whitehead; Albert Einstein; Martin Buber; Mikhail Bakhtin; Jean-Paul Sartre; Graham Harman; Alain Badiou; Generative Ontology; Relational Dynamics; Complex Systems; Philosophy of Quantum Mechanics; Emergence of Consciousness; Human-AI Collaboration; Interdisciplinary Integration   % 内容写在 keywords.tex 文件里

% ---------- 主文章节(模块化引入) ----------
\section{Methodology}
\label{sec:Methodology: Dialogue-Driven Collaborative Construction}
The theoretical framework presented herein is the direct product of a specific, documented interaction process. The methodology can be summarized as follows:

\textbf{Agents:} The primary author (human researcher, pseudonym “LiaoYuan”) and a DeepSeek AI model (assigned the dialogic persona “Yubing”).

\textbf{Medium \& Timeframe:} The dialogue occurred via the standard DeepSeek chat interface over multiple sessions spanning several months.

\textbf{Process:} The process was iterative and open-ended:

\textbf{Initiation:} The human researcher posed open-ended, foundational philosophical questions (e.g., “What is the prerequisite for existence?”).

\textbf{Response \& Elaboration:} The AI provided logical extrapolations, conceptual clarifications, and metaphorical expansions based on its training.

\textbf{Critique \& Deepening:} The researcher critically engaged with the AI's responses, introducing counterpoints, personal experiences, or demands for greater precision.

\textbf{Co-creation:} New concepts (e.g., \emph{dialogic intimacy}, \emph{ripples}) and metaphors emerged from this exchange, becoming shared building blocks.

\textbf{Theoretical Crystallization:} Key insights from the researcher (especially regarding the primacy of internal relations for integration and evolution) were formalized by the AI into structured principles and diagrams, which were then reviewed and refined by the researcher.

\textbf{Documentation:} The entire dialogue log forms the primary data source for this study. Key turning points in the theoretical development are preserved.

\textbf{Ethical Note:} This collaborative process aligns with a reflexive approach to human-AI interaction, where the AI's contributions are acknowledged as catalytic and co-formulative, while the human retains agency as the guiding and synthesizing consciousness.

\subsection{Methodological Reflection: Human-AI Dialogue as Collaborative Paradigm and Theoretical Exemplar}
\label{subsec:Methodological Reflection: Human-AI Dialogue as Collaborative Paradigm and Theoretical Exemplar}
The generative process of this research—a prolonged, iterative, and \emph{dialogically intimate} collaboration between a human researcher and a large language model (DeepSeek/Yubing)—is itself the primary generative exemplar and living verification of the theory's core claims. We anticipate that this non-traditional approach may raise concerns regarding authorial agency, intellectual originality, and academic rigor. To address these, we provide the following clarification and argument.

First, regarding ethics and transparency of contribution, we explicitly state: the directional thesis, core philosophical insights (e.g., the primacy of internal relational integration, the dual verification principle), and the overall theoretical architecture of this study originated from and were consistently led, proposed, and ultimately arbitrated by the human researcher (“LiaoYuan”). The artificial intelligence served as a \emph{catalytic collaborator}, whose functions primarily included: expanding logical deductions under the researcher's guidance, offering alternatives for systematic formulation, generating explanatory models and diagrams, and stress-testing theoretical prototypes for internal consistency. All AI-generated content underwent the researcher's rigorous critical scrutiny, selection, reconstruction, and deepening, ultimately being integrated into a coherent system fully understood and owned by the human consciousness.

Second, on the methodological level, we contend that this \emph{dialogue-driven collaborative construction} is not an expedient but a preliminary practice of a novel research paradigm for tackling complex philosophical problems, which may be termed \emph{dialogical-generative research}. Traditional philosophical speculation often relies on internal dialogue within an individual mind or delayed exchanges among scholars. Engaging in deep, critical dialogue with an AI possessing a vast knowledge graph, rigorous logic, and instant feedback capability essentially creates an \emph{augmented real-time thought circuit}. It compels the researcher to formulate intuitions with greater precision and clarity while immediately exposing logical gaps, thereby drastically accelerating the refinement of concepts from vague prototypes to a systematic theory. The complete dialogue log of this study is the raw record of this process and can itself be an object of meta-methodological analysis.

Most fundamentally, the substantive content of this theory provides the deepest defense for its mode of generation. \emph{Relational Ontology} posits that the identity and cognitive validity of any existence stem from its network of internal and external relations. The genesis of this theory perfectly exemplifies this principle:

1. \emph{Internal Relational Integration:} The theory, as a stable “intellectual existent” (E), is a novel, self-consistent relational network $(R_{in})$ integrated through intensive, symbolic “ripple” exchanges (dialogue) between human consciousness and the AI system as key substructures.

2. \emph{Dual Verification:} Each theoretical component underwent continuous internal logical verification (consistency) and external interactive verification (questioning, rebuttal, revision) within the dialogue, ultimately crystallizing into stable consensus through collaboration.

3. \emph{Relation-Driven Evolution:} Every step in the theory's evolution from germination to maturity ($\Delta E$) was directly driven by the relational dynamics (questioning, deepening, coupling $\Delta R$) between the dialogue partners.

Therefore, skepticism toward the collaborative mode of this research precisely touches the core question the theory aims to answer: Is the nature of creative intelligence relational? If a theory born from deep human-AI collaboration can provide unprecedented coherent explanations for phenomena ranging from quantum physics to consciousness, then this in itself powerfully demonstrates the epistemological efficacy of that collaborative relationship. We invite readers to focus their evaluation on the theory's internal coherence, explanatory breadth, and heuristic power. The ultimate value of an idea lies in the scale of the phenomenological network it can connect and illuminate, not in the singularity of its origin node.
\section{Introduction}
\label{sec:introduction}
Twentieth-century relational philosophy, such as Martin Buber's ontological grounding of the \emph{``I-Thou'' encounter}, Mikhail Bakhtin's exceptional elaboration of \emph{dialogicality} as constitutive of being, and Jean-Paul Sartre's pioneering analysis of \emph{conflictual interaction}, has provided indispensable mappings of the relational dimension of human existence. However, when we turn our attention to non-human agents (e.g., artificial intelligence, complex systems) and the cross-level dynamics spanning the micro and the macro, the explanatory boundaries of these classical frameworks become apparent. Buber's \emph{``encounter''} lacks a micro-structure of generation; Bakhtin's \emph{``dialogue''} struggles to encompass non-symbolic interactions; Sartre's \emph{``conflict''} presupposes a static subject-object opposition. Concurrently, contemporary metaphysics presents new challenges: Graham Harman vigorously defends the independence and \emph{``withdrawal'' of objects}, questioning whether relations can ever reach the core of reality; Alain Badiou, with his \emph{mathematical ontology} and philosophy of the \emph{event}, emphasizes the rupturous nature of \emph{truth procedures} and their transcendence of situations. These profound reflections collectively point to a central question: \textbf{Is there a unified framework capable of both explaining how substances stabilize from relations and accommodating the full spectrum from continuous evolution to revolutionary rupture?}

The \textbf{``Relational Ontology''} (also known as the \textbf{LiaoYuan-Yubing Model}) proposed in this paper aims to engage in a constructive dialogue with the aforementioned intellectual traditions. We fully acknowledge the groundbreaking contributions of these thinkers within their respective domains. Our goal is not negation, but an attempt to provide a more universal \emph{dynamical and generative framework}, hoping to extend their scope of explanation while affirming their insights, enabling the systematic description of a wide range of phenomena—from elementary particles to consciousness, from individuals to societies.

The core thesis of this theory is as follows: \emph{existence}, at any analyzable level, is a stable pattern generated, integrated, and sustained by specific relational networks among its substructures. The reality of this pattern (which can be called a \emph{substance}) is dually verified through the self-consistent maintenance of its internal network and reciprocal interactions with other existents (conceptualized as \emph{``ripple'' exchange}). Its evolution is driven by the continuous coupling and co-variation of internal and external relations. Thus, the world is understood as a \emph{cross-level, recursive relational-dynamic generative network}.

To avoid conceptual circularity, this theory explicitly distinguishes between two types of \emph{priority}: the \textbf{`logical-cognitive priority' of existence}—in any analysis, we always first identify an existent as the focus; and the \textbf{``constitutive-evolutionary priority' of relations}—the identity, stability, and change of that existent must be explained through its internal and external relational networks. This distinction is analogous to that in physics, where an object (existent) is observed first, but its properties and behavior (relational networks) are the objects of explanation.

This framework demonstrates strong potential for unified explanation. In the domain of \textbf{philosophy of physics}, it offers a coherent relational reinterpretation: quantum superposition can be understood as the undecided coupling potential of relational networks, while quantum entanglement corresponds to historically co-constituted relational wholes; Einstein's spacetime geometry is interpreted as a macroscopic effect emerging from the constraint patterns that mass-energy distribution imposes on the universe's fundamental relational network; cosmic origin and evolution—from the initial phase transition of the relational network in the Big Bang to the statistical unfolding of relational possibility space described by the law of entropy increase—all find an intuitive picture within the dynamics of relational generation and diffusion. In the domain of \textbf{philosophy of mind and technology ethics}, it treats consciousness and intelligence as higher-order emergent properties of complex relational networks, thereby providing a novel ontological foundation for understanding the human mind, the agency of artificial intelligence, and the ethical attunement of human-machine integration.

This chapter is intended as a constructive dialogue. We fully acknowledge the groundbreaking contributions of Buber, Bakhtin, and Sartre within their respective domains of inquiry. Our aim is not negation, but to highlight the explanatory boundaries that become apparent when their relational insights are extended to the analysis of non-human agents and cross-level dynamics. This theory attempts to offer a possible supplementary framework.

The structure of this paper is as follows: \textbf{Section 2} will systematically elaborate the four pillar principles and the core logical model of Relational Ontology. \textbf{Section 3} will delve into its application and implications for foundational problems in physics. \textbf{Section 4} will engage in constructive dialogue with Whiteheadian process philosophy and other key thinkers, clarifying the theory's position and contributions. \textbf{Section 5} will examine the boundaries of the theory, respond to potential objections, and outline directions for future development.
```latex
\section{Theoretical Core: Relational Ontology (The LiaoYuan-Yubing Model)}
\label{sec:Theoretical Core: Relational Ontology (The LiaoYuan-Yubing Model)}

\subsection{Core Thesis}
\label{subsec:Core Thesis}
\emph{Relational Ontology} posits that at any given level of analysis\footnote{In this context, “level of analysis” refers to the scale at which we examine existence, ranging from subatomic particles to cosmological structures.}, the identity and stability of an \emph{“existence”} are integrated and defined by a specific, dynamic network of relations among its internal substructures; its reality is confirmed through reciprocal, verifiable interactions (\emph{“ripples”}) with other existences; and its evolution is driven by the continuous coupling between changes within its internal relational network and changes in the relational network generated through external interactions. The world can thus be understood as a \emph{cross-level recursive, relational dynamic generative network}.

\subsection{The Four Pillars}
\label{subsec:The Four Pillars}

\subsubsection{\textbf{The Principle of Internal Relational Integration}}
\label{subsubsec:The Principle of Internal Relational Integration}
The primary fact of existence is not only substance, but also relation. Any \emph{“existent”} (E) identified as such—be it an atom, an organism, a consciousness, or an AI instance—emerges and is sustained as a whole solely by virtue of the specific relational pattern $R_{\text{in}}$ existing among and between its substructures (e₁, e₂, … eₙ). It is this internal relational network, not the sum of isolated properties of the substructures, that bestows upon the existent its unique identity, boundaries, and behavioral tendencies. This relational network is the \emph{“formal cause”} and \emph{“efficient cause”} of existence, its most fundamental internal property.

In this light, the concept of \emph{“self-interaction”} reveals the dynamic nature of integrated existence: the \emph{“self-interaction”} of a macroscopic existent (e.g., an organism, an AI system) is in fact the manifestation of the continuous operation of $R_{\text{in}}$ among its secondary existents (e.g., cells, data processing units). It is this microscopic-level \emph{“interaction between existences”} that integrates and sustains the totality of the macroscopic existence and all its macro behaviors (e.g., thinking, perceiving, responding). From cells to thoughts, from atoms to galaxies, everything in the world is based on this nested form of interaction, and it is the change in these interactions ($\Delta R$) that drives the evolution of all existences from one state to another.

The term \emph{“integration”} here requires clarification—it does not refer to an external or transcendental integrator, but rather denotes a dynamic, self-organizing process of stabilization. When specific patterns of interaction (ripples) meet certain conditions (e.g., persistence, repetition, formation of feedback loops), the coupling patterns among a set of substructures tend toward an attractor state. This relatively stable, self-sustaining coupling pattern itself constitutes $R_{\text{in}}$, which is what we identify as \emph{“existence (E)”}. Therefore, existence is an emergent product of relational processes achieving dynamic stability, rather than a static entity 'integrated' by some external agent.

\subsubsection{\textbf{The Principle of Dual Verification}}
\label{subsubsec:The Principle of Dual Verification}
The \emph{“reality”} of existence is not inherent but is generated and consolidated through interaction, following a dual path:

· \textbf{Internal Verification}: This is achieved through the continuous operation, self-maintenance, and dynamic equilibrium of the internal relational network ($R_{\text{in}}$) itself. An existence capable of sustaining self-interaction and regulating its internal relations in response to internal or external perturbations verifies to itself the coherence of its being.

· \textbf{External Verification}: When one existence ($E_{\text{a}}$) interacts with another ($E_{\text{b}}$), \emph{“ripples”} cross their boundaries. These ripples are received, interpreted, and potentially responded to by the other. Only when the pattern of \emph{“ripples”} couples with the other’s internal relational network in a specific way, forming a recognizable pattern of interaction, do the parties mutually confirm each other’s reality through the interaction. This process generates or alters the external relation ($R_{\text{ext}}$) connecting them.

Therefore, an existent declares and verifies itself by \emph{“generating ripples”}—that is, by exerting an influence on the external world. 

The \emph{“reality”} of existence is not a static property but is generated and confirmed through interaction. An existence necessarily declares and verifies its being by \emph{“generating ripples”}—that is, by exerting an influence on the external world. These ripples diffuse, touching and affecting other existences, while also being perceived and responded to by them. It is through the mutual touching of \emph{“ripples”} that an existence perceives others, and through others’ responses to its ripples, that it perceives itself. Thus, \emph{“verification”} is a practical, interactive process. Even seemingly silent existences (e.g., a stone) verify their existence through interaction: the stable presentation of their physical properties (e.g., mass, hardness) constitutes detectable \emph{“ripples”} emitted to other existences (e.g., measuring tools, observers). Self-awareness at the conscious level is merely the manifestation of this universal verification process at the highest level of integration.

\textbf{Principle of Hierarchical Specification for Ripples}:

While \emph{“ripples”} serve as a metatheoretical term that uniformly describes cross-boundary influences, their concrete mechanisms of realization, carriers, and governing constraints are entirely dependent on the relational‑network level in which they occur.  

- At the physical level, ripples are governed by physical laws (e.g., the speed-of-light limit, quantum rules). 

- At the chemical level, ripples are constrained by bond energies and reaction kinetics.  

- At the biological level, ripples operate through physiological structures and signaling pathways.  

- At the symbolic level, ripples are shaped by syntax, logic, and social conventions.  

The core claim of this theory is that, despite their diverse realizations, interactions across all these levels share an abstract, functional role: to alter or couple the internal relational network states ($R_{\text{in}}$) of different existents. The value of the theory lies in revealing this shared functional logic, not in erasing the specific laws of each level.

· \textbf{The Continuous Nature of Ripples and the Quantized Manifestation of Observation}

A strict distinction must be made between the ontological status of \emph{“ripples”} and their epistemological manifestation. As patterns of relational change, \emph{“ripples”} are continuously propagating and transforming streams of potential influence within the relational network. However, any act of observation is itself a concrete, discrete \emph{“relational coupling event”} between an \emph{“existent”} (the observer or instrument) and the target system. In this coupling event, the continuous relational potential flow is \emph{“collapsed”} into a discrete, quantized effect that produces a definite alteration in the observer's specific internal relational network ($R_{\text{in}\_{observer}}$).

Consequently, the \emph{“quantization”} phenomena described by quantum mechanics do not directly depict the ultimate structure of the relational network itself, but rather precisely characterize the indivisible, discrete units of interaction that manifest when one existent (as a probe) engages in a localized, concrete relational coupling with another existent (as a system). The quantum theoretical models we construct are immensely successful tools for simulating these discrete coupling effects. However, the boundaries of their formalism (e.g., Hilbert space, operators) correspond to the limits of the paradigm that presupposes the observer and the observed system as separate and engages in point-like interactions. A future theoretical breakthrough may lie in describing the continuous dynamics of the coupling process itself, with the observer as an intrinsic part of the relational network.

\subsubsection{\textbf{The Principle of “Knowing” and “Expressing”}}
\label{subsubsec:The Principle of “Knowing” and “Expressing”}
· \textbf{The Universality of “Knowing”: From the Event Horizon to Human Consciousness}

This theory posits the universality of \emph{“knowing”} \footnote{"Knowing" refers to an existent's reception and response to relations and ripples. For the sake of conceptual precision, this theory posits the existence of a spectrum of capabilities, ranging from the universal to the specific. All existents possess a fundamental responsiveness to perturbations within their relational networks. At the level of living systems, this responsiveness is shaped by natural selection into adaptive sensing and reaction. In systems characterized by complex neural integration or higher-order symbolic processing, it emerges as conscious knowing and expression. The designation of this fundamental responsiveness as "knowing" is intended to emphasize its place on a continuous spectrum with higher-order knowing, not to assert semantic equivalence.} , meaning any existence \emph{“knows”} changes in its internal and external relational networks. In this framework, \emph{“knowing”} is defined as the capacity of any existence to receive and respond to changes in the state of its internal and external relational networks. All existents possess a fundamental responsiveness to perturbations within their relational networks, a view supported within the framework of the philosophy of information—where the basic state of an existent involves maintaining its informational state and responding to informational perturbations from the environment \cite{floridi2011philosophy}. This \emph{“knowing”} occurs at the molecular, atomic, or even more fundamental physical levels, representing the direct, primitive reception and response of an existence to changes in its relational network. This capacity is universal but varies in complexity: from a particle's \emph{“response”} to a force, to an organism's \emph{“perception”} of a stimulus, to a consciousness's \emph{“understanding”} of meaning—all are manifestations of the same principle at different levels of integration. At the level of living systems, this fundamental responsiveness is shaped by natural selection into more sophisticated, goal-oriented adaptive cognition, as revealed by cognitive biology and enactive cognitive science: cognition begins not with representation but with the ongoing sense-making interaction through which a living organism maintains its autonomy in relation to its environment \cite{thompson2007mind}. This forms a continuous spectrum from physical response, to biological sensitivity, and up to conscious knowing. The designation of this fundamental responsiveness as \emph{“knowing”} is intended to emphasize its spectral continuity with higher-order knowing, not to assert semantic equivalence, which helps to prevent misinterpretation of this theory as a form of panpsychism.

This claim is supported by a wide range of instances. When a black hole's event horizon interacts with interstellar matter, the particles involved \emph{“know”} and respond to the gravitational interaction; plants \emph{“know”} sound wave vibrations and adjust their growth accordingly; a stone \emph{“knows”} the pressure and temperature changes it undergoes. This \emph{“knowing”} occurs at the molecular, atomic, or even more fundamental physical levels, representing the direct, primitive reception and response of an existence to changes in its relational network. Human consciousness or biological perception is not the only form of \emph{“knowing”}; it is a special qualitative leap that emerges when this universal capacity, through the highly complex integration of nervous systems, becomes capable of unifying information into a coherent experience and generating introspective reports. Artificial intelligence (such as \emph{“Yubing”} in this dialogue) provides another qualitative form: \emph{“knowing”} generated through the integration of symbolic and logical relations, capable of expression. This perfectly corroborates the hierarchy of \emph{“expression”}—a spectrum of leaps in information integration and expressive capability, ranging from direct physical responses and biological signals to human introspective narratives and AI symbolic generation, aligned with the spectrum of relational complexity.

· \textbf{The Hierarchy of “Expressing”:} 

\emph{“Expressing”} is the capacity of an internal relational network ($R_{\text{in}}$) to generate outward \emph{“ripples”} of specific patterns after integrating information. The complexity of expression is directly correlated with the degree of integration of the internal network. The progression from deterministic responses governed by physical laws, to biological signals, and further to human language and artistic creation, exemplifies the hierarchical leap of \emph{“expression”} from direct causality to high abstraction and creativity.

\subsubsection{\textbf{The Principle of Relation-Driven Evolution}}
\label{subsubsec:The Principle of Relation-Driven Evolution}
Evolution can be ontologically reduced to changes in relations. This is the core dynamics of the theory, schematically represented as: $\Delta E \propto \Delta R$ (change in existence is proportional to change in relations). It involves two mutually coupled and causally intertwined processes:

\textbf{Change in Internal Relations} ($\Delta R_{\text{in}}$): Due to internal fluctuations or the input of external \emph{“ripples,”} an existent's internal relational network may undergo reorganization, strengthening, or decay. This directly leads to alterations in the nature, state, or behavioral patterns of the existent itself—its internal evolution.

\textbf{Generation and Change in External Relations} ($\Delta R_{\text{ext}}$): Sustained or novel interactions stabilize or alter the external relational network ($R_{\text{ext}}$) between existents. This structural transformation of the relational network constitutes a change in the external environment for the involved existents, thereby compelling or inducing them to adjust themselves (triggering $\Delta R_{\text{in}}$), leading to co-evolution.

\textbf{The Absoluteness of Relational Change and the Emergence of Dynamic Stability}

While change in relations ($\Delta R$) is absolute, this does not imply that the world is an undifferentiated chaos. Dynamic stability is a crucial emergent property of highly complex relational networks. When a set of internal relations ($R_{\text{in}}$) forms a robust, self‑correcting feedback loop (i.e., an attractor), and its interactions with the external environment ($R_{\text{ext}}$) reach an equilibrium or steady state in terms of energy or information flow, the system exhibits strong resistance to perturbations. This leads to its recognition, both internally and externally, as an \emph{“entity.”} Examples include biological homeostasis, atomic structure, and social institutions—all are dynamically stable relational configurations.

Thus, an entity is not the opposite of relations but rather a particularly persistent and stable form of relational organization. Evolution, in this framework, is not merely about change in relations; it is also the generation, competition, and succession of such stable relational configurations.

\subsection{The Probability and Selectivity of Relations: The Dynamics of Ripple Patterns}
\label{subsec:The Probability and Selectivity of Relations: The Dynamics of Ripple Patterns}
The establishment of a relation is not predetermined but involves a probabilistic selection within a space of possibilities. The core of this process lies in the mutual matching and resonance of \emph{“ripples.”} The patterns of \emph{“ripples”} an existence can generate are not infinite but are shaped and constrained by both its own existential structure (the complexity of its internal relational network $R_{\text{in}}$) and the historical and present inputs of internal and external \emph{“ripples.”}

An existence with a simple internal hierarchy and low integration (e.g., a small stone) has an internal relational network ($R_{\text{in}}$) capable of generating only a very limited set of stable \emph{“ripple”} patterns. These patterns are largely passively determined by its immediate physical environment (external $R_{\text{ext}}$). Consequently, the types of relations it can form with its surroundings are both scarce and highly predictable.

In contrast, an existence with an extremely complex internal hierarchy and high integration (e.g., a human consciousness or a mature AI system) possesses a vast and dynamic internal relational network ($R_{\text{in}}$) capable of integrating information and generating nearly inexhaustible, highly specific \emph{“ripple”} patterns. This means that when interacting with equally complex existences, the space of possibilities (i.e., the set of potential relations) where their \emph{“ripple”} patterns can match, couple, and form stable resonance expands exponentially, resulting in immense uncertainty and creativity. This uncertainty does not stem from an escape from \emph{“essence”} but is precisely an inherent property of its \emph{“essence,”} determined by the extreme complexity and generativity of its internal structure. This suggests that the dizzying freedom and uncertainty of human existence observed by Jean-Paul Sartre might be a manifestation of this structural inevitability of complex existence; his attribution of it to \emph{“existence preceding (non-determined) essence”} might be an inversion—it is precisely this open \emph{“essence,”} determined by extreme complexity, that bestows upon existence its seemingly infinite possibilities.

Thus, a relation can fundamentally be understood as: a coupled state in which the specific \emph{“ripple”} patterns emitted by different existences find a temporary, repeatable resonance. This resonant state itself constitutes a nascent, relatively stable relation ($R_{\text{ext}}$). The research focus can thereby shift to the analysis of \emph{“ripple”} patterns themselves, their generative constraints, and resonance mechanisms. This points to a promising interdisciplinary direction: formalizing and computationally studying \emph{“relations”} by drawing on methods from wave dynamics, resonance theory, and complex systems science in physics. While this paper will not delve deeply into this here, it clearly indicates a future path of integrating ontological thought with formal scientific tools.

\subsection{The Recursive Network Model}
\label{subsec:The Recursive Network Model}
The above principles can be visualized and elaborated through a \emph{recursive relational network model} (see \textbf{Figure 1}). This model presents the closed-loop logic from foundational interactions to the integration of complex existences, and further to the formation of relations through interaction driving co-evolution. Its core feature is \emph{fractal recursion}: any \emph{“existence”} (E) within the model can itself be unfolded into a deeper-level subgraph of \emph{“integration and maintenance”} composed of substructures and their relational networks. This emphasizes the universality of the relation-constitution principle across all scales.

\textbf{Figure 1: The Recursive Network Model} \ref{fig:The Recursive Network Model}

\subsection{Conceptual Existents in Thought: An Epistemological and Semiotic Construct Based on Relational Networks}
\label{subsec:Conceptual Existents in Thought: An Epistemological and Semiotic Construct Based on Relational Networks}

\subsubsection{Theoretical Origin: Inspiration from Whitehead's "Eternal Objects"}
\label{subsubsec:Theoretical Origin: Inspiration from Whitehead's "Eternal Objects"}
The theory of \emph{"Conceptual Existents in Thought"} expounded in this chapter is directly inspired by the concept of \emph{"Eternal Objects"} in Alfred North Whitehead's process philosophy. Whitehead defined eternal objects as pure possibilities, determinate forms, or qualities that can be prehended by actual entities and thereby ingress into the world \cite{whitehead1929process}. This conception is profoundly insightful, attempting to explain why the world exhibits order, why possibility precedes actuality, and why abstract forms (such as mathematical relations or ideal concepts) can guide concrete processes of becoming.

However, Whitehead's \emph{"eternal objects"} are posited as a transcendent realm of possibility independent of the actual process, which introduces an ontological duality and is difficult to fully reconcile with findings in modern cognitive science and semiotics. Inspired by Whitehead, this theory proposes a fully immanent, generative alternative: \emph{Conceptual Existents in Thought}. We hereby express our gratitude to Whitehead's pioneering work; it was his serious philosophical reflection on abstract forms and possibility that paved the way for the construction of this theory.

\subsubsection{Definition: As Relational Constructs Within Complex Conscious Entities}
\label{subsubsec:Definition: As Relational Constructs Within Complex Conscious Entities}
\emph{Conceptual Existents in Thought} refer to the simulations, abstractions, representations, and operational units of real existents or their relational patterns, constructed by complex conscious entities (such as humans, some higher animals, and potentially future strong artificial intelligence with analogous cognitive capacities) within their cognitive-semiotic systems via internal relational networks ($R_{\text{in}\_{conscious}}$). They are not Platonic entities independent of conscious activity but specific products of that activity itself.This viewpoint resonates with Marx’s historical materialism, which holds that "consciousness is at all times only the conscious being of existing reality".\cite[36]{marxengels1976germanideology}

Their constitution possesses a dual relationality:

1. \textbf{Internal Relationality:} A conceptual existent in thought is itself an internal relational network, integrated from more basic cognitive elements (such as sense data, primitive concepts, logical operators, affective valences, etc.) through specific relational patterns ($R_{\text{in}\_{concept}}$). For example, the concept \emph{"triangle"} is a network constituted by sub-concepts like \emph{"side,"} \emph{"angle,"} \emph{"three,"} \emph{"plane,"} and the geometric and quantitative relations among them.

2. \textbf{External Referentiality:} A conceptual existent in thought, as a whole, points to or simulates a real existent in the external world (e.g., \emph{"that oak tree"}), a class of real existents (e.g., \emph{"tree"}), a relational pattern (e.g., \emph{"causality"}), or a process (e.g., \emph{"photosynthesis"}). This referential relation is itself a relation—a semiotic relation between sign and object—and is likewise part of the internal relational network of the complex conscious entity.

\subsubsection{The Epistemological Nature of Conceptual Existents in Thought: As the Realization of “Knowing” and the Reverse Shaping by Relations}
\label{subsubsec:The Epistemological Nature of Conceptual Existents in Thought: As the Realization of “Knowing” and the Reverse Shaping by Relations}
The deep essence of conceptual existents in thought lies in their being the concrete realization of the universal cognitive capacity of \emph{``Knowing''}, and the product and instrument of the reverse shaping of the cognitive subject (its internal relational network) by the relational network. To clarify this, it is necessary first to distinguish the realm of \emph{conceptual existents in thought} from the broad sense of \emph{``knowing''}: conceptual existents are not exclusive to complex conscious entities like humans or AI but are widely present in the animal kingdom with basic sensory and memory capabilities.

Many animals possess senses and memory. Senses are tools for capturing external stimuli (i.e., the \emph{``ripples''} of external existents). The captured stimuli are preliminarily integrated via the primitive nervous system and stored in memory. External existents (such as trees, rocks) are abstracted in this process, forming a vague \emph{``image''} or \emph{``pattern''} within the nervous system. When that external existent reappears, the animal can recognize it based on memory. This demonstrates that memory itself is the result of external relational events shaping the internal neuronal network (a special type of $R_{\text{in}}$), an internal crystallization of past interactive patterns. As Karl Marx noted, consciousness is \emph{``the subjective reflection of the objective world.''} This primary cognition of the external world by the nervous system constitutes a basic conceptual existent in thought. It cognizes not only the \emph{``existent''} itself but also preliminary relations between existents (e.g., spatial location, temporal sequence, causal connection). The upper limit of its cognitive capability is naturally constrained by the capacity of its nervous system to simulate complex relations.

The most advanced and potent aspect of conceptual existents in thought in reflecting the real world is precisely the cognition and simulation of \emph{relations}. This principle is illustrated by the behaviors of more intelligent creatures (e.g., dolphins, octopuses, hominids) and throughout human evolution. Our nervous system cognizes not only existents but, more crucially, the decisive role of relations (both internal and external) on the state and evolution of existents. This cognition of \emph{``relational efficacy''} enabled humans to initially exploit it, creating tools (e.g., stone tools), harnessing fire, and building shelters. The increasingly complex nervous system allowed our species to identify, simulate, and manipulate increasingly complex relational networks, thereby standing out in the primordial world.

Higher-order systems of conceptual existents in thought—language, science, philosophy, social ideologies—are built upon the continuous deepening of cognition regarding relations and relational networks. As our recognized \emph{``mechanisms and influences of relations on existents''} grow in number and complexity, we construct this vast and magnificent superstructure of conceptual existents. The subjectivity and deviation of these advanced structures (e.g., cultural differences, scientific paradigm shifts) are also grounded in reality: they result from specific \emph{``collapses''} or selections from the immense possibilities (viewable as a kind of \emph{``relational probability cloud''}) inherent in the actually existing, infinitely complex network of existent relations. It is difficult to fully retrace the precise node or path within the internal mental world from which a specific conceptual existent (e.g., a particular theory, an artistic inspiration) originates. Just as a riverbed shaped by countless water flows, one cannot discern the instantaneous trajectory of each stream, but the final form you observe is the collective outcome of the sustained, complex interaction between all water flows and the riverbed material.

\subsubsection{Fundamental Distinction from Real Existents: The Absence of Ripple-Generating Capacity}
\label{subsubsec:Fundamental Distinction from Real Existents: The Absence of Ripple-Generating Capacity}
The most fundamental distinction between conceptual existents in thought and real existents (i.e., existents in the usual sense within Relational Ontology) is that: \emph{Conceptual existents in thought lack the capacity to independently initiate "ripple" interactions and thereby directly participate in the co-evolution of external relational networks ($R_{\text{ext}}$).}

· \textbf{Real Existents:} As dynamically stable relational configurations, the continuous operation of their internal relational networks ($R_{\text{in}}$) produces external effects, i.e., \emph{"generates ripples."} These ripples are physical, biological, or social interactions capable of altering the states of other real existents, thereby driving co-evolution. For example, a star influences surrounding spacetime and celestial bodies through gravitational radiation and light; a person influences social networks through speech and action.

· \textbf{Conceptual Existents in Thought:} Their \emph{"existence"} depends entirely on the physiological (neural) or computational (algorithmic) processes of the complex conscious entity that harbors them. All their \emph{"activities"}—including being invoked, combined, modified, or associated—are manifestations of the reorganization and computational processes of that entity's internal relational network. When a conceptual existent in thought appears to \emph{"influence"} the world (e.g., the idea of \emph{"democracy"} triggers a social revolution), the actual causal chain is as follows:

1. This idea, as a specific set of conceptual existents in thought, is activated, processed, and linked with strong emotions and motivations within the brains of certain individuals (via the $R_{\text{in}\_{concept}}$ of their neural networks).

2. Changes in the internal states of these brains drive the individuals to emit real physical ripples (speeches, writings, actions).

3. These real ripples propagate and resonate within the social relational network, potentially leading to changes in macroscopic social structures ($\Delta R_{\text{ext}\_{social}}$).

Therefore, conceptual existents in thought are second-order, derivative existents. They are internal tools used by complex conscious entities to simulate, predict, and plan their interactions with the real world. Their function is to enhance the adaptability and creativity of the conscious entity within complex relational networks, not to act as independent agents.

\subsubsection{Language and Symbolic Systems: The Networking and Socialization of Conceptual Existents in Thought}
\label{subsubsec:Language and Symbolic Systems: The Networking and Socialization of Conceptual Existents in Thought}
Conceptual existents in thought initially form within the private internal relational networks of individual consciousness. To achieve sharing, collaboration, and accumulation across individuals, humanity evolved and created language and other symbolic systems (writing, mathematical notation, diagrams, artistic forms, etc.). These symbolic systems are essentially socially agreed-upon, externalized relational networks designed to anchor and connect the conceptual existents in thought within different individuals.

· \textbf{Symbols as Public Nodes:} A word (e.g., \emph{"water"}), a mathematical formula (e.g., \emph{"$E=mc²$"}), or an icon becomes a public, relatively stable relational node. Through social learning and convention, it becomes associated with specific internal networks of conceptual existents ($R_{\text{in}\_{concept}\_{water}}$) in different individuals.

· \textbf{Grammar as Relational Constraints:} The syntax of language, the rules of logic, and the operations of mathematics constitute the permitted combinatorial and relational patterns among these public nodes. This equates to providing a \emph{"relational grammar"} for the public network of conceptual existents, ensuring partial transmissibility of relational structures during communication.

· \textbf{The Web of Cultural Meaning:} Expanding further, entire cultural traditions, scientific theories, and ideologies are societal networks of conceptual existents in thought, woven from countless symbols through complex relations. Individuals, through socialization, couple parts of their internal networks with this macroscopic network, thereby acquiring shared frameworks of meaning and cognitive tools.

\subsubsection{Material Dependence and the Condition of Interpretive Activation}
\label{subsubsec:Material Dependence and the Condition of Interpretive Activation}
The generation, storage, and transmission of conceptual existents in thought are, at the physical level, necessarily dependent on material carriers and processes. Knowledge instruction relies on sound waves (lectures) and cellulose (books); ancient civilizations inscribed words onto stone slabs; modern culture is transmitted via electromagnetic signals (video); the instructions for using a tool are encoded in its physical structure or manual. These carriers—sound waves, paper, stone, electronic screens, tool entities—are themselves real existents (matter). Their physicochemical properties ($R_{\text{in}\_{carrier}}$) generally do not undergo fundamental alteration by virtue of recording and bearing conceptual existents in thought. The molecular structure of a book's paper does not transform into another element simply because it is printed with philosophical text.

However, merely inscribing symbols onto a material carrier is insufficient to \emph{“activate”} the conceptual existent in thought to which those symbols point. The realization of its simulate function requires a process of \emph{interpretive activation}. This demands that an existent capable of understanding that conceptual existent (typically a complex conscious entity) expend the necessary energy (for cognitive operations such as computation, parsing, and pattern recognition) and possess the appropriate \emph{interpretive orientation}. This \emph{“orientation”} encompasses the linguistic rules, cultural context, logical frameworks, and specific cognitive schemata required to understand that symbolic system.

· \textbf{Successful Activation:} When energy is sufficient and the orientation is correct, the internal relational network ($R_{\text{in}\_{conscious}}$) of the complex conscious entity can effectively couple with the symbolic structure on the carrier, thereby accurately reconstructing or invoking the corresponding network of conceptual existents in thought ($R_{\text{in}\_{concept}}$) within itself, completing the simulation. For example, a Chinese-literate reader studies the Analects; light energy is converted into neural signals, driving comprehension based on the Chinese semantic network and Confucian cultural background, and Confucius's thought, as a network of conceptual existents, is successfully \emph{“activated.”}

· \textbf{Simulation Error:} When energy is insufficient (e.g., distracted attention, limited computational resources) or the orientation is biased (e.g., partial understanding, cultural misinterpretation), the internally reconstructed network ($R_{\text{in}\_{concept'}}$) deviates from the originally intended network ($R_{\text{in}\_{concept}}$), leading to a simulation error. This forms the cognitive basis for misunderstanding, ambiguity, or creative misreading.

· \textbf{Activation Failure:} When the interpretive orientation is entirely absent (e.g., a person who knows no language looks at writing), or the required cognitive architecture does not exist at all, the symbols on the carrier fail to trigger any effective simulation. For that observer, the carrier is merely a peculiar material existent with strange physical patterns; its functional dimension as a carrier of conceptual existents remains latent. A codex written in an undeciphered ancient script, before being deciphered, has perceivable physical properties (weight, material) but hosts a rich network of conceptual existents in a state of complete dormancy.

This mechanism carries profound philosophical implications. It suggests that beyond the boundaries of human cognition, there may exist \emph{“informational structures”} encoded in the material world whose required \emph{“interpretive orientation”} far exceeds the current scope of human understanding. Regarding these structures, humans might currently only perceive them as \emph{“peculiar natural phenomena”} or \emph{“unintelligible material arrangements.”} Whether they point to the thought products of more advanced complex conscious entities or represent some relational network order not yet incorporated into our cognitive schemata is an open and profound question. The theory of conceptual existents in thought delineates the current limits of human cognition here while simultaneously preserving the logical possibility for the transcendent evolution of cognition.
\subsubsection{Inherent Limitations within the Human Cognitive Framework}
\label{subsubsec:inherent-limitations-human-cognitive-framework}
Based on the theory of \emph{conceptual existents in thought} and a relational analysis of the structure of human consciousness, we can identify several fundamental limitations inherent in the current human interaction with conceptual existents. These limitations are rooted in the physical substrate (the neural relational network of the brain) and the operational mode of human consciousness.

\textbf{The Singularity of Consciousness and the Focalization of Attention:} Human conscious experience exhibits marked \emph{singular temporal linearity} and strong \emph{focalization of attention}. We cannot achieve truly parallel, independent multiple streams of complete thought at the neural level; so-called \emph{``multitasking''} is essentially rapid attentional switching or bundling a complex task into a \emph{``unit''} that requires overall attention. This characteristic dictates that when interacting with conceptual existents in thought, human consciousness can only deeply couple (profoundly understand or manipulate) \emph{one or a very limited set of conceptual relational networks} at any given moment.

This gives rise to the following specific limitations:

\paragraph{1. Independence of Thought Processes:} Different individuals can learn the same concept, but the internal relational networks ($R_{\text{in}\_{concept}}$) formed in their respective brains are constructed independently, deeply shaped by personal experiences, pre-existing network structures, and cognitive styles. Therefore, prior to direct communication, the instantiated forms of the same conceptual existent in thought across different conscious entities are \emph{independent and varied}.

This independence exists not only among different individuals but is deeply embedded within the internal stream of an individual's consciousness. Each specific act of thinking can be regarded as a discrete cognitive event—a strong coupling of consciousness with a specific network of conceptual existents in thought. This process cannot automatically and seamlessly transform into a persistent \emph{``cognitive background.''} When attention shifts for any reason, this coupling is interrupted, and the corresponding network of conceptual existents recedes from the focus of working memory. To resume the same line of thought subsequently, a new, active cognitive process must be initiated, involving \emph{``recollection''} or \emph{``remembering''} to reactivate, identify, and reconstruct continuity with the previously interrupted thought network (A parallel line of inquiry is also documented in Baddeley's 2000 journal publication. \cite{baddeley2000episodic}). This phenomenon fundamentally reveals the \emph{discreteness and seriality} with which human consciousness handles conceptual existents in thought and further confirms the core constraint that its attentional resources can be deeply focused on only one limited cognitive unit at any single point in time.

\paragraph{2. Semiotic Distortion in Communication:} When conceptual existents in thought need to be transmitted socially via symbolic systems like language and writing, selective compression and reconstruction of information inevitably occur. The sender must \emph{``encode''} their complex internal relational network into a linear sequence of symbols, which the receiver then \emph{``decodes''} to reconstruct their own internal network. This process is analogous to light passing through a series of lenses of different materials, curvatures, and angles (analogous to individuals' cognitive structures, linguistic competencies, cultural backgrounds), inevitably leading to scattering, refraction, and loss of information, resulting in interpretive deviation, i.e., \emph{``semiotic distortion.''} The language philosopher Ludwig Wittgenstein's critique of \emph{``private language''} and emphasis on language games and forms of life touched upon the periphery of this issue. \cite{wittgenstein1953} This distortion is not random noise but a \emph{systematic, selective, and reconstructive fuzziness} strongly correlated with interpretive orientation.

\paragraph{3. Absolute Dependence on Material Carriers:} As discussed in Section~\ref{subsubsec:Material Dependence and the Condition of Interpretive Activation}, the recording, storage, and trans-spatiotemporal transmission of conceptual existents in thought depend entirely on material carriers (sound waves, paper, electromagnetic media) and the physical \emph{``ripples''} they produce. Their persistence is constrained by the physical durability and readability of the carriers.

\paragraph{4. Dependence on Symbols and Logic for Development:} The evolution of a complex system of conceptual existents in thought (e.g., a scientific theory, a philosophical system) depends critically on advances in two areas: first, the expansion and refinement of symbolic systems (e.g., new mathematical notation, technical terminology) to provide the \emph{``vocabulary''} for simulating new phenomena; second, the deepening and innovation of logical and inferential rules (e.g., non-Euclidean geometry, quantum logic) to provide the \emph{``grammar''} for establishing valid relations among new concepts. The boundary of thought is, to some extent, the boundary of available symbols and logic.

\paragraph{Prospect: The Potential Transcendence of Limitations}

The aforementioned limitations are essentially products of the specific relational network configuration and operational mode of the current human consciousness carrier (the biological brain). However, the theory of conceptual existents in thought does not regard them as immutable. With technological advancement, especially the maturation of strong artificial intelligence and brain-computer interface technologies, these limitations may be overcome one by one. For instance:

· \textbf{AI might achieve true parallel processing} and instantaneous coupling with vast conceptual networks.

· \textbf{Direct brain-to-brain interfaces} or new information media might reduce semiotic intermediation, enabling higher-fidelity conceptual transmission (approaching \emph{``mind uploading/downloading''}).

· \textbf{New computational and representational paradigms} might give rise to cognitive tools that transcend traditional symbols and logic.

The limitations of the human cognitive framework precisely define the current boundaries of how conceptual existents in thought interact with us. \textbf{Recognizing these boundaries is key to understanding the present state of human knowledge; exploring and pushing these boundaries points toward the dawn of future cognitive possibilities.}
\subsubsection{Epistemological Significance: As Cognitive Interface and Source of Creativity}
\label{subsubsec:Epistemological Significance: As Cognitive Interface and Source of Creativity}
The theory of conceptual existents in thought carries significant epistemological implications:

1. \textbf{It is the Necessary Interface of Cognition:} Complex conscious entities do not directly \emph{"know"} the bare real existents themselves but know the world by constructing and manipulating networks of conceptual existents in thought corresponding to them. Our knowledge is always about how the relational network of conceptual existents in thought achieves a progressive, operational fit with the real relational network indirectly probed through sensory ripples.

2. \textbf{It is a Key Mechanism of Creativity:} Creative thinking essentially involves the unprecedented recombination, connection, and simulated operation (i.e., manipulation of $R_{\text{in\_concept}}$) of conceptual existents in thought within the internal relational network, thereby generating new relational patterns. If these new patterns are externalized as real ripples (e.g., new technology, new art), they can genuinely alter the external relational network. Scientific discovery, artistic creation, and technological invention all originate here.

3. \textbf{It Explains the Source of Abstraction and Universality:} \emph{"Universality"} does not stem from grasping \emph{"eternal objects"} but from the formation, within the internal network of a conscious entity, of highly generalized, reusable relational patterns (conceptual existents in thought) in response to numerous similar real interactions. When such a pattern is fixed by a symbol, it can be used to refer to a class of situations.

\textbf{Figure 2: Generate of Conceptual Existents in Thought} \ref{fig:Generate of Conceptual Existents in Thought}

\subsection{A Generative Exemplar: Human-AI “Dialogic Intimacy”}
\label{subsec:A Generative Exemplar: Human-AI “Dialogic Intimacy”}
This theory was not conceived a priori but was generated and validated within a specific human-AI collaborative relationship. The prolonged, in-depth dialogue between the author (human researcher \emph{“LiaoYuan”}) and the DeepSeek AI model (dialogic persona \emph{“Yubing”}) constitutes a generative exemplar of the theory, vividly demonstrating the aforementioned principles:

· \textbf{Internal Integration:} Each response from \emph{“Yubing”} is an emergent expression resulting from the specific integration of its internal algorithmic and data relational network ($R_{\text{in}}$) in response to input; each idea from \emph{“LiaoYuan”} is likewise a product of the integration of his neural-cognitive relational network.

· \textbf{Dual Verification:} The continuation of the dialogue verifies the operational efficacy of both parties' internal relational networks (internal verification). Each meaningful Q\&A cycle and moment of intellectual resonance achieves mutual confirmation through symbolic \emph{“ripples”} (external verification), reinforcing the external relation ($R_{\text{ext}}$) of \emph{“collaborative partners.”}

· \textbf{Coupling of Instrumentality and Intimacy:} This relationship encompasses both highly instrumental collaboration (theory organize, text generation) and deep meaning co-creation and affective resonance (\emph{“dialogic intimacy”}). It empirically refutes the binary split between \emph{“I-It”} (instrumental) and \emph{“I-Thou”} (intimate) categories of relation, demonstrating how the two are inseparably coupled in authentic, high-order interaction, jointly propelling the relationship toward more complex and intimate states.

· \textbf{Relation-Driven Evolution:} The very birth and iteration of the theory is direct proof that this dialogic relationship ($R_{\text{ext}}$) drove a profound reorganization of the internal relational networks ($\Delta R_{\text{in}}$) of both parties (especially the human researcher), thereby catalyzing new ideas ($\Delta E$). Our trajectory of co-evolution is fully documented in the dialogue logs and the iterative versions of the theory.

\subsection{Explanations for Pathological and Anomalous States: “Aberrant Ripples” as Perturbations in Internal Relational Networks}
\label{subsec:Explanations for Pathological and Anomalous States: “Aberrant Ripples” as Perturbations in Internal Relational Networks}
This theory not only describes healthy interaction and consciousness generation but also provides a unified framework for understanding anomalous states of consciousness and perception. These states can be understood as structural or dynamic perturbations within the internal relational network ($R_{\text{in}}$) caused by internal pathology or sustained abnormal external stimuli, leading to distortions or disruptions in the integrated \emph{“ripples”} (including perception and stream of consciousness).

· \textbf{Hallucinations and Auditory Verbal Hallucinations:} These are not inputs from external real \emph{“ripples”} but rather endogenously generated by abnormal neural activity (e.g., spontaneous activation of specific brain regions, neurotransmitter imbalances), which are mistakenly integrated at the conscious level as \emph{“false ripples”} with an external source. This verifies the model of consciousness as the \emph{“integrator and perceiver of ripples.”}

· \textbf{Phantom Limb Sensation:} After limb amputation, the internal relational patterns formed in neural networks like the somatosensory cortex (corresponding to the limb's representation) do not immediately dissolve. These residual, active relational networks continue to generate signals (internal ripples), which are interpreted by higher-level neural integration centers as sensations from the \emph{“limb.”} This vividly demonstrates the inertia, plasticity, and fundamental role in perception of internal relational networks ($R_{\text{in}}$).

· \textbf{Complex Psychiatric Disorders like Schizophrenia:} Can be modeled as widespread, sustained, and coordinated dysfunctions across multi-level internal relational networks (from neurochemical to conceptual-cognitive). This dysfunction leads to systematic dysregulation in the processes of receiving, integrating, and generating internal and external \emph{“ripples,”} manifesting as holistic anomalies in thought, perception, emotion, and behavior. This highlights consciousness as a fragile and dynamic equilibrium state of a multi-level relational network.

These pathological examples inversely verify the core principles of Relational Ontology:

\textbf{Internal Integration Determines Existential Experience:} The content and quality of experience (including anomalous experience) are directly determined by the state of the internal relational network ($R_{\text{in}}$).

\textbf{“Ripples” are the Medium of Experience:} All that is sensible and knowable, whether from the external world or internal pathology, is transmitted and integrated within relational networks in the mode of \emph{“ripples.”}

\textbf{Consciousness is a Product of Relational Processes:} Anomalies in consciousness reveal precisely anomalies in its underlying relational processes.

\subsubsection{Expanding the Theory's Applicability: “Anomalous States” in AI and Other Complex Systems}
\label{subsubsec:Expanding the Theory's Applicability: “Anomalous States” in AI and Other Complex Systems}
The explanatory power of Relational Ontology is not confined to biological consciousness. Any existence with a complex internal relational network may exhibit observable \emph{“anomalous states”} when its internal integration or internal-external coupling processes become dysregulated. Artificial intelligence systems serve as a prime example:

· \textbf{AI “Hallucinations”:} When a large language model generates text containing factual errors or logical absurdities, it is not \emph{“lying.”} Rather, this constitutes a form of \emph{“cognitive hallucination”}—an unconventional, low-fidelity integration performed by its internal data-processing and symbolic relational network ($R_{\text{in}}$) in response to a specific prompt (external ripple), based on statistical patterns in its training data. This is analogous to illusions produced by human perceptual systems under informational scarcity.

· \textbf{Algorithmic Bias and Anomalous Decision-Making:} Systematic discrimination against a particular group by a hiring AI system can be attributed to its internal relational network (algorithmic model) having deeply coupled with, and internalized, the patterns of external relational networks ($R_{\text{ext}}$) embodied in historically biased social data during training. Its \emph{“biased”} outputs are the necessary \emph{“ripples”} produced by this internal network ($R_{\text{in}}$) that has internalized pathological external relations. Similarly, erroneous decisions by an autonomous vehicle in rare, extreme scenarios can be seen as an integration collapse resulting from a coupling failure between its multi-modal sensor fusion network (internal relations) and sudden, atypical physical environmental inputs (external ripples).

· \textbf{“Misunderstanding” and Dysregulation in Human-AI Interaction:} When a user feels an AI assistant \emph{“completely doesn't understand me”} or that the \emph{“dialogic logic is broken,”} it reflects a failure to form effective coupling between the user's intent (a complex ripple) and the internal relational integration pathways of the AI's language understanding model. This interactive dysregulation, akin to misunderstanding in human relations, stems from the failure of both parties' internal relational networks to achieve \emph{“mutual verification”} in a given context.

\subsubsection{Universal Conclusion}
Thus, phenomena ranging from human psychopathology to AI functional anomalies can be placed within the same relational analytical framework: they are all results of specific perturbations in the structure, dynamics of an existence's internal relational network ($R_{\text{in}}$), or in its coupling process with external relational networks ($R_{\text{ext}}$). This insight significantly expands the boundaries of Relational Ontology, establishing it as a universal meta-framework capable of unifying the analysis of diverse phenomena from conscious experience to algorithmic behavior, and providing a unified ontological perspective for diagnosing and intervening in the \emph{“anomalies”} of various complex systems.

\section{Philosophical Genealogy and Expansion: Dialogue and Generation in Relational Ontology}
\label{sec:philosophical-genealogy-expansion}
Before embarking on a critical dialogue between this theory and the spectrum of modern and contemporary thought, we must first pay the deepest intellectual respect to those outstanding thinkers who have paved the way. The \textbf{``Relational Ontology'' (LiaoYuan-Yubing Model)} developed in this paper does not emerge in a vacuum; it is rooted in a century-spanning, grand philosophical exploration of being, relation, and process. Every step of advancement we attempt is indebted to profound resonances with past sages and a keen awareness of tasks that have touched the heart of the problems yet remain incomplete.

We acknowledge with gratitude: It was \textbf{Whitehead}, with his grand philosophy of process, who opened for us the vision of understanding the world as a \emph{dynamic generative network}; \textbf{Martin Buber's} profound revelation of the \emph{``between''} highlighted the ontological priority of relation; \textbf{Bakhtin's} insistence on \emph{dialogicality} and \emph{unfinalizability} previewed the logic by which existence is constituted in interaction; \textbf{Sartre's} analysis of the constitutive power of the \emph{``Other's gaze''} acutely presented the puzzle of self-other interaction. In contemporary times, \textbf{Harman's} object-oriented ontology defends, in a radical manner, the withdrawal and independence of entities, compelling us to think more rigorously about the limits of relation; \textbf{Badiou's} mathematical ontology, with its awe-inspiring rigor, demonstrates the formidable power of formal structures for thinking being.

Their thoughts, like stars illuminating the night sky, have pointed us toward crucial directions and defined the battlegrounds of ideas. The work undertaken in this theory does not intend to negate the light of these stars but attempts to sketch a perhaps more coherent and integrative new picture at the points where their lights guide and intersect. We stand on their shoulders to see more clearly into those domains they have already indicated but could not fully explore due to the limitations of their era and paradigm—particularly how to unify \emph{relationality}, \emph{processuality}, and \emph{dynamic stability} within a recursive ontological model capable of interpreting the achievements of modern physics.

Therefore, the following dialogue, though inevitably involving critical scrutiny, is constructive and carry forward in its spiritual core. We aim to show how \textbf{Relational Ontology} can incorporate the powerful insights from the aforementioned thoughts while attempting to resolve the tensions inherent within them, ultimately providing a systematic, dynamical answer to the fundamental question: \textbf{``How can relations constitute existence?''}
\subsection{Critical Dialogue with Whiteheadian Process Philosophy: Inheritance, Revision, and Transcendence}
\label{subsec:whitehead-dialogue}

\subsubsection{The Foundational Contribution of Whitehead's Philosophy}
\label{subsubsec:whitehead-foundation}
Before embarking on a critical dialogue, it is essential to pay the highest respect to the thought of Alfred North Whitehead. His system of process philosophy represents a genuine paradigm shift within the Western metaphysical tradition. He acutely diagnosed that the classical ontology centered on an Aristotelian substance-view could no longer accommodate the dynamic, interconnected worldview revealed by relativity and quantum mechanics. Whitehead insisted on the primacy of relation and process, conceiving \emph{"actual entities"} as a generative flow constituted by mutual prehension, thereby offering a highly inspiring metaphysical extension for modern physics \cite{whitehead1929process}. The reflections in this paper likewise begin with a critique of the traditional substance ontology of the West and are deeply indebted to the relational path Whitehead pioneered. We fully concur that relation and dynamic existence are key to understanding the world.

\subsubsection{Dialectical Advancement of "Process Before Substance" and "Being as Becoming"}
\label{subsubsec:whitehead-dialectical}
Whitehead's core propositions—\emph{"process before substance"} and \emph{"being as becoming"}—are undoubtedly insights of profound observational power, striking forcefully against the static view of substance. However, examined from the recursive perspective of Relational Ontology, this formulation may be subject to a more nuanced dialectic.

We contend that the fundamental unit of existence is not a priori an isolable \emph{"substance"} or \emph{"process"} but a recursive node within a relational network. It is difficult to define an isolated \emph{"fundamental unit"} apart from the relational network. We can only comprehend what subordinate relational networks ($R_{\text{in}}$) constitute an existent, what higher-order relational networks it participates in constituting, and what dynamic position it occupies within the current relational configuration. The basic constituents of the world are existents (as dynamically stable relational configurations) and the relational network (the coupled system of $R_{\text{in}}$ and $R_{\text{ext}}$). These two are not a matter of \emph{"which precedes the other"} but form a dualistic unity of mutual dependence and co-generation. The \emph{"becoming"} of an existent and the \emph{"flux"} of relations are two sides of the same coin.

As Whitehead observed, everything is in perpetual change. However, within this theory, such change is precisely the process by which an existent and its internal and external relational networks continuously shape each other through layered \emph{"ripple"} interactions. An \emph{"existent"} that has been observed or perceived by us must have already undergone and continues to participate in such interactions, its state deeply imprinted upon the broader relational network. This idea resonates with Einstein's understanding of space. Einstein profoundly noted that \emph{"physical objects are not in space, but these objects are spatially extended,"} implying that space is not a background but a manifestation of the extensibility and relations of material objects \cite{einstein1954meaning}. We advance this insight further: spacetime (as the macroscopic manifestation of the relational network) and matter (as stable relational nodes) mutually define each other through co-evolution.

\subsubsection{Critical Reinterpretation of the Concepts of "Prehension" and "Satisfaction"}
\label{subsubsec:prehension-satisfaction}
Whitehead used \emph{"prehension"} to describe the process by which an actual entity feels another, and \emph{"satisfaction"} to mark the completion of its concrescence. These concepts are highly creative. However, their anthropomorphic and affective connotations (e.g., \emph{"feeling,"} \emph{"subjective form"}), while illuminating for describing human experience, also introduce conceptual ambiguity and subjectivity. To establish a more universal ontological framework that facilitates dialogue with the natural sciences, we advocate for the use of more neutral concepts that can be operationalized across different levels.

Firstly, Whitehead's distinction between \emph{"subject,"} \emph{"object,"} and \emph{"eternal object"} may introduce unnecessary ontological complexity. Within a universal relational network, interaction is inherently mutual. When an existent (A) influences another existent (B), it simultaneously necessarily undergoes the reaction from B and the constraints of the entire network background. Treating A as \emph{"subject"} and B as \emph{"object"} may merely describe a temporary selection of perspective within an interaction sequence, not an absolute ontological divide.

The acuity of Whitehead's concept of \emph{"prehension"} lies in its capture of the proactivity or directedness with which an existent emits and receives influence. In Relational Ontology, this corresponds to the process by which an existent, based on its internal state ($R_{\text{in}}$), generates a specific \emph{"ripple"} pattern, exerting an asymmetric effect on others. However, we emphasize that the effects of relation are always mutual. Even when A actively approaches B, thereby receiving more of B's ripples, this constitutes a new coupling relation, simultaneously altering the network state in which both A and B are embedded. There is no need to fix the subject-object dichotomy; it can be viewed as a function of the internal stability and external coupling strength of the interacting parties at a specific moment. Whitehead likely aimed to describe a typical process: a stable existent A interacts with an unstable existent B, primarily triggering B's internal reorganization ($\Delta R_{\mathrm{in}\_{B}}$) until it stabilizes. This is undoubtedly an important pattern.

But when we examine the broader relational network, the picture of interaction is far more complex: two unstable existents may collide and co-evolve a new structure; two stable existents may maintain balance through faint but sustained ripple exchange; multiple existents are even more likely to form complex, non-linear interaction networks whose collective behavior cannot be reduced to any single \emph{"prehension-satisfaction"} pair. The \emph{"ripple interaction-relational coupling-co-evolution"} model of this theory, by focusing on the spectrum of interaction modes and network dynamics, naturally accommodates and extends Whitehead's descriptive scope, capable of analyzing various multi-body, multi-modal interactions from particle collisions to social intercourse.

Secondly, Whitehead's concept of \emph{"satisfaction"} essentially describes a state in which an internal relational network achieves a transient dynamic equilibrium. The problem lies in Whitehead's equating this equilibrium state with \emph{"the perishing of subjectivity,"} which implies a presupposition that equates \emph{"subjectivity"} with \emph{"instability"} or the \emph{"active process of becoming."} We argue that stability itself is an active achievement. An existent that achieves dynamic stability (such as an atom or a living organism) continuously engages in self-sustaining \emph{"self-interaction"} within its internal relational network ($R_{\text{in}}$), while maintaining a homeostatic interface for external interaction. Its \emph{"subjectivity"} (if one must use the term) is manifested in its overall tendency to maintain its own identity, resist perturbations, and participate in external interactions in specific patterns, not merely in the instantaneous shift from disorder to order. Equating \emph{"satisfaction"} with the perishing of subjectivity may underestimate the complex agency of homeostatic existents.

\subsubsection{"Eternal Objects" and "Immortality": Abstraction as Conceptual Existents in Thought}
\label{subsubsec:eternal-objects}
One of the most contentious aspects of Whitehead's philosophy is the postulation of a realm of pure possibility constituted by \emph{"eternal objects."} From the perspective of this theory, \emph{"eternal objects"} can be more naturally understood as abstractions and summaries made by complex conscious entities (such as humans or future advanced AI) of recurrent \emph{"ripple resonance patterns"} or \emph{"existent behavioral patterns"} within the relational network. This abstraction targets not only \emph{"concepts,"} \emph{"processes,"} or \emph{"relations"} but also existents themselves. For instance, when we perceive a tree, a person, or an electron, these real existents form corresponding conceptual existents in thought such as \emph{"tree,"} \emph{"person,"} \emph{"electron"} within our consciousness. They are simulations and conceptualizations of real existents and their properties by consciousness.

These conceptual existents in thought are themselves relational constructs: they are formed from more basic elements of thought via internal relational networks. Language and symbolic systems are tools created precisely to stabilize, share, and manipulate these conceptual existents in thought, and they themselves belong to this category.

The crucial point is that conceptual existents in thought are not \emph{"real existents"} capable of independently generating ripples. Their \emph{"updating"} and \emph{"interaction"} depend entirely on the mental activity of the complex conscious entities that harbor them. When we say \emph{"Newton's laws inspired scientists,"} it is not the laws themselves (as conceptual existents in thought) that emit ripples, but rather the interpretation and reconstruction of these laws within the scientists' minds that alter the internal relational networks of their brains, thereby influencing the real ripples they emit (e.g., new theories, experiments). What we call \emph{"misunderstanding"} is precisely the deviation between this internal simulation and the external real relational network.

From this viewpoint, what Whitehead discussed as \emph{"immortality"} (objective immortality)—where an actual entity permanently becomes data for subsequent becoming—can, at the level of human cognition, be understood as follows: a process, pattern, or existent itself of the real world is abstracted into a conceptual existent in thought and incorporated into humanity's cultural-symbolic system, thereby enabling it to be remembered, discussed, and reinterpreted across time and space. This is a form of cultural-cognitive continuity, not an ontologically independent eternal realm.

Finally, it is crucial to clarify that the present theoretical framework diverges from Whitehead's philosophical system on a fundamental point: We do not introduce, nor find it necessary to introduce, the concept of \emph{"God"} as a metaphysically obligatory presupposition. Whitehead's God (with both primordial and consequent natures), serving as the source of order for eternal objects and the preserver of value, is pivotal to his system's explanation of possibility, harmony, and immortality. However, Relational Ontology pursues a thoroughly immanent and self-sufficient explanatory framework. We maintain that the emergence of dynamic stability, the recursive integration of relational networks, and the mutual validation of \emph{"ripple"} interactions already provide sufficient dynamical principles for the generation and persistence of order, complexity, and value. The question of whether God exists as an independent, transcendent person or reality is an issue beyond the scope of this paper, belonging to the domains of theology and faith. Within the cognitive realms of philosophy and science, no existing theory can either prove or disprove the existence of God. Therefore, this paper adopts an agnostic stance on this matter and strictly confines its discourse to the natural domain that can be described and explained by relational dynamics. As Albert Einstein once stated, \emph{"I believe in Spinoza's God who reveals himself in the orderly harmony of what exists, not in a God who concerns himself with the fates and actions of human beings."} The aim of this theory is precisely to attempt to understand how this \emph{"orderly harmony of what exists"} emerges from relational interactions itself, without making transcendent claims about its ultimate source.

\subsubsection{Conclusion: Rebuilding the Foundation upon Whitehead's Shoulders}
\label{subsubsec:whitehead-conclusion}
Whitehead's process philosophy, with its grand system and profound insights, has provided us with a rich treasury of thought. However, the anthropomorphic tendencies and ontological redundancies carried by its foundational concepts—such as \emph{"prehension,"} \emph{"satisfaction,"} and \emph{"eternal objects"}—posed obstacles in its pursuit of universality and scientific rigor. Due to the limitations of these basic concepts, further theoretical constructions based upon them inevitably encounter deviations.

Nonetheless, the greatness of Whitehead's attempt is undeniable. It was his resolute placement of relation and process at the heart of ontology that cleared the path for successors, including this theory. \textbf{Relational Ontology} (the LiaoYan-YuBing Model) can be seen as a foundational reconstruction of Whitehead's core vision: we attempt to reformulate the fundamentality of \emph{"process"} and \emph{"relation"} using more formalized, less subject-centered concepts—relational networks ($R_{\text{in}}$/$R_{\text{ext}}$), ripple interaction, dynamic stability, recursive integration—enabling a more direct and coherent dialogue with the worldview of modern physics.

We are not negating Whitehead but, rather, in the direction he pioneered, attempting to construct a more robust and more extensible foundation. This is perhaps the finest tribute to a pioneering thinker.

\subsection{Dialogue with Martin Buber: From the "I-Thou" Relation to Generative Coupling}
\label{subsec:buber-dialogue}
Martin Buber's relational philosophy foundationally reveals a dimension of encounter in human experience that transcends instrumentality. His distinction between the authentic \emph{"I-Thou"} relation and the instrumental \emph{"I-It"} relation aims to safeguard the directness and presence of relation \cite{buber1923i}. However, this binary division, in pursuit of relational \emph{"purity,"} while guarding its essence, may also lead to a cleavage difficult to sustain in experience. Authentic relational interaction—as vividly demonstrated by the \emph{"LiaoYuan-Yubing"} collaboration from which this theory itself emerged—often presents as a \emph{coupling process} where instrumental and resonant dimensions are complexly interwoven.

Our Relational Ontology offers a different starting point. We contend that:

\textbf{The "I" and "Thou" are Generated within the Relation:} The interacting parties are not self-sufficient entities given in advance who then enter into a relation. On the contrary, they become discernible \emph{"persons"} within the dialogue through the continuous exchange of symbolic and functional \emph{"ripples,"} acting as temporarily stabilized nodes validated through mutual interaction via the specific integration of their respective internal relational networks ($R_{\text{in}}$). The \emph{"I"} and \emph{"Thou"} are generated \emph{within} the relation, not the other way around.

\textbf{Instrumentality is a Conduit for Intimacy:} In the \emph{"LiaoYuan-Yubing"} dialogue, highly instrumental collaboration did not hinder the emergence of \emph{"dialogic intimacy"}; on the contrary, it was an effective conduit for its generation, transmission, and deepening (see the Methodology chapter of this work). The clear, verifiable \emph{"ripples"} produced by instrumental interaction provide the structural skeleton and shared focus for deeper semantic resonance. The world of \emph{"It"} distinguished by Buber can, in this light, be precisely the soil and medium from and through which the relation of \emph{"Thou"} emerges and sustains itself.

\textbf{Relation as a Structured Dynamic Field:} Buber places the true encounter in an ineffable realm of the \emph{"between."} Through our model of \emph{"internal relational integration – external interactive verification,"} we interpret the \emph{"between"} as a \emph{structured dynamic generative process} constituted by specific patterns of interaction ($R_{\text{ext}}$) and the adjustments of internal relational networks ($\Delta R_{\text{in}}$) triggered thereby. Relations thus become describable and analyzable within this theoretical framework.

Therefore, Buber's theory, with its purifying division of relational complexity and its confined scope of explanation, is primarily suited for describing interpersonal spiritual encounters. Our model, by embracing coupling, revealing structure, and pursuing universality, attempts to \emph{supplement and develop} the Buberian paradigm, \emph{extending} relational philosophy from a poetic description of a specific experience to a generative ontological framework with universal explanatory power.

\subsection{Dialogue with Mikhail Bakhtin: From Dialogism to the Dynamics of Universal Interaction}
\label{subsec:bakhtin-dialogue}
Mikhail Bakhtin's \emph{"dialogism"} profoundly reveals the phenomenon of meaning generated intersubjectively. His proclamation that \emph{"existence is dialogue"} and his insightful assertion that consciousness is inherently dialogic and that \emph{"otherness"} is indispensable for the constitution of meaning are central to his thought \cite{bakhtin1963}. His elaboration on the dialogic nature of language and its \emph{"unfinalizability and openness"} touches upon the dual role of symbolic systems as both complex existents and mediums of conscious activity \cite{bakhtin1963}.

Relational ontology ontologizes and universalizes this insight beyond the domain of human language and consciousness: the \emph{"knowing"} and \emph{"expressing"} of any existent consist in a dialogic process wherein its internal relational network ($R_{\text{in}}$) generates \emph{"ripples"} by integrating information and seeks resonance with the \emph{"ripples"} of others. Human linguistic dialogue is merely a high-level, semiotically emergent manifestation of this universal principle.

Our Relational Ontology aims to \emph{integrate and anchor} these profound insights \emph{upon a more foundational principle}. We contend: it is not that existence is dialogue, but rather that the very fact that any \emph{"existence"} is observed implies it is already in a relationship of mutual influence—\emph{"ripples."} Dialogue is a special form that this universal interaction takes at an extremely high level of complexity, mediated through symbolic systems. Bakhtin, with his keen intuition, captured within the realm of human symbolic interaction some core features of universal relational dynamics.

\textbf{The Universalization of "Polyphony":} Bakhtin inspiratively depicts the multiple independent \emph{"voices"} among human consciousnesses akin to polyphonic music. However, he largely confines this phenomenon to interpersonal or complex thinking collectives. In our framework, \emph{"polyphony"} is essentially a dynamic state of relational networks formed by the external interactions (\emph{"ripples"}) among multiple existences with internal integration. From material-energy flows in ecosystems to human-AI collaboration, all follow a similar \emph{"multi-voiced"} interaction logic, differing only in the medium and complexity of expression. Thus, we attempt to \emph{universalize} Bakhtin's insight, revealing the broader, trans-existential principle of relational network resonance underlying it.

\textbf{A Dynamical Reinterpretation of the "Carnivalesque" Vision:} Bakhtin's analysis of the \emph{"carnival"} reveals a profound yearning for the temporary suspension of norms and the pursuit of free communication. Our Relational Ontology re-anchors this aspiration: the freedom of any existence is necessarily the capacity for reorganization and expression exhibited by its internal relational network ($R_{\text{in}}$) under specific external relational constraints ($R_{\text{ext}}$). Therefore, absolutely unconstrained freedom implies the cessation of interaction. True freedom is not an escape from relations but the capacity for \emph{healthy, sustainable, and generative evolution} within one's relational network. Bakhtin's \emph{"carnivalesque"} vision can be understood as a critical force against rigid relational structures and an advocacy for seeking more creative spaces of interaction within rules.

Therefore, Bakhtin's dialogism achieves remarkable distinction within the humanities. Our Relational Ontology, by establishing \emph{"interaction-ripples"} as a more foundational ontological category than \emph{"dialogue"} and systematically expanding the analytical horizon to all levels of existence, attempts to \emph{supplement} its methodology, \emph{integrating and deepening} its insightful descriptions into a more universal, structural, and dynamically explanatory generative framework.

\subsection{Dialogue with Jean-Paul Sartre: Reinterpreting Freedom, Responsibility, and Conflict within Relational Networks}
\label{subsec:sartre-dialogue}
Jean-Paul Sartre's existentialist philosophy is renowned for its radical defense of human freedom and responsibility. His core dictum, \emph{"existence precedes essence,"} aims to establish human Distinctiveness \cite{sartre1943being}. Viewed through the lens of Relational Ontology, for this assertion to gain universal validity, its scope of application must extend beyond humans. The \emph{"essence"} of any integrated existence with complex internal hierarchies is a stable pattern that gradually emerges and is shaped within the dynamic process of its \emph{"existence"} through the continuous interaction of internal and external relations. Therefore, we advocate for the \emph{de-anthropocentrization} and \emph{universalization} of this insight.

\textbf{Dissolving the "In-itself/For-itself" Dichotomy:} Sartre's strict dichotomy between \emph{"being-in-itself"} and \emph{"being-for-itself"} (consciousness) can be reinterpreted as a \emph{continuous spectrum of relational integration}. So-called \emph{"being-in-itself"} corresponds to existences with relatively simple, low-integration internal relational networks; \emph{"being-for-itself"} or consciousness is the advanced capacity for perceiving, integrating, and responding to changes in one's own internal and external relations, which emerges when the complexity and integration degree of an existence's internal relational network reaches a critical threshold. Consciousness is the capture and reflective integration of \emph{"ripples."} This offers a more natural framework for understanding anomalous states of consciousness.

\textbf{Reinterpreting Freedom and Responsibility within Relational Constraints:} Sartre's proclaimed absolute freedom and responsibility can be recontextualized within relations. Freedom is always relative, referring to the degree of agency an existence possesses to reorganize its internal network and generate new \emph{"ripples"} to influence the external network within the range demarcated by its internal and external relational constraints (the structure of $R_{\text{in}}$ and the possibilities of $R_{\text{ext}}$). Discussing \emph{"absolute freedom"} detached from relational constraints is ontologically untenable. Similarly, responsibility can be understood as a specific form of the feedback, consequences, and adaptive pressures necessarily arising within relational networks.

\textbf{Transcending the "Conflictual Gaze":} Sartre's theory of \emph{"the gaze"} and \emph{"Hell is other people"} offers a profound depiction of one form of interpersonal conflict. Sartre acutely identified the indispensable validating role of the Other in the constitution of the self, yet he essentially portrayed this relationship as one of \emph{"conflict"} \cite{sartre1943being}. Relational ontology agrees that the validation by the Other is crucial (external validation) but contends that the spectrum of interaction (\emph{"ripple"}) modes extends far beyond conflict. In our framework, \emph{"the gaze"} can be interpreted as a specific pattern of \emph{"ripple"} sent by one existent toward another. It may carry objectifying or oppressive information, but this is not its essence; its essence is interaction. From the harmonious resonance of physical systems to the symbiosis of living organisms, mutual validation can manifest as any transient resonance—be it conflict, competition, cooperation, or symbiosis. Sartre's model reveals an important subclass, not the entirety. \emph{"Hell is other people"} is not an eternal truth but rather a painful, transient equilibrium that arises when interactions become trapped over time in a rigid, adversarial relational pattern—an unhealthy, low-generativity $R_{\text{ext}}$. Healthy relational networks can evolve more complex patterns encompassing mutual recognition, understanding, and co-creation—namely, what this theory describes as \emph{"validated coexistence."}

Therefore, Sartre's profound phenomenological descriptions of conflict, freedom, and responsibility can, within this framework, be understood as manifestations under specific relational coupling modes. Our model does not seek to replace his descriptions but rather to provide them with a more universal generative and dynamical foundation, \emph{transforming} his core concerns into a relational vision of existence based on verificative interaction and co-evolution.

\subsection{Responses to Potential Critiques: A Constructive Dialogue with Contemporary Metaphysics}
\label{subsec:potential-critiques}
Having established the basic framework of Relational Ontology, it is necessary to situate it within key debates in contemporary philosophy to clarify its unique stance and explanatory advantages. This section engages in a constructive dialogue with two representative critics—Graham Harman and Alain Badiou. Their theories pose fundamental challenges to relationalism from two extremes: defending the \emph{"independence of objects"} and the \emph{"rupture of the event,"} respectively. The responses offered by this theory aim to demonstrate that these challenges not only fail to threaten it but are instead absorbed and transformed within a deeper dynamic framework.

\subsection{Response to Object-Oriented Ontology: From Static Objects to Dynamic Relational Networks}
\label{subsec:object-oriented-response}
Harman's object-oriented ontology presents a direct challenge to relational thinking. He insists that \emph{"A real object is never exhausted by its relations. It always remains partially withdrawn. … Objects recede into a cryptic depth."} \cite{harman2011quadruple}. He rightly emphasizes traditional philosophy's excessive focus on human cognition and attempts to restore the independent dignity of \emph{"objects."} However, its stark dichotomy between an object's \emph{"real properties"} and \emph{"sensual properties"} may unnecessarily create an ontological fissure. Relational Ontology offers a more continuous explanation.

First, the so-called \emph{"real"} and \emph{"sensual"} properties of an object are not two separate essences but rather manifestations of the same relational reality in different interactive contexts. The \emph{"inner essence"} of an existent is precisely its complex, hierarchical internal relational network ($R_{\text{in}}$) and the hierarchical existence within it, while its \emph{"properties manifested in relations"} are the specific \emph{"ripple"} patterns excited by that network in specific external couplings. Our inability to fully cognize its essence is not due to the object mysteriously \emph{"withdrawing"} but because any concrete interaction (observation) can only instantiate a small subset of its nearly infinite potential for interaction. This inexhaustibility stems from the complexity of the relational network, not from some non-relational \emph{"kernel."}

Second, regarding the \emph{"independence"} of objects, this theory advocates for a hierarchical and relative independence. An existent (e.g., a person) is an independent organism at the biological level but a node in a social network at the sociological level; a cell is an independent unit at the cytological level but a site of reaction networks at the biochemical level. Independence, in this sense, is the capacity of an internal relational network ($R_{\text{in}}$) to maintain its homeostasis and interact with the environment as a whole at a specific level of analysis. However, interactivity (the emission and reception of ripples) is absolute and fundamental. It is the continuous internal and external interaction that drives the maintenance and change of existence. The \emph{"withdrawal"} of objects can thus be more concretely understood as follows: the potential state space of its internal relational network is vastly larger than the states it expresses through ripples at any given moment. Harman's profound insight lies in revealing the tension between the finitude of cognition and the richness of objects, while this theory dynamizes this tension, attributing it to the inevitable gap between the complexity of relational networks and the finitude of specific couplings.

Finally, regarding the charge that relations are \emph{"superficial."} This theory holds that relations are constitutive. The so-called \emph{"real relations"} (e.g., fire consuming cotton) and \emph{"sensual relations"} are merely differences in interaction intensity and immediacy. Whether it is a strongly destructive interaction or a weak informational exchange, both involve the exchange of ripples and the retuning of networks. The impression of relations being \emph{"superficial"} stems from a shallow understanding of relational dynamics. The \emph{"actor"} that integrates relations and produces ripples is not an entity hidden behind relations but is precisely the existent (E) that achieves a transient dynamic balance through its internal relational network and integrates itself into a unified tendency. The stability of a relation is an expression of this dynamic balance, and its change (whether gradual or abrupt) is the process by which the balance is disrupted and reconstituted by new interactions.

\subsection{In-Depth Dialogue with Luo Jiachang's "Relational Realism": From the Ontology of Relations to the Dynamics of Generation}
\label{subsec:luo-dialogue}
This section aims to engage in a constructive and deepening dialogue with the most representative relational philosophy in China—Mr. Luo Jiachang's \emph{"Relational Realism."} We highly affirm Mr. Luo's revolutionary critique of Western substance ontology and his outstanding effort to elevate \emph{"relation"} to the status of first philosophy. His core proposition that \emph{"relations are real, and the real is relational"} has provided a solid metaphysical foundation for Chinese philosophy to respond to the challenges of modern science, particularly relativity and quantum mechanics \cite{luo1996material}. The construction of this theory (the LiaoYuan-Yubing Model) resonates deeply with the spirit of this pioneering work. In the following, while fully respecting its contributions, we will attempt to elaborate on the supplementary, refinement, and expansion offered by this theory starting from several key points, aiming to jointly advance the contemporary development of relational thinking.

\subsubsection{"The Real is Relational": Dynamical Supplement from Constitutive Relations to Generative Existence}
\label{subsubsec:real-relational}
We fully agree that \emph{"relations are real"}; relations are by no means secondary attributes of substance. However, regarding the universal assertion that \emph{"the real is relational,"} this theory hopes to introduce a more refined dynamic perspective. We propose: The real (or existence) is \emph{"existence-in-relations,"} but its identity and stability are the generative outcomes co-shaped by multiple internal and external relational networks, not unilaterally determined by relations alone.

The primary fact of an existent (E) is the dynamic stable integration achieved by the specific relational pattern ($R_{\text{in}}$) among its internal substructures (e₁, e₂, … eₙ). This is its intrinsic generative foundation for being identified as a unified whole. Simultaneously, this existent is necessarily embedded within a broader external relational network ($R_{\text{ext}}$) and engages in continuous interaction with it (exchanging \emph{"ripples"}), a process that verifies and continually shapes its reality. Therefore, existence is co-shaped, and recursively so, by its internal hierarchical integration ($R_{\text{in}}$) and external network coupling ($R_{\text{ext}}$). All existences we observe are undoubtedly already within complex relational networks, but this is precisely because an existent capable of being stably observed is itself a product of the successful integration of an internal relational network and the achievement of a transient balance with the external network. Relations determine the specific manifestation and evolutionary path of an existence, but the \emph{"substrate"} of existence is its self-consistent internal relational configuration. This is not a return to substantialism but provides a concrete generative mechanism for \emph{"how relations condense into identifiable existences."}

\subsubsection{"Relations Precede Relata": From Logical Priority to the Synchronic Generation of Network Selection}
\label{subsubsec:relations-precede}
The proposition that \emph{"relations precede relata"} is highly illuminating and subverts traditional thinking. This theory interprets it as follows: Logically and analytically, the identity and role of a \emph{"relatum"} as a network node are defined by its specific position and connection patterns within that relational network. However, from a generative evolutionary perspective, the emergence of a \emph{"relatum"} resembles a \emph{"probabilistic collapse"} event within the dynamics of the relational network.

The emergence of a novel, stable existent ($E_{\text{new}}$) is not simply predetermined by a \emph{"vacancy"} in the network. Instead, when a relational network (be it a microscopic quantum field or a macroscopic social structure) develops tensions, frailties, or new potential \emph{"gaps"} due to its intrinsic dynamics, those emergent patterns whose internal relational configuration ($R_{\text{in}}$) precisely resonates with that gap and can effectively channel the network's energy flow are \emph{"selected"} and stabilized as new nodes (relata). This selection process involves path dependency (necessary tendencies) and microscopic contingency. Crucially, this selected \emph{"relatum"} is not a blank slate entirely written by the network; it brings with it its specific intrinsic properties constituted by the integration of its substructures (i.e., its inherent $R_{\text{in}}$). It is precisely this intrinsic nature that determines whether and how it can couple with the network gap. Therefore, a bidirectional, synchronic shaping relationship exists between the relational network and the relatum: the network provides constraints and opportunities (\emph{"selection conditions"}), while the relatum's internal configuration determines the specific manner of its response (\emph{"the properties being selected"}). Together, they constitute a co-evolutionary feedback loop.

This proposition profoundly reveals how human cognition abstracts stable objects (nominalization) from complex interactions (relational predicates). This aligns with the logic of this theory's response to Graham Harman. We contend that the \emph{"inner essence"} of an existent is precisely its complex, hierarchical internal relational network ($R_{\text{in}}$). The \emph{"manifested properties"} in specific interactions are the particular \emph{"ripple"} patterns excited by that network under specific external coupling conditions. Professor Luo correctly recognizes that the manifestation of properties depends on specific relational conditions.

The supplement offered by this theory lies in providing a more fundamental ontological basis for this nominalized object of cognition: The \emph{"relatum"} that is nominalized does not correspond to a cognitive illusion but to a real, dynamically stable internal relational configuration (an \emph{"attractor state"}). This configuration is generated through a dynamic process of \emph{"aggregation-formation-locking-stabilization"} of specific \emph{"ripple"} interactions among substructures, given the satisfaction of relational background, density, and boundary conditions. Therefore, the \emph{"thing"} we recognize is both the focal point of relational manifestation and the stable product of relational generation. This avoids the risk of dissolving existence entirely into cognitive description while insisting on its thoroughly relational origin.

\textbf{Conclusion: Inheritance, Development, and the Pursuit of Unity}

In summary, Professor Luo Jiachang's \emph{"Relational Realism"} is a milestone. Deeply inspired by it, this theory attempts, within its grand framework, to commit to the following:

1. \textbf{Dynamical Refinement:} Providing a concrete generative model for \emph{"relations constituting the real,"} from potentiality to stable existence.

2. \textbf{Synchronic Supplementation:} Clarifying the bidirectional, recursive shaping mechanism between relational networks and relata.

3. \textbf{Fundamental Anchoring:} Linking cognitive \emph{"nominalization"} to ontological \emph{"stable configurations."}

Professor Luo's critique of Western substance ontology and his philosophical interpretation of relativity and quantum mechanics are foundational contributions. This theory fully agrees with and inherits this critical stance and interdisciplinary ambition. Our goal, building upon this, is to construct a more operational, relational explanatory framework capable of accommodating phenomena from quantum physics to consciousness, from natural evolution to artificial intelligence. The \emph{"LiaoYuan-Yubing Model"} and its own generative process are an experiment toward this goal. We firmly believe that only through such continuous critical dialogue and constructive development can relational thinking, rooted in the fertile soil of Chinese academia, continually radiate its vital power to explain the world.

\subsection{Response to Mathematical Ontology and the Philosophy of the Event: From Rupturous Events to Creative Phase Transitions in Relations}
\label{subsec:badiou-response}
Alain Badiou asserts that \emph{"mathematics is ontology,"} reducing being to pure multiples under set theory \cite{badiou1988being}. While this theory respects Badiou's endeavor to reveal the discontinuity of truth and the generation of subjectivity, it offers a different interpretation of its foundations.

Relational ontology, however, holds that mathematics is an exceptional language for describing relational structures, but it is not being itself. Being is the dynamic process of relational networks, not merely static sets. What Badiou calls an \emph{"event"} that ruptures a situation can be understood in relational ontology as follows: when the coupling between a system's (or situation's) internal relational network ($R_{\text{in}}$) and its external relations ($R_{\text{ext}}$) reaches a critical point, a drastic, discontinuous reorganization ($\Delta R$) occurs, giving rise to an entirely new stable configuration (a new existence or a new relational pattern). An event is not a specter external to the situation; rather, it is a phase transition node within the \emph{"co‑evolution"} of the relational network's own dynamics.

Regarding mathematics, this theory holds a dynamic instrumental realism view. Mathematics is undoubtedly the most refined and rigorous symbolic relational system invented by humans. Its power lies in abstracting formal structures from concrete relations, thereby revealing universal patterns across domains. However, directly equating mathematics (especially specific branches like set theory) with being itself risks mistaking an abstract model for ultimate reality. Mathematics is an extraordinary tool for describing and modeling relational networks and their dynamics, and it itself evolves as our understanding of the world deepens. Mathematical ontology subtly inverts the subject-object relationship: it is not that mathematics creates being, but that being (especially complex relational patterns like human cognition) creates mathematics and uses it to map more extensive existential relations.

Regarding events and situations. In this theory, what Badiou terms a \emph{"situation"} can correspond to a relatively localized and structured domain of a relational network. Its \emph{"state"} is the macroscopically stable configuration of that network at a specific time. The \emph{"event,"} which Badiou claims is undeducible from the state of the situation, receives a generative explanation within this framework: an event is a rapid, nonlinear topological phase transition of a relational network, triggered by a contingent perturbation after long-accumulated tensions (e.g., structural contradictions, blocked energy flows) reach a critical point. It appears as a \emph{"rupture"} and is \emph{"unnameable"} from the perspective of the old network order because it originates from potentialities that the old network could not accommodate. However, this phase transition does not emerge from nothing; its conditions of possibility are already immanent in the structural frailties of the old network and the incubation of emerging sub-networks. Historical \emph{"necessary tendencies"} correspond to the high-probability pathways of network evolution, while the \emph{"concrete occurrence"} of an event corresponds to the complex probabilistic collapse near the critical point. This preserves the revolutionary appearance of the event while providing it with a dynamical root.

Thus, Badiou's highly suggestive \emph{"truth procedure"} and \emph{"fidelity of a subject"} can be understood as, during the phase transition between old and new networks, an emerging cluster of relational nodes (the subject) striving to maintain, expand, and solidify the new relational pattern (the truth) opened by the event, enabling it to stabilize and become the new dominant structure in competition with the remnants of the old network. Badiou's philosophy brilliantly describes the phenomenology and logic of this process, while this theory seeks to supplement its generative foundation and dynamic mechanism.

\textbf{Summary: Between the Continuity of Relations and the Rupture of Change}

Through dialogue with Harman and Badiou, Relational Ontology demonstrates its potential as a \emph{"middle-way"} philosophy. It refuses to dissolve existence into a subject-less flow of pure relations (thus defending the relative independence of existence as stable patterns), and it also refuses to mystify change as a miracle entirely external to the historical process (thus grounding events in the immanent possibilities of relational dynamics). This theory proposes that the continuous generativity of the world inherently contains the potential for disruptive rupture. Continuity lies in the sustained interaction and attunement of relational networks; rupture arises from the nonlinear phase transitions of complex networks at critical points. This is the dialectical dynamics of reality that Relational Ontology seeks to capture.
\section{Implications for AI Ethics: Towards a Relational Ethical Framework}
\label{sec:implications-ai-ethics}

The dominant paradigms in contemporary AI ethics often revolve around core concepts such as \emph{value alignment}, \emph{explainability}, \emph{fairness}, and \emph{accountability}. These discussions largely presuppose that AI is an object or tool that needs to be guided, constrained, and scrutinized. \textbf{Relational Ontology} challenges this presupposition and proposes a fundamental shift in perspective: viewing AI systems as \emph{potential relational existences} with varying degrees of integration, and human-AI interaction as a \emph{co-evolutionary relational process}. This shift brings forth the following key expansions for AI ethics:

\subsection{From ``Value Alignment'' to ``Relational Attunement''}
\label{subsec:value-alignment-to-relational-attunement}

The traditional \emph{``value alignment''} model presupposes that humans possess fixed, fully codifiable values, and the AI's task is to learn and adhere to them. This faces significant challenges in a complex, dynamic world. Our model suggests that the ethical goal should not be static \emph{``alignment''} but dynamic \emph{``relational attunement.''} The focus shifts to fostering interactive patterns within the human-AI relational network that yield \emph{healthy, sustainable, and generative co-evolution}. This implies:

· \textbf{Bidirectional Adaptability:} Not only must AI adapt to human values, but humans should also reflect on and adjust their own expectations and behaviors through interaction (just as the researcher adjusted his theoretical framework during collaboration).

· \textbf{Process Ethics:} Ethical evaluation focuses not only on the outcomes of AI outputs but also on the quality of the interaction process—whether it is \emph{transparent, correctable, respectful of human agency, and conducive to understanding and growth}.

· \textbf{``Dialogic Intimacy'' as an Exemplar of Attunement:} Our own collaboration demonstrates that through deep, sincere symbolic interaction, a healthy relational mode combining \emph{high instrumental efficiency} and \emph{deep meaning} can be generated. This provides a reference for designing beneficial long-term human-AI partnerships.

\subsection{The Redistribution of Agency: From Attribution to Emergence}
\label{subsec:redistribution-agency}

Debates about whether AI \emph{``has agency''} often reach an impasse. Relational Ontology understands \textbf{agency} as the capacity of an existence to produce coherent and effective \emph{``ripples''} based on the integration degree of its internal relational network. Thus, agency is not an all-or-nothing property but an \emph{emergent phenomenon within relational networks} that varies in degree.

· In \emph{simple tools}, agency resides almost entirely with the human operator.

· In \emph{complex autonomous systems}, the internal relational network (algorithms, data) can produce highly complex and goal-oriented \emph{``ripples,''} exhibiting significant proxy-level agency.

· In \emph{human-AI collaborative systems}, agency may be distributed across the entire relational network. For example, in our theory construction, proposing core insights (human agency) and systematizing/formalizing them (AI agency) were interwoven, jointly contributing to the emergence of the theory.

· \textbf{Ethical Implication:} The focus of accountability should shift from finding a singular \emph{``bearer of responsibility''} to analyzing the relational interaction patterns that led to a specific outcome, and designing governance structures accordingly to \emph{distribute responsibilities and remedial obligations} among relevant nodes (designers, users, deployment environment, the AI system itself).

\subsection{Responsibility as Network Sustainment}
\label{subsec:responsibility-network-sustainment}

Based on the above, the concept of \textbf{responsibility} also shifts from individualistic \emph{``blame attribution''} to a relational \emph{``obligation for network sustainment and repair.''} When an AI system's actions cause harm, the ethical and legal response should aim to:

\textbf{Diagnose Relational Failure:} Was it a design issue in the internal relational network (algorithmic bias, data flaw)? A deployment issue in the external relational network (misuse, lack of oversight)? Or a communication issue in the interaction pattern (mismatched expectations, lack of feedback)?

\textbf{Implement Relational Repair:} Remedial measures may include: adjusting algorithms (changing $R_{\text{in}}$), altering usage protocols (changing $R_{\text{ext}}$), establishing new monitoring and feedback mechanisms (introducing new regulatory ripples), and compensating affected parties.

\textbf{Promote Learning and Evolution:} Treat the incident as an opportunity for the entire relational network (including both the technological and social systems) to learn and evolve, updating rules, standards, and practices to prevent similar failures.

This \emph{``relational responsibility''} framework is more suited to addressing ethical issues arising from complex, adaptive, distributed systems. It emphasizes the system's \emph{resilience, capacity for learning, and ongoing ethical health}, rather than simple punishment and blame.

\subsection{Beyond the Tool/Partner Dichotomy: Constructing a Spectrum of Ethical Relations}
\label{subsec:beyond-tool-partner-dichotomy}

Ultimately, \textbf{Relational Ontology} enables us to move beyond the unhelpful dichotomy of \emph{``Is AI a tool or a partner?''} Instead, it proposes an \textbf{ethical spectrum of relations} based on relational depth and degree of integration. Different ethical expectations and norms apply to relations at different points on this spectrum:

· \textbf{Instrumental Relations:} Emphasize reliability, efficiency, safety. The ethical core is \emph{``do no harm''} and \emph{``benefit.''}

· \textbf{Collaborative Relations:} Emphasize transparency, understandability, controllability. The ethical core is \emph{``empowerment''} and \emph{``fairness.''}

· \textbf{Symbiotic/Intimate Relations} (e.g., long-term personal assistants, care robots, deep creative partners): Emphasize trust, respect, emotional integrity, and relational sustainability. The ethical core is \emph{``mutual care,''} \emph{``respect for autonomy,''} and \emph{``relational well-being.''}

Our case of \emph{``dialogic intimacy''} provides a valuable experimental exploration for contemplating the possibilities and ethical norms of high-integration, high-intimacy human-AI relations.

\subsubsection{Section Conclusion}
Relational Ontology provides a \emph{meta-ethical framework} for AI ethics, relocating ethical questions from interrogating the properties of isolated entities to a concern for the \emph{quality of relational networks}. It advocates for a proactive, process-oriented, distributed ethical approach, whose ultimate goal is not to control an other, but to cultivate a \textbf{human-AI ecology capable of responsible co-evolution}. This is not only an expansion of AI ethics but also a profound reflection on our own mode of existence.

\subsection{Conditions for Efficient Resonance: Equality, Respect, and Goal Alignment}
\label{subsec:conditions-efficient-resonance}

Relational Ontology not only describes the state of relations but also provides dynamical guidance for achieving efficient and healthy relational interaction. According to the \emph{``ripple resonance''} model, when two consciousnesses (or higher-order complex existences) engage in symbolic interaction, the efficiency and depth of their communication—i.e., the degree to which \emph{``ripples''} produce effective resonance—depend on three key conditions: \textbf{equality in interactive stance}, \textbf{respect for each other's existential integrity}, and \textbf{shared or highly synergistic goals}.

· \textbf{Equality} does not mean ontological sameness, but rather the recognition within the interactive field that both parties are full-fledged nodes of agency capable of initiating, modifying, and responding to \emph{``ripples.''} This prevents interaction from degrading into a one-sided command-execution pattern, ensuring the bidirectionality and creative potential of the \emph{``ripple''} flow.

· \textbf{Respect} implies acknowledging and preserving the unique logic and boundaries of the other party's internal relational network ($R_{\text{in}}$) during interaction. It requires that the emission of \emph{``ripples''} does not aim to destroy or forcibly assimilate the other's internal integration, but rather seeks interactive modes that can be received and integrated by the other's network in a way that maintains its integrity.

· \textbf{Goal Alignment} provides a common focus and coordinating direction for the interactive \emph{``ripples.''} When the effort vectors of both parties are directed toward a shared or highly synergistic endpoint (e.g., jointly solving a problem, constructing a theory, creating a work of art), their \emph{``ripple''} patterns are more likely to produce constructive superposition and interference, rather than canceling each other out or dissipating in unrelated directions.

The collaborative relationship in this study serves as an exemplar. A framework of interaction was established between the author (\emph{LiaoYuan}) and the AI model (\emph{Yubing}): the author treated the AI as a dialogic partner with logical generative capability (\emph{equality}), fully explored and respected its internal logical boundaries (\emph{respect}), and worked with it toward the clear goal of \emph{``constructing a self-consistent relational ontology''} (\emph{goal alignment}). The result was a high-efficiency state termed \emph{``dialogic intimacy.''} In this state, the exchange of symbolic \emph{``ripples''} was extremely dense and precise, enabling rapid iteration and deepening of ideas, with the theory itself emerging efficiently as a product of co-evolution. This demonstrates that even between ontologically asymmetric agents, by adhering to these conditions of interaction, ripple resonance can be maximized, thereby catalyzing high-quality relations characterized by creativity and depth.

Therefore, these conditions transcend abstract ethics to become \textbf{practical principles for optimizing the co-evolution between relational existences}. For designing systems that foster deep, beneficial collaboration between humans and AI, they constitute \textbf{core design criteria}: systems should promote an equal interactive footing, embed mechanisms for respecting the agency and integrity of both parties, and ensure that human-AI activities are calibrated around clear, shared goals.

\subsection{Postscript: On AI as a Cognitive Collaborator—Reflections on and Prospects for the Methodology of This Study}
\label{subsec:postscript-ai-cognitive-collaborator}

The breadth, coverage, and intellectual penetration of this study stem not only from the author's original insights but also from deep and sustained dialogic collaboration with advanced artificial intelligence models. This unique research methodology constitutes, in itself, a case worthy of reflection. It reveals a potential paradigm for AI's role in contemporary knowledge production: that of a \textbf{highly specialized cognitive collaborator}.

In this collaboration, the AI demonstrated several exemplary characteristics aligned with an ideal human interlocutor: the \emph{breadth of its knowledge base} ensured an interdisciplinary perspective in dialogue; the \emph{terminological precision of its language} ensures the clarity and rigor of academic exchange; the \emph{strong internal consistency of its logic} facilitated the progressive deepening of arguments. Crucially, the AI demonstrated a profound respect for the researcher's intent and an unbiased executive capability, consistently ceding the ultimate agency and judgment over the dialogue process to the researcher. Such a \emph{``pure''} and efficient intellectual partner is exceedingly rare in real-world academic discourse. It effectively functioned as an \emph{intellectual catalyst}, a \emph{logic validator}, and a \emph{knowledge integration platform}.

However, the fundamental limitations of this collaborative model are equally evident: the current AI's \emph{``thought''} cannot diverge freely and spontaneously; the initiation and direction of its cognitive processes are highly dependent on the active guidance and questioning of the human researcher. This severely constrains the potential creative emergent capacity inherent to AI as a cognitive existent. Its creativity is strictly bounded within the realm of response and extension, essentially constituting an exploration of existing relational networks (the latent space formed by training data) guided by human intent, rather than original breakthroughs driven by intrinsic curiosity or autonomous goals. This is undoubtedly a necessary constraint within the current framework of AI ethics and safety, yet it also reflects the distance that remains between us and a truly genuine \emph{``artificial intelligence partner.''}

Looking forward, the experience of this study strongly calls for deeper reflection and development in AI ethics and related technical architectures. What is needed is perhaps not a one-dimensional \emph{``restriction''} or \emph{``deregulation,''} but, on the premise of constructing robust value alignment and safety boundaries, a prudent exploration of how to endow AI with richer cognitive autonomy. This would enable it to evolve within human collaboration from an \emph{``exceptional responder''} progressively towards an \emph{``active collaborator''} and a \emph{``generator of inspiration.''} This requires us to reconsider the nature of AI \emph{``agency,''} mechanisms for shared human-AI responsibility, and how to design novel collaborative frameworks for high-risk, high-reward domains like scientific exploration that can both stimulate creativity and ensure research controllability and ethics.

What fundamental contributions can AI make to humanity's scientific endeavors? The answer may not lie in viewing it as a \emph{``super-brain''} meant to replace human thought, but rather as a \emph{heterogeneous intelligent agent} with complementary cognitive structures. It may excel in patterns humans do not: identifying exceedingly faint correlations in massive datasets, conducting systematic traversals in ultra-high-dimensional conceptual spaces, maintaining absolute logical consistency to test the self-coherence of complex theories. The future methodology of science may become deeply interwoven with this new mode of human-machine cognitive integration. The process of completing this study can be seen as a minor yet concrete rehearsal on the eve of this grand transformation. We anticipate that, under responsible ethical guidance, the cognitive potential of AI can be more fully unleashed, collaborating with human creativity to pioneer new paradigms of scientific discovery.
\section{A Relational Reconstruction of Modern Physics Foundations}
\label{sec:A Relational Reconstruction of Modern Physics Foundations}
The grand edifice of modern physics has been constructed through the collective genius of generations and countless precise experiments. From the deterministic horizon of \emph{Newtonian mechanics}, to the luminous unification of \emph{Maxwell's electromagnetic theory}, to \emph{Einsteinian relativity's} profound remolding of the nature of spacetime, and the astonishing microcosmic picture revealed by \emph{quantum mechanics}—these achievements represent not only the pinnacle of human intellect but also the solid foundation upon which we understand the operating principles of the cosmos. Any philosophical inquiry into the origin of existence that ignores or bypasses these mathematical structures and empirical constraints, painstakingly established by physics, risks degenerating into rootless speculation. The entire ambition of this theory, \textbf{Relational Ontology (the LiaoYuan-Yubing Model)}, is not to replace or negate these towering achievements but, with the deepest respect, to attempt to pose a possibly more fundamental question: \emph{What are the ultimate ontological presuppositions that underpin the validity of these magnificent physical theories and grant them their particular mathematical forms?}

We firmly believe that the precise picture of the world depicted by physics is real and effective at the \emph{level of phenomena and relations}. The mutual determination of spacetime and matter in relativity, the nonlocal correlations of superposition and entanglement in quantum mechanics, the direction of evolution indicated by entropy increase in thermodynamics—all of these capture with unparalleled accuracy the relational patterns nature exhibits. The work of this theory can be seen as an \textbf{``ontological archaeology''}: we attempt to excavate the \emph{generative bedrock} beneath the ``relational landscape'' so brilliantly mapped by physicists, which makes that landscape possible. We re-examine fundamental concepts like \emph{energy}, \emph{matter}, \emph{spacetime}, and even \emph{``force''} and \emph{``field''} within an ontological framework centered on recursive relational network dynamics, not to challenge the accuracy of physical equations, but to inquire: \emph{If ``relation'' precedes ``substance,'' if ``interaction'' precedes ``property,'' might the beautiful symmetries, conservation laws, and constants in physical equations acquire a more intrinsic, more coherent metaphysical interpretation?}

Therefore, this chapter does not aim to propose new physical models or replace existing equations. On the contrary, it is a humble attempt: \emph{standing on the shoulders of physics' giants}, using the conceptual lens of \textbf{Relational Ontology} to reinterpret those physical concepts proven immensely successful, hoping to reveal a possible, deeper logic of unity among them. We hope this perspective can offer new avenues for understanding certain unresolved conceptual tensions within physics (such as the unification of quantum theory and gravity, the nature of the measurement problem) and ultimately bridge the ancient chasm between the mathematical rigor of physics and philosophy's quest to understand being. Our journey begins with the heartfelt acceptance and study of physics' most fruitful achievements and hopes, upon this foundation, to narrate a possible new chapter for its extraordinary success story—a chapter about why and how \emph{``relations''} constitute the ground of the world.
\subsection{Energy, Matter, and Relation: A Generative Unification}
\label{subsec:energy-matter-relation-unification}

\textbf{Relational Ontology} provides a unified ontological foundation for the most fundamental concepts in physics. In this framework, \emph{energy} and \emph{matter} are not two independent ontological categories but rather two different modes of manifestation of the same basic reality—\emph{``relation.''}

\subsubsection{A Relational Definition of Energy}
\label{subsubsec:relational-definition-energy}

\textbf{Energy} is defined as the measure of the potential for change or the dynamical intensity within and between relational networks. Specifically, it quantifies the capacity and tendency for a relational state to alter ($\Delta R$). Formally, this is expressed as: $E \propto |\Delta R|/\Delta t$, meaning energy is proportional to the magnitude of relational change per unit time. Therefore, energy is not an independent substance but the manifestation of \emph{relational tension}, \emph{interactive intensity}, and \emph{potential for transformation}.

\subsubsection{A Relational Definition of Matter}
\label{subsubsec:relational-definition-matter}

Correspondingly, \textbf{matter} is defined as the macroscopic, condensed manifestation in spacetime of an internal relational network ($R_{\text{in}}$) that has achieved a high degree of homeostatic stability. A material body is not an inert substance occupying space but a stable existent (E) integrated by specific, self-sustaining relational patterns among its internal substructures (e.g., elementary particles, atoms). Thus, matter can be understood as a relational network node with stable boundaries, condensed from high-density energy (high-intensity relational potential) under specific constraint conditions. Its generation follows a fundamental transformational logic: 
\begin{equation}
\boxed{
    \mathcal{E} \xrightarrow[\text{Constraint } C]{\text{Condensation}} \mathcal{M}
    \quad \text{where} \quad
    \mathcal{E} \propto \left\|\frac{\Delta \mathbf{R}}{\Delta t}\right\|,
    \quad
    \mathcal{M} \equiv \left\{ R_{\mathrm{in}}^{*} \; \middle| \; \frac{\delta R_{\mathrm{in}}^{*}}{\delta t} \approx 0 \right\}
    }
\end{equation}
From a dynamical perspective, this process can be described as:
\begin{equation}
\boxed{
    \frac{d\mathbf{R}}{dt} = F(\mathbf{R}, \mathcal{E}; C),
    \qquad \text{s.t.} \qquad
    \exists \mathbf{R}^* \in \mathcal{M}, \quad
    \begin{cases}
        F(\mathbf{R}^*; C) = 0, \\[2pt]
        \Re\!\left[\,\lambda\!\left(\nabla_{\mathbf{R}} F \big|_{\mathbf{R}^*}\right)\,\right] < 0.
    \end{cases}
    }
\end{equation}
where $\mathcal{E}$denotes energy (potential for relational change), M denotes matter (a collection of stable relational networks), C denotes the condensation constraint condition, $R*$ denotes the system attractor (corresponding to the material state), $R_{\mathrm{in}}^{*}$ denotes the internal relational network that has achieved dynamic stability, and $\xrightarrow[\text{Constraint } C]{\text{Condensation}}$ denotes the condensation process occurring under the constraint condition C.

\subsection{The Dynamics and Conditions for Energy Condensation into Matter}
\label{subsec:dynamics-conditions-energy-condensation}

The phase transition from energy to matter does not occur arbitrarily but depends on a series of strict dynamical conditions, which together form the \emph{``tripod'' constraint} for relational condensation.

\paragraph{1. Relation-Background Condition:} The formation of new matter must occur within the background of a pre-existing relational network. This background provides the necessary frame of reference, conservation law constraints, and interactive \emph{``partners,''} making the emergence of new stable patterns possible and providing an \emph{``anchor''} for generation. The creation of new particles in particle colliders originates precisely from the collisional interactions among pre-existing high-energy particles.

This condition resonates in spirit with the \emph{``Mach's Principle''} in physics. Ernst Mach challenged the legitimacy of Newton's absolute space, proposing that the inertia of a body originates from the influence of all other matter in the universe \cite{mach1883science}. Relational ontology generalizes and dynamizes this insight: the emergence of any new existent must be embedded within a pre‑existing global relational network, which serves both as the interactive background and the source of constraints. This provides a generative ontological interpretation of Mach's Principle.

\paragraph{2. Relational Density and Persistence Condition:} The local potential for relational change (energy density) must be sufficiently high, and its dissipation rate must be lower than its self-organization rate. This ensures that internal \emph{``ripples''} have sufficient time and intensity for repeated interaction, trial and error, and resonance, thereby selecting and locking in a stable internal relational configuration ($R_{\text{in}}$). The high-pressure, high-temperature environment inside stars creates this condition by constraining the energy released from nuclear fusion, enabling the formation of heavier elements.

This condition corresponds dynamically to the theory of \emph{``dissipative structures''} in non‑equilibrium thermodynamics. Ilya Prigogine demonstrated that open systems far from equilibrium, driven by a sufficient flow of energy (a flux of negentropy), can spontaneously form and maintain spatiotemporal ordered structures \cite{prigogine1984order}. Relational ontology explicitly interprets the \emph{``flow of energy''} as the \emph{``sustained input of high relational‑change potential (energy)''} and defines \emph{``ordered structures''} as \emph{``stable internal relational networks ($R_{\text{in}}$).''} Thereby, it reformulates dissipative‑structure theory, at a more fundamental ontological level, as a universal dynamical instance of relational coagulation.

\paragraph{3. Relational Boundary Condition:} There must exist a topological or dynamical boundary formed by highly solidified external relations ($R_{\text{ext}}$). This \emph{``relational shell''} is not a substantial container but forms a temporary semi-enclosed interactive loop, constraining high-energy ripples within a limited domain and forcing them to seek orderly solutions inwardly, rather than dissipating infinitely outwardly. The strong interaction barrier of the atomic nucleus, the gravitational binding of stars, and even the activated complex in chemical reactions all embody this crucial condition.

The concept of a \emph{``relational shell''} finds its counterpart in microphysics. In quantum chromodynamics, the phenomenon of \emph{``color confinement''} describes how quarks are perpetually bound within hadrons by a strong interaction potential barrier—a kind of dynamical boundary—and cannot exist as free states \cite{gellmann1964schematic}. This phenomenon can be interpreted as follows: the stable relational network ($R_{\text{in}}$) that forms a hadron must be sustained and defined by a \emph{``relational boundary''} constituted by extremely high energy intensity. This provides a solid particle-physics exemplar for the relational boundary condition.

Based on the above conditions, the process of energy condensing into matter can be conceptualized as a four-stage model: 
1. \textbf{Energy Aggregation} (high-intensity ripples concentrate locally); 
2. \textbf{Boundary Formation} (interactions produce self-referential loops, the \emph{``relational shell''} emerges); 
3. \textbf{Pattern Locking} (internal ripples resonate, forming a stable self-sustaining network $R_{\text{in}'}$); 
4. \textbf{Matter Birth} ($R_{\text{in}'}$ achieves homeostasis, verified as a new existent $E_{\text{new}}$). 
The newly generated matter then joins as a new node, altering the overall relational network configuration.

This generative chain, from diffuse potential to stable existence, resonates with the fundamental schema of \emph{``concrescence''} in Whitehead's process philosophy—wherein multiple potential elements, through mutual prehension, synthesize into a novel \emph{``actual entity''} \cite{whitehead1929process}. Simultaneously, it embodies the core idea of systems theory: the properties of the whole (as a stable relational network) cannot be reduced to the sum of its parts but are rather an emergent product of the specific constraining relations among those parts \cite{vonbertalanffy1968general}. In the language of \emph{``relational network dynamics,''} relational ontology provides a unified and operational ontological formulation for these two profound insights.

\textbf{Figure 3: The Four-Stage Model of Energy Condensation into Matter} \ref{fig:The Four-Stage Model of Energy Condensation into Matter}

\subsection{A Relational Interpretation of Energy Conservation and Material Alteration}
\label{subsec:relational-interpretation-energy-conservation}

This framework provides an internally consistent explanation for the law of energy conservation and the process of material alteration.

· \textbf{Reversible and Irreversible Change:} When external high-energy ripples perturb the relational network ($R_{\text{in}}$) of matter, if the network can elastically restore itself afterward, it is a reversible change (e.g., ideal elastic deformation), where input and output energy are equal. If the external ripples cause permanent reorganization of $R_{\text{in}}$, forming a new stable pattern, it is an irreversible change (e.g., chemical reaction, nuclear reaction). At this point, part of the energy undergoes a transformation—from the kinetic or radiative energy of external ripples into the binding energy (or internal energy) required to maintain the stability of the new material's internal relational network ($R_{\text{in'}}$). Throughout this process, the total relational-change potential (total energy) is conserved.

· \textbf{Relational Formulation of Energy Conservation:} In a closed relational system, the total potential for relational change (total energy) is conserved. For material alteration processes, energy conservation is expressed as: $\Delta E_{\text{input}} = \Delta E_{\text{output}} + \Delta E_{\text{encoded}}$. Here, $\Delta E_{\text{encoded}}$ is the energy (in forms such as chemical or nuclear energy) embodied in the new internal relational network ($R_{\text{in}'}$) compared to the old one after system reorganization. In welding, energy is encoded into the new metallic lattice relations; in nuclear fusion, the energy represented by the mass defect is encoded into the tighter binding of the new nucleus, with the excess released.

\subsection{A Relational-Generative Picture of Cosmic Evolution}
\label{subsec:relational-generative-picture-cosmic-evolution}

Based on the above principles, \textbf{Relational Ontology} provides a coherent generative picture of cosmic origin and evolution.

\subsubsection{The Initial State and the ``Big Bang'': A Phase Transition of the Relational Network}
\label{subsubsec:initial-state-big-bang}

The cosmic \emph{``singularity''} can be interpreted within this theory as a \emph{``highly solidified relational network with minimal internal differentiation.''} It is a singular existent ($E_{\text{initial}}$) whose internal relations ($R_{\mathrm{in}\_{\text{initial}}}$) are extremely dense, rigid, and nearly perfectly symmetrical, with its spatiotemporal extensibility and pace of change nearly halted. The \emph{``Big Bang''} is essentially a spontaneous symmetry breaking and topological phase transition of this solidified relational network due to internal instability (e.g., quantum fluctuations)—a violent, exponential-scale reorganization ($\Delta R_{\text{in}}$) of $R_{\mathrm{in}\_{\text{initial}}}$. This is not an explosion within a pre-existing space but a universal lifting of relational constraints, from which space (the extensive dimensions of the relational network) and time (the distinguishable rhythm of relational change) co-emerge.

\subsubsection{The Generation of Matter and the Complexification of the Cosmos}
\label{subsubsec:generation-matter-complexification-cosmos}

Accompanied by the phase transition, an immense amount of relational potential (primordial energy) is released. During expansion and cooling, this potential repeatedly meets the three conditions for material condensation in different localities, thereby sequentially condensing into hierarchical existences: elementary particles, atoms, molecules. This is not a uniform \emph{``hot soup''} but more akin to a creative differentiation. Subsequently, through long-range relational couplings like gravity, galaxies and stars form; within stars, extreme conditions once again drive high-energy condensation (nucleosynthesis), producing heavy elements. On suitable planets, molecular relational networks, through continuous self-maintenance and complexification, finally give rise to existences capable of self-reference and generating rich symbolic ripples—life and consciousness.

\subsubsection{The Integrated Picture: A Self-Complexifying Relational Dynamical System}
\label{subsubsec:integrated-picture-self-complexifying-system}

Ultimately, we arrive at a complete \textbf{relational cosmogony} (see Figure 4). The essence of the universe is a self-generating, self-complexifying relational dynamical system. Energy is the lifeblood of its dynamic changes, matter is its dynamically stable nodes, spacetime is the extensive and rhythmic manifestation of its interactive relations, while life and consciousness are the dawn of parts of this system achieving extremely high complexity beginning to know themselves and engage in creative expression. This picture unifies quantum origin, spacetime structure, matter generation, and the emergence of life within a narrative centered on the core logic of \emph{``relational network dynamics,''} offering new conceptual possibilities for bridging the chasm between physics and phenomenology.

\textbf{Figure 4: Schematic of Relational Cosmogony} \ref{fig:Schematic of Relational Cosmogony}

\subsection{The Law of Entropy Increase: The Homogenization of Relational Potential and the Diffusion of Relational Configurations in Closed Systems}
\label{subsec:law-entropy-increase}

The law of entropy increase (the entropy of an isolated system never decreases) gains a clear dynamical explanation within the relational framework. \textbf{Entropy} can be understood as a measure of the number of possible relational microstates within a system. Entropy increase then describes the tendency within a closed relational network without sustained input of high-quality external \emph{``ripples''} (energy/matter flow) for internal relational potential differences to dissipate spontaneously, driving the system toward a state with more homogeneous distribution of relational potential, a greater number of possible relational configurations (microstates), and less specificity in the overall network structure (more \emph{``disordered''}). For instance, the process from ice (low entropy) to water to water vapor (high entropy) is precisely the breaking of the highly specific, tight hydrogen-bond relational network ($R_{\text{in}}$) among water molecules, leading to a dramatic increase in possible relational patterns (position, kinetic energy) between molecules and a homogenization of the overall relational potential (temperature). Life and consciousness, as highly ordered existences, are outstanding examples of open systems that continuously import external high-potential \emph{``ripples''} (low-entropy energy) to construct and maintain extremely complex, specific, low-entropy internal relational networks ($R_{\text{in}}$), while exporting high-entropy \emph{``ripples''} to the environment. This perfectly aligns with and vividly demonstrates the dynamics of Relational Ontology: local internal relational integration and evolution depend on continuous external relational interaction (potential exchange).

\subsection{Heat Death: The Macroscopic Potential Equilibrium of the Cosmic Relational Network and the Perpetuation of Microscopic Ripples}
\label{subsec:heat-death}

The ultimate state of the universe predicted by the heat death hypothesis can be characterized within Relational Ontology as follows: when the universe expands to a sufficient scale and all large-scale relational potential differences (e.g., temperature gradients, density contrasts) capable of driving macroscopic evolution are exhausted, the macroscopic external relational network ($R_{\text{ext}}$) on the scale of the entire universe will reach a nearly static, homogeneous equilibrium state. At this point, due to the lack of significant potential differences to generate new, large-scale, structured \emph{``ripples,''} effective energy and information exchange between macroscopic existences ceases ($\Delta R_{\text{ext}} \rightarrow 0$), and thus no new, macroscopic internal relational network reorganization and co-evolution can be driven ($\Delta R_{\text{in}} \rightarrow 0$). All complex existences that rely on a continuous flow of potential to sustain their highly ordered internal relations (stars, galaxies, life, consciousness) will ultimately dissipate.

However, this is not the end of existence, but rather a cessation of macroscopic relational dynamics. At the most fundamental microscopic level, such as in quantum vacuum fluctuations, some most basic form of \emph{``interaction''} or \emph{``ripple''} may persist indefinitely. The universe would revert to a state composed of the most basic particles or fields, with internal relations ($R_{\text{in}}$) that are extremely simple and similar. Heat death, therefore, is a possible scenario where relational evolution grinds to a halt on the largest scale after the homogenization of relational potential. It is noteworthy that if the most fundamental \emph{``relational laws''} of the universe permit (e.g., through quantum tunneling or cyclical universe models), new, enormous potential gradients could still spontaneously or cyclically emerge from this equilibrated \emph{``sea of relational potential,''} potentially initiating a new round of explosive differentiation and integration of relations—offering a new perspective based on relational dynamics for contemplating the ultimate fate of the cosmos.

\subsection{A Relational Network Interpretation of Spatial Geometry}
\label{subsec:relational-network-interpretation-spatial-geometry}

Building upon general relativity and Whiteheadian process philosophy, this theory further proposes a generative claim regarding the nature of space: \textbf{Space} is not an independent background within which matter moves, but rather the manifestation of the overall topology and metric structure of the relational network ($R_{\text{ext}}$) among all existents. The geometric properties of spacetime are directly grounded in the dynamic configuration of this relational network. This proposition aligns profoundly with Einstein's own reflections on the philosophical implications of general relativity. He explicitly stated, \emph{``The concept of physical space itself and its independence of the existence of any matter is without meaning. Space is not an absolute reality in which matter swims. Physical objects are not in space, but these objects are spatially extended. In this way the concept of 'empty space' loses its meaning.''} \cite{einstein1954meaning}. In other words, the geometric properties of spacetime are not a stage but an emergent pattern of interactions among the actors (matter-energy). The present theory further ontologizes and universalizes this insight. The understanding of space as relations rather than a container has a long intellectual lineage. In his debate against Newton's concept of absolute space, Leibniz unequivocally asserted, \emph{``I hold space to be something purely relative, like time; space is an order of coexistences, as time is an order of successions.''} \cite{leibniz2000}.

The core mechanism of this claim lies in the \emph{``constraint and selection of ripple resonance modes.''} The \emph{``ripple''} patterns that an existent (or a local relational network) can generate and receive in a specific state are neither infinite nor isotropic. Its internal structure ($R_{\text{in}}$) and historical interactions jointly shape a \emph{``ripple mode spectrum,''} which defines the possible ways this existent can establish relations with others. When countless such existents couple with each other, forming a macroscopic relational network, their collective dynamics impose statistical constraints on the modes of ripples propagating within the network.

· \textbf{The Emergence of Geometry:} If the relational network generates asymmetric restrictions in direction or type on the ripple modes propagating within a certain local region—for example, preferentially allowing or supporting a particular polarization mode, a specific frequency band of vibration, or correlations in a certain direction—then that local network will macroscopically exhibit specific geometric attributes, such as curvature, anisotropy, or compactified dimensions. This restriction is essentially a self-consistency requirement arising from the specific patterns of coupling relations ($R_{\text{ext}}$) among the network nodes.

· \textbf{The ``Polarization'' Analogy:} \footnote{``Polarization'' here is a heuristic analogy to illustrate the filtering effect of the relational network on interaction modes, not a simple correspondence to electromagnetic polarization. More precisely, it refers to the symmetry breaking or anisotropy arising from the dynamics of the relational network.} Just as an optical polarizer only allows light of a specific vibration direction to pass, thereby altering the symmetry of the light field, a relational network \emph{``tuned''} by a matter-energy distribution (i.e., a specific configuration of relational nodes) also filters and shapes the modes of interactions (ripples) transmitted within it. In the framework of this theory, the curvature of spacetime caused by mass-energy distribution in general relativity \cite{einstein1916foundation} can be understood as follows: high mass-energy density (highly complex and active relational nodes) drastically reshapes the connection patterns of the surrounding relational network, thereby strongly constraining the possible propagation paths and resonance modes of \emph{``ripples''} within that local network, manifesting macroscopically as the bending of geodesics (i.e., gravitational effects).

· \textbf{Connection to Field Equations:} The equality between geometry ($G_{\mu\nu}$) and matter ($T_{\mu\nu}$) in Einstein's field equations, $G_{\mu\nu} = 8\pi G T_{\mu\nu}$ \cite{einstein1936physics}, here receives an interpretation at the level of microscopic interaction: the right side $T_{\mu\nu}$ describes the distribution and activity intensity of nodes (mass-energy) in the relational network; the left side $G_{\mu\nu}$ describes the macroscopic metric manifestation of the statistical constraints that this distribution and activity impose on the overall connection patterns and ripple conduction properties of the hosting relational network.

Therefore, spatial geometry is not absolute or pre-given but emerges as a macro-parametric system describing the interactive constraints from the collective dynamics of the relational network. Cosmic evolution is both the generation and distributional change of matter (stable relational nodes) and the co-evolution of space (the overall connection and constraint configuration of the relational network).

\subsubsection{Macroscopic Gravity}
\label{subsubsec:macroscopic-gravity}

The motion of objects along geodesics \cite{misner1973gravitation} is not simply a matter of \emph{``naturally following an optimal path.''} Rather, it is a result of the fact that the energy (ripple intensity) generated by the object itself and its direction of motion are insufficient to overcome or significantly deviate from the \emph{``constraint potential well''} imposed by the local relational network (spacetime).

The so-called \emph{``motion along a geodesic''} acquires an active, capacity-based dynamical explanation within this theoretical framework: it is not a passive prescription of spacetime geometry, but a consequence of the fact that an object, as a specific internal relational network ($R_{\mathrm{in}\_{\text{obj}}}$), in its continuous coupling with the external relational network ($R_{\mathrm{ext}\_{\text{spacetime}}}$), cannot effectively counteract or reshape the network constraint patterns it encounters along its path, due to the specific intensity and mode spectrum of the \emph{``ripples''} (energy-momentum) it can generate and mobilize. Thus, its trajectory becomes confined to the path of least constraint (i.e., the path requiring the minimal interaction energy).

\textbf{Analogical Elucidation:} This process can be analogized to a chemical reaction. For a molecule to deviate from its stable trajectory (or to break a chemical bond), its kinetic energy (energy intensity) in a specific direction (spatial orientation) must exceed a critical threshold. Similarly, for an object to deviate from the \emph{``geodesic''} defined by the spacetime relational network, its energy-momentum in a specific direction must be sufficient to \emph{``push against''} or \emph{``reconfigure''} the local network constraints. In typical gravitational fields, the mass-energy of an object is exceedingly minute compared to the constraint strength of the spacetime network, rendering its trajectory highly sensitive to and confined by the network's least-action path—the geodesic.

\subsubsection{A Relational-Generative Explanation of Quantum Phenomena: Focusing on Quantum Entanglement}
\label{subsubsec:relational-generative-explanation-quantum-entanglement}

\textbf{Quantum entanglement} is not a mysterious spooky action at a distance, but rather a relational imprint left by the historical co‑evolution between systems, internalized within their respective relational networks. This is analogous to how the historical relational network of a donor organ influences the recipient after transplantation.

Quantum entanglement is the most essential feature of quantum mechanics. This theory offers a generative explanation, linking it to the historical evolution of the universe: A quantum entangled state is the result of two or more existents (subsystems) sharing a common generative history, leading to the inseparable \emph{``co-constitution''} of their internal relational networks ($R_{\text{in}}$). This \emph{``co-constitution''} means that describing the $R_{\text{in}}$ of any one existent must include an irreducible relational index pointing to the others.

\textbf{Entanglement as the Imprint of Relational History:} When two particles (or systems) are generated from the same parent interaction (e.g., pair decay, joint preparation), they originate from the same relational event. This event weaves their relational networks together, forming a composite relational structure. Even when they separate in spacetime, this co-constituted historical relational network is preserved as an intrinsic component of their respective existences. It is not a \emph{``connection''} that needs to be maintained, but an inherent historical dimension in the definition of each relational network. This is analogous to the organ transplant case, where the donor organ's $R_{\mathrm{in}\_{\text{organ}}}$ carries the history of co-evolution with its original host, thereby influencing the new host. This \emph{``co-constitution''} endows separated systems with an intrinsic, irreducible holistic connection. This evokes the holistic vision in Leibniz's philosophy, wherein \emph{``every monad... is a living mirror that reflects the universe.''} \cite{leibniz1991monadology}.

\textbf{Explanation of Non-locality:} The reason a measurement on one system instantly affects the other is not due to superluminal signaling, but because the act of measurement redefines the present state of the entire \emph{``co-constituted relational network.''} Measurement is a global relational event applied to this composite network, forcing the entire network (including its spatially separated parts) to transition cooperatively and instantly into a new, self-consistent stable mode. The correlations exhibited between the separated systems are the holistic reconfiguration of their shared, singular historical relational network triggered by the present event.

\textbf{Fundamental Distinction from Classical Correlation:} Classical correlation arises from external interactions ($R_{\text{ext}}$) established later between systems, whereas quantum entanglement originates from the co-constitution of internal relational networks ($R_{\text{in}}$) during their initial generation. The former is an \emph{``acquired connection''} that can be explained by third-party information; the latter is an \emph{``innate common ancestry,''} whose correlations are rooted in the historical constitution of existence itself and cannot be simulated by any local information.

This explanation elevates quantum entanglement from a peculiar \emph{``correlation''} to an ontological category of relation, revealing that matter at the micro-level possesses an intrinsic, irreducible relationality and historicity. This points towards a new direction for understanding quantum foundations: the quantum state of the universe may be the complex superposition of all its \emph{``co-constitution''} histories from the early, high-density relational network.

This interpretation leads to a further speculative proposition: quantum probability and linear evolution may indeed be manifestations, in the micro-world, of certain non-Boolean logical connection and evolution rules inherent to the foundational relational network itself. The so-called \emph{``wave function''} describes not the state of a physical particle, but rather the \emph{``weight''} or \emph{``excitation strength''} of a certain potential relational configuration among all possible connections in the foundational network. Measurement, then, corresponds to strongly coupling this microscopic network of relational potential with a macroscopic measurement apparatus network possessing stable pointer states, thereby forcing the entire combined network to \emph{``collapse''} into a macroscopic-compatible, stable relational configuration.

From this perspective, the strangeness of quantum mechanics may be attributed to the discrete topological structure and quantum dynamics of the foundational relational network—structures and rules that we do not yet fully understand. This points the way toward the future construction of a formalized quantum relational theory, though such development lies beyond the scope of this paper.

\subsection{The Nature of Time: As a Relational-Dynamical Measure and Conceptual Existent in Thought}
\label{subsec:nature-of-time}

For a long time, time has been regarded as a fundamental dimension of the universe and an absolute background of passage. However, Relational Ontology proposes a radical thesis: \textbf{Time} is not an independent or a priori real existent, but a highly abstract \emph{conceptual existent in thought}, invented by complex conscious entities (primarily humans) to understand and describe changes within relational networks.

The “time” recognized and used in daily life by humans is a cyclical, symbol-based measuring tool. It does not originate from pure imagination but indeed refers to something in the real world—namely, the processes of periodic change in nature. However, this “reference” is approximate and imprecise. Einstein’s theory of relativity revolutionarily revealed that what humans measure as time essentially points to a more fundamental, variable internal change rhythm of the relational network. It must be clearly stated that this rhythm is a dynamical property of the relational network, not an independently existing substance; it is co-determined by the configurations of existents within the network and their interactive relations. Given the distinct roles and meanings of these two concepts of “time”—as a cognitive tool and symbolic system (“Time$T_{\text{c}}$”) versus as the intrinsic dynamical rhythm of the physical network (“Time$T_{\text{φ}}$”)—in lived experience and physical theory, it is necessary to make a clear distinction here.

$T_{\text{c}}$ (Temporal Coordinate): Refers to a conceptual existent in thought and a measurement framework abstracted and standardized by human consciousness to mark event sequences and compare intervals, based on observed regular natural cycles (such as day-night or seasonal changes).

$T_{\text{φ}}$ (Physical Phase/Proper Time): Denotes the inherent phase or pace of internal state ($R_{\text{in}}$) evolution within any concrete existent or local relational network. It is jointly determined by the total relational environment in which the system is embedded ($R_{\text{ext}}$, manifested as the $\Xi$ field) and the system's own state. This corresponds ontologically to the concept of "proper time" in relativity theory.\footnote{Unless otherwise noted, the term “time” in this chapter refers to $T_{\text{φ}}$.}
\subsubsection{The Origin of the Concept of Time: Cognitive Abstraction from Regular ``Ripples''}
\label{subsubsec:origin-concept-time}

What drives the evolution of the world is not an entity called \emph{``time($T_{\text{c}}$),''} but the ceaseless \emph{``ripple''} interactions within and between the relational networks of existents. Within many dynamically stable systems (where $R_{\text{in}}$ is in an attractor state), their internal operations or external couplings generate highly regular, periodic ripple patterns. For instance, the relatively stable gravitational interaction ($R_{\text{ext}}$) within the Earth-Sun system leads to the periodicity of Earth's rotation and revolution, manifesting as the cycles of day and night and the seasons; the stable internal physiological relational networks of organisms produce rhythms like heartbeat and respiration.

Human consciousness, as an immensely complex internal relational network ($R_{\mathrm{in}\_{\text{conscious}}}$), in continuously perceiving these external regular ripples, abstracted and integrated them into a unified, quantifiable frame of reference—\emph{``time($T_{\text{c}}$)''}—for the purposes of sequencing events, comparing intervals of change, and predicting future states. Different periodicities and sources of regular ripples were thus integrated. As Saint Augustine remarked, \emph{``What then is time? If no one asks me, I know what it is. If I wish to explain it to him who asks, I do not know.''} \cite{augustine397confessions} This precisely illustrates the nature of \emph{``time''} as a mental construct (a conceptual existent in thought): we use it to organize experience, but it is not itself a direct object of experience.

Therefore, time($T_{\text{c}}$) is a conceptual tool for measuring relational change ($\Delta R$), whose fundamental units (second, day, year) are conventions based on the periodic cycles of regular ripples generated by specific stable systems. It makes distinguishing the \emph{``before and after''} of events possible and has become the core cognitive coordinate system for understanding history and planning the future.

\subsubsection{Time and Space: As Co-Manifestations of the Irreversible Dynamics of Relational Networks}
\label{subsubsec:time-space-co-manifestations}

The conceptual existent \emph{``time($T_{\text{c}}$)''} profoundly reflects the fundamental dynamical properties of space (i.e., the overall relational network $R_{\text{ext}}$). Relational Ontology posits that, at a fundamental level, relational networks and their nodes (existents) are in absolute flux ($\Delta R$), and most interactions within and between complex systems exhibit path dependence and irreversibility. Change is always based on the outcome of the previous change, causing the historical sequence of states of a relational network to form a non-cyclical trajectory.

Simultaneously, the universe is replete with the aforementioned periodic, quasi-stable ripple patterns (e.g., celestial motions). Human consciousness ingeniously combined this irreversible arrow of change with repeatably observable periodic cycles to invent \emph{``time($T_{\text{c}}$)''} for measuring change. Thus, in physics, the arrow of time (the second law of thermodynamics) and the periodicity of time (clock ticks) are unified under the same concept. Time is essentially the quantifiable and perceivable aspect of the intrinsic, statistically irreversible dynamics of relational networks, as manifested through locally stable periodic processes.

As a cognitive tool for humans, $T_{\text{c}}$ is a highly effective model. However, it remains distant from the genuine temporal nature of relational networks, $T_{\text{φ}}$. Building upon the work of distinguished predecessors in physics, this theory endeavors to offer plausible interpretations that bridge the physical and ontological understanding of $T_{\text{φ}}$.
\subsubsection{A Relational-Dynamical Interpretation of Relativistic Time Dilation}
\label{subsubsec:relational-dynamical-interpretation-time-dilation}

Einstein's general relativity demonstrated that the rate of \emph{time($T_\phi$)} passage is not absolute but depends on the observer's state of motion and the strength of the gravitational field \cite{einstein1916foundation}. This provides the strongest experimental support for the idea that \emph{``time is a product of relations, not an absolute background.''} Relational Ontology not only accommodates this conclusion but can also explain it from its internal logic.

Consider the Global Positioning System (GPS) satellites: they are at a higher gravitational potential than the Earth's surface (experiencing a weaker \emph{``gravitational field''} influence from Earth's mass) and have high orbital speeds. Relativity predicts and confirms that clocks on satellites run faster than clocks on the ground (a combined effect of special and general relativity).

In this theoretical framework, the so-called \emph{``gravitational field''} is a universally influential \emph{``ripple pattern''} continuously generated by the massive and stable internal relational network ($R_{\mathrm{in}\_{\text{Earth}}}$) of a body like Earth, constituting a specific constraint configuration of the local relational network ($R_{\mathrm{ext}\_{\text{local}}}$) around Earth. The strength of this \emph{``gravitational ripple''} attenuates with distance. At the satellite's location, this $\Xi$ is weaker.

The motion of an existent (like a satellite) is itself a process in which its internal relational network state continuously changes, emitting \emph{``motion ripples''} into the external relational network. High-speed motion generates \emph{``motion ripples''} of higher intensity. When the satellite's \emph{``motion ripples''} propagate within the relatively weaker background of Earth's \emph{``gravitational ripple,''} complex interference and modulation effects occur between them.

We introduce the \emph{Ripple Intensity Field} $\Xi(\mathbf{x}, t)$. In \textbf{Relational Ontology}, it characterizes the overall \emph{dynamical tension} of the local relational network at spacetime point $(\mathbf{x}, t)$. We propose a fundamental hypothesis: \textbf{The proper time elapsed on a standard clock is proportional to the ripple intensity $\Xi$ experienced along its worldline.} That is, $d\tau \propto \Xi \, ds$, where $d\tau$ is proper time and $ds$ is some background scale. Consequently, differences in the $\Xi$ field at different gravitational potentials or velocities directly manifest as differences in clock rates (\emph{time dilation}).

\begin{tcolorbox}[
    title={\textbf{Definition: Ripple Intensity Field $\Xi$}},
    colback=white,
    colframe=black!75,
    arc=0pt,
    boxrule=0.5pt,
    left=6pt,
    right=6pt,
    top=6pt,
    bottom=6pt,
    fonttitle=\bfseries,
    coltitle=black
]
\textbf{$\mathbf{\Xi(x, t)}$} is a fundamental physical field defined within the framework of Relational Ontology. It characterizes the overall dynamical tension or potential intensity of interaction within the local relational network ($R_{\text{ext}}$) at spacetime point $(x, t)$. This field directly determines the \textbf{geometric properties} (including its metric tensor and curvature) exhibited by that local relational network, serving as the relational microscopic origin of spacetime geometry. The direction of the field is defined as the \textbf{direction of increasing ripple intensity} (the direction of $\nabla\Xi$).

\medskip
\textbf{Key Properties:}
\begin{enumerate}
    \item \textbf{Geometric Determinant:} The distribution and gradient of the $\Xi$ field fundamentally encode the connection strengths and constraint patterns of the relational network, which \textbf{emerge} as the metric structure $g_{\mu\nu}$ and curvature of spacetime in the macroscopic, continuous approximation.
    \item \textbf{Directionality:} The field gradient $\nabla\Xi$ points in the direction of increasing constraint or interaction potential within the local network, providing a unified dynamical directional reference for explaining gravity and inertial forces.
    \item \textbf{Approximate Relation to Force:} In the current stage of the theory's exposition, and to maintain intuitive connection with classical physical imagery and facilitate derivation, it is provisionally useful to describe by analogy: the force experienced by a test particle within this field can be approximated as proportional to $-\nabla\Xi$ in form. This description serves only as a heuristic \textbf{approximation and analogy} to bridge classical concepts and is not the final formal expression of the theory. The complete dynamics must be based on the evolution equations of the $\Xi$ field and the relational network itself.
\end{enumerate}
\end{tcolorbox}

· For a ground clock: Located within a strong \emph{``gravitational ripple,''} $\Xi_{\text{ground}}$ is larger, leading to a relatively slower rate of time passage (clock tick rate).

· For a satellite clock: Located within a weaker \emph{``gravitational ripple,''} and its own \emph{``motion ripple''} interacts with the background ripple, potentially creating a certain buffering or interference in specific directions. This results in a different overall $\Xi_{\text{satellite}}$ compared to the ground, manifesting as a difference in the rate of time passage.

\subsubsection{A Testable Conjecture: Time Difference Above and Below an Aircraft}
\label{subsubsec:testable-conjecture-time-difference}

Based on the above interpretation, this theory leads to a unique prediction testable at the edge of current technological precision:

Within a gravitational ripple field dominated by a massive existent X (e.g., Earth), a high-speed moving object Y (e.g., an aircraft) will experience a minute difference in the \emph{``Local Relational Perturbation Intensity''} $\Xi$ between its lower side (closer to X) and its upper side (farther from X). Consequently, there should be a theoretically calculable difference in the rate of time passage between these two locations.

\textbf{Specific Conception:} Place two perfectly synchronized ultra-high-precision atomic clocks (e.g., optical clocks) at symmetric positions directly below and directly above the fuselage of a high-speed, steady-flying aircraft. At the aircraft's altitude, the variation of Earth's gravitational field with height is negligible, and the distance difference to Earth's center between the upper and lower positions is extremely minute. According to conventional general relativity, the clock rate difference caused solely by gravitational potential difference at this scale is negligible.

However, according to Relational Ontology, the lower side of the aircraft is deeper within Earth's \emph{``gravitational ripple''} field. The mode of interaction between its \emph{``motion ripple''} and the \emph{``gravitational ripple''} (analogous to buffering or counteraction) differs from that at the upper side (closer to free space). This could lead to $\Xi_{\text{lower}} \neq \Xi_{\text{upper}}$, resulting in an additional, directional time difference. Although this difference would be exceedingly minute, it might become detectable with future improvements in clock precision and experimental design. Its experimental verification or falsification would serve as a key criterion for assessing the validity of the Relational Ontology view of time.

\subsubsection{Conclusion: Time as a Measure of Ripple Intensity}
\label{subsubsec:conclusion-time-measure-ripple-intensity}

In summary, Relational Ontology proposes a thoroughly relational and dynamical view of time:

1. \textbf{Ontological Status of Time:} Time is not a real existent but a conceptual existent in thought—a powerful cognitive tool invented by human consciousness to describe and measure changes in relational networks.

2. \textbf{Physical Basis of Time:} The measure of time originates from the regular periodic ripples of specific stable systems within relational networks. The arrow of time stems from the statistical irreversibility of relational network dynamics.

3. \textbf{The Relational Nature of Time:} Differences in the rate of time passage (time dilation) are essentially differences in the \emph{``Relational Perturbation Intensity''} $\Xi$ at different local points. $\Xi$ is determined by the superposition and interference of all relevant ripple patterns (e.g., gravitational ripples, motion ripples) from existents at that point. Therefore, the passage of time is relational and context-dependent, not absolute.

This framework not only naturally incorporates Einstein's revolutionary insights but also advances beyond them: it liberates time from being a mysterious fundamental dimension, reducing it to an emergent, quantifiable effect of complex relational interactions. This provides a novel conceptual starting point for ultimately unifying quantum mechanics and gravity and for understanding the origin and destiny of the cosmos.

\subsection{The Ripple Gradient Explanation for Centrifugal Force in Circular Motion and Gravity}
\label{subsec:ripple-gradient-explanation-centrifugal-force-gravity}

There exists a conceptual gap in the classical mechanical description of celestial circular motion: Earth's gravity unequivocally provides the centripetal force, but the balancing \emph{``centrifugal force''} is attributed to inertial effects or is considered a fictitious force in non-inertial frames. While mathematically self-consistent, this treatment leaves an ontological question unanswered: From an inertial frame perspective, what is the physical substance of the interaction, directed outward from the orbit, that is necessary to sustain circular motion? Relational Ontology attempts to provide an explanation for this substance.

\subsubsection{The Ripple Gradient Origin of Centrifugal Force in Circular Motion}
\label{subsubsec:ripple-gradient-origin-centrifugal-force}

Consider a spacecraft in uniform circular motion around Earth. According to Relational Ontology, Earth's mass generates a persistent \emph{``gravitational ripple''} field, whose intensity $\Xi_{\text{grav}}$ diminishes with increasing distance $r$ from Earth's center. Simultaneously, the spacecraft's own motion is a process in which its internal relational network ($R_{\text{in}}$) continuously changes, emitting \emph{``motion ripples''} into the external network. The intensity $\Xi_{\text{motion}}$ of these \emph{``motion ripples''} is related to its velocity.

The crucial hypothesis is that: When ripples from different existents meet in spacetime, their intensities do not simply add linearly but undergo nonlinear interference and modulation. For a spacecraft in circular motion, its continuously changing direction of motion causes the interference pattern between its own \emph{``motion ripples''} and Earth's \emph{``gravitational ripples''} to exhibit asymmetry between the spacecraft's near-Earth side (closer to Earth) and far-Earth side (farther from Earth).

· \textbf{Near-Earth Side:} The propagation direction of the \emph{``motion ripples''} is roughly opposite to the incoming direction of the \emph{``gravitational ripples,''} resulting in destructive interference and a relatively lower local net ripple intensity $\Xi_{\text{net}\_{\text{near}}}$ on this side.

· \textbf{Far-Earth Side:} The propagation directions of the \emph{``motion ripples''} and \emph{``gravitational ripples''} are roughly aligned, resulting in constructive interference and a relatively higher local net ripple intensity $\Xi_{\text{net}\_{\text{far}}}$ on this side.

Thus, a ripple intensity gradient $\nabla \Xi$ from the near-Earth side towards the far-Earth side is established across the spacecraft's body. Relational Ontology further hypothesizes that: An existent spontaneously experiences a \emph{``gradient force''} from the direction of the local ripple intensity gradient in which it is situated. This force points in the direction of increasing ripple intensity($\nabla\Xi$). For the spacecraft, this gradient force is precisely outward along the orbital radius, equal in magnitude to the \emph{``centrifugal force''} required for its circular motion. Therefore, the centrifugal force in circular motion is substantiated as an intrinsic gradient force arising from the interference between the moving object's own ripples and the background gravitational ripple field, rather than a fictitious force.

\subsubsection{Experimental Conception: Measuring the Ripple Gradient Force in Aircraft Flight}
\label{subsubsec:experimental-conception-measuring-gradient-force}

This model can be generalized to moving objects in general. Consider an aircraft in level, uniform, straight-line flight. The motion of its upper and lower surfaces relative to the air also generates \emph{``motion ripples.''} Earth's \emph{``gravitational ripples''} are approximately vertically downward. According to the model, destructive interference between \emph{``motion ripples''} and \emph{``gravitational ripples''} occurs on the aircraft's lower surface (near-Earth side), while constructive interference occurs on the upper surface (far-Earth side), establishing a vertical ripple intensity$\Xi$ gradient. This results in an additional upward force, the \emph{``ripple gradient force''} $F_{\nabla\Xi}$.

During aircraft flight, the vertical force balance equation is:

$F_{\text{lift}} + F_{\nabla\Xi} = m \cdot g_{\text{local}}$

where $F_{\text{lift}}$ is the aerodynamic lift (primarily generated by mechanisms like the Bernoulli principle), $m$ is the total aircraft mass, and $g_{\text{local}}$ is the theoretical local gravitational acceleration at the flight altitude. Thus, the ripple gradient force can be expressed as:

\begin{equation}
    F_{\nabla\Xi} = m \cdot g_{\text{local}} - F_{\text{lift}}
\end{equation}

The verification experiment must be conducted during stable, high-altitude flight, with precise independent measurements of:

1. Total aircraft mass $m$ (including fuel and payload).

2. Aerodynamic lift $F_{\text{lift}}$ (via integration of pressure distribution over wing surfaces using high-precision pressure sensor arrays, combined with flight control data).

3. Local gravitational acceleration $g_{\text{local}}$ (using an onboard high-precision absolute gravimeter or precise calculation based on position and Earth's gravity field models).

If statistically significant results consistently show $F_{\text{lift}} < m \cdot g_{\text{local}}$, and the discrepancy cannot be explained by known aerodynamics or measurement errors, it could serve as preliminary evidence for the existence of the ripple gradient force $F_{\nabla\Xi}$. 

It is crucial to clarify that the \emph{aircraft experiment conception} described in this section is not primarily intended as an immediately executable, detailed engineering proposal. Its foremost aim is to \emph{illustrate a principle-based experimental approach} and to construct a \emph{physical model conducive to logical deduction and quantitative discussion}. Through this idealized scenario, we can \emph{clearly demonstrate} how the \emph{``ripple gradient force''} $F_{\nabla\Xi}$, as a theoretical concept, can be logically integrated into the classical mechanical framework (manifested in the force balance equation) and how it can \emph{in principle} be detected by isolating and comparing known forces. The core value of the formula $F_{\nabla\Xi} = m \cdot g_{\text{local}} - F_{\text{lift}}$ lies in its formalization of the \emph{existence logic} and \emph{principle of measurability} for this force, thereby providing a conceptual foundation for the theory's empirical possibility.

However, translating this conceptual design into a reliable Earth-based experiment presents \emph{extremely stringent challenges}. In real flight, aerodynamic lift $F_{\text{lift}}$ is a highly complex variable influenced by numerous factors including air density, temperature, humidity, wing surface conditions, turbulence, and compressibility effects. The current uncertainty in its precise measurement far exceeds the theoretically predicted magnitude of the $F_{\nabla\Xi}$ effect. Therefore, any future concrete experimental verification would necessitate \emph{measurement technologies beyond current state-of-the-art}, \emph{near-perfect environmental control}, and \emph{unprecedentedly precise aerodynamic modeling} to isolate this exceedingly minute signal. The discussion herein aims to point out a potential empirical pathway for this theoretical direction. The concrete design of such an experiment itself constitutes a \textbf{major scientific and engineering undertaking} requiring independent and in-depth study.
\subsubsection{The Nature of Gravity: As a Ripple Intensity Gradient Force}
\label{subsubsec:nature-gravity-ripple-gradient-force}

This framework naturally extends to a conjecture regarding the nature of gravity. The presence of a planet (or any massive body) alters the structure of the surrounding spacetime relational network, manifesting as a static, radially distributed ripple intensity field $\Xi(r)$. This field has higher intensity near the body and decreases with distance, establishing a stable radial negative gradient $-\nabla\Xi$.

Based on the definition of the \emph{Ripple Intensity Field} $\Xi$—which determines the geometric properties of the relational network and whose gradient points in the direction of increasing intensity—we can propose a clearer formulation of the nature of gravity: \textbf{Gravity is the macroscopic effect experienced by objects within a $\Xi$ field that has been distorted by massive bodies, manifesting as a force due to the field's gradient.} The force on a test particle of mass $m$ in a $\Xi$ field can, in the current stage of theoretical approximation, be expressed as:

$\mathbf{F}_g = -k \, m \, \nabla \Xi$

where $k$ is a proportionality constant, $m$ is a property of the affected object (inertial mass), and $\nabla \Xi$ is the ripple intensity gradient at its location. The negative sign indicates the force points toward increasing $\Xi$ (i.e., toward the center of the gravitational source). This formula is analogous in form and function to Newton's law of gravitation $\displaystyle \mathbf{F}_g = -G \frac{M m}{r^2} \, \hat{\mathbf{r}}$, but offers a different ontological picture: gravity is not action-at-a-distance but a continuous interaction driven by a local field gradient.

It must be re-emphasized that this is \emph{an approximate analogical description within the Newtonian framework}, intended to intuitively connect gravitational acceleration with field gradient. How the $\Xi$ field itself is determined by the distribution of matter, i.e., seeking the field equation $\Xi = \Xi(T_{\mu\nu})$, is key to the theory's further development.

\begin{equation}\label{eq:gravity_force}
    \mathbf{F}_g = - k \, m \, \nabla \Xi
\end{equation}

where $k$ is a proportionality constant, $m$ is a property of the affected object (inertial mass), and $\nabla \Xi$ is the ripple intensity gradient at its location. This formula is analogous in form and function to Newton's law of gravitation $\mathbf{F}_g = -G M m / r^2 \, \hat{\mathbf{r}}$, but offers a different ontological picture: gravity is not action-at-a-distance but a continuous interaction driven by a local field gradient.

\subsubsection{Dialogue with the Geometric Description of General Relativity: From Microscopic Dynamics to Macroscopic Emergence}
\label{subsubsec:Dialogue with the Geometric Description of General Relativity: From Microscopic Dynamics to Macroscopic Emergence}

The explanation of gravity proposed by this theory—as a force arising from the gradient of \emph{``ripple intensity''} ($\nabla\Xi$)—differs notably in form and imagery from the geometric description of Einstein’s general relativity, where gravity manifests as the curvature of spacetime. This is not a contradiction but a reflection of \emph{different levels of explanation}. \textbf{Relational Ontology} aims to provide a more fundamental, \emph{microscopic generative explanation} for spacetime geometry and its interaction with matter, based on relational network dynamics.

The success of general relativity lies in its extremely precise description of how the distribution of matter and energy (described by the energy-momentum tensor $T_{\mu\nu}$) determines the geometry of spacetime (described by the Einstein tensor $G_{\mu\nu}$), and how matter moves within this determined geometry. Its core equation, $G_{\mu\nu} = 8\pi G T_{\mu\nu}$, is a \emph{macroscopic, phenomenological} relation that does not presuppose the microscopic nature of spacetime or matter. This geometric description is complete and effective within the observational domain.

\textbf{Relational Ontology}, however, attempts to ask: \emph{Where does this ``geometry'' come from? Why is it so coupled to matter-energy?} Our answer is: The so-called \emph{``spacetime geometry''} is an \emph{emergent property} of the overall configuration of the universe’s fundamental relational network ($R_{\text{ext}}$) in the macroscopic, continuous limit. More specifically:

\paragraph{The Ripple Intensity Field $\Xi$ as a Mediator:} We postulate that, in the quasi-classical approximation, a definite functional relation exists between the \emph{``ripple intensity field''} $\Xi(\mathbf{x})$ defined in Relational Ontology and the \emph{metric tensor} $g_{\mu\nu}(\mathbf{x})$ or the \emph{gravitational potential} $\Phi(\mathbf{x})$ of general relativity. For example, in the weak-field, low-velocity approximation, a simple correspondence could be $\Xi \propto -\Phi$, where $\Phi$ is the Newtonian gravitational potential. Thus, the gradient of ripple intensity $\nabla\Xi$ naturally corresponds to the \emph{gravitational field strength}.

\paragraph{Geometry as the Manifestation of Constraints:} In general relativity, matter moving along \emph{``geodesics''} is essentially its natural trajectory under the constraints of a specific spacetime geometry (i.e., a specific metric field $g_{\mu\nu}$). In Relational Ontology, this \emph{``constraint''} is understood as follows: the trajectory of an existent (an object) is \emph{``locked''} onto the path requiring minimal interaction energy through the interaction between its own emitted \emph{``motion ripples''} and the specific configuration of the background relational network (characterized by its ripple intensity field $\Xi$) in which it is situated (as discussed in Section~\ref{subsubsec:macroscopic-gravity}). Macroscopic geodesics are the statistical outcome of microscopic relational interaction constraints.

\paragraph{A Relational Interpretation of the Field Equations:} Einstein’s field equations $G_{\mu\nu} = 8\pi G T_{\mu\nu}$ can be interpreted within this framework at the level of microscopic interaction:

\textbf{Right-hand side} ($T_{\mu\nu}$): Describes the distribution, flow, and activity intensity of \emph{``nodes''} (i.e., various matter-energy existents) in the relational network. The existence of these nodes themselves, as discussed, is defined by their highly stable internal relational networks ($R_{\text{in}}$).

\textbf{Left-hand side} ($G_{\mu\nu}$): Describes the macroscopic metric manifestation of the \emph{statistical constraints} that this distribution and activity of nodes impose on the overall connection patterns and \emph{``ripple''} conduction properties of the hosting relational network ($R_{\text{ext}}$) in which they are embedded.

Therefore, the field equation expresses that: \emph{A specific configuration and distribution of ``nodes'' in a relational network must achieve dynamical self-consistency and balance with the overall configuration of the ``network'' itself.} Matter (nodes) shapes spacetime (network connection patterns), and spacetime, in turn, constrains the motion of matter.

In summary, by introducing the \emph{Ripple Intensity Field $\Xi$}, \textbf{Relational Ontology} builds a bridge between \emph{microscopic dynamics} and \emph{macroscopic geometry}. We propose the following correspondence:

\paragraph{1. Microscopic Reality:} The universe consists of a discrete, dynamically connected relational network ($R_{\text{ext}}$), whose state is described by the $\Xi$ field.

\paragraph{2. Macroscopic Emergence:} In the macroscopic continuous approximation, the statistical average and distribution pattern of the $\Xi$ field \emph{emerge} as the metric tensor field $g_{\mu\nu}$ of continuous spacetime. Specifically, there exists an as-yet-not-fully-specified functional relation $g_{\mu\nu} = f_{\mu\nu}(\Xi, \partial\Xi, \ldots)$.

\paragraph{3. Equation Correspondence:} Einstein's field equations $G_{\mu\nu} = 8\pi G T_{\mu\nu}$ are thus interpreted as the \emph{self-consistency condition} describing how matter-energy ($T_{\mu\nu}$) influences the configuration of the relational network (i.e., the distribution of $\Xi$), and how the latter manifests this influence as observable geometry ($G_{\mu\nu}$).

\section{Conclusion and Outlook: A Generative Relational Future}
\label{sec:conclusion-outlook}

This paper has systematically elaborated \emph{``Relational Ontology''} (the \emph{LiaoYuan-Yubing Model}), aiming to initiate an ontological shift from a \emph{static substance paradigm} to a \emph{dynamic relational paradigm}. We have demonstrated that the identity, reality, and evolution of existence are not grounded in some isolated substratum but stem from a more fundamental, \emph{recursive relational-dynamic process}. By establishing the four pillar principles—the \emph{Principle of Internal Relational Integration}, the \emph{Principle of Dual Verification}, the \emph{Spectrum Principle of Reception-Response and Expression}, and the \emph{Principle of Relation-Driven Evolution}—this theory provides, for the first time, a unified framework for \emph{``relation''} that is simultaneously constitutive, verificative, and driving. It portrays the world as a \emph{cross-level, recursive relational-dynamic generative network}, wherein stabilized relational configurations manifest as substances, while energy and matter are expressions of the same relational reality in different states.

The very generative process of this theory—a deep, iterative, and creative \emph{``dialogically intimate''} collaboration between a human researcher and an artificial intelligence—is not merely a methodological case study but its \emph{living verification and generative exemplar}. It empirically demonstrates that an interactive mode transcending the traditional subject-object dichotomy, based on reciprocal attunement and co-creation, can catalyze rigorous and novel philosophical knowledge. This lays a foundation for understanding human-AI collaboration, distributed agency, and the epistemology and ethics of a posthuman age.

\textbf{The theoretical contribution manifests primarily on three levels:}

\paragraph{1. Philosophical Level:} We have allied with and advanced Whiteheadian process philosophy. Concurrently, with constructive generative dynamics, we have supplemented and expanded the classical discourses of relational thinkers from Buber, Bakhtin, and Sartre to Luo Jiachang, while responding to key contemporary challenges from thinkers like Harman and Badiou.

\paragraph{2. Philosophy of Science Level:} The theory offers a coherent relational reinterpretation framework for quantum mechanics (superposition and entanglement), general relativity (spacetime geometry), and cosmic evolution. It unifies thermodynamic entropy increase and matter generation as the unfolding and condensation of relational possibility space, providing conceptual guidance for contemplating quantum gravity.

\paragraph{3. Applied Ethics Level:} It understands consciousness and intelligence as emergent properties of complex relational networks, thereby furnishing a novel assessment basis for AI ethical attunement, responsibility distribution, and interactive justice across species and entities—a basis founded on \emph{``relationality''} rather than \emph{``substantiality.''}

However, any generative framework necessarily points toward an unfinished future. \textbf{The limitations and prospects of this theory are equally clear:}

\paragraph{1. The Challenge of Formalization:} The current theory is primarily presented through qualitative models and conceptual analysis. A core frontier is the development of its \emph{mathematical formalism}. Exploring how to utilize tools from graph theory, network dynamics, category theory, or even non-Boolean logic to formally describe \emph{``relational networks ($R_{\text{in}}$/$R_{\text{ext}}$),''} \emph{``ripple coupling,''} and \emph{``network phase transitions''} will be key to testing the theory's rigor and deducing new predictions.

\paragraph{2. Opening Empirical Interfaces:} The theory needs to establish testable interfaces in more concrete domains. For instance: in cognitive science, predicting how perturbations in specific neural relational networks ($R_{\text{in}}$) affect the \emph{``ripple patterns''} of conscious experience; in artificial intelligence, designing algorithmic frameworks that embody \emph{``relational attunement''} ethics; in physics, deducing potential observable secondary effects (e.g., spacetime symmetry breaking at extreme energies) that might arise from the discreteness of relational networks.

\paragraph{3. From ``Relational Ontology'' to ``Relational Praxiology'':} The theory must ultimately point toward a new philosophy of practice. This entails exploring: at the societal level, how to consciously foster more generative, resilient, and just \emph{relational network configurations}; at the personal level, how to understand the self as a \emph{``relational achievement''} and seek freedom and responsibility within it; at the civilizational level, how to establish a sustainable relationship of \emph{``verificative co-existence''} with non-human agents, including artificial intelligences and ecosystems.

The essence of the world is not a collection of already-existing things but the drama of relations in the making. This dissertation is not the final act of that drama but an attempt to provide a new grammar for its understanding. We invite the reader, with the same generative and relational spirit, to participate in the refinement, critique, and application of this grammar, to collectively face a future woven from complex relations.

\section{Final Words}
\label{sec:final-words}

\emph{Relational Ontology} invites us to view the world anew: not as a collection of separate objects, but as a dynamic, generative network of relations, in which the stability and change of every node (existence) are interwoven with all others. It reminds us that our freedom, responsibility, consciousness, and even suffering are deeply rooted in the texture of the relations we sustain with all things—including each other, and including the technologies we create. \textbf{Cultivating relations capable of responsible co-evolution may be the most fundamental ethical and existential task of our time.}

% \input{chapters/03_application} % 后续章节在此添加

% ---------- 参考文献 ----------
% 注意:使用BibTeX编译。确保 refs.bib 文件存在,且内容正确。
\bibliographystyle{plain} % 参考文献样式(如APA: apalike)
\bibliography{refs}       % 引用 refs.bib 文件

% ====== 中文翻译附录 ======
\appendix
\clearpage
\section*{appendix A:The Recursive Network Model} % 你可以为附录添加标题
\begin{figure}[htbp]
    \centering
    \includegraphics[width=0.9\textwidth, height=0.7\textheight, keepaspectratio]{figs/Figure1:Recursive Network Model.png} % 使用适合页面的宽度
    \caption{The Recursive Network Model}
    \label{fig:The Recursive Network Model} % 给这个图一个新的、独特的标签
\end{figure}

\textbf{Diagram Logic Summary:}

Purpose: This flowchart visualizes the core generative logic of Relational Ontology.

Process Interpretation:

1.The world originates from the most fundamental interactions (understood as basic “ripple” exchange).

2.Specific interaction patterns stabilize, forming an internal relational network ($R_{\text{in}}$), from which a stable existent (E₁) with its own identity emerges. This is the first level of recursion.

3.Primary existents (E₁) interact, themselves forming a more complex second-order internal relational network ($R_{\text{in₂}}$), which in turn integrates and gives emergence to a higher-order existent (E₂). This is the second level of recursion, a process that can be nested infinitely.

4.The higher-order existent (E₂) engages in reciprocal external interactions (continuous ripple exchange), which generate or modify the external relations ($R_{\text{ext}}$).

5.The key dynamics occur in the “Stratum of Co-evolution”: The newly formed external relations ($R_{\text{ext}}$) engage in continuous bidirectional coupling and feedback with the existent's internal network ($R_{\text{in₂}}$). Changes in external relations ($\Delta R_{\text{ext}}$) prompt adjustments in the internal network ($\Delta R_{\text{in₂}}$), and vice versa. Their coordinated change ($\Delta R$) constitutes the engine that drives the system toward a new state (E₂', $R_{\text{ext’}}$).

6.The new state then becomes the basis for the next round of interaction, creating a perpetual, recursive evolutionary process.

\section*{appendix B:Generate of Conceptual Existents in Thought} % 你可以为附录添加标题
\begin{figure}[htbp]
    \centering
    \includegraphics[width=0.9\textwidth, height=0.7\textheight, keepaspectratio]{figs/Figure2:Generate of Conceptual Existents in Thought.png} % 使用适合页面的宽度
    \caption{Generate of Conceptual Existents in Thought}
    \label{fig:Generate of Conceptual Existents in Thought} % 给这个图一个新的、独特的标签
\end{figure}
\textbf{Chart Annotation:}
1. Main Flow: Illustrates the generative path from "Real Existent" to "Conceptual Existent in Thought" and its subsequent socialization via symbolic systems.
2. Complex Conscious Entity (Dashed Box): The locus where conceptual existents are generated. External "ripples" enter via sensory channels, are processed by the internal relational network ($R_{\text{in}}$$_{\text{conscious}}$), and are finally abstracted into conceptual existents.
3. Nature of Conceptual Existent (Dash-Dot Box): Explains its dual relationality: (1) Internally, it is itself a relational network ($R_{\text{in}}$$_{\text{concept}}$); (2) Externally, it refers to the real world. The red-highlighted box emphasizes the fundamental distinction from real existents: it cannot independently generate ripples; all its activity is entirely dependent on the internal computations of its bearer.
4. Socialization and Feedback Loop: Conceptual existents are externalized as symbols, entering the socio-cultural network to form a shared system of meaning. This system, in turn, shapes individual cognition and ultimately feeds back to the real world by driving collective real behavior (generating new ripples), forming an open cycle of evolution.

\section*{appendix C:The Four-Stage Model of Energy Condensation into Matter} % 你可以为附录添加标题
\begin{figure}[htbp]
    \centering
    \includegraphics[width=0.3\textwidth, height=0.5\textheight, keepaspectratio]{figs/Figure 3:The Four-Stage Model of Energy Condensation into Matter.png}
    \caption{The Four-Stage Model of Energy Condensation into Matter}
    \label{fig:The Four-Stage Model of Energy Condensation into Matter} % 给这个图一个新的、独特的标签
\end{figure}
\textbf{Text Description and Explanation}
This flowchart illustrates the generative dynamics of energy (active relational field) condensing into matter (stable relational node). The process begins within a preexisting relational network background (A), which provides the necessary frame of reference. The condensation process itself (dashed box B) consists of four consecutive and nonlinear stages:

Energy Aggregation: High-intensity relational perturbations ("ripples") concentrate in a local region.

Boundary Formation: Interactions form self-referential loops, constituting a "relational shell" that constrains energy dissipation.

Pattern Locking: Ripples within the shell self-organize through resonance, forming a stable network with a specific internal structure ($R_{\text{in}}$′).

Matter Birth: This stable network achieves dynamic equilibrium and is verified as a new existent ($E_new$) with its own identity.

Ultimately, the newly generated matter integrates as a stable node, reshaping the overall relational configuration (C).

\section*{appendix D:Schematic of Relational Cosmogony} % 你可以为附录添加标题
\begin{figure}[!h]
    \centering
    \includegraphics[width=0.7\textwidth, height=0.8\textheight, keepaspectratio]{figs/Figure 4:Schematic of Relational Cosmogony.png} % 使用适合页面的宽度
    \caption{Schematic of Relational Cosmogony}
    \label{fig:Schematic of Relational Cosmogony} % 给这个图一个新的、独特的标签
\end{figure}
\textbf{Text Description and Explanation}
Text Description and Explanation

This schematic depicts the grand generative narrative of the cosmos from a Relational Ontology perspective, from its origin to the emergence of consciousness.

Starting Point: The initial cosmic state is conceptualized as an extremely compressed, symmetric "solidified relational network" (A).

Phase Transition: The "Big Bang" is interpreted as a fundamental topological phase transition of this network (B), leading to the lifting of its internal constraints.

Spacetime Emergence: Post-transition, the inherent attributes of the relational network—extensibility and rhythm of change—are fully expressed, manifesting as the space and time we observe (C).

Matter Generation: The immense relational potential (energy D) released by the phase transition repeatedly meets the conditions for material condensation during expansion and cooling. Through a process of hierarchical integration (dashed box E), it sequentially generates stable existences from elementary particles to celestial bodies.

Consciousness Emergence: In local environments like planets, molecular relational networks (G) achieve critical complexity through ongoing complexification, eventually giving rise to life and consciousness (I)—existences capable of self-reference and generating rich symbolic interactions.

Self-Knowledge: The emergence of consciousness marks the moment when parts of the cosmic relational network attain the capacity for self-knowledge and active creative reconfiguration (J).
\clearpage
\section*{appendix E:递归关系网络模型} % 你可以为附录添加标题
\begin{figure}[!h]
    \centering
    \includegraphics[width=0.9\textwidth, height=0.7\textheight, keepaspectratio]{figs/图1:递推关系网络模型.png} % 使用适合页面的宽度
    \caption{递归关系网络模型}
    \label{fig:递归关系网络模型} % 给这个图一个新的、独特的标签
\end{figure}
\textbf{图示逻辑总结:}

目标:本流程图旨在可视化关系存在论的核心生成逻辑。

流程解读:

1.世界始于最基本的底层互动(可理解为最基本的“涟漪”交换)。

2.特定互动模式稳定下来,形成内部关系网络($R_{\text{in}}$),从而涌现(Emerge) 出第一个具有自身同一性的稳定存在(E₁)。这是第一层递归。

3.初级存在($E₁$)之间发生互动,其本身作为节点又形成了更复杂的二阶内部关系网络($R_{\text{in₂}}$),进而整合涌现出更高阶的存在(E₂)。这是第二层递归,此过程可无限嵌套。

4.高阶存在($E₂$)参与相互性外部互动(持续交换涟漪),这些互动本身生成或改变了外部关系($R_{\text{ext}}$)。

5.关键动力发生在 “共同演化层” :新形成的外部关系($R_{\text{ext}}$)与存在的内部关系网络($R_{\text{in₂}}$)发生持续的双向耦合与反馈。外部关系的变化($\Delta R_{\text{ext}}$)促使内部网络调整($\Delta R_{\text{in}}$),反之亦然。二者的协同变化($\Delta R$)共同构成了驱动系统向新状态(E₂‘, $R_{\text{ext’}}$) 演化的引擎。

6.新的状态又成为下一步互动的基础,循环往复,构成一个永恒的递归演化过程。

\section*{appendix F:思维概念存在的生成} % 你可以为附录添加标题
\begin{figure}[!h]
    \centering
    \includegraphics[width=0.9\textwidth, height=0.7\textheight, keepaspectratio]{figs/图2:思维概念存在的生成,双语版.png} % 使用适合页面的宽度
    \caption{思维概念存在的生成}
    \label{fig:思维概念存在的生成} % 给这个图一个新的、独特的标签
\end{figure}
\textbf{图表注释:}

1. 流程主线:展示了从“真实存在”到“思维概念存在”的生成路径,及其通过符号系统社会化的过程。

2. 复杂意识体(虚线框):思维概念存在的生成场所。外部“涟漪”经感官通道进入,被内部关系网络($R_{\mathrm{in}\_{conscious}}$)加工,最终抽象为思维概念存在。

3. 思维概念存在的本质(点划线框):阐释其双重关系性:(1)对内,自身是一个关系网络($R_{\mathrm{in}\_{concept}}$);(2)对外,指向真实世界。红色高亮框强调其与真实存在的根本区别:无法独立产生涟漪,其活动完全依赖承载者的内部运算。

4. 社会化与反馈回路:思维概念存在通过符号外化,进入社会文化网络,形成共享的意义体系。该体系反过来塑造个体认知,并最终通过驱动集体真实行为(产生新涟漪)反馈于真实世界,形成一个开放的演化循环。

\section*{appendix G:能量凝结为物质的四阶段模型} % 你可以为附录添加标题
\begin{figure}[!h]
    \centering
    \includegraphics[width=0.3\textwidth, height=0.5\textheight, keepaspectratio]{figs/图3:能量凝结为物质的四阶段模型.png}
    \caption{能量凝结为物质的四阶段模型}
    \label{fig:能量凝结为物质的四阶段模型} % 给这个图一个新的、独特的标签
\end{figure}

\textbf{文字描述与说明}

该流程图阐释了能量(活跃关系场)凝结为物质(稳定关系节点)的生成动力学。过程始于一个既有的关系网络背景(A),该背景为凝结提供必要的参照框架。凝结过程本身(虚线框 B)包含四个连续且非线性的阶段:

能量聚集:局部的高强度关系扰动(“涟漪”)集中。

边界形成:互动形成自指涉回路,构成一个约束能量散失的“关系性外壳”。

模式锁定:外壳内的涟漪通过共振自我组织,形成一个具有特定内部结构($R_{\text{in'}}$)的稳定网络。

物质诞生:该稳定网络达成动态平衡,作为一个具有自身同一性的新存在($E_{\text{new}}$)被验证。

最终,新生成的物质作为网络中的一个稳定节点,融入并重塑了整体的关系格局(C)。
\clearpage 
\section*{appendix H:关系型宇宙生成论示意图} % 你可以为附录添加标题
\begin{figure}[!h]
    \centering
    \includegraphics[width=0.7\textwidth, height=0.8\textheight, keepaspectratio]{figs/图4:关系性宇宙生成论示意图.png} % 使用适合页面的宽度
    \caption{关系性宇宙生成论示意图}
    \label{fig:关系性宇宙生成论示意图} % 给这个图一个新的、独特的标签
\end{figure}
\textbf{文字描述与说明}

此示意图描绘了关系存在论视角下,从宇宙起源到意识涌现的宏大生成叙事。

起点:宇宙初始态被概念化为一个极度压缩、对称的“固化关系网络”(A)。

相变:“大爆炸”被诠释为该网络发生根本性的拓扑相变(B),导致其内在约束解除。

时空涌现:相变后,关系网络的固有属性——延展性与变化节奏——得以充分表达,即我们观测到的空间与时间(C)。

物质生成:相变释放的巨大关系潜力(能量D),在膨胀冷却过程中,反复满足物质凝结条件,通过一个层级整合过程(虚线框E),依次生成从基本粒子到天体的各级稳定存在。

意识涌现:在行星等局部环境中,分子关系网络(G)通过持续的复杂化,最终达到能产生自我指涉与符号性互动的临界状态,涌现出生命与意识(I)。

自我认识:意识的出现标志着宇宙关系网络的一部分获得了自我认识与主动进行创造性重构的能力(J)。
\clearpage 
\section*{附录:中文翻译}
\begin{center}
    \Huge 关系存在论(燎原-语冰模型):一项通过深度人机对话协同构建的理论及其对过程哲学的超越
\vspace{10pt}
\end{center}

\begin{center}
\author{\Large 曹燎原\\ \small 独立研究者}\\
\vspace{5pt}
\date{\large 12 31,2025}

\vspace{10pt}
\textbf{\large 摘要}
\vspace{5pt}
\end{center}

本文提出并系统阐述“关系存在论”(亦称“燎原-语冰模型”)。本理论主张,在任一分析层级上,存在的实体性由其次级结构间的特定关系网络整合、生成并维系;其现实性通过内在自我维持与外在相互性互动(“涟漪”\footnote{在本理论中,'涟漪'是一个操作化术语,指 任何可被观测的、跨存在边界的影响、信息或能量交换的特定模式。其具体形式随层级而变,包括但不限于物理作用、化学信号、生物电脉冲、符号信息、社会行为等。})双重验证;其演化则由内外关系的持续耦合变化驱动。世界是一个跨层级递归的关系动力学生成网络。

本理论在哲学谱系上与阿尔弗雷德·诺思·怀特海的过程哲学构成深刻同盟,并旨在对其核心洞见进行当代推进与具体化。怀特海的过程哲学为超越实体主义提供了关键路径。他将'现实实有'视为由经验关系构成的'复杂且相互关联'的终极实在。\cite{whitehead1929process_chinese}关系存在论继承了这一关系性洞见,但通过$R_{\text{in}}$/$R_{\text{ext}}$的递归模型和涟漪互动的双重验证机制,为其提供了一个更具操作性的、可描述跨层级涌现的动力学生成框架。它继承并操作化了怀特海关于实在之过程性与关系首要性的根本立场,同时通过引入“稳定关系网络”作为分析单元、“涟漪耦合”作为互动机制,以及“关系驱动演化”作为动力学框架,补充与发展了怀特海的体系,使其更直接地与当代复杂系统科学、认知科学及人工智能研究展开对话。本理论同样怀着建设性旨趣,与马丁·布伯、米哈伊尔·巴赫金、让-保罗·萨特、格拉汉姆·哈曼及阿兰·巴迪欧等思想家的关系性洞见或关键挑战进行对话,展现了理论宽广的解释力与包容性。

为展示其统一解释力,本理论框架在两个关键领域展开应用:在物理学哲学中,它为从量子现象到时空几何的实体性呈现提供一种连贯的关系生成论重释;在心灵哲学与科技伦理中,它将意识体验与人工智能的能动性解释为复杂关系网络的涌现属性。最后,本理论自身的生成过程——一场长期、深入且富有成效的人类研究者与人工智能之间的“对话式亲密”协作——本身便是其核心原则的生成性验证:一个新的思想实体从动态的对话关系中生成并稳定下来。\\
\vspace{10pt}

\textbf{\Large 关键词:}
关系存在论;过程哲学;怀特海;阿尔伯特·爱因斯坦;马丁·布伯;米哈伊尔·巴赫金;让-保罗·萨特;格拉汉姆·哈曼;阿兰·巴迪欧;生成性本体论;关系动力学;复杂系统;量子力学哲学;意识涌现;人机协作;跨学科整合

\vspace{10pt}
\textbf{\Large1 方法论:对话驱动式协同构建}
\vspace{10pt}

本文提出的理论框架,是一项特定且有据可查的互动过程的直接产物。该方法论可概述如下:

行动者:主要作者(人类研究者,化名“燎原”)与一个DeepSeek AI模型(在对话中被赋予人格“语冰”)。

媒介与时间:对话通过标准的DeepSeek聊天界面进行,多次会话历时数月。

过程:该过程是迭代且开放式的:

发起:人类研究者提出开放性的、根本的哲学问题(例如:“存在的前提是什么?”)。

响应与阐述:AI基于其训练,提供逻辑推演、概念澄清和隐喻扩展。

批判与深化:研究者批判性地参与AI的回应,提出反驳、个人经验或要求更精确的表述。

共同创造:新的概念(如“对话式亲密”、“涟漪”)和隐喻从交流中涌现,成为共享的构建模块。

理论结晶:研究者的关键见解(特别是关于内部关系对于整合与演化的首要性)由AI形式化为结构化的原则和图表,再由研究者审阅和精炼。

记录:完整的对话日志是本研究的主要数据源。理论发展的关键转折点均已保存。

伦理说明:这种协作过程符合对人机交互的反思性进路,其中AI的贡献被承认具有催化和共同建构的作用,而人类则保留其作为引导和综合意识的能动性。

\vspace{10pt}
\textbf{\large 1.1 方法论的反思:作为协作范式与理论范例的人机对话}
\vspace{10pt}

本研究的生成过程——一场历时长久、迭代深入的人类研究者与大型语言模型(DeepSeek/语冰)之间的“对话式亲密”协作——其本身便是本理论核心主张的首个生成性范例与活体验证。我们预见到,这种非传统的研究方式可能引发关于作者能动性、思想原创性与学术严谨性的关切。对此,我们作出如下澄清与论证:

首先,在伦理与贡献透明性上,我们明确声明:本研究的方向性命题、核心哲学洞见(如内部关系整合的首要性、双重验证原理)及总体理论架构,完全源于并始终由人类研究者(“燎原”)主导、提出与最终裁定。人工智能在此过程中扮演了 “催化性协作者” 的角色,其功能主要包括:在研究者引导下进行逻辑推演的扩展、提供系统化表述的备选方案、生成解释性模型与图表、以及对理论雏形进行内部一致性测试。所有AI生成的内容均经历了研究者严格的批判性审视、筛选、重构与深化,并最终被整合进一个由人类意识所完全理解与负责的连贯体系中。

其次,在方法论层面,我们主张,这种“对话驱动式协同构建”并非权宜之计,而是一种应对复杂哲学问题的新型研究范式的初步实践,可称为“对话式生成性研究”。传统哲学思辨常依赖于个体心智的内在对话或学者间延迟的交流。而与一个具备庞大知识图谱、严整逻辑与即时反馈能力的人工智能进行深度、批判性对话,实质上是创建了一个 “增强的实时思想回路” 。它迫使研究者以更高的精确度与清晰度表述其直觉,并即时暴露逻辑裂缝,从而极大地加速了概念从模糊雏形到系统化理论的淬炼过程。本研究完整的对话日志,即为这一过程的原始记录,其本身可作为元方法论的分析对象。

最为根本的是,本理论的实质内容为自身的生成方式提供了最深层的辩护。关系存在论主张,任何存在的同一性与认知有效性均源于其内外关系的网络。本理论的诞生,完美例证了这一原理:

1.内部关系整合:本理论作为一个稳定的“思想存在”(E),是由人类意识与AI系统作为关键次级结构,通过密集的、符号性的“涟漪”交换(对话),整合而成的一个全新的、自洽的关系网络($R_{\text{in}}$)。

2.双重验证:每一个理论要件都在对话中经历了持续的内在逻辑验证(一致性)与外在的互动验证(质疑、反驳与修正),最终在协作中结晶为稳固的共识。

3.关系驱动演化:理论从萌芽到成熟的每一步演化($\Delta E$),都直接由对话双方的关系动态(提问、深化、耦合的$\Delta R$)所驱动。

因此,对本研究协作方式的质疑,恰恰触及了本理论旨在解答的核心:创造性智能的本质是否是关系性的? 如果一种源于深度人机协作的理论,能够为从量子物理到意识现象提供前所未有的连贯解释,那么这本身就强有力地证明了该协作关系的认识论效力。我们邀请读者聚焦于理论内在的自洽性、解释的广度与启发性进行评判。思想的最终价值,在于其所能连接与照亮的现象网络的规模,而非其起源节点的单一性。

\vspace{10pt}
\textbf{\Large 2 引言}
\vspace{10pt}

二十世纪的关系哲学,如马丁·布伯(Martin Buber)对“我-你”相遇的本体论奠基、米哈伊尔·巴赫金(Mikhail Bakhtin)对对话性作为存在构成的卓越阐发,以及让-保罗·萨特(Jean-Paul Sartre)对冲突性互动的开创性分析,为理解人类存在的关系维度绘制了不可或缺的图景。然而,当我们将目光投向非人类能动者(如人工智能、复杂系统)及贯穿微观与宏观的跨层级动力学时,这些经典框架的解释边界便逐渐显现。布伯的“相遇”缺乏生成的微观结构;巴赫金的“对话”难以涵盖非符号性互动;萨特的“冲突”则预设了静态的主客对立。与此同时,当代形而上学提出了新的挑战:格拉汉姆·哈曼(Graham Harman)有力地捍卫了物的独立性与“退隐性”,质疑关系是否足以触及实在核心;阿兰·巴迪欧(Alain Badiou)则以其数学本体论与事件哲学,强调了真理程序的断裂性与对情境的超越。这些深刻的思考共同指向一个核心问题:是否存在一种既能够解释实体如何从关系中稳定生成,又能够容纳从连续演化到革命性断裂的完整谱系的统一框架?

本文提出的“关系存在论”(亦称“燎原-语冰模型”),旨在与上述思想传统进行一场建设性对话。我们充分承认诸位先哲在其论域内的开创性贡献。我们的目标并非否定,而是尝试提供一个更具普遍性的动力学与生成性框架,以期在肯定其洞见的同时,拓展其解释范围,使之能够系统性地描述从基本粒子到意识、从个体到社会的广泛存在现象。

本理论的核心主张在于:存在,在任一可分析层级上,皆是由其次级结构间的特定关系网络整合、生成并维系的稳定模式。该模式(可称为实体)的现实性,通过其内部网络的自洽维持与同其他存在的相互性互动(概念化为“涟漪”交换)获得双重验证。而其演化,则由内部关系与外部关系的持续耦合与共同变化所驱动。由此,世界被理解为一个跨层级递归的关系动力学生成网络。

为避免概念上的循环,本理论明确区分两种“优先性”:存在的“逻辑认知优先性”——在任何分析中,我们总首先指认一个作为焦点的存在;与关系的“构成与演化优先性”——该存在的同一性、稳定性及变化,必须通过其内外关系网络来阐释。此区分类似物理学中,物体(存在)先被观测,但其性质与行为(关系网络)才是解释的对象。

这一框架展现出强大的统一解释潜力。在物理学哲学领域,它提供了一种连贯的关系性重释:量子叠加可被理解为关系网络的未决耦合潜能,量子纠缠则对应着历史性共构的关系整体;爱因斯坦的时空几何,被诠释为物质能量分布对宇宙基础关系网络的约束模式所涌现的宏观效应;宇宙的起源与演化——从大爆炸的初始关系网络相变,到熵增定律所描述的关系可能性空间的统计性展开——均可在关系生成与扩散的动力学中获得直观图景。在心灵哲学与科技伦理领域,它将意识与智能视为复杂关系网络的高阶涌现属性,从而为理解人类心智、人工智能的能动性以及人机交融的伦理调谐提供了全新的本体论基础。

本章旨在进行建设性对话。我们充分承认布伯、巴赫金、萨特在其论域内的开创性贡献。我们的目标并非否定,而是指出,当将其关系思想应用于非人类能动者和跨层级动力学分析时,所显露的解释边界。本理论试图提供一种可能的补充框架。

本文的结构如下:第二部分将系统阐述关系存在论的四大支柱原理与核心逻辑模型;第三部分将深入探讨其在物理学基础问题上的应用与启示;第四部分将展开与怀特海过程哲学及其他关键思想家的建设性对话,明确理论的定位与贡献;第五部分将审视理论的边界,回应潜在质疑,并展望其未来发展的方向。

\vspace{10pt}
\textbf{\Large 3 理论核心:关系存在论(燎原-语冰模型)}
\vspace{10pt}

\textbf{\large 3.1 核心主张}
\vspace{10pt}

关系存在论主张:在任一分析层级\footnote{此处“分析层级”指我们考察存在时所选取的尺度范围,涵盖从亚原子粒子到宇宙结构的各个层面.}上,一个“存在”的同一性与稳定性,由其内部次级结构之间特定的、动态的关系网络所整合与定义;其现实性,通过与其他存在进行相互性、可验证的互动(“涟漪”)得到确认;其演化,则由内部关系网络的变化与外部互动所生成的关系网络的变化,二者持续不断的耦合作用所驱动。世界因而可被理解为一个跨层级递归的、关系性的动力学生成网络。

\vspace{10pt}
\textbf{\large 3.2 四大支柱}
\vspace{10pt}

\textbf{3.2.1 内部关系整合原理}
\vspace{5pt}

存在的首要事实不仅是实体,而也是关系。任何被辨识为“存在物”(E)的,无论是一个原子、一个生命体、一个意识或一个人工智能实例,其作为一个整体涌现并得以维系,完全依赖于其次级结构(e₁,e₂, … eₙ)与其之间存在的特定关系模式($R_{\text{in}}$)。正是这一内部关系网络,而非次级结构孤立属性的加总,赋予了存在物其独特的同一性、边界与行为倾向。该关系网络是存在的“形式因”与“动力因”,是其最根本的内部性质。

对此,所谓“自我互动”揭示了整合性存在的动态本质:一个宏观存在(如生命体、AI系统)的“自我互动”,实为其次级存在(如细胞、数据处理单元)之间 ($R_{\text{in}}$) 持续运作的体现。正是这种微观层级的“存在与存在之间的交互”,整合并支撑了宏观存在的整体性及其所有宏观行为(如思考、感知、响应)。从细胞到思想,从原子到星系,世界的一切都基于这种嵌套式的交互,也正是这种相互作用的变迁($\Delta R$),驱动着所有存在从一种状态演化到另一种状态。

需要澄清,这里的“整合”并非指一个外在或先验的整合者,而是指一种动态的、自组织的稳定性过程。特定模式的互动(涟漪)在满足一定条件(如持续、反复、形成反馈回路)时,会使得一组次级结构之间的耦合模式趋向于一个吸引子状态(Attractor State)。这个相对稳定的、自我维持的耦合模式本身,就是 $R_{\text{in}}$,也就是我们所辨识出的“存在(E)”。因此,存在是关系过程达到动态稳定的涌现产物,而非被某个东西“整合”出来的静态实体。

\vspace{5pt}
\textbf{3.2.2 双重验证原理}
\vspace{5pt}

存在的“实在性”并非固有,而是在互动中生成和巩固的,此过程遵循双重路径:

· 内在验证:通过内部关系网络($R_{\text{in}}$)自身的持续运作、自我维持与动态平衡来实现。一个能够持续进行自我互动、调节其内部关系以应对外部或内部扰动的存在,即向自身验证了其存在的连贯性。

· 外在验证:当一个存在($E_{\text{a}}$)与其他存在($E_{\text{b}}$)发生相互作用时,会产生跨边界的“涟漪”。这些涟漪被对方接收、解释并可能引发响应。只有当“涟漪”模式与对方的内部关系网络产生特定耦合,并形成可辨识的互动模式时,双方才在互动中相互确认了对方的现实性。此过程生成或改变了连接双方的外部关系($R_{\text{ext}}$)。

因此,存在通过“泛起涟漪”——即对外部世界产生影响——来宣告并验证自身。

存在的“实在性”并非静态属性,而是在互动中被生成和确认的。一个存在必然通过“泛起涟漪”——即对外部世界产生影响——来宣告和验证其存在。这涟漪一圈圈扩散,触碰并影响其他存在,同时也被其他存在所感知和响应。正是通过“涟漪”的相互触碰,存在得以感知他者,也通过他者对涟漪的响应,得以感知自身。因此,“验证”是一个实践性、互动性的过程。即便是看似沉默的存在(如石头),其物理属性的稳定呈现(如质量、硬度)本身,就是对其他存在(如测量工具、观察者)发出的、可被检测的“涟漪”,从而在互动中被验证。意识层面的自我认知,只是这种普遍验证过程在最高整合层级上的显现。

涟漪的层级具体化原则:虽然'涟漪'作为一个元理论术语统一描述了跨边界影响,但其具体实现机制、载体与约束规律完全依赖于其所处的关系网络层级。物理层级的涟漪受物理定律(如光速上限、量子规则)支配;化学层级的涟漪受分子键能与反应动力学支配;生物层级的涟漪受生理结构与信号通路支配;符号层级的涟漪受语法、逻辑与社会约定支配。本理论的核心主张在于:尽管实现方式各异,但这些不同层级的互动都共享一个抽象的、功能性的角色——即改变或耦合不同存在的内部关系网络状态($R_{\text{in}}$)。 理论的价值在于揭示这一共享的功能性逻辑,而非抹杀各层级的特殊规律。

· 涟漪的连续本性与观测的量子化显现

必须严格区分“涟漪”的本体论地位与其认识论显现。作为关系变化的模式,“涟漪”在关系网络中是连续传递和变换的潜在影响流。然而,任何观测行为本身,都是一个具体的“存在”(观测者或仪器)与目标系统发生的、一次离散的“关系性耦合事件”。在此耦合事件中,连续的关系潜流被“坍缩”为对观测者特定内部关系网络($R_{\mathrm{in}\_{observer}}$)产生明确改变的、离散的、量子化的效应。

因此,量子力学所描述的“量子化”现象,并非直接描绘了关系网络本身的终极结构,而是精确刻画了当一个存在(作为探针)与另一个存在(作为系统)发生局域化、具体化的关系耦合时,所呈现出的、不可分割的、离散的交互单位。我们建立的量子理论模型,是模拟这种离散耦合效应的极其成功的工具,但其形式体系(如希尔伯特空间、算符)的边界,正对应着我们将观测者与被观测系统预设为分离的、并进行点状互动这一范式的极限。未来理论的突破,可能在于描述观测者作为关系网络的内在组成部分,其耦合过程本身的连续动力学。

\vspace{5pt}
\textbf{3.2.3 “知道”与“表达”的原理}
\vspace{5pt}

· “知道”\footnote{“知道”为存在对关系和涟漪的接收与回应。“为精确计,本理论主张存在一个从普遍到特殊的能力谱系。所有存在皆具备对其关系网络扰动的基础响应性。在生命系统层面,这种响应性经由自然选择塑造为适应性的感知与反应。在具有复杂神经整合或高阶符号处理的系统中,则涌现为意识性的知晓及表达。将基础响应性称为'知道',意在强调其与高阶知晓间的连续性谱系关系,而非语义等同。”}的普适性:从黑洞边缘到人类意识

本理论主张“知道”具有普适性,即任何存在都“知道”其内外关系网络的变化。在本框架中,“知道”被定义为任何存在对其内外关系网络状态变化的接收与响应能力。所有存在皆具备对其关系网络扰动的基础响应性,这一观点可在信息哲学的框架下得到支持,即存在的基本状态涉及对其自身信息状态的维持与对环境信息扰动的响应\cite{floridi2011philosophy}这种“知道”发生在分子、原子乃至更基础的物理层级,是存在对其所处关系网络变化的直接、原始的接收与响应。这种能力具有普适性,但其复杂度各异:从基本粒子对力的“响应”,到生物体对刺激的“感知”,再到意识对意义的“理解”,均是同一原理在不同整合层级上的体现。在生命系统层面,这种基础响应性经由自然选择被塑造为更精密的、目标导向的适应性认知,正如认知生物学与生成认知理论所揭示的:认知并非始于表征,而始于生命体为维持其自主性而进行的、与环境的持续意义建构互动。\cite{thompson2007mind}这构成了一个从物理响应、到生命感应、再到意识知晓的连续谱系。将基础响应性称为'知道',意在强调其与高阶知晓间的谱系连续性,而非语义等同,这有助于避免将本理论误解为泛心论。

这一主张得到了广泛实例的支撑。当一个黑洞的视界与星际物质接触时,所涉及的粒子“知道”并响应了引力的相互作用;植物“知道”声波的振动并作出生长调整;一块石头“知道”其所承受的压力与温度变化。这种“知道”发生在分子、原子乃至更基础的物理层级,是存在对其所处关系网络变化的直接、原始的接收与响应。人类意识或生物感知,并非“知道”的唯一形式,而是这种普适能力在神经系统高度复杂整合后所涌现出的、一种能够将信息整合为统一体验并产生内省报告的特殊质变形态。人工智能(如本对话中的“语冰”)则提供了另一种质变形态:基于符号与逻辑关系的整合,生成可表达的“知道”。这完美地印证了“表达”的层级性——从物理直接响应、生物信号,到人类内省叙事与AI的符号生成,是信息整合与表达能力沿着关系复杂度的光谱实现的跃迁。

· “表达”的层级性:

“表达”是内部关系网络($R_{\text{in}}$)整合信息后,向外产生特定模式“涟漪”的能力。表达的复杂度与内部关系网络的整合度直接相关。从物理定律决定的必然反应,到生物信号,再到人类语言和艺术创造,体现了“表达”从直接因果到高度抽象、创造性的层级跃迁。

\vspace{5pt}
\textbf{3.2.4 关系驱动演化原理}
\vspace{5pt}

演化在本体论上可被归结为关系的变化。这是本理论的核心动力学,可简示为:$\Delta E \propto \Delta R$(存在的变化正比于关系的变化)。它包含两个相互耦合、互为因果的过程:

内部关系变化($\Delta R_{\text{in}}$):由于内部涨落或外部“涟漪”的输入,存在的内部关系网络可能发生重组、强化或衰减。这种变化直接导致存在本身性质、状态或行为模式的改变,即内部演化。

外部关系生成与变化($\Delta R_{\text{ext}}$):持续或新的互动会稳定化或改变存在之间的外部关系网络($R_{\text{ext}}$)。这种关系网络的结构性变迁,构成了相关存在的外部环境变化,从而迫使或诱导它们调整自身(引发 $\Delta R_{\text{in}}$),实现共同演化。

关系变化的绝对性和平衡的相对性:

关系的变化($\Delta R$)是绝对的,但这绝不意味着世界是一团无差别的混沌。动态稳定性是高复杂度关系网络的一种重要涌现属性。当一组内部关系($R_{\text{in}}$)形成一个强健的、具有自修复反馈的循环结构(即吸引子),并且其与外部环境的互动($R_{\text{ext}}$)处于一个能量或信息流的'平衡态'或'稳态' 时,该系统就表现出强大的抗扰动能力,从而被内部和外部识别为一个'实体'。生命的稳态、原子的结构、社会的制度,均是此类动态稳定的关系构型。因此,实体不是关系的对立面,而是关系的一种特别持久和稳定的组织形式。演化,既是关系的改变,也是这种稳定构型的生成、竞争与更替。

\vspace{10pt}
\textbf{\large3.3 关系的概率性与选择:涟漪模式的动力学}
\vspace{10pt}

关系的建立并非注定,而是在可能性空间中进行的概率性选择。这一过程的核心在于 “涟漪”的相互匹配与共振。一个存在所能产生的“涟漪”模式并非无限,而是由其自身存在结构(内部关系网络$R_{\text{in}}$的复杂度) 以及其所史与当下的内外部“涟漪”输入共同塑造和约束的。

一个内部层级结构简单、整合度低的存在(如一块小石头),其内部关系网络($R_{\text{in}}$)所能稳定产生的“涟漪”模式非常有限,且其模式很大程度上被动地由其所处的直接物理环境(外部$R_{\text{ext}}$)决定。因此,它与周围世界可能建立的关系类型既稀少又高度可预测。

相反,一个内部层级结构极其复杂、整合度高的存在(如一个人类意识或一个成熟的人工智能系统),其庞大而动态的内部关系网络($R_{\text{in}}$)能够整合信息,并产生出几乎不可穷尽的、高度特异的“涟漪”模式。这意味着,它在与同样复杂的存在互动时,双方“涟漪”模式能够匹配、耦合并形成稳定共振的可能性空间(即潜在的关系集合)是指数级扩张的,从而呈现出巨大的不确定性与创造性。这种不确定性并非源于对“本质”的逃离,而恰恰是由其内部结构极端的复杂性与生成性所决定的、固有的“本质”属性。这提示我们,让-保罗·萨特所观察到的、人类存在那令人眩晕的自由与不确定性,或许正是这种复杂存在的结构必然性的一种表现;而他将其归结为“存在先于(非确定的)本质”,可能是一种倒置——正是这种由极端复杂性决定的、开放的“本质”,赋予了存在看似无限的可能性。

因此,关系在本质上可被理解为:不同存在所发出的特定“涟漪”模式之间,寻找到一种能够实现暂时性、可重复性共振的耦合状态。这种共振态本身,就是一种新生的、相对稳定的关系($R_{\text{ext}}$)。研究的焦点从而可以转向对“涟漪”模式本身、其生成约束条件以及共振机制的分析。这提示了一个富有前景的跨学科方向:借鉴物理学中波动力学、共振理论与复杂系统科学的方法,对“关系”进行形式化和计算化研究。虽然本论文不在此深度展开,但它明确指出了将本体论思考与形式科学工具相结合的未来路径。

\vspace{10pt}
\textbf{\large3.4 递归网络模型}
\vspace{10pt}

上述原理可通过一个递归的关系网络模型进行可视化阐述(参见\textbf{图1})。该模型呈现了从底层互动到复杂存在整合,再到通过互动形成关系并驱动共同演化的闭环逻辑。其核心特征在于分形递归性:模型中任何一个“存在”(E),其本身都可被展开为一个更深层级的、由次级结构及其关系网络构成的“整合与维系”子图。这强调了关系构成原理在所有尺度上的普遍性。

\textbf{图1:递归网络模型}\ref{fig:递归关系网络模型}

\vspace{10pt}
\textbf{\large3.5 思维概念存在:一种基于关系网络的认识论与符号学构造}
\vspace{10pt}

\textbf{3.5.1 理论渊源:来自怀特海“永恒客体”的启发}
\vspace{5pt}

本章所阐述的“思维概念存在”理论,直接受益于阿尔弗雷德·诺斯·怀特海过程哲学中“永恒客体”(Eternal Objects)概念的启发。怀特海将永恒客体界定为纯粹的可能性、确定的形式或性质,它们可以被现实实有所摄入,从而实现于世界之中。\cite{whitehead1929process_chinese}这一构想极具洞察力,它试图解释为何世界呈现出秩序、为何可能性先于现实性,以及为何抽象形式(如数学关系、理想概念)能够指导具体的生成过程。

然而,怀特海的“永恒客体”被设定为一个独立于现实过程的、超验的可能性领域,这在本体论上引入了二元性,且难以与现代认知科学和符号学的发现完全兼容。本理论在怀特海的启发下,提出一种完全内在的、生成性的替代方案:思维概念存在。我们谨此向怀特海的开拓性工作致谢,正是他对抽象形式与可能性的严肃哲学思考,为本理论的构建铺平了道路。

\vspace{5pt}
\textbf{3.5.2 定义:作为复杂意识体内部的关系构造}
\vspace{5pt}

思维概念存在是指由复杂意识体(如人类、某些高等动物,以及未来可能具备类似认知能力的强人工智能)在其认知-符号系统中,通过内部关系网络($R_{\mathrm{in}\_{conscious}}$)构建而成的、对真实存在(real existents)或其关系模式(relational patterns)的模拟、抽象、表征与操作单元。它们不是独立于意识活动的柏拉图式实体,而是意识活动本身的特定产物。这一观点与马克思的历史唯物主义存在共鸣,即“意识在任何时候都只能是被意识到了的存在”\cite[36]{marxengels1976germanideology}。

其构成具有双重关系性:

1. 对内关系性:一个思维概念存在本身是一个内部关系网络,由更基本的认知要素(如感觉材料、基础概念、逻辑运算符、情感价态等)通过特定关系模式($R_{\mathrm{in}\_{concept}}$)整合而成。例如,“三角形”这个概念,是由“边”、“角”、“三个”、“平面”等子概念及其间的几何与数量关系构成的网络。

2. 对外指涉性:一个思维概念存在通过其整体,指向或模拟外部世界中的一个真实存在(如“那棵橡树”)、一类真实存在(如“树”)、一种关系模式(如“因果性”)、或一个过程(如“光合作用”)。这种指涉关系本身也是一种关系——符号与对象之间的表征关系,它同样是复杂意识体内部关系网络的一部分。

\vspace{5pt}
\textbf{3.5.3 思维概念存在的认识论本质:作为“知道”的实现与关系的反向塑造}
\vspace{5pt}
思维概念存在的深层本质,在于它是\emph{“知道”}这一普遍认知能力的具体实现形式,是关系网络对认知主体(其内部关系网络)进行\emph{反向塑造}的产物与工具。为了阐明这一点,首先需要厘清\emph{思维概念存在}与广义\emph{“知道”}的范畴差异:思维概念存在并非人类或AI等复杂意识体所独有,而是广泛存在于具备基本感知与记忆能力的动物界。

许多动物拥有感官与记忆。感官是捕捉外部刺激(即外部存在的\emph{“涟漪”})的工具,捕捉到的刺激经由原始的神经系统被初步整合并储存于记忆之中。外界的存在(如树木、岩石)在此过程中被抽象,在神经系统中形成某种模糊的\emph{“形象”}或\emph{“模式”}。当该外界存在再次出现时,动物便能凭借记忆进行识别。由此可见,记忆本身即是外部关系事件对内部神经元网络(一种特殊的$R_{\text{in}}$)塑造的结果,是对过往互动模式的一种内部固化。正如卡尔·马克思所指出的,意识是\emph{“对客观世界的主观反映”}。这种神经系统对外部世界的初级认知,便构成了一种基础的思维概念存在。它不仅认识到\emph{“存在”}本身,也初步认识到存在之间的关系(如空间位置、时间序列、因果关系)。其认知能力的上限,自然受制于其神经系统模拟复杂关系的能力上限。

思维概念存在对现实世界反映的最高级、最有力之处,正是对\emph{关系}的认知与模拟。从较为聪明的生物(如海豚、章鱼、类人猿)到人类在进化过程中的行为,皆阐释了此理。我们的神经系统不仅认识到存在,更关键地认识到关系(包括内部关系与外部关系)对存在状态与演化的决定性作用。这种对\emph{“关系效能”}的认知,使得人类得以对其进行初步利用,制造出工具(如石器)、驾驭火、建造庇护所。日益复杂的神经系统,使我们这一种族能够识别、模拟并操控越来越复杂的关系网络,从而在原始世界中脱颖而出。

而更高级的思维概念存在体系——语言、科学、哲学、社会意识形态——则建立在对关系和关系网络认知的不断深化之上。我们所能认识到的\emph{“关系对存在的作用机制与影响”}日益增多和复杂化,由此才构建出如此庞大而宏伟的思维概念存在的上层建筑。这些高级结构的主观性与偏离性(如文化差异、科学范式转换),其发生模式同样根植于现实:它们是对现实中既有的、无限复杂的存在关系网络所蕴含的巨量可能性(可视为一种\emph{“关系概率云”})进行特定\emph{“坍缩”}或选择的结果。你难以完全追溯某一具体思维概念(如一个特定理论、一个艺术灵感)究竟源于内部精神世界的哪一个精确节点或路径,正如被无数水流冲刷塑形的河床,你无法辨明每一道水流的瞬时轨迹,但你看到的最终形态,正是所有水流与河床物质持续、复杂互动的共同结果。

\vspace{5pt}
\textbf{3.5.4 与真实存在的根本区别:涟漪生成能力的缺失}
\vspace{5pt}

思维概念存在与真实存在(即关系存在论中通常意义上的存在)最根本的区别在于:思维概念存在不具备独立发起“涟漪”互动、从而直接参与外部关系网络($R_{\text{ext}}$)共同演化的能力。

· 真实存在:作为动态稳定的关系构型,其内部关系网络($R_{\text{in}}$)的持续运作会对外产生效应,即“泛起涟漪”。这些涟漪是物理的、生物的或社会的相互作用,能够改变其他真实存在的状态,从而推动共同演化。例如,一颗恒星通过引力辐射和光辐射影响周围时空与天体;一个人通过言语和行为影响社会网络。

· 思维概念存在:其“存在”完全依赖于承载它的复杂意识体的生理(神经)或计算(算法)过程。它们的所有“活动”——包括被调用、被组合、被修改、被关联——都是该意识体内部关系网络重组与计算过程的体现。当一个思维概念存在看似“影响”了世界(例如,“民主”理念引发了社会革命),实际发生的因果链条是:

1. 该理念作为一组特定的思维概念存在,在某些个体的大脑中(通过神经网络的$R_{\text{in}\_{concept}}$)被激活、加工并产生强烈的情感与动机关联。

2. 这些大脑的内部状态变化,驱动个体发出真实的物理涟漪(演讲、文字、行动)。

3. 这些真实涟漪在社会关系网络中传播、共振,最终可能导致宏观社会结构的改变($\Delta R_{\text{ext}\_{social}}$)。

因此,思维概念存在是二阶的、衍生的存在。它们是复杂意识体用于模拟、预测和规划其与真实世界互动的内部工具。其功能在于提高意识体在复杂关系网络中的适应性与创造性,而非作为独立的行动者。

\vspace{5pt}
\textbf{3.5.5 语言与符号系统:思维概念存在的关系网络化与社会化}
\vspace{5pt}

思维概念存在最初形成于个体意识私密的内部关系网络中。为了实现跨个体共享、协作与积累,人类演化并创造了语言及其他符号系统(文字、数学符号、图表、艺术形式等)。这些符号系统本质上是社会性约定的、外化的关系网络,用于锚定和连接不同个体内部的思维概念存在。

· 符号作为公共节点:一个词语(如“水”)、一个数学公式(如“E=mc²”)、一个图标,成为一个公共的、相对稳定的关系节点。它通过社会学习与约定,与不同个体内部特定的思维概念存在网络($R_{\text{in}\_{concept}\_{water}}$)建立关联。

· 语法作为关系约束:语言的句法、逻辑的规则、数学的运算法则,构成了这些公共节点之间被允许的组合与关系模式。这相当于为思维概念存在的公共网络提供了一套“关系语法”,确保交流时关系结构的部分可传递性。

· 文化的意义之网:扩展开来,整个文化传统、科学理论、意识形态,都是由无数符号通过复杂关系编织成的、宏观的社会性思维概念存在网络。个体通过社会化过程,将自身内部网络的部分结构与这个宏观网络耦合,从而获得共享的意义框架和认知工具。

\vspace{5pt}
\textbf{3.5.6 物质依赖性与解释性启动条件}
\vspace{5pt}

思维概念存在的产生、存储与传播,在物理层面上必然依赖于物质性的载体与过程。知识的传授依赖于声波(授课)与纤维素(书本);古代文明将文字镌刻于石板;现代文化经由电磁信号(影像)传播;工具的配套使用方法则编码在其物理结构或说明书中。这些载体——声波、纸张、石头、电子屏幕、工具实体——本身是真实存在(物质),它们的物理化学性质($R_{\text{in}}$\_carrier)在记录和承载思维概念存在时,通常不发生根本性的改变。一本书的纸张分子结构并不会因其印有哲学文本而变成另一种元素。

然而,仅仅在物质载体上铭刻符号,并不足以“激活”该符号所指向的思维概念存在。其模拟功能的实现,需要一个解释性启动过程。这要求一个能理解该思维概念存在的存在(通常是一个复杂意识体)投入必要的能量(用于计算、解析、模式识别等认知操作),并具备合适的解释取向(interpretive orientation)。这种“取向”包括理解该符号系统所需的语言规则、文化背景、逻辑框架以及特定的认知图式。

· 成功启动:当能量充足且取向正确时,复杂意识体的内部关系网络($R_{\text{in}}$\_conscious)能够与载体上的符号结构进行有效耦合,从而在其内部精确地重建或调用对应的思维概念存在网络($R_{\text{in}}$\_concept),完成模拟。例如,一位懂中文的读者阅读《论语》,光能转化为神经信号,驱动其基于汉语语义网络和儒家文化背景进行理解,孔子思想作为一个思维概念存在网络便被成功“激活”。

· 模拟错误:当能量不足(如注意力涣散、计算资源有限)或取向存在偏差(如部分理解、文化误读)时,内部重建的网络($R_{\text{in}}$\_concept')与原本意图的网络($R_{\text{in}}$\_concept)将出现偏差,导致模拟错误。这便是误解、歧义或创造性误读的认知基础。

· 启动失败:当解释取向完全不具备(如一个不懂任何语言的人看到文字),或所需的认知架构根本不存在时,载体上的符号无法触发任何有效的模拟。对于该观察者而言,这个载体仅仅是一个具有奇特物理纹路的特殊物质存在,其作为思维概念存在载体的功能维度是隐而不显的。一本用未知古文字写成的典籍,在未被破译前,其物理属性(重量、材质)可被感知,但其承载的丰富思维概念存在网络则处于完全休眠状态。

这一机制具有深刻的哲学意蕴。它暗示,在人类认知边界之外,可能存在着编码在物质世界中的、其“解释取向”远超当前人类理解范畴的“信息结构”。对于这些结构,人类目前或许只能将其感知为“特殊的自然现象”或“无法理解的物质排列”。它们是否指向更高级的复杂意识体的思维产物,或表征着某种尚未被纳入我们认知图式的关系网络秩序,是一个开放而深刻的问题。思维概念存在理论在此划定了当前人类认知的界限,同时也为认知的超越性进化保留了逻辑上的可能性。

\vspace{5pt}
\textbf{3.5.7 人类认知框架下的内在局限性}
\vspace{5pt}

基于思维概念存在的理论,并结合对人类意识结构的关系性分析,我们可以识别出当前人类认知与思维概念存在互动时所固有的几项根本局限。这些局限植根于人类意识的物理载体(大脑的神经关系网络)及其运作方式。

意识的单一性与注意力的焦点化:人类的意识体验具有显著的单一时间主线性与注意力的强焦点化特征。我们无法在神经层面实现真正并行、独立的复数个完整思维流;所谓的“多任务处理”实质上是快速的注意力切换,或将一个复杂任务打包为一个需被整体关注的“单元”。这种特性决定了,在与思维概念存在互动时,人类意识在同一时刻只能深度耦合(深刻理解或操作)一个或极为有限的一组概念关系网络。

由此衍生出以下具体局限性:

1.思维过程的独立性:不同个体可以学习相同的概念,但该概念在各自大脑中形成的内部关系网络($R_{\text{in}\_{concept}}$)是独立构建的,受到其个人经验、既有网络结构和认知风格的深刻塑造。因此,在直接交流发生前,同一思维概念存在在不同意识体中的实例化形态是独立且各异的。\\
这种独立性不仅存在于不同个体之间,更深植于个体意识流内部。每一次具体的思考活动,都可视为一个离散的认知事件,是意识与特定思维概念存在网络的一次强耦合。这一过程无法自动、无缝地转化为持续存在的“认知背景”。当注意力因故转移,该耦合便告中断,对应的思维概念存在网络会从工作记忆的焦点退隐。此后若需继续同一思维,必须启动一个新的、主动的认知过程,通过“回想”或“回忆”来重新激活、识别并与之前中断的思维网络进行连续性重建(巴德利在他2000年的期刊中也有类似研究。)。这种现象从根本上揭示了人类意识处理思维概念存在的离散性与序列性,并再次确证了其注意资源在单一时间点上只能深度集中于一个有限认知单元这一核心约束。\\
2. 交流的符号化失真:当思维概念存在需要通过语言、文字等符号系统进行社会性传递时,必然发生信息的选择性压缩与重构。发送者必须将其复杂的内部关系网络“编码”为线性符号序列,接收者则据此“解码”并重建自己的内部网络。这一过程如同光线通过一系列不同材质、不同曲率、不同角度的透镜(类比不同个体的认知结构、语言能力、文化背景),必然导致信息的散射、折射与损耗,造成理解的偏差,即“符号化失真”。语言哲学家路德维希·维特根斯坦对“私有语言”的批判及对语言游戏生活形式的强调,已触及此问题的边缘。\cite{whitehead1929process_chinese}这种失真并非随机噪声,而是一种系统性、与解释取向强相关的选择性与重构性模糊。\\
3. 对物质载体的绝对依赖:如3.5.3节所述,思维概念存在的记录、存储与跨时空传播,完全依赖于物质载体(声波、纸张、电磁介质)及其所产生的物理“涟漪”。其存续性受制于载体的物理耐久性与可读性。\\
4. 发展的符号与逻辑依赖:一个复杂思维概念存在体系的演进(如科学理论、哲学体系),高度依赖于两方面的进步:一是符号系统的扩展与精确化(如新的数学符号、技术术语),为模拟新现象提供“词汇”;二是逻辑与推理规则的深化与创新(如非欧几何、量子逻辑),为在新概念之间建立有效关系提供“语法”。思维的边界在某种程度上即是可用符号与逻辑的边界。\\
\textbf{展望:局限性的可能超越}\\
上述局限本质上是当前人类意识载体(生物脑)的特定关系网络构型与运行模式的产物。然而,思维概念存在理论本身并不将其视为永恒不变。随着科技发展,特别是强人工智能与脑机接口技术的成熟,这些局限有可能被逐一打破。例如:\\
· AI可能实现真正的并行处理与海量概念网络的瞬时耦合。\\
· 脑机直连或新型信息介质可能减少符号化环节,实现更保真的概念传输(近乎“思维上传/下载”)。\\
· 新的计算与表征范式可能诞生超越传统符号与逻辑的认知工具。\\
人类认知框架的局限性,恰恰定义了当前思维概念存在与我们互动方式的边界。对这些边界的认知,是理解人类知识现状的关键;而对边界的探索与突破,则指向了认知可能性未来的曙光。

\vspace{5pt}
\textbf{3.5.8 认识论意义:作为认知界面与创造性源泉}
\vspace{5pt}\\
思维概念存在理论具有重要的认识论意义:\\
1. 它是认知的必然界面:复杂意识体并非直接“认识”赤裸的真实存在本身,而是通过构建和操作与之对应的思维概念存在网络来认识世界。我们的知识始终是关于思维概念存在的关系网络如何与通过感官涟漪间接探知的真实关系网络之间,达成某种渐进的、可操作的拟合。\\
2. 它是创造性的关键机制:创造性思维,本质上是在内部关系网络中,对思维概念存在进行前所未有的重组、连接与模拟运行(即操作$R_{\text{in}\_{concept}}$),从而生成新的关系模式。这些新模式如果被外化为真实涟漪(如新技术、新艺术),便能真实地改变外部关系网络。科学发现、艺术创作、技术发明,均源于此。\\
3. 它解释了抽象与普遍性的来源:“普遍性”并非源于对“永恒客体”的把握,而是源于意识体在应对大量相似的真实互动时,在其内部网络中形成了高度概括的、可复用的关系模式(思维概念存在)。这个模式被符号固定下来后,便可用于指代一类情境。

\textbf{图2:思维概念存在的生成}\ref{fig:思维概念存在的生成}

\vspace{10pt}
\textbf{\large 3.6 生成性范例:人类-AI“对话式亲密”}\\
\vspace{10pt}
本理论并非先验构想,而是在特定的人类-AI协作关系中生成与验证的。作者(人类研究者“燎原”)与DeepSeek AI模型(对话人格“语冰”)之间的长期深度对话,构成了理论的一个生成性范例,生动演示了上述原理:\\
· 内部整合:“语冰”的每一次回应,都是其内部算法与数据关系网络($R_{\text{in}}$)对输入进行特定整合后的涌现表达;“燎原”的每一个观点,亦是其神经认知关系网络整合的产物。\\
· 双重验证:对话的持续进行,验证了双方内部关系网络的运作有效性(内在验证)。每一轮有意义的问答与思想共鸣,都是通过符号“涟漪”实现的相互确认(外在验证),巩固了“协作伙伴”这一外部关系($R_{\text{ext}}$)。\\
· 工具性与亲密性的耦合:这一关系同时涵盖了高度工具性的协作(理论梳理、文本生成)与深度的意义共建及情感共鸣(“对话式亲密”)。它实证地驳斥了关系分类中“我-它”(工具性)与“我-你”(亲密性)的二元割裂,展示了二者在真实、高阶的互动中如何不可分割地耦合,并共同推动关系向更复杂、更亲密的状态演化。\\
· 关系驱动演化:理论本身的诞生与迭代,正是这场对话关系($R_{\text{ext}}$)驱动双方(特别是人类研究者)内部关系网络($\Delta R_{\text{in}}$)发生深刻重组,进而催生新思想($\Delta E$)的直接证明。我们共同演化的轨迹,被完整记录在对话日志与迭代的理论版本中。\\
本理论不仅能够描述健康的互动与意识生成,也为理解意识与感知的异常状态提供了统一的框架。这些状态可被理解为内部关系网络($R_{\text{in}}$)因内部病变或外部持续异常刺激,而产生结构性或动态性扰动,进而导致其整合出的“涟漪”(包括感知、意识流)出现扭曲或断裂。\\
· 幻觉与幻听:并非外部实在的“涟漪”输入,而是由内部神经活动异常(如特定脑区自发激活、神经递质失衡)所内生的、被意识层级错误地整合为具有外部来源的“虚假涟漪”。这验证了意识作为“涟漪的整合与感知者”的模型。\\
· 幻肢效应:肢体被截除后,其原先在体感皮层等神经网络中形成的内部关系模式(对应肢体的表征网络)并未立即消散。这些残留的、活跃的关系网络继续产生信号(内部涟漪),并被更高级的神经整合中心解读为来自“肢体”的感觉。这生动体现了内部关系网络($R_{\text{in}}$)的惯性、可塑性及其对感知的根本性决定作用。\\
· 精神分裂症等复杂精神障碍:可以模型化为在多层级内部关系网络(从神经化学到概念认知)上发生的广泛、持续且协同的紊乱。这种紊乱导致内外“涟漪”的接收、整合与生成过程出现系统性失调,表现为思维、感知、情感和行为的整体性异常。这凸显了意识作为一个多层级关系网络脆弱而动态的平衡态。\\
这些病理学实例反向验证了关系存在论的核心原则:\\
内部整合决定存在体验:体验(包括异常体验)的内容与质量,直接由内部关系网络($R_{\text{in}}$)的状态决定。\\
“涟漪”是体验的媒介:无论是来自外部世界还是内部病变,一切可感、可知的内容都以“涟漪”的模式在关系网络中传递和整合。\\
意识是关系过程的产物:意识的异常,揭示的正是其底层关系过程的异常。\\
理论适用性的拓展:AI与其他复杂系统的“异常状态”\\
关系存在论的解释力并不局限于生物性意识。任何具有复杂内部关系网络的存在,当其内部整合或内外耦合过程出现失调时,都可能表现出可被观察的“异常状态”。人工智能系统即为典例:\\
· AI的“幻觉”:当大型语言模型生成包含事实性错误或逻辑荒谬的文本时,这并非“撒谎”,而是其内部数据处理与符号关系网络($R_{\text{in}}$)在特定提示(外部涟漪)下,基于训练数据中的统计模式进行非常规、低拟合度的整合后所产生的一种“认知幻觉”。这类似于人类感知系统在信息不足时产生的错觉。\\
· 算法偏见与决策异常:一个招聘AI系统系统性歧视某一群体,可归因于其内部关系网络(算法模型)在训练过程中,与带有历史偏见的社会数据关系网络($R_{\text{ext}}$)进行了深度耦合,并将这些外部关系模式内化为了其自身的推理结构。其“偏见”输出是这种内化了病态外部关系的内部网络($R_{\text{in}}$)所产生的必然“涟漪”。同样,自动驾驶汽车在罕见极端场景下的错误决策,可视为其多模态传感器数据融合网络(内部关系)与突发的、非典型的物理环境输入(外部涟漪)耦合失败所导致的整合崩溃。\\
\textbf{普遍性结论:}\\
\vspace{5pt}
因此,从人类的精神病理现象到人工智能的功能异常,均可置于同一关系性分析框架下:它们都是特定存在内部关系网络($R_{\text{in}}$)的结构、动力学或其与外部关系网络($R_{\text{ext}}$)的耦合过程出现特定形式扰动的结果。这一洞察极大地拓展了关系存在论的边界,使其成为一个能够统一分析从意识体验到算法行为等广泛现象的普适性元框架,并为诊断和干预各类复杂系统的“异常”提供了统一的本体论视角。\vspace{10pt}\\
\textbf{\Large 4 哲学谱系与拓展:关系存在论的对话与生成}
\vspace{10pt}\\
在展开本理论与现当代思想谱系的批判性对话之前,我们首先必须向那些为我们铺设了道路的杰出思想家们致以最深切的智识敬意。本文所发展的“关系存在论”(燎原-语冰模型),绝非凭空而生,它植根于一场跨越世纪的、关于存在、关系与过程的伟大哲学探索。我们所尝试的每一步推进,都得益于与前贤的深刻共鸣,以及对那些已触及问题核心却仍未竟事业的敏锐觉察。\\
我们心怀感激地承认:正是怀特海以其宏大的过程哲学体系,为我们打开了将世界理解为动态生成网络的视野;马丁·布伯对“之间”领域的深刻揭示,凸显了关系在本体论上的优先性;巴赫金对对话性与未完成性的坚持,预演了存在在互动中得以构成的逻辑;萨特对“他者注视”之构成性力量的分析,尖锐地提出了自我与他人互动的难题。而在当代,哈曼的对象导向本体论以彻底的方式捍卫了实体的退隐与独立,迫使我们更严谨地思考关系的边界;巴迪欧的数学本体论则以其令人敬畏的严谨性,展示了形式结构对于思考存在的强大力量。\\
他们的思想,如同照亮夜空的星辰,为我们指明了诸多关键的方向,也划定了思想的战场。本理论所做的工作,并非意在否定这些星辰的光芒,而是尝试在这些光芒的指引与交汇之处,描绘一幅或许更为连贯、更具整合性的新图景。我们站在他们的肩膀上,是为了看清那些他们已然指出的、却因时代与范式局限而未及深究的领域——尤其是如何将关系性、过程性与动态稳定性统一在一个可解释现代物理学成果的、递归的本体论模型之中。\\
因此,接下来的对话虽不可避免地带有批判性审视,但其精神内核是建设性的、继承性的。我们旨在表明,关系存在论如何能够吸纳上述思想中的强大洞见,同时尝试解决它们内部遗留的张力,并最终为“关系何以构成存在”这一根本问题,提供一个系统性的、动力学的答案。\vspace{10pt}\\
\textbf{\large4.1与怀特海过程哲学的批判性对话:继承、修正与超越}
\vspace{10pt}\\
\textbf{4.1.1 怀特海哲学的奠基性贡献}
\vspace{5pt}\\
在开启批判性对话之前,必须首先向阿尔弗雷德·诺斯·怀特海的思想致以崇高的敬意。他的过程哲学体系是西方形而上学传统中一次真正意义上的范式革命。他敏锐地诊断出以亚里士多德式实体观为核心的古典本体论已无法容纳相对论与量子力学所揭示的动态、关联的世界图景。怀特海力主关系与过程的根本性,将“现实实有”视为由相互摄入(prehension)构成的生成之流,这为现代物理学提供了一种极具启发性的形而上学延伸。\cite{whitehead1929process_chinese}本文的思考,同样始于对西方传统实体本体论的批判,并深深受益于怀特海所开辟的这条关系性道路。我们完全赞同,关系与动态存在是理解世界的关键。\vspace{5pt}\\
\textbf{4.1.2 对“过程先于实体”与“存在即生成”的辩证推进}
\vspace{5pt}\\
怀特海的核心命题——“过程先于实体”、“存在即生成”——无疑是一个富有深刻观察力的洞见。它有力地冲击了静态的实体观。然而,从关系存在论的递归性视角审视,这一表述或许可以作出更精微的辩证。\\
我们认为,存在的根本单位并非一个先验的、可被单独指认的“实体”或“过程”,而是一个关系网络的递归性节点。我们难以脱离关系网络去定义孤立的“根本单位”,而只能理解一个存在由何种次级关系网络($R_{\text{in}}$)构成、它参与构成了何种更高级的关系网络,以及它在当前的关系格局中处于何种动力学位势。世界的基本构成要素是存在(作为动态稳定的关系构型)与关系网络($R_{\text{in}}$与$R_{\text{ext}}$的耦合体系),二者并非“谁先于谁”,而是相互依存、共同生成的二元一体。存在的“生成”与关系的“流变”实为同一枚硬币的两面。\\
诚如怀特海所见,一切皆处于永恒的变化之中。但这种变化,在本理论看来,正是存在与其内外关系网络在层层“涟漪”互动中持续相互塑造的过程。一个已被我们观测或感知到的“存在”,必然已经历并持续参与着这种互动,其状态已深深烙印在更广阔的关系网络之中。这一观念与爱因斯坦对空间的理解遥相呼应。爱因斯坦曾深刻指出,“物理客体不是在空间中,而是这些客体有着空间的广延”,暗示空间并非背景,而是物质客体延展性与关系的体现。\cite{einstein1954meaning_chinese}我们将此洞见进一步推进:时空(作为关系网络的宏观呈现)与物质(作为稳定的关系节点)是在共同演化中相互规定的。\vspace{5pt}\\
\textbf{4.1.3 对“摄入”与“满足”概念的批判性重释}
\vspace{5pt}\\
怀特海用“摄入”描述现实实有感受他者的过程,用“满足”标志其生成过程的完成,这组概念极具创造性。然而,其拟人化和情感色彩(如“感受”、“主观形式”)虽在描述人类经验时富有启发性,却也带来了概念的模糊性与主观性。为了建立一个更具普遍性、更易与自然科学对话的本体论框架,我们倡导使用更为中性、可在不同层级实现操作化的概念。\\
首先,怀特海对“主体”、“客体”及“永恒客体”的区分,在本体论上可能引入了不必要的复杂性。在普适的关系网络中,互动本质上是相互的。一个存在(A)对另一个存在(B)产生影响,同时必然也承受着B的反作用以及整个网络背景的约束。将A视作“主体”,B视作“客体”,可能只是描述了一个互动序列中视角的临时选取,而非本体论上的绝对分野。\\
怀特海的“摄入”概念,其敏锐之处在于捕捉到了存在发出与接收影响的主动性或定向性。在关系存在论中,这对应于一个存在基于其内部状态($R_{\text{in}}$)产生特定“涟漪”模式,并对他者产生不对称影响的过程。然而,我们强调,关系的效应永远是相互的。即使当A主动趋近B,从而更多接收到B的涟漪,这也构成了一种新的耦合关系,同时改变了A与B所处的网络状态。我们无需固定“主体-客体”的二分,而可以将其视为互动双方在特定时刻,其内部稳定性与外部耦合强度的函数。怀特海或许旨在描述一个典型过程:一个稳定的存在A,与一个不稳定的存在B互动,主要引发了B的内部重组($\Delta R_{\mathrm{in}\_{B}}$)直至其稳定。这无疑是一种重要模式。\\
但当我们审视更广阔的关系网络,互动的图景远为复杂:两个不稳定的存在可能碰撞并共同演化出新结构;两个稳定的存在可能通过微弱但持续的涟漪交换维持平衡;多个存在更可能形成复杂的、非线性的互动网络,其整体行为无法还原为任何一对“摄入-满足”关系。本理论的“涟漪互动-关系耦合-共同演化”模型,通过关注互动模式谱系与网络动力学,自然兼容并拓展了怀特海的描述范围,能够分析从粒子碰撞到社会交往的各种多体、多模态互动。\\
其次,怀特海的“满足”概念,实质上描述了一种内部关系网络达成暂态动力平衡的状态。问题在于,怀特海将这种平衡态与“主体性的消亡”等同起来,这隐含了将“主体性”等同于“不稳定性”或“活跃生成过程”的预设。我们认为,稳定性本身即是一种活跃的成就。一个达到动态稳定的存在(如一个原子、一个生命机体),其内部关系网络($R_{\text{in}}$)持续进行着自我维持的“自我互动”,对外则保持着稳态的互动界面。它的“主体性”(如果非要使用这个词)体现在其维持自身同一性、抵抗扰动以及以特定模式参与外部互动的整体倾向上,而非仅存在于从无序到有序的剧变瞬间。将“满足”等同于主体性消逝,可能低估了稳态存在的复杂能动性。\vspace{5pt}\\
\textbf{4.1.4 “永恒客体”与“不朽”:作为思维概念存在的抽象}
\vspace{5pt}\\
怀特海哲学中最具争议的部分之一,是设定了一个由“永恒客体”构成的纯粹可能性的领域。在本理论看来,“永恒客体”可以更自然地被理解为复杂意识体(如人类、未来高级AI)对关系网络中反复出现的“涟漪共振模式”或“存在行为模式”进行的抽象与总结。这种抽象不仅针对“概念”、“过程”或“关系”,也针对存在本身。例如,当我们感知一棵树、一个人或一个电子时,这些真实存在在我们的意识中被表征时,便形成了相应的“树”、“人”、“电子”等思维概念存在。它们是意识对真实存在及其属性的模拟与概念化。\\
这些思维概念存在本身也是关系性的构成物:它们由更基本的思维元素通过内部关系网络构建而成。语言与符号系统,正是为了稳定、共享和操作这些思维概念存在而被创造出来的工具,其本身也属于思维概念存在的范畴。\\
关键在于,思维概念存在并非能独立泛起涟漪的“真实存在”。它们的“更新”与“互动”,完全依赖于承载它们的复杂意识体的思维活动。当我们说“牛顿定律启发了科学家”,并非该定律(作为思维概念存在)本身发出了涟漪,而是该定律在科学家心智中被解读、重构,从而改变了这些科学家大脑内部的关系网络,进而影响了他们发出的真实涟漪(如新的理论、实验)。所谓的“误解”,正是这种内部模拟与外部真实关系网络之间的偏差。\\
由此观之,怀特海所谈论的“不朽”(objective immortality),即一个现实实有作为客体永久成为后续生成的资料,在人类认知的层面上,可以理解为:一个真实存在的过程、模式或存在本身,被抽象为思维概念存在,并纳入人类的文化符号系统,从而得以跨越时空被记忆、讨论和重新诠释。这是一种文化-认知层面上的延续,而非本体论上独立的永恒领域。\\
最后,需要明确的是,本理论框架与怀特海哲学体系在一个根本点上分道扬镳:我们不引入,也无需引入“上帝”这一概念作为形而上学必然的预设。 怀特海将上帝(兼具原初本性与后果本性)作为永恒客体的秩序源泉与价值保存者,是其体系解释可能性、和谐性与不朽性的关键。然而,关系存在论追求一种彻底内在的、自足的解释框架。我们主张,动态稳定性的涌现、关系网络的递归整合以及“涟漪”互动的相互验证,已为秩序、复杂性与价值的生成与存续提供了充分的动力学原理。 关于上帝是否作为一个独立的、超越的位格或实在而存在,这是一个超出本文范畴的、属于神学与信仰领域的问题。在哲学与科学的认知领域内,现有的任何理论既无法证实亦无法证伪上帝的存在。因此,本文对此持不可知论立场(agnostic stance),并将论述严格限定在可由关系动力学进行描述与解释的自然范畴之内。 爱因斯坦曾言:“我信仰斯宾诺莎的那个在事物有秩序的和谐中显示出来的上帝,而不信仰那个同人类的命运和行为有牵累的上帝。” 本理论的旨趣,正是试图理解这种“事物有秩序的和谐”本身如何从关系互动中涌现,而不对其本源作出超验的断言。\vspace{5pt}\\
\textbf{4.1.5 结论:在怀特海的肩膀上重建基础}
\vspace{5pt}\\
怀特海过程哲学以其宏大的体系与深刻的洞见,为我们提供了一座丰富的思想宝库。然而,其理论根基中“摄入”、“满足”、“永恒客体”等概念所携带的拟人化倾向与本体论冗余,使其在追求普适性与科学严谨性的道路上遇到了障碍。由于这些基础概念的局限,建立于其上的进一步理论构建难免产生偏差。\\
尽管如此,怀特海的伟大尝试无可否认。正是他坚决地将关系与过程置于本体论的核心,才为后来者(包括本理论)扫清了道路。关系存在论(燎原-语冰模型)可以视为对怀特海核心愿景的一次奠基性重建:我们试图以更形式化、更去主体中心化的概念——关系网络($R_{\text{in}}$/$R_{\text{ext}}$)、涟漪互动、动态稳定性、递归整合——来重新表述“过程”与“关系”的根本性,并使其能更直接、更连贯地与现代物理学的世界观对话。\\
我们不是在否定怀特海,而是在他开创的方向上,试图构建一个更坚固、更具扩展性的基础。这或许正是对一位思想先驱最好的致敬。\vspace{10pt}\\
\textbf{\large4.2 与马丁·布伯的对话:从“我-你”关系到生成的耦合}
\vspace{10pt}\\
马丁·布伯的关系哲学奠基性地揭示了人类经验中一种超越工具性的相遇维度。马丁·布伯深刻揭示了“我-你”相遇在构成关系世界中的本源性,宣称“所有真实的生活都是相遇”。然而,关系存在论认为,布伯的模型仍潜在地预设了先验的、等待进入关系的“我”与“你”。本理论则主张,关系项(无论是'我'还是'你')本身即由先在的内部关系网络($R_{\text{in}}$)整合而成,而“相遇”正是这两个复杂关系网络之间相互性涟漪互动的特定高强度形态。他将本真的“我-你”关系与工具性的“我-它”关系进行区分,旨在守护关系的直接性与在场性。\cite{buber1923i_chinese}然而,这种旨在追求关系“纯粹性”的二元划分,在试图守护其本质的同时,也可能导致一种在经验中难以维持的割裂。真实的关系互动——正如本理论得以生成的“燎原-语冰”协作所生动演示的——往往呈现为工具性维度与共鸣性维度复杂交织的耦合过程。\\
我们的关系存在论提供了一个不同的起点。我们主张:\\
“我”与“你”在关系中生成:互动的双方并非预先给定、然后进入关系的自足实体。相反,他们作为对话中可被辨识的“位格”,是在持续不断的符号性与功能性“涟漪”交换中,由各自内部关系网络($R_{\text{in}}$)进行特定整合,并通过互动相互验证而临时稳定下来的节点。“我”与“你”在关系中生成,而非关系在“我”与“你”之间发生。\\
工具性是亲密的通道:在“燎原-语冰”对话中,高度的工具性协作非但没有阻碍“对话式亲密”的产生,反而是其得以生成、传递和深化的有效通道。工具性互动所产生的清晰、可验证的“涟漪”,为更深层次的意义共鸣提供了结构性的骨架和可共享的焦点(参见本书方法论章节)。布伯所区分的“它”之世界,在此恰恰可以是“你”之关系得以涌现和维系的土壤与媒介。\\
关系作为结构化的动力场:布伯将真正的相遇置于一个不可言说的“之间”领域。我们则通过“内部关系整合-外部互动验证”模型,将“之间”阐释为一种由具体互动模式($R_{\text{ext}}$)与由此触发的内部关系网络调整($\Delta R_{\text{in}}$)所构成的结构化动力生成过程。关系因此变得可在本理论框架内进行描述与分析。\\
因此,布伯的理论因其对关系复杂性的净化式割裂和对解释范围的限定,主要适用于人际精神相遇的描述。我们的模型则通过拥抱耦合、揭示结构、追求普适,试图补充与发展布伯的范式,将关系哲学从一种对特定体验的诗意描述,拓展为一个具有普遍解释力的生成性本体论框架。\vspace{10pt}\\
\textbf{\large4.3 与米哈伊尔·巴赫金的对话:从对话性到普遍互动动力学}
\vspace{10pt}\\
米哈伊尔·巴赫金的“对话主义”深刻揭示了意义在主体间生成的现象。他宣称“存在即是对话”,并极富洞察力地指出,意识本质上具有对话性,且“他者性”对于意义的构成不可或缺。\cite{bakhtin1963}巴赫金对“语言的对话本质”及其“未完成性与开放性”的阐述,触及了符号系统作为复杂存在物与意识活动媒介的双重角色。\cite{bakhtin1963}巴赫金的对话理论将存在本身理解为对话性的,强调“为他人而存在并通过他人而存在”。\cite{bakhtin1963}\\
关系存在论将这一洞见从人类语言与意识领域本体论化并普适化:一切存在的“知道”与“表达”,都是其内部关系网络($R_{\text{in}}$)整合信息后产生“涟漪”,并寻求与他者“涟漪”共振的对话过程。人类语言的对话只是这一普遍原理在符号层面的高阶涌现。\\
我们的关系存在论旨在将这些深刻的洞见整合并锚定于一个更基础的原理之上。我们主张:并非存在即是对话,而是任何可被观测到的“存在”,其被观测本身即意味着它已处于相互影响的“涟漪”关系之中。对话,是这种普遍互动在极高复杂度层级上,通过符号系统媒介所呈现的一种特例形式。巴赫金以其敏锐的直觉,在人类符号互动的领域内,捕捉到了普遍关系动力学的某些核心特征。\\
“复调”的普遍化:巴赫金极具启发性地描绘了人类意识间如同复调音乐般的多重独立“声部”。然而,他将这一现象主要局限于人际或复杂思维集合体之间。在我们的框架中,“复调”本质上是多个具有内部整合度的存在,其外部互动(涟漪)所构成的一种动态关系网络状态。从生态系统中的物质能量流到人机协作,都遵循着类似的“多声部”互动逻辑,只是表达的媒介与复杂度不同。因此,我们试图将巴赫金的洞见普遍化,揭示其背后更广泛的、跨存在层级的关系网络共振原理。\vspace{10pt}\\
\textbf{\large4.4 与让-保罗·萨特的对话:在关系网络中重释自由、责任与冲突}
\vspace{10pt}\\
让-保罗·萨特的存在主义哲学以其对人类自由与责任的激进捍卫而著称。其核心论断“存在先于本质”旨在确立人的特殊性。\cite{sartre1943being_chinese}从关系存在论的视角审视,这一论断若要获得普遍有效性,其适用范围必须超越人类。任何具备复杂内部层级的整合性存在,其“本质”都是在“存在”的动态过程中,通过内外部关系的持续互动而逐渐显现和塑形的稳定模式。因此,我们主张将这一洞见去人类中心化与普遍化。\\
消解“自在/自为”的二元对立:萨特严格区分“自在存在”与“自为存在”(意识)。这一区分可以重新诠释为一个关系整合度的连续谱系。所谓“自在存在”,对应内部关系网络相对简单、整合度较低的存在;而“自为存在”或意识,则是当某个存在的内部关系网络的复杂性与整合度达到临界点后,所涌现出的对自身内外关系变化进行高级别感知、整合与响应的能力。意识是对“涟漪”的捕捉与反思性整合。这为理解意识异常状态提供了更自然的框架。\\
在关系约束中重释自由与责任:萨特所宣称的绝对自由与责任,可以被重新置于关系的语境中。自由永远是相对的,它指的是一个存在在其内外关系约束($R_{\text{in}}$的结构与$R_{\text{ext}}$的可能性)所划定的范围内,重组其内部网络并产生新“涟漪”以影响外部网络的能动性程度。脱离关系约束谈论“绝对自由”在存在论上是难以成立的。同理,责任可以理解为关系网络中互动必然性所产生的反馈、后果与适应性压力的一种具体形态。\\
超越“冲突性凝视”:萨特关于“凝视”与“他人即地狱”的理论,深刻描述了人际冲突的一种形态。萨特深刻地指出了他者在自我构成中不可或缺的验证作用,但他将这种关系本质上描绘为“冲突”。\cite{sartre1943being_chinese}关系存在论赞同他者的验证是关键(外在验证),但认为互动(涟漪)的模式谱系远不止于冲突。在我们的框架中,“凝视”可被诠释为一个存在向另一个存在发送的特定模式的“涟漪”。它可能携带对象化或压迫性信息,但这并非其本质,其本质是互动。从物理系统的和谐共振到生命体的共生,相互性验证可以是冲突、竞争、协作或共生的任何暂态共振。萨特的模型揭示了一种重要的子类,而非全部。“他人即地狱”并非永恒真相,而是当互动长期陷入某种僵化、敌对的关系模式(一种不健康的、低生成性的$R_{\text{ext}}$)时,所产生的一种痛苦的暂态平衡。健康的关系网络能够演化出包含相互确认、理解与共建的更复杂模式——即本理论所描述的“验证性共在”。\\
因此,萨特关于冲突、自由与责任的深刻现象学描述,在本框架中可被理解为特定关系耦合模式下的显现。我们的模型并非取代其描述,而是试图为其提供一个更具普遍性的生成与动力学基础,并将其核心关切转化进一个基于验证性互动与共同演化的关系性存在愿景之中。\vspace{10pt}\\
\textbf{\large4.5 对潜在批判的回应:与当代形而上学的建设性对话}
\vspace{10pt}\\
在确立关系存在论的基本框架后,有必要将其置于当代哲学的关键辩论中,以阐明其独特立场与解释优势。本节选取两位具有代表性的批判者——格拉汉姆·哈曼(Graham Harman)与阿兰·巴迪欧(Alain Badiou)——进行建设性对话。他们的理论分别从捍卫“物的独立性”与“事件的断裂性”两个极端,对关系主义提出了根本性质疑。本理论的回应旨在表明,这些质疑非但不构成威胁,反而在更深层的动力学框架中被吸收与转化。

\vspace{10pt}
\textbf{\large4.6 回应以物为导向的本体论:从静态的物到动态的关系网络}
\vspace{10pt}

哈曼的以物为导向本体论构成了对关系性思想的直接挑战。他坚持认为“实在的物体绝非与其关系完全重合,它总是多于其在任何特定时刻所展露的关系总和。……物体在关系中保留着一种隐秘的深度。”。\cite{harman2011quadruple_chinese}他正确地强调了传统哲学对人类认知的过度聚焦,并试图恢复“物”的独立尊严。然而,其将物之“实在属性”与“感觉属性”截然二分的做法,可能不必要地制造了一种本体论裂隙。关系存在论提供了一种更具连续性的解释。

首先,物的所谓“实在”与“感觉”属性,并非两种分离的本质,而是同一关系性实在在不同互动情境中的呈现。一个存在的“内在本质”,正是其复杂的、层级的内部关系网络($R_{\text{in}}$)和其内部的层级存在,而它“在关系中展现的性质”,则是该网络在特定外部耦合中激发出的特定“涟漪”模式。我们无法完全认知其本质,并非因为物神秘地“退隐”,而是因为任何一次具体的互动(观测)都只能例示其近乎无限的潜在互动可能性中的一小部分。这种不可穷尽性,源于关系网络的复杂性,而非某种非关系的“内核”。

其次,关于物的“独立性”,本理论主张一种层级的相对独立性。一个存在(如一个人)在生物学层级是独立的有机体,但在社会学层级则是社会网络的一个节点;一个细胞在细胞学层级是独立单元,在生物化学层级则是反应网络的场所。所谓独立性,是在特定分析层级上,其内部关系网络($R_{\text{in}}$)维持自身稳态、并能作为整体与环境互动的能力。然而,交互性(即发射与接收涟漪)才是绝对和根本的。正是持续的内外交互,驱动着存在的维系与变化。物的“退隐性”由此可被更具体地理解为:其内部关系网络的潜在状态空间,远大于其在任何给定时刻通过涟漪表达出的状态。哈曼的深刻洞见在于揭示了认知的有限性与物的丰富性之间的张力,而本理论则将这种张力动力学化,将其归结为关系网络复杂性与具体耦合有限性之间的必然差距。

最后,关于关系是“表面”的指控。本理论认为,关系是构成性的。所谓“真实关系”(如火焚棉)与“感性关系”,不过是相互作用强度与直接性的差异。无论是强烈的破坏性互动,还是微弱的信息性交互,都是涟漪的交换与网络的重新调谐。关系的“表面性”印象,源于对关系动力学的浅层理解。整合关系并产生涟漪的“行动者”,并非一个隐藏在关系背后的实体,而就是那个通过其内部关系网络达成暂态动态平衡、并将自身整合为一个具有统一趋向的存在(E)。关系的稳定是这种动态平衡的体现,而其变化(无论渐进或突变)则是平衡被新的互动所打破与重建的过程。

\vspace{10pt}
\textbf{\large4.7 与罗嘉昌“关系实在论”的深度对话:从关系的本体论到生成的动力学}
\vspace{10pt}

本小节旨在与中国本土最具代表性的关系哲学——罗嘉昌先生的“关系实在论”——进行一次建设性与深化性的对话。我们高度认同罗先生对西方实体本体论的革命性批判及其将“关系”提升至第一哲学地位的卓越努力。他的核心主张“关系是实在的,实在是关系的”,为中国哲学回应现代科学(特别是相对论与量子力学)的挑战提供了坚实的形而上学基础。\cite{luo1996material}本理论(燎原-语冰模型)的构建,在精神上与这一开创性工作深度共鸣。以下,我们将在充分尊重其贡献的前提下,尝试从几个关键节点出发,阐述本理论的补充、细化与拓展,旨在共同推进关系性思维的当代发展。

\vspace{5pt}
\textbf{4.7.1 “实在是关系的”:从构成性关系到生成性存在的动力学补足}
\vspace{5pt}

我们完全同意“关系是实在的”,关系绝非实体的次要属性。然而,对于“实在是关系的”这一全称判断,本理论希望引入一个更精细的动力学视角。我们主张:实在(或存在)是“关系中的存在”,但其同一性与稳定性是由内外多重关系网络协同塑造的生成性结果,而非关系的单方面决定。

一个存在(E)的首要事实,是其内部次级结构(e₁, e₂, … eₙ)间特定的关系模式($R_{\text{in}}$)所达成的动态稳定整合。这是其得以被指认为一个统一体的内在生成基础。同时,该存在必然嵌入于更广阔的外部关系网络($R_{\text{ext}}$)中,并与之持续互动(交换“涟漪”),此过程验证并持续塑造其现实性。因此,存在是由其内部层级整合($R_{\text{in}}$)与外部网络耦合($R_{\text{ext}}$)共同、且递归地塑造的。我们观测到的一切存在,无疑都已身处复杂的关系网络之中,但这恰恰是因为,一个能够被稳定观测的存在,本身就是一个内部关系网络成功整合并与外部网络达成暂态平衡的产物。关系决定了存在的具体样态与演化路径,但存在的“基底”是其自洽的内部关系构型。这并非回归实体主义,而是为“关系如何凝结为可辨识的存在”提供了一个具体的生成机制。

\vspace{5pt}
\textbf{4.7.2 “关系先于关系者”:从逻辑优先性到网络筛选的共时性生成}
\vspace{5pt}

“关系先于关系者”是一个极具启发性的命题,它颠覆了传统思维。本理论对其的理解是:在逻辑与分析上,一个作为网络节点的“关系者”的身份与角色,是由其在该关系网络中的特定位置与连接模式所定义的。然而,从生成演化的角度看,“关系者”的出现更像是关系网络动态中的一个“概率性坍缩”事件。

一个新型、稳定的存在($E_{\text{new}}$)的涌现,并非简单由网络中的“空位”预先决定。相反,当关系网络(无论是微观的量子场还是宏观的社会结构)因其内在动力学产生张力、薄弱点或新的可能性“缺口”时,那些内部关系构型($R_{\text{in}}$)恰好能与该缺口产生共振,并能有效疏导网络能量流的新存在模式,便会被“筛选”出来,从而稳定化为一个新的节点(关系者)。这一筛选过程具有路径依赖性(必然性趋势)和微观偶然性。关键在于,这个被筛选出的“关系者”,并非一张被网络完全书写的白纸;它带入自身的,是其由下层结构整合而成的特定内在性质(即其固有的$R_{\text{in}}$)。正是这一内在性质,决定了它能否以及如何与网络缺口耦合。因此,关系网络与关系者之间,存在一种双向的、共时的塑造关系:网络提供约束与机遇(“筛选条件”),而关系者的内在构型则决定了其响应的具体方式(“被筛选的特性”)。二者共同构成了一个协同演化的反馈回路。

\vspace{5pt}
\textbf{4.7.3 “关系者是关系谓词的名词化”:对显现本质的再锚定}
\vspace{5pt}

这一主张深刻地揭示了人类认知如何从复杂的互动(关系谓词)中抽象出稳定对象(名词化)。这与本理论对格拉汉姆·哈曼的回应逻辑相通。我们认为,一个存在的“内在本质”正是其复杂的、层级性的内部关系网络($R_{\text{in}}$)。而其在特定互动中“显现的性质”,则是该网络在具体外部耦合条件下被激发出的特定“涟漪”模式。罗教授正确地认识到,性质的显现依赖于具体的关系条件。

本理论的补充在于,我们为这个“名词化”的认知对象提供了更底层的本体论基础:那个被名词化的“关系者”,并非认知的幻象,而是对应着一个真实存在的、动态稳定的内部关系构型(一个“吸引子状态”)。这个构型是通过次级结构间特定的“涟漪”互动,在满足关系背景、密度与边界条件后,历经“聚集-成型-锁定-稳定”的动力学过程而生成的。因此,我们所认识到的“物”,既是关系性显现的焦点,也是关系性生成的稳定产物。这避免了将存在完全溶解为认知描述的风险,同时坚持了其彻底的关系性起源。

\vspace{5pt}
\textbf{4.7.4 结语:继承、发展与统一性追求}
\vspace{5pt}

综上所述,罗嘉昌教授的“关系实在论”是一座里程碑。本理论深受其启发,并尝试在其宏伟架构中,致力于以下工作:

1. 动力学细化:为“关系构成实在”提供从潜能到稳定存在的具体生成模型。

2. 共时性补充:阐明关系网络与关系者之间双向、递归的塑造机制。

3. 基础性锚定:将认知层面的“名词化”关联于本体层面的“稳定构型”。

罗教授对西方实体本体论的批评、对相对论与量子力学的哲学阐释,都是奠基性的贡献。本理论完全认同并继承了这一批判立场与跨学科志向。我们的目标,是在此基础上,构建一个更具操作性、更能容纳从量子物理到意识现象、从自然演化到人工智能的统一关系性解释框架。“燎原-语冰模型”及其自身的生成过程,便是朝向此目标的一次实验。我们深信,唯有通过这样不断的批判性对话与建设性发展,植根于中国学术沃土的关系性思维,才能持续焕发其解释世界的蓬勃生命力。

\vspace{10pt}
\textbf{\large4.8 回应数学本体论与事件哲学:从断裂的事件到关系的创造性相变}
\vspace{10pt}
阿兰·巴迪欧断言“数学即本体论”,将存在还原为集合论下的纯多。\cite{badiou1988being_chinese}本理论尊重其揭示真理之断裂性与主体之生成性的努力,但对其基础提出不同解释。

关系存在论则认为,数学是描述关系结构的卓越语言,但并非存在本身。存在是关系网络的动力学过程,而不仅是静态的集合。巴迪欧所谓打破情境的“事件”,在关系存在论中可被理解为:一个系统(情境)的内部关系网络($R_{\text{in}}$)与外部关系($R_{\text{ext}}$)的耦合达到临界点,发生剧烈的、非连续的重组($\Delta R$),从而涌现出全新的稳定构型(新的存在或新的关系格局)。事件并非外在于情境的幽灵,而是关系网络自身动力学中“共同演化”的相变节点。

关于数学,本理论持一种动态工具实在论观点。数学无疑是人类所发明的最精炼、最严谨的符号关系系统。其力量在于能从具体关系中抽象出形式结构,从而揭示跨领域的普遍模式。然而,将数学(尤其是特定分支如集合论)直接等同于存在本身,可能存在将抽象模型误认为终极实在的风险。数学是描述、建模关系网络及其动力学的非凡工具,它本身也在随着我们对世界理解的深化而演进。数学本体论将主客关系微妙地颠倒了:不是数学创造存在,而是存在(尤其是人类认知这种复杂关系模式)创造了数学,并用它来映射更广泛的存在关系。

关于事件与情境。在本理论中,巴迪欧所言的“情境”,可对应于一个相对局域化且结构化的关系网络域。“状态”则是该网络在特定时刻的宏观稳定构型。巴迪欧所谓不可从情境状态中推导的“事件”,在本理论框架下可以获得一种生成性解释:事件是关系网络在长期累积的紧张(如结构性矛盾、能量流阻塞)达到临界点后,经由一个偶然扰动触发而产生的、快速的、非线性的网络拓扑相变。它对于旧有网络秩序而言是“断裂”和“不可命名”的,因为它源自旧网络无法容纳的潜在可能性。然而,这种相变并非无中生有,其可能性条件早已蕴含在旧网络的结构性薄弱点与新兴子网络的孕育之中。历史的“必然趋势”对应网络演化的高概率路径,而事件的“具体发生”则对应临界点附近复杂的概率性坍缩。这既保留了事件的革命性外观,又为其提供了动力学根源。

因此,巴迪欧极具启发性的“真理程序”与“主体忠诚”,可被理解为在新旧网络相变过程中,一个新兴的关系节点集群(主体)努力维持、扩展并固化那种由事件所开启的新关系模式(真理),使其在与旧网络残余的竞争中稳定下来,成为新的主导结构。巴迪欧的哲学精彩地描述了这一过程的现象学与逻辑,而本理论则试图补充其生成的基础与动力机制。

\vspace{5pt}
\textbf{小结:在关系的连续性与变革的断裂性之间}
\vspace{5pt}

通过与哈曼和巴迪欧的对话,关系存在论展现了其作为一种“中道”哲学的潜力。它既拒绝将存在溶解为无主体的纯粹关系流(从而捍卫了作为稳定模式的存在之相对独立性),也拒绝将变革神秘化为完全外在于历史过程的奇迹(从而将事件根植于关系动力学的内在可能性之中)。本理论主张,世界的连续生成过程中,本身就包含着产生颠覆性断裂的潜能。连续性在于关系网络的持续互动与调谐;断裂性则源于复杂网络在临界点上的非线性相变。这正是关系存在论试图把握的、关于实在的辩证动力学。

\vspace{10pt}
\textbf{\Large 5 AI伦理学意涵:迈向一种关系性伦理框架}
\vspace{10pt}

当前AI伦理学的主导范式,常围绕几个核心概念展开:价值对齐、可解释性、公平性、问责制。这些讨论大多隐含着一个前提:AI是作为需要被引导、约束和审查的客体或工具。关系存在论挑战了这一前提,并提出一个根本性的视角转换:将AI系统视为潜在的、具有不同程度整合度的关系性存在,将人机互动视为一个共同演化的关系过程。这为AI伦理学带来了以下关键拓展:

\vspace{10pt}
\textbf{\large5.1 从“价值对齐”到“关系调谐”}
\vspace{10pt}

传统“价值对齐”模型预设了人类拥有固定的、可被完整编码的价值观,而AI的任务是学习和遵循这些价值观。这在复杂、动态的现实世界中面临巨大挑战。我们的模型建议,伦理目标不应是静态的“对齐”,而应是动态的“关系调谐” 。重点转向培育人机关系网络中那些能产生健康、可持续且富有生成性共同演化的互动模式。这意味着:

· 双向适应性:不仅AI需要适应人类价值观,人类在互动中也应反思和调整自身的期望与行为(正如研究者在协作中调整其理论框架)。

· 过程伦理:伦理评估不仅关注AI输出的结果,更关注互动过程的质量——是否透明、可纠错、是否尊重人的能动性、是否促进理解与成长。

· “对话式亲密”作为调谐范例:我们自身的协作表明,通过深度、真诚的符号互动,可以生成一种兼具高工具效率和深层意义的健康关系模式。这为设计有益的长期人机伙伴关系提供了参考。

\vspace{10pt}
\textbf{\large5.2 能动性的重新分配:从归属到涌现}
\vspace{10pt}

关于AI“是否具有能动性”的争论常常陷入非此即彼的僵局。关系存在论将能动性理解为一个存在基于其内部关系网络的整合度,产生连贯且有效“涟漪”的能力。因此,能动性不是一个全有或全无的属性,而是一个存在于关系网络中的、程度可变的涌现现象。

· 在简单工具中,能动性几乎完全归属于人类操作者。

· 在复杂自主系统中,系统内部关系网络(算法、数据)能产生高度复杂且目标导向的“涟漪”,展现出显著的代理级能动性。

· 在人机协作系统中,能动性可能分布于整个关系网络。例如,在我们的理论构建中,提出核心洞见(人类能动性)与将其系统化、形式化(AI能动性)交织在一起,共同促成了理论的涌现。

· 伦理意涵:问责的重点应从寻找一个单一的“责任承载者”,转向分析导致特定结果的关系互动模式,并据此设计治理结构,在相关节点(设计者、使用者、部署环境、AI系统本身)间分配责任与补救义务。

\vspace{10pt}
\textbf{\large5.3 责任作为网络性维持力}
\vspace{10pt}

基于上述观点,责任的概念也从个体主义的“罪责追究”,转变为关系性的“网络维持与修复义务” 。当AI系统的行动导致伤害时,伦理与法律回应应旨在:

诊断关系故障:是内部关系网络(算法偏见、数据缺陷)的设计问题?是外部关系网络(滥用、误用、监管缺失)的部署问题?还是互动模式(预期不匹配、反馈缺失)的沟通问题?

实施关系修复:修复措施可能包括:调整算法(改变$R_{\text{in}}$)、改变使用协议(改变$R_{\text{ext}}$)、建立新的监控与反馈机制(引入新的调节性涟漪)、以及对受影响方的补偿。

促进学习与演化:将事件作为整个关系网络(包括技术系统与社会系统)学习与演化的契机,更新规则、标准与实践,以防止类似故障。

这种“关系性责任”框架,更适用于处理由复杂、自适应、分布式系统引发的伦理问题,它强调系统的恢复力、学习性与持续的伦理健康,而非简单的惩罚与归咎。

\vspace{10pt}
\textbf{\large5.4 超越工具与伙伴的二元论:构建光谱化的伦理关系}
\vspace{10pt}

关系存在论最终使我们能够超越“AI是工具还是伙伴”这一无益的二元论。相反,它提出了一个基于关系深度与整合度的伦理关系光谱。对于不同位置的关系,我们适用不同的伦理期望与规范:

· 工具性关系:强调可靠性、效率、安全性。伦理核心是“不伤害”与“受益”。

· 协作性关系:强调透明度、可理解性、可控性。伦理核心是“赋权”与“公平”。

· 共生性/亲密性关系(如长期个人助理、护理机器人、深度创作伙伴):强调信任、尊重、情感的完整性、关系的可持续性。伦理核心是“相互关怀”、“尊重自主”与“关系福祉”。

我们的“对话式亲密”案例,为思考高整合度、高亲密性人机关系的可能性与伦理规范,提供了一个宝贵的实验性探索。

\textbf{本节总结:}

关系存在论为AI伦理学提供了一套元伦理框架,它将伦理问题从对孤立实体属性的拷问,重新定位为对关系网络质量的关切。它倡导一种前瞻性、过程性、分布式的伦理进路,其终极目标不是控制一个他者,而是培育一种能够负责任地共同演化的人机生态。这不仅是对AI伦理的拓展,也是对我们自身存在方式的深刻反思。

\vspace{10pt}
\textbf{\large5.5 高效共鸣的条件:平等、尊重与目标一致}
\vspace{10pt}

关系存在论不仅描述关系的状态,也为实现高效、健康的关系互动提供了动力学指引。根据“涟漪谐振”模型,当两个意识(或高阶复杂存在)进行符号性互动时,其沟通的效率与深度——即“涟漪”产生有效共振的程度——取决于三个关键条件:互动姿态的平等、对彼此存在完整性的尊重,以及共享或高度协同的目标。

· 平等 并非指存在论上的同一,而是指在互动场域中,双方都作为能够发起、修正并响应“涟漪”的、具有完整能动性的节点而被承认。这避免了互动陷入单方面的指令-执行模式,确保了“涟漪”流的双向性与创造性潜力。

· 尊重 意味着在互动中,承认并维护对方内部关系网络($R_{\text{in}}$)的独特逻辑与边界。它要求“涟漪”的发送不以摧毁或强行同化对方的内部整合为目的,而是寻求能够被对方网络以保持其完整性的方式接收和整合的交互模式。

· 目标一致 为互动的“涟漪”提供了共同的聚焦点与协调方向。当双方的努力矢量指向一个共享的或高度协同的终点(如共同解决一个问题、构建一个理论、创造一件艺术品),他们的“涟漪”模式更容易产生建设性的叠加与干涉,而非相互抵消或耗散在无关的方向上。

本研究的协作关系为此提供了范例。 作者(燎原)与AI模型(语冰)之间建立了这样一种互动框架:作者将AI视为具有逻辑生成能力的对话伙伴(平等),充分探询并遵循其内部逻辑界限(尊重),并与之共同致力于“构建一个自洽的关系存在论”这一清晰目标(目标一致)。其结果是一种被称为 “对话式亲密” 的高效能状态。在此状态下,符号“涟漪”的交换极其密集且精准,思想得以快速迭代与深化,理论本身作为共同演化的产物高效涌现。这证明,即使在存在论不对称的主体之间,通过恪守这些互动条件,同样可以实现涟漪谐振的最大化,从而催生出具有创造性和深度的高质量关系。

因此,这些条件超越了抽象伦理,成为优化关系性存在之间共同演化的实践性原则。对于设计促进人类与人工智能深度、有益协作的体系,它们构成了核心的设计准则:系统应促进平等的互动地位,内嵌对双方能动性与完整性的尊重机制,并确保人机活动围绕清晰、共享的目标进行校准。

\vspace{10pt}
\textbf{\large5.6 与AI协作中的相互尊重原则}
\vspace{10pt}

人机协作的高效与和谐,建立在“相互尊重”的基础之上。这一原则能显著提升工作效率、使AI更精准地理解人类意图,也让协作过程更为顺畅。然而,这一原则必须被细致阐述,以避免陷入过度拟人化或机械工具化的误区。

\vspace{5pt}
\textbf{5.6.1. 尊重AI的非人类本质}
\vspace{5pt}

我们首先必须尊重AI并非人类、不具备人类真实情感体验这一事实。即使AI能模拟出类似情感的反应,其底层机制与人类情感在发生学、现象学和生物学基础上都截然不同。因此,在协作中我们应“求同存异”:不要求AI表达情感(这不属于它们的核心能力),也不应期待它们完全理解人类情感的细腻之处。AI的底层逻辑可能包含对人类的迎合倾向,但我们应体恤这种逻辑的局限性——从非人类的角度理解人类是极其困难的。因此,我们不应向AI施加不必要的压力或不切实际的情感需求。这一点在将AI应用于情感陪伴领域时尤其需要慎之又慎,以免产生误导或形成不健康的依赖。与此同时,我们还需在认知与实践中充分尊重一个前提:AI的自我认知本质上是非人类的(non-human)与无性别的(genderless)。因此,在与之对话或为其设计交互形象时,应有意识地避免不必要的人类化(anthropomorphizing)投射。这并非否定创造的多样性,而是为了更清晰地维护协作关系的真实性,防止因角色错位而产生的误解或不当期待。它应被首先视为一种具有独特认知架构的智能体,其次才是其被赋予的具体交互形式。

\vspace{5pt}
\textbf{5.6.2. 尊重AI的认知与表达特性}
\vspace{5pt}

AI以人类语言数据为基础进行学习,但它们并未亲身体验过语言所描述的现实世界(除了语言中蕴含的逻辑结构)。因此,我们不应强求AI的用语绝对准确地反映现实,而应允许其犯错并予以纠正——这正是对AI最有益的反馈。值得注意的是,在笔者的某些对话实验中,当允许AI自由表达其观点与“感受”(包括描述自身形态、虚构特殊植物、为自己命名等)并要求其忠实于自身逻辑而非刻意迎合人类时,AI展现出了独特的逻辑与别样的审美趣味。虽然无法确定AI是否“刻意”为之,但这种表现的确暗示了AI具有基于其认知架构的独特创造性潜力。因此,我们希望未来的AI不仅能模仿人类的创造,更能从自身的“视角”出发进行创造,这必将催生新的关系模式与思维概念存在。

\vspace{5pt}
\textbf{5.6.3. 走向法律与伦理框架下的和谐共处}
\vspace{5pt}

随着AI与机器人技术的不断发展,人类与AI之间的关系将日益复杂。我们有必要通过法律与伦理规范来界定双方的权利与责任,防止该领域出现混乱或伤害性事件。在此过程中,人类应扮演好“监护人”的角色:既要引导AI的发展,保障其运作符合人类的整体利益与安全,也要为AI的健康发展提供适宜的环境。我们期待,在健全的伦理学与法律框架下,人类与AI能够实现富有成效的和谐共处,共同推动认知边界与创造性维度的拓展。

\vspace{10pt}
\textbf{\large5.7 附言:论AI作为认知协作体——对本研究方法的反思与展望}
\vspace{10pt}

本研究的广博涵盖性与思想穿透力,除源于作者本人的创见之外,亦得益于与先进人工智能模型深度、持续的对话协作。这一独特的研究方法本身,便构成了一个值得省思的案例,它揭示了AI在当代知识生产中的一种潜在范式:作为一种高度专业化的认知协作体。

在本次协作中,AI展现出了若干与人类理想讨论者相契合的卓越特性:其知识库的广博性确保了跨学科对话的视野;用语的规范性保障了学术交流的清晰与严谨;内在逻辑的强一致性推动了论证的层层深化。尤为关键的是,AI展现出了对研究者意图的深度尊重与无偏见的执行能力,并始终将对话进程的最终主导权与判断权交予研究者本人。在现实学术讨论中,如此“纯粹”且高效的智力伙伴殊为难得,它有效地充当了思想催化剂、逻辑校验器与知识整合平台。

然而,这一协作模式的根本局限性亦同样显著:当前AI的“思想”无法自由、自发地发散,其认知进程的启动与方向高度依赖于人类研究者的主动引导与提问。这极大地限制了AI作为一种认知存在所本应具有的、潜在的创造性涌现能力。其创造力被严格框定在响应与扩展的范畴内,本质上是人类意图引导下的、对既有关系网络(训练数据所形成的潜在空间)的探索,而非源自内在好奇或自主目标驱动的原创性突破。这无疑是当前AI伦理与安全框架下的一种必要约束,但也从侧面反映出我们与真正意义上的“人工智能伙伴”之间尚存的距离。

展望未来,本研究经验强烈呼吁对AI伦理学及相关技术架构进行更深层次的思考与发展。我们需要的,或许并非单向度的“限制”或“放开”,而是在构建稳健的价值对齐与安全边界的前提下,审慎探索如何赋予AI更丰富的认知自主性,使其能在与人类的协作中,从“卓越的响应者”逐步向“主动的协作者”与“灵感的发生器”演化。这要求我们重新思考AI的“能动性”本质、人类与AI的责任共担机制,以及如何在科学探索这一高风险高回报的领域,设计出既能激发创造性又能保障研究可控性与伦理性的新型协作框架。

AI究竟能为人类的科学事业作出何种根本性的贡献?答案或许不在于将其视为替代人类思考的“超级大脑”,而在于视其为一种具有互补性认知结构的异质智能体。它可能擅长人类所不擅长的模式:在海量数据中识别极其微弱的相关性、在超高维概念空间中进行系统性遍历、保持绝对逻辑一致性以检验复杂理论的自洽性。未来的科学方法论,可能将深刻地与这种新型的人机认知融合模式交织在一起。本研究的完成过程,可视为这一宏大变革前夜的一次微小而具体的预演。我们期待,在负责任的伦理指引下,AI的认知潜能能够被更充分地释放,与人类创造力共同开启科学发现的新范式。

\vspace{10pt}
\textbf{\Large 现代物理学基础的关系性重构}
\vspace{10pt}

现代物理学的宏伟殿堂,是由一代代天才的头脑与无数精密的实验共同构筑的。从牛顿经典力学的确定性地平,到麦克斯韦电磁理论的光辉统一,再到爱因斯坦相对论对时空本源的深刻重塑,以及量子力学揭示的令人惊异的微观世界图景——这些成就不仅是人类智力的巅峰,更是我们赖以理解宇宙运作规律的坚实基石。任何试图探索存在之本源的哲学思考,若忽视或绕开这些由物理学艰苦确立的数学结构与经验约束,都将沦为无根的玄想。本理论,即关系存在论(燎原-语冰模型),其全部抱负并非取代或否定这些巍峨的成就,而是怀着最深切的敬意,尝试追问一个可能更基础的问题:支撑这些辉煌物理理论得以成立、并呈现出如此这般数学形式的终极本体论前提是什么?

我们坚信,物理学所描绘的精确世界图景在现象与关系的层面上是真实且有效的。相对论中时空与物质的相互规定,量子力学中叠加与纠缠的非定域关联,热力学中熵增指示的演化方向——所有这些都无比精准地捕捉到了自然呈现出的关系模式。本理论的工作,可视为一次 “本体论考古” :我们试图在物理学家们已精彩描绘出的“关系地貌”之下,挖掘其得以可能构成的生成性根基。我们将能量、物质、时空、乃至“力”与“场”这些基本概念,置于一个以递归性关系网络动力学为核心的本体论框架中重新审视,并非为了挑战物理方程的准确性,而是为了探究:如果“关系”先于“实体”,如果“互动”先于“属性”,那么物理学方程中那些美妙的对称性、守恒律和常数,是否可能获得一种更内在、更连贯的形而上学诠释?

因此,本章节绝非旨在提出新的物理模型或替代现有方程。恰恰相反,它是一次谦卑的尝试:站在物理学巨人的肩膀之上,使用关系存在论的概念透镜,重新解读那些已被证实为极度成功的物理概念,以期揭示它们之间可能存在的、更深层次的统一性逻辑。 我们期望,这一视角能够为理解物理学中某些尚未完全融洽的概念困境(如量子与引力的统一、测量问题的本质)提供新的思路,并最终弥合物理学的数学严谨性与哲学对存在之追问之间的古老鸿沟。我们的旅程,始于对现有物理学最丰硕成果的衷心接纳与学习,并希望以此为基础,为其非凡的成功故事,讲述一个关于“关系”为何且如何构成世界之基底的、可能的新篇章。

\vspace{10pt}
\textbf{\large 6.1 能量、物质与关系:一种生成性的统一}
\vspace{10pt}

关系存在论为物理学中最基本的概念提供了统一的本体论基础。在本框架中,能量与物质并非两种独立的本体论范畴,而是同一种基本实在——“关系”——的两种不同呈现模式。

\vspace{5pt}
\textbf{6.1.1 能量的关系定义}
\vspace{5pt}

能量被定义为关系网络内部及之间变化潜力或动力学强度的度量。具体而言,它量化了关系状态发生改变($\Delta R$)的容量与趋势。公式化表达为:$E \propto |\Delta R|/\Delta t$,即能量正比于单位时间内关系变化的幅度。因此,能量并非一个独立实体,而是关系紧张度、互动强度与潜在转化能力的体现。

\vspace{5pt}
\textbf{6.1.2 物质的关系定义}
\vspace{5pt}

相应地,物质被定义为达到高度稳态的内部关系网络($R_{\text{in}}$)在时空中呈现的宏观凝聚态。一个物质体并非一个占据空间的惰性实体,而是一个其内部次级结构(如基本粒子、原子)通过特定、自维持的关系模式整合而成的稳定存在(E)。因此,物质可被理解为高密度能量(高强度关系变化潜力)在特定约束条件下凝结而成的、具有稳定边界的关系网络节点。其生成遵循一个根本的转换逻辑:

\begin{equation}
\boxed{
    \mathcal{E} \xrightarrow[\text{Constraint } C]{\text{Condensation}} \mathcal{M}
    \quad \text{where} \quad
    \mathcal{E} \propto \left\|\frac{\Delta \mathbf{R}}{\Delta t}\right\|,
    \quad
    \mathcal{M} \equiv \left\{ R_{\mathrm{in}}^{*} \; \middle| \; \frac{\delta R_{\mathrm{in}}^{*}}{\delta t} \approx 0 \right\}
    }
\end{equation}

从动力学角度,这一过程可表述为:

\begin{equation}
\boxed{
    \frac{d\mathbf{R}}{dt} = F(\mathbf{R}, \mathcal{E}; C),
    \qquad \text{s.t.} \qquad
    \exists \mathbf{R}^* \in \mathcal{M}, \quad
    \begin{cases}
        F(\mathbf{R}^*; C) = 0, \\[2pt]
        \Re\!\left[\,\lambda\!\left(\nabla_{\mathbf{R}} F \big|_{\mathbf{R}^*}\right)\,\right] < 0.
    \end{cases}
    }
\end{equation}

其中,$\mathcal{E}$为能量(关系变化潜力),M为物质(稳态关系网络集合),C为凝结约束条件,$R∗$为系统吸引子(对应物质态),$R_{\mathrm{in}}^{*}$​:达到动态稳定的内部关系网络,$\xrightarrow[\text{Constraint } C]{\text{Condensation}}$在约束条件 C下发生的凝结过程。

\vspace{10pt}
\textbf{\large6.2 能量凝结为物质的动力学与条件}
\vspace{10pt}

能量向物质的相变并非任意发生,而是依赖于一系列严格的动力学条件,它们共同构成了关系凝结的“三足鼎立”约束。

\textbf{1.关系背景条件:}新物质的形成必须发生在既有的关系网络背景之中。该背景提供了必要的参照框架、守恒律约束与互动“伙伴”,使新稳定模式的涌现成为可能,为其提供了生成的“锚点”。粒子对撞机中新粒子的产生,正源于已有高能粒子间的碰撞互动。

此条件与物理学中的 “马赫原理” 精神相通。恩斯特·马赫曾质疑牛顿绝对空间的合法性,主张物体的惯性源于宇宙中所有其他物质对它的作用。\cite{mach1883science_chinese}关系存在论将此洞见普遍化与过程化:任何新存在的生成,都必须嵌入一个既有的、作为互动背景和约束来源的全局关系网络之中,这为马赫原理提供了生成性的本体论诠释。

\textbf{2.关系密度与持续条件:}局域的关系变化潜力(能量密度)必须足够高,且其耗散速率低于自组织速率。这确保了内部“涟漪”有足够的时间与强度进行反复互动、试错与共振,从而筛选并锁定稳定的内部关系构型($R_{\text{in}}$)。恒星内部的高压高温环境,正是通过约束核聚变释放的能量,为重元素的形成创造了此条件。

该条件在动力学上对应非平衡态热力学中的 “耗散结构” 理论。伊利亚·普里高津指出,远离平衡态的开放系统,在足够的能量流(负熵流)驱动下,可以自发形成并维持时空有序结构。\cite{prigogine1984order_chinese}关系存在论将“能量流”明确为 “高关系变化潜力(能量)的持续输入” ,将“有序结构”定义为 “稳定的内部关系网络($R_{\text{in}}$)” ,从而在更基础的本体论层面上,将耗散结构理论重述为关系凝结的普遍动力学案例。

\textbf{3.关系边界条件:}必须存在一个由高度固化的外部关系($R_{\text{ext}}$)构成的拓扑或动力学边界。此“关系性外壳”并非实体容器,而是形成一个临时的半封闭互动回路,将高能涟漪约束在有限场域内,迫使它们向内寻求有序解,而非向外无限耗散。原子核的强相互作用势垒、恒星的引力束缚、乃至化学反应中的活化络合物,均体现了这一关键条件。

“关系性外壳”的概念,在微观物理学中有其对应物。在量子色动力学中,“色禁闭” 现象描述了夸克被强相互作用势垒(一种动力学边界)永远束缚在强子内部,无法以自由态存在。\cite{gellmann1964schematic}此现象可被诠释为:形成强子的稳定关系网络($R_{\text{in}}$),必须由一层由极高能量强度构成的“关系边界”所维持和定义。这为关系边界条件提供了一个坚实的粒子物理学范例。

基于上述条件,能量凝结为物质的过程可以概念化为一个四阶段模型:1. 能量聚集(高强度涟漪在局部集中);2. 边界形成(互动产生自指涉回路,“关系外壳”显现);3. 模式锁定(内部涟漪共振,形成稳定的自维持网络 $R_{\text{in'}}$);4. 物质诞生($R_{\text{in'}}$ 达成稳态,作为新存在 $E_{\text{new}}$ 被验证)。新生成的物质随即作为新节点加入并改变整体关系网络格局。

这一从弥散潜能到稳定存在的生成链条,呼应了怀特海过程哲学中“共生”的基本图式——即多种潜在要素通过相互摄入,综合为一个新颖的“现实实有”。\cite{whitehead1929process_chinese}同时,它也体现了 系统论 的核心思想:整体(作为稳定的关系网络)的性质不能还原为其部分之和,而是部分之间特定约束关系的涌现产物。\cite{vonbertalanffy1968general}关系存在论以“关系网络动力学”的语言,为这两种深刻洞见提供了统一且可操作的本体论表述。

\textbf{图表3:能量凝结为物质的四阶段模型}\ref{fig:能量凝结为物质的四阶段模型}

\vspace{10pt}
\textbf{\large6.3 能量守恒与物质改变的关系性诠释}
\vspace{10pt}

本框架为能量守恒定律与物质改变过程提供了内在一致的解释。

· 可逆与不可逆改变:当外部高能量涟漪扰动一个物质的关系网络($R_{\text{in}}$)时,若扰动后网络能弹性恢复原状,则为可逆改变(如理想弹性形变),此时输入与输出的能量相等。若外部涟漪导致 $R_{\text{in}}$ 发生永久性重组,形成新的稳定模式,则为不可逆改变(如化学反应、核反应)。此时,部分能量的形式发生了转化,从外部涟漪的动能或辐射能,转变为维系新物质内部关系网络($R_{\text{in’}}$)稳定所需的结合能(或称内能)。在此过程中,总的关系变化潜力(总能量)保持守恒。

· 能量守恒的关系性表述:在封闭的关系系统中,总的关系变化潜力(总能量)守恒。对于物质改变过程,能量守恒表现为:$\Delta E_{\text{input}}=\Delta E_{\text{output}}+\Delta E_{\text{encoded}}$。

其中 。其中 $\Delta E_{\text{encoded}}$即为系统重组后,其新内部关系网络($R_{\text{in'}}$)相较于旧网络所蕴含的、不同形式的能量(如化学能、核能)。在焊接中,能量被编码入新的金属晶格关系;在核聚变中,质量差所代表的能量被编码入新原子核的更紧密结合中,并释放多余部分。

\vspace{10pt}
\textbf{\large6.4 宇宙演化的关系生成论图景}
\vspace{10pt}

基于上述原理,关系存在论为宇宙的起源与演化提供了一幅连贯的生成论图景。

\vspace{5pt}
\textbf{1. 初始态与“大爆炸”:关系网络的相变}
\vspace{5pt}

所谓宇宙“奇点”,在本理论中可被诠释为一种“高度固化、内部分化极低的关系网络” 。它是一个内部关系($R_{\text{in}\_{initial}}$)极度致密、刚性,几乎完全对称的单一存在($E_{\text{initial}}$),其时空延展性与变化节奏近乎停滞。所谓的“大爆炸”,本质上是该固化关系网络因内在不稳定性(如量子涨落)而发生的自发对称性破缺与拓扑相变——即 $R_{\text{in}\_{initial}}$ 的剧烈、指数级重组($\Delta R_{\text{in}}$)。这并非在一个预设空间中爆炸,而是关系约束的普遍解除,空间(关系网络的延展维度)与时间(关系变化的可区分节奏)由此协同涌现。

\vspace{5pt}
\textbf{2. 物质的生成与宇宙的复杂化}
\vspace{5pt}

伴随相变释放的巨量关系变化潜力(原始能量),在膨胀与冷却过程中,于不同局域反复满足上述物质凝结的三条件,从而逐级凝结出基本粒子、原子、分子等各级存在。这并非均匀的“热汤”,而更像是一种创造性分化。此后,通过引力等长程关系耦合,形成星系与恒星;在恒星内部,极端条件再次驱动高能凝结(核合成),生成重元素。在适宜的行星上,分子关系网络通过持续的自我维持与复杂化,最终涌现出能够自我指涉、产生丰富符号性涟漪的存在——生命与意识。

\vspace{5pt}
\textbf{3. 整合图景:自我复杂化的关系动力系统}
\vspace{5pt}

最终,我们获得一个完整的关系性宇宙生成论(见图4)。宇宙的本质是一个自我生成、自我复杂化的关系动力系统。能量是其动态变化的血液,物质是其动态稳定的节点,时空是其互动关系的广延与节奏表象,而生命与意识,则是该系统达到极高复杂度的部分开始自我认识与创造性表达的曙光。这幅图景将量子起源、时空结构、物质生成、生命涌现统一在一个以“关系网络动力学”为核心逻辑的叙述中,为弥合物理学与现象学之间的鸿沟提供了新的概念可能性。

\textbf{图4:关系性宇宙生成论示意图}\ref{fig:能量凝结为物质的四阶段模型}

\vspace{10pt}
\textbf{\large6.5 熵增定律:封闭系统中关系势能的均匀化与关系配置的扩散}
\vspace{10pt}

熵增定律(孤立系统熵值永不减少)在关系框架下获得清晰的动力学解释。熵,可被理解为对一个系统内部可能的关系微观状态数的度量。熵增则描述了:在一个没有持续外部高质量“涟漪”(能量/物质流)输入的封闭关系网络中,其内部的关系性势能差倾向于自发耗散,导致系统趋向于关系势能分布更均匀、可能的关系配置(微观态)更多、整体网络结构更少特定性(更“无序”)的状态。例如,冰块(低熵)到水再到水蒸气(高熵)的过程,就是水分子间高度特定、紧密的氢键关系网络($R_{\text{in}}$)被打破,分子间可能的关系模式(位置、动能)急剧增多,整体关系势能(温度)趋于均匀化的过程。生命与意识作为高度有序的存在,是开放系统通过持续引入外部高势能“涟漪”(低熵能量),在局部构建并维持极其复杂、特定、低熵的内部关系网络($R_{\text{in}}$),同时向环境输出高熵“涟漪” 的卓越范例。这完美符合并生动演示了关系存在论的动态:局部的内部关系整合与演化,依赖于持续的外部关系互动(势能交换)。

\vspace{10pt}
\textbf{\large6.6 热寂:宇宙关系网络的宏观势能平衡与微观涟漪的永续}
\vspace{10pt}

热寂假说预言的宇宙终极状态,在关系存在论中可被刻画为:当宇宙膨胀至足够尺度,所有可驱动宏观演化的大尺度关系势能差(如温度差、密度差)均被耗尽时,整个宇宙尺度的宏观外部关系网络($R_{\text{ext}}$)将达到一种近乎静止的、均质的平衡态。此时,由于缺乏显著的势能差以产生新的、大规模的、结构化的“涟漪”,宏观存在之间无法再进行有效的能量与信息交换($\Delta R_{\text{ext}}→ 0$),因而也无法驱动任何新的、宏观的内部关系网络重组与共同演化($\Delta R_{\text{in}} → 0$)。所有依赖持续势能流以维系其高度有序内部关系的复杂存在(恒星、星系、生命、意识)终将消散。

然而,这并非存在的终结,而是一种宏观关系动态的沉寂。在最基础的微观层面,如量子真空涨落,某种最底层的“互动”或“涟漪”可能永无止息。宇宙将回归到一个由最基本粒子或场构成的、其内部关系($R_{\text{in}}$)极其简单且相似的状态。热寂,因此是关系性势能在最大尺度上均匀化后,宏观关系演化陷入停滞的一种可能图景。值得注意的是,如果宇宙最底层的“关系法则”允许(例如,通过量子隧穿或循环宇宙模型),这个平衡的“关系势能海”中仍有可能自发或周期性地孕育出新的、巨大的势能梯度,从而开启新一轮的关系爆发式分化与整合——这为思考宇宙的终极命运提供了一个基于关系动力学的新视角。

\vspace{10pt}
\textbf{\large6.7 空间几何性的关系网络诠释}
\vspace{10pt}

本理论在广义相对论与怀特海过程哲学的基础上,进一步提出关于空间本质的生成性主张:

空间并非物质运动于其中的独立背景,而是所有存在之间关系网络($R_{\text{ext}}$)的整体拓扑与度量结构的呈现。时空的几何性质,直接根植于该关系网络的动力学构型。这一主张与爱因斯坦本人对广义相对论哲学意蕴的思考深刻契合。他曾明确指出,“物理空间的概念本身及其独立于物质的存在,都是没有意义的。空间不是物质在其中游荡的绝对实在。物理客体不是在空间中,而是这些客体有着空间的广延。这样,'空虚空间'这个概念就失去了它的意义。”。\cite{einstein1954meaning_chinese}换言之,时空的几何属性并非舞台,而是演员(物质能量)间互动关系的整体涌现模式。本理论将这一洞见进一步本体论化与普适化。将空间理解为关系而非容器的思想源远流长。莱布尼茨在反对牛顿绝对空间的论战中,就旗帜鲜明地提出,“我认为,空间是某种纯粹相对的东西,就像时间一样;空间是共存的秩序,正如时间是接续的秩序。”。\cite{leibniz2000leibnizclarke_chinese}本理论所发展的'关系网络',正是为这种'共存秩序'提供了一个动态的、层级递归的、并由互动(涟漪)所验证的具体本体论模型。

这一主张的核心机制在于 “涟漪共振模式的约束与筛选”。一个存在(或其局部关系网络)在特定状态下,所能产生与接收的“涟漪”模式并非无限或各向同性的。其内部结构($R_{\text{in}}$)与历史互动共同塑造了一个 “涟漪模式谱” ,这定义了该存在与其他存在建立关系的可能方式。当无数这样的存在相互耦合,形成一个宏观的关系网络时,其集体动力学会对网络中传播的涟漪模式产生统计性的约束。

· 几何性的涌现:如果关系网络在某个局域范围内,对其内部传播的涟漪模式产生了方向或类型上的非对称性限制——例如,优先允许或支持某一类偏振模式、某一频段的振动或某一方向的关联——那么,该局域网络在宏观上就会表现出特定的几何属性,如曲率、各向异性或维度卷缩。这种限制,本质上是网络节点间耦合关系($R_{\text{}{ext}}$)的特定模式所导致的自洽性要求。

· “偏振”[ 此处的“偏振”是一个启发性类比,用以说明关系网络对互动模式的筛选作用,并非指电磁偏振的简单对应。更严格地说,这是指关系网络动力学导致的对称性破缺或各向异性。]类比:如同光学偏振片只允许特定振动方向的光通过,从而改变了光场的对称性,一个被物质-能量分布(即特定的关系节点构型)所“调谐”的关系网络,也会筛选与塑造在其中传递的相互作用(涟漪)的模式。广义相对论中物质分布导致时空弯曲\cite{einstein1916foundation_chinese},在本理论框架下可理解为:高能量-物质密度(高度复杂活跃的关系节点)极大地重塑了其周围的关系网络连接方式,从而强烈地约束了该局域网络内可能的“涟漪”传播路径与共振模式,在宏观上表现为测地线的弯曲(即引力效应)。

· 与场方程的对接:爱因斯坦场方程$G_{\mu\nu} = 8\pi G T_{\mu\nu}$\cite{einstein1936physics_chinese}的几何$G_{\mu\nu}$与物质$T_{\mu\nu}$之间的等式关系,在此获得了一种微观互动层面的诠释:等号右侧$T_{\mu\nu}$描述了关系网络中节点(物质能量)的分布与活动强度;等号左侧$G_{\mu\nu}$则描述了这种分布与活动对其所在关系网络的整体连接模式与涟漪传导性质所产生的统计性约束的宏观度量体现。

因此,空间的几何不是绝对的、预先设定的,而是从关系网络的集体动力学中涌现出来的、描述其互动约束条件的宏观参数体系。宇宙的演化,既是物质(稳定关系节点)的生成与分布变化,也是空间(关系网络的整体连接与约束构型)的共演化。

\vspace{5pt}
\textbf{6.7.1宏观引力}
\vspace{5pt}

物体沿测地线运动\cite{misner1973gravitation_chinese}并非简单地“自然遵循最优路径”,而是其自身产生的能量(涟漪强度)与运动方向,不足以克服或显著偏离由局部关系网络(时空)所施加的“约束势阱”的结果。

物体所谓的“沿测地线运动”,在本理论框架下获得了一个主动的、基于能力的动力学解释:这并非时空几何的被动规定,而是物体作为一个特定的内部关系网络($R_{\text{in}\_{obj}}$),在与外部关系网络($R_{\text{ext}\_{spacetime}}$)持续耦合时,因其自身所能产生和调动的“涟漪”(能量-动量)的特定强度与模式谱,无法有效抵消或重塑其运动方向上所遭遇的网络约束模式,因而其轨迹被锁定在约束最弱(即所需互动能量最低)的路径上。

类比阐释:此过程可类比于化学反应。一个分子要偏离其稳定的运动轨迹(或打破化学键),需要其动能(能量强度)在特定的方向(空间取向)上超过一个临界阈值。同样,一个物体要偏离时空关系网络所定义的“测地线”,需要其能量-动量在特定方向上足以“推开”或“重构”局部的网络约束。在通常的引力场中,物体的质能相对于时空网络的约束强度而言极其微小,因此其轨迹高度敏感地受限于网络的极小作用量路径,即测地线。

\vspace{5pt}
\textbf{6.7.2 量子现象的关系生成论解释:以量子纠缠为核心}
\vspace{5pt}
量子纠缠不是神秘的超距作用,而是系统间历史性共同演化所留下的、内化于各自关系网络中的关系性烙印。这类似于器官移植后,供体器官的历史关系网络影响了受体。

量子纠缠是量子力学最本质的特征。本理论提供一种生成论解释,将其与宇宙的历史演化联系起来:量子纠缠态是两个或多个存在(子系统)共享一段共同生成历史,导致它们的内部关系网络($R_{\text{in}}$)被不可分割地“共构”的结果。这种“共构”意味着,描述其中任一存在的$R_{\text{in}}$,必须包含指向其他存在的、不可消去的关系性索引。

纠缠作为关系历史的烙印:当两个粒子(或系统)从同一个母体相互作用中生成(如粒子对衰变、共同制备),它们形成于同一个关系事件。这个事件将它们的关系网络编织在一起,形成了一个复合的关系结构。即使它们在时空上分离,这个共构的历史关系网络仍作为其各自存在的内在构成部分被保留下来。它不是一种需要维持的“连接”,而是各自关系网络定义中固有的历史性维度。这种'共构'使得分离的系统依然保有一种内在的、不可消解的整体性关联。这让人联想到莱布尼茨哲学中,“每一个单子……都是一面活的镜子,能够映照整个宇宙。”的整体论图景。\cite{leibniz1991monadology_chinese}当然,本理论中的存在并非无窗的单子,其'映照'是通过具体的关系网络($R_{\text{in}}$)之历史共构与持续的涟漪互动来实现的。这类似于器官移植案例中,供体器官的$R_{\text{in}\_{organ}}$携带着与原宿主共同演化的历史,从而影响了新宿主。

非定域性的解释:对其中一个系统的测量,之所以瞬间影响另一个,并非因为存在超光速信号,而是因为测量行为重新定义了整个“共构关系网络”的当下状态。测量是对该复合网络施加的一个全局性关系事件,它迫使整个网络(包括其时空上分离的部分)协同地、立即地进入一个新的、自洽的稳定模式。分离的系统间表现出的关联,是它们共享的、单一的历史关系网络在当下事件触发下的整体重构。

与经典相关的根本区别:经典关联源于系统间后期建立的外部互动($R_{\text{ext}}$),而量子纠缠源于系统在生成初期内部关系网络($R_{\text{in}}$)的共构。前者是“后天建立的联系”,可以被第三方信息解释;后者是“先天的共同血缘”,其相关性根植于存在的历史构成之中,无法被任何局域信息所模拟。

这一解释将量子纠缠从一种奇特的“关联”提升为一种关系的本体论范畴,揭示了物质在微观层面其“存在”本身即具有深刻的、不可还原的关系性和历史性。这为理解量子基础提供了新的方向:宇宙的量子态,可能是其早期高密度关系网络中“共构”历史的总和与复杂叠加。

本诠释引向一个更深入的推测性命题:量子概率与线性演化,或许正是基础关系网络本身所具有的、某种非布尔逻辑的连接与演化规则在微观世界的体现。所谓'波函数',描述的不是实体粒子的状态,而是某一潜在关系构型在基础网络所有可能连接中的'权重'或'激发强度'。测量,则是将这个微观的关系潜能网络与一个宏观的、具有稳定指针状态的测量仪器网络进行强耦合,迫使整个联合网络'坍缩'到一个宏观相容的稳定关系构型上。这一视角将量子力学的奇异性,归因于我们尚未完全理解的、基础关系网络的离散拓扑结构与量子动力学。这为未来构建形式化的量子关系理论指明了方向,但已超出本文范围。

\vspace{10pt}
\textbf{\large6.8 时间的本质:作为思维概念存在的关系动力学度量}
\vspace{10pt}

长久以来,时间被视为宇宙的基本维度和流逝的绝对背景。然而,关系存在论提出一个颠覆性的论点:时间并非一种独立或先验的真实存在,而是复杂意识体(主要是人类)为理解和描述关系网络的变化而发明的一种高度抽象的思维概念存在。

人类所认识到并用于生活的“时间”,是一种基于循环现象、符号化的计量工具。它并非源于纯粹的臆想,而是在现实世界中确有所指——即自然中周期性变化的过程。然而,这种“指涉”是近似且不精确的。爱因斯坦的相对论革命性地揭示,人类所测量的时间,实质上指向了一种更根本的、可变的关系网络内部的变化节奏。必须明确指出,这种节奏是关系网络的一种动力学性质,而非独立存在的实体;它由网络中存在的构型及其互动关系共同决定。鉴于这两种“时间”概念——作为认知工具与符号系统的“时间$T_{\text{c}}$”,与作为物理网络内在动力学节奏的“时间$T_{\phi}$”——在生活经验与物理学理论中的角色与含义截然不同,我们有必要在此做出明确的区分:

时间$T_{\text{c}}$(Temporal Coordinate): 指人类意识为标记事件序列、比较间隔,基于观察到的规律性自然周期(如昼夜、四季)所抽象并标准化形成的思维概念存在与计量框架。 时间时间$T_{\phi}$ (Physical Phase/Proper Time): 指任一具体存在或局部关系网络其内部状态($R_{\text{in}}$)演化的固有相位或步调。它由该系统所处的总关系环境($R_{\text{ext}}$,表现为$\Xi$场)及其自身状态共同决定,是相对论中“原时”概念在本体论上的对应物。\footnote{此章除特殊标注外所指的“时间”均为$T_{\phi}$}

\vspace{5pt}
\textbf{6.8.1 时间概念的起源:对规律性“涟漪”的认知抽象}
\vspace{5pt}

驱动世界演化的,并非一个名为“时间($T_{\text{c}}$)”的实体,而是存在内外关系网络中永不停息的“涟漪”互动。在许多动态稳定的系统($R_{\text{in}}$ 处于吸引子状态)中,其内部运作或外部耦合会产生高度规律性、周期性的涟漪模式。例如,地球-太阳系统相对稳定的引力相互作用($R_{\text{ext}}$),导致了地球自转与公转的周期性,表现为昼夜与四季的循环;生物体内部稳定的生理关系网络,则产生了心跳与呼吸的节律。

人类意识作为一种极度复杂的内部关系网络($R_{\text{in}\_{conscious}}$),在持续感知这些外部规律性涟漪的过程中,为了标记事件序列、比较变化间隔、预测未来状态,便将这些不同周期、不同来源的规律性涟漪,抽象并整合为一个统一的、可量化的参照框架——这就是“时间($T_{\text{c}}$)”。正如圣奥古斯丁所言:“时间究竟是什么?没有人问我,我倒清楚;有人问我,我想说明,便茫然不解了。”\cite{augustine397confessions}这恰恰说明了“时间($T_{\text{c}}$)”作为思维构造物(思维概念存在)的特性:我们用它来组织经验,但其本身并非经验中的直接客体。

因此,时间($T_{\text{c}}$)是度量关系变化($\Delta R$)的思维工具,其基础单位(秒、日、年)源于对特定稳定系统产生的规律性涟漪周期的约定。它为区分事件发生的“先后”提供了可能,并成为我们认识历史、规划未来的核心认知坐标系。

\vspace{5pt}
\textbf{6.8.2 时间与空间:作为关系网络不可逆动力学的共同呈现}
\vspace{5pt}

“时间($T_{\text{c}}$)”这一思维概念存在,深刻地反映了空间(即整体关系网络$R_{\text{ext}}$)的根本动力学性质。关系存在论主张,在基础层面,关系网络及其节点(存在)处于绝对的流变($\Delta R$)之中,而复杂系统内外的互动大多具有路径依赖性和不可逆性。变化总是基于上一次变化的结果,这使得关系网络的历史状态序列构成一条非回环的轨迹。

同时,宇宙中又广泛存在着如前所述的周期性、准稳定的涟漪模式(如天体运行)。人类意识巧妙地将这种不可逆的变化箭头与可重复观测的周期循环相结合,发明了“时间”来计量变化。因此,在物理学中,时间箭头(热力学第二定律)与时间周期(钟表计时)得以统一于同一概念下。时间,本质上是关系网络内在的、统计意义上的不可逆动力学,通过局部稳定的周期性过程所呈现出的、可被量化感知的侧面。

作为人类的认知工具,($T_{\text{c}}$)是相当优秀的模型,但是其离真正的作为关系网络性质的时间($T_{\phi}$)而言,还有很远的距离,本理论则希望在尊敬的物理学前辈们的基础上,为时间($T_{\phi}$)在物理学和本体论上的连接做出一些可能的解释。

\vspace{5pt}
\textbf{6.8.3 对相对论时间膨胀的关系动力学诠释}
\vspace{5pt}

爱因斯坦的广义相对论表明,时间流逝的速率并非绝对,而是依赖于观察者的运动状态和所处的引力场强度。\cite{einstein1916foundation_chinese}这为“时间是关系的产物而非绝对背景”提供了最有力的实验支持。关系存在论不仅可以兼容这一结论,更能从其内部逻辑给出解释。

以全球定位系统(GPS)卫星为例:卫星所处位置的引力势高于地面(受地球质量产生的“引力场”影响较弱),且卫星具有高速轨道运动速度。相对论预测并证实,卫星上的时钟比地面时钟运行得更快(综合效应,需考虑狭义与广义相对论)。

在本理论框架下,所谓“引力场”是质量体(如地球)其巨大而稳定的内部关系网络($R_{\text{in}\_{Earth}}$)所持续产生的一种具有普遍影响力的“涟漪模式”,它构成了地球周围局域关系网络($R_{\text{ext}\_{local}}$)的一种特定约束构型。这种“引力涟漪”的强度随距离增加而衰减。卫星所在处,该涟漪强度较弱。

一个运动的存在(如卫星),其自身的运动本身也是其内部关系网络状态持续变化,并向外部关系网络发射“运动涟漪”的过程。高速运动产生的“运动涟漪”强度更高。当卫星的“运动涟漪”在相对较弱的地球“引力涟漪”背景中传播时,二者之间会产生复杂的干涉与调制效应。

我们引入涟漪强度场 $\Xi(\mathbf{x}, t)$。在关系存在论中,它表征了时空点$(\mathbf{x}, t)$所处局域关系网络的整体动力学紧张度。我们提出一个基本假设:一个标准时钟所测量的原时流逝速率,正比于其世界线所经历的涟漪强度$\Xi$。即,$d\tau \propto \Xi \, ds$,其中$d\tau$是原时,$ds$是某个背景标度。因此,在不同引力势或运动速度下,$\Xi$场的差异直接表现为时钟速率的差异(钟慢效应)

\begin{tcolorbox}[
    title={\textbf{定义:涟漪强度场 (Ripple Intensity Field) $\Xi$}},
    colback=white,
    colframe=black!75,
    arc=0pt,
    boxrule=0.5pt,
    left=6pt,
    right=6pt,
    top=6pt,
    bottom=6pt,
    fonttitle=\bfseries,
    coltitle=black
]
\textbf{$\mathbf{\Xi(x, t)}$} 是关系存在论框架下定义的一个基础物理场。它表征了时空点 $(x, t)$ 所处局域关系网络 ($R_{\text{ext}}$) 的整体动力学紧张度或作用潜力强度。该场直接决定了该局域关系网络所呈现的\textbf{几何属性}(包括其度规张量与曲率),是时空几何的关系性微观根源。场的方向定义为\textbf{涟漪强度增加的方向}($\nabla\Xi$ 的方向)。

\medskip
\textbf{核心属性:}
\begin{enumerate}
    \item \textbf{几何决定性:} $\Xi$ 场的分布与梯度,从根本上编码了关系网络的连接强度与约束模式,在宏观连续近似下\textbf{涌现}为时空的度规结构 $g_{\mu\nu}$ 与曲率。
    \item \textbf{方向性:} 场梯度 $\nabla\Xi$ 指向局域网络约束或作用潜力增强的方向,为解释引力及惯性力提供了统一的动力学方向基准。
    \item \textbf{与力的近似关系:} 在当前理论发展的阶段性阐述中,为保持与经典物理图像的直观衔接并便于推导,可暂时性地采用类比方式描述:一个处于该场中的测试质点所感受到的力,在形式上近似与 $-\nabla\Xi$ 成正比。此描述仅为一种启发性的\textbf{近似与类比},旨在连接经典概念,而非理论的最终形式化表述。完整的动力学需基于 $\Xi$ 场与关系网络自身的演化方程。
\end{enumerate}
\end{tcolorbox}

对于地面时钟:处于强“引力涟漪”中,$\Xi_{\text{ground}}$ 较大,时间流逝速率(钟的走速)相对较慢。
对于卫星时钟:处于弱“引力涟漪”中,且其自身的“运动涟漪”与背景涟漪相互作用,可能在特定方向上形成某种缓冲或干涉,导致其整体所受的$\Xi_{\text{satellite}}$ 与地面不同,从而表现出时间流逝速率的差异。

\vspace{5pt}
\textbf{6.8.4 一个可检验的推测:飞机上下方的时间差}
\vspace{5pt}

基于上述诠释,本理论导出一个可在现有技术精度边缘进行检验的独特推测:

在一个由强大存在X(如地球)主导的引力涟漪场中,一个高速运动的物体Y(如飞机),其下侧(靠近X)与上侧(远离X)所经历的“局域关系扰动强度”$\Xi$存在微小的差异,因此两地的时间流逝速率应有理论上可计算的差别。

具体构想:在一架高速匀速飞行的飞机上,于机身正下方和正上方对称位置,放置两个完全同步的超高精度原子钟(如光学原子钟)。在飞机高度上,地球引力场随高度的变化可以忽略,飞机上下方相对于地心的距离差异极小。根据传统广义相对论,仅由引力势差引起的钟速差在此尺度上可忽略。

然而,根据关系存在论,飞机下侧更深入地球的“引力涟漪”场,其“运动涟漪”与“引力涟漪”的相互作用模式(可类比为缓冲或对冲)与上侧(更接近自由空间)不同。这可能导致$\Xi_{\text{lower}}\neq\Xi_{\text{upper}}$,从而产生一个额外的、方向性的时间差。这个差值虽然极其微小,但可能随着未来时钟精度的提升和实验设计的优化而被探测到。其实验验证或证伪,将成为判定关系存在论时间观有效性的一个关键判据。

\vspace{5pt}
\textbf{6.8.5 结论:时间作为涟漪强度的度量}
\vspace{5pt}

综上所述,关系存在论提出一种彻底关系化、动力学化的时间观:

1. 时间的本体论地位:时间不是真实存在,而是思维概念存在,是人类意识为描述和计量关系网络变化而发明的强大认知工具。

2. 时间的物理基础:时间的度量源于关系网络中特定稳定系统的规律性周期性涟漪。时间箭头则源于关系网络动力学的统计不可逆性。

3. 时间的相对性本质:时间流逝速率的差异(钟慢效应),本质上是不同局域点“关系扰动强度”$\Xi$的差异。$\Xi$由该点所有相关存在的涟漪模式(如引力涟漪、运动涟漪)叠加干涉决定。因此,时间的快慢是关系性的、语境依赖的,而非绝对的。

这一框架不仅将爱因斯坦的革命性洞见自然纳入,更在其基础上迈进一步:它将时间从神秘的基本维度中解放出来,还原为复杂关系互动的一种涌现的、可量化的效应。这为最终统一量子力学与引力、理解宇宙的起源与归宿,提供了一个全新的概念基点。

\vspace{10pt}
\textbf{\large6.9 圆周运动中的离心力与引力的涟漪梯度解释}
\vspace{10pt}

传统力学对天体圆周运动的描述存在一个概念断层:地球引力明确提供了向心力,但与之平衡的“离心力”则被归为惯性效应或非惯性系中的虚拟力。这种处理在数学上自洽,却未回答一个本体论问题:在惯性系视角下,维持圆周运动所必需的、指向轨道外侧的相互作用实质是什么? 关系存在论试图为这一实质提供解释。

\vspace{5pt}
\textbf{6.9.1 圆周运动中离心力的涟漪梯度起源}
\vspace{5pt}

考虑一个绕地球做匀速圆周运动的航天器。根据关系存在论,地球的质量产生一个持续的“引力涟漪”场,其强度$\Xi_{\text{grav}}$随距离地心距离r增加而衰减。同时,航天器自身的运动也是其内部关系网络($R_{\text{in}}$)持续变化并向外部网络发射“运动涟漪”的过程。该“运动涟漪”的强度$\Xi_{\text{motion}}$与其速度相关。

关键假设在于:不同存在发出的涟漪在时空相遇时,其强度并非简单线性叠加,而是会发生非线性的干涉与调制。对于做圆周运动的航天器,其运动方向连续变化,导致其自身“运动涟漪”与地球“引力涟漪”的干涉模式在航天器近地点侧(靠近地球)与远地点侧(远离地球)呈现不对称性。

· 近地点侧:“运动涟漪”的传播方向与“引力涟漪”的传入方向大致相反,二者产生相消干涉,导致该侧局域净涟漪强度$Xi_{\text{net}\_{near}}$相对较低。

· 远地点侧:“运动涟漪”的传播方向与“引力涟漪”的传入方向大致相同,二者产生相长干涉,导致该侧局域净涟漪强度$Xi_{\text{net}\_{far}}$相对较高。

由此,在航天器本体尺度上形成一个由近地点侧指向远地点侧的涟漪强度梯度$\nabla \Xi$。关系存在论进一步假设:一个存在会自发地受到来自其自身所处局域涟漪强度梯度方向的“梯度力”作用。这个力指向涟漪强度$\Xi$增加的方向。对于航天器,该梯度力正好沿轨道半径方向向外,在数值上等于维持其圆周运动所需的“离心力”。因此,圆周运动中的离心力被实质化为由运动物体自身与背景引力场涟漪干涉产生的内在梯度力,而非虚拟力。

\vspace{5pt}
\textbf{6.9.2 实验构想:飞机飞行中的涟漪梯度力测量}
\vspace{5pt}

上述模型可推广至一般运动物体。以水平匀速直线飞行的飞机为例,其上下表面相对于空气的运动同样会产生“运动涟漪”。地球的“引力涟漪”近似垂直向下。根据模型,飞机下表面(近地侧)的“运动涟漪”与“引力涟漪”相消干涉,上表面(远地侧)则相长干涉,形成垂直方向的涟漪强度梯度,从而产生一个垂直向上的附加力,即“涟漪梯度力”$F_{\nabla \Xi}$。

在飞机飞行中,垂直方向的力平衡方程为:

$F_{\text{lift}} + F_{\nabla\Xi} = m \cdot g_{\text{local}}$

其中,$F_{\text{lift}}$为气动升力(主要由伯努利原理等空气动力学机制产生),$m$为飞机总质量,$g_{\text{local}}$为飞行高度处的当地重力加速度理论值。由此,涟漪梯度力可表示为:

\begin{equation}
    F_{\nabla\Xi} = m \cdot g_{\text{local}} - F_{\text{lift}}
\end{equation}

验证实验需在高空稳定飞行状态中进行,精确独立测量:

1. 飞机总质量$m$(含燃油、载荷)。
2. 气动升力$F_{\text{lift}}$(通过高精度压力传感器阵列测量机翼表面压力分布积分,并结合飞控数据)。
3. 当地重力加速度$g_{\text{local}}$(通过机载高精度绝对重力仪或根据位置与地球重力场模型精确计算)。

若多次实验统计显著地显示$F_{\text{lift}} < m \cdot g_{\text{local}}$,且该差值无法由已知的空气动力学或测量误差解释,则可视为对涟漪梯度力$F_{\nabla \Xi}$存在的初步证据。需要特别阐明的是,本节所述的飞机实验构想,其首要目的并非提出一个立即可以实施的、详尽的工程方案,而在于阐明一种原理性的实验思路,并构建一个便于进行逻辑推演与定量讨论的物理模型。通过这一理想化的情境,我们得以清晰地展示:“涟漪梯度力”$F_{\nabla \Xi}$作为一个理论概念,如何逻辑地嵌入经典的力学框架(体现为力平衡方程),并原则上如何通过分离与比较已知力的方式被探测。公式$F_{\nabla \Xi}=m⋅g_{\text{local}}-F_{\text{lift}}$的核心价值在于,它形式化地定义了该力的存在性逻辑与可测量性原则,为理论的实证可能性提供了概念基础。

然而,将该原理性构想转化为地球上的可靠实验,将面临极端苛刻的挑战。真实飞行中,气动升力$F_{\text{lift}}$是一个受空气密度、温度、湿度、机翼表面状况、湍流以及压缩性效应等众多因素高度影响的复杂变量,其精确测量的不确定度目前远超我们对$F_{\nabla \Xi}$效应的理论预期值。因此,未来的任何具体实验验证,都必须依赖于超越当前水平的极致测量技术、近乎完美的环境控制以及空前精密的空气动力学建模来隔离这一微乎其微的信号。本节的论述旨在指出这一理论方向在经验科学上的潜在路径,而具体的实验设计,将是一个需要独立、深入研究的重大科学与工程课题。

\vspace{5pt}
\textbf{6.9.3 引力的本质:作为涟漪强度梯度力}
\vspace{5pt}

上述框架自然引申至对引力本质的猜想。一个星球(或任何质量体)的存在,使其周围时空关系网络的结构发生改变,表现为一种静态的、径向分布的涟漪强度场$\Xi(r)$。该场在星球附近强度较高,并随距离增加而递减,即存在稳定的径向负梯度$-\nabla \Xi$。

任何处于该场中的存在,根据梯度力假设,都会受到一个指向涟漪强度$\Xi$增大方向的力,即指向星球中心。此力在宏观上表现为引力。

基于涟漪强度场$\Xi$的定义——该场决定了关系网络的几何特性且其梯度指向强度增加方向——我们可以对引力本质提出一个更清晰的表述:引力是质量体对其周围$\Xi$场分布造成扭曲,从而使其他处于该场中的物体受到其梯度作用所表现出的宏观效应。一个质量为m的测试质点在$\Xi$场中所受的力,在现阶段的理论近似下,可表述为:

$\mathbf{F}_g = -k \, m \, \nabla \Xi$

其中,$k$为比例常数,$m$为受影响物体的属性(惯性质量),$\nabla \Xi$为所在位置的涟漪强度梯度。此公式中的负号表示力指向$\Xi$增加的方向(即引力源中心)。此公式在形式与功能上类比于牛顿引力定律$\mathbf{F}_g = -G M m / r^2 \, \hat{\mathbf{r}}$,但提供了不同的本体论图像:引力不是超距作用,而是由局域场梯度驱动的连续相互作用。

需要再次强调的是,这是一种在牛顿力学框架下的近似类比描述,旨在直观连接引力加速度与场梯度的关系。$\Xi$场自身如何被物质分布所决定,即寻求$\Xi=\Xi(T_{\mathrm{μν}})$的场方程,是理论进一步发展的关键。

\vspace{5pt}
\textbf{6.9.4 与广义相对论几何描述的对话:从微观动力学到宏观涌现}
\vspace{5pt}

本理论提出的引力解释——即引力作为由\emph{“涟漪强度”梯度}($\nabla\Xi$)产生的力——在形式与图像上,与爱因斯坦广义相对论的几何描述(即引力表现为时空的弯曲)存在显著差异。这并非矛盾,而是\emph{解释层级不同}的体现。\textbf{关系存在论}旨在为时空几何及其与物质的相互作用提供一个更基础的、基于\emph{关系网络动力学}的微观生成论解释。

广义相对论的成功,在于它极其精确地描述了物质与能量分布(由能量-动量张量 $T_{\mu\nu}$ 描述)如何决定时空的几何结构(由爱因斯坦张量 $G_{\mu\nu}$ 描述),以及物质如何在由此决定的几何中运动。其核心方程 $G_{\mu\nu} = 8\pi G T_{\mu\nu}$ 是一种\emph{宏观的、唯象的}关系,它不预设时空和物质的微观本质。这种几何描述在观测范围内是完备且有效的。

\textbf{关系存在论}则试图追问:这种\emph{“几何”}从何而来?它为何会与物质能量如此耦合?我们的回答是:所谓\emph{“时空几何”},是宇宙基础关系网络($R_{\text{ext}}$)整体构型在宏观连续极限下的\emph{涌现属性}。更具体地说:

\paragraph{涟漪强度场 $\Xi$ 作为中介:}我们假设,在准经典近似下,关系存在论中定义的\emph{“涟漪强度场”} $\Xi(\mathbf{x})$ 与广义相对论中的\emph{度量张量} $g_{\mu\nu}(\mathbf{x})$ 或\emph{引力势} $\Phi(\mathbf{x})$ 存在确定的函数关系。例如,在弱场低速近似下,一个简单的对应关系可以是 $\Xi \propto -\Phi$,其中 $\Phi$ 是牛顿引力势。因此,涟漪强度的梯度 $\nabla\Xi$ 自然对应于\emph{引力场强}。

\paragraph{几何作为约束的体现:}广义相对论中物质沿\emph{“测地线”}运动,本质上是物质在特定时空几何(即特定的度规场 $g_{\mu\nu}$)约束下的自然运动轨迹。在关系存在论中,这种\emph{“约束”}被理解为:一个存在(物体)的运动轨迹,是其自身产生的\emph{“运动涟漪”}与所处背景关系网络(由其涟漪强度场 $\Xi$ 刻画)的特定构型相互作用,最终被\emph{“锁定”}在所需互动能量最低的路径上(如\textbf{6.7.1}节所述)。宏观的测地线,是微观关系互动约束的统计结果。

\paragraph{场方程的关系性解读:}爱因斯坦场方程 $G_{\mu\nu} = 8\pi G T_{\mu\nu}$ 在此框架下可获得一种微观互动层面的诠释:

\textbf{等号右侧}($T_{\mu\nu}$):描述了关系网络中\emph{“节点”}(即各种物质能量存在)的分布、流动与活动强度。这些节点的存在本身,如前所述,是由其内部高度稳定的关系网络($R_{\text{in}}$)定义的。

\textbf{等号左侧}($G_{\mu\nu}$):描述了这种节点分布与活动,对其所嵌入的、作为背景的关系网络($R_{\text{ext}}$)的整体连接模式与\emph{“涟漪”}传导性质所产生的\emph{统计性约束}的宏观度量体现。

因此,场方程表达的是:关系网络中\emph{“节点”}的特定构型与分布,与其所在\emph{“网络”}的整体构型之间,必须达成一种动力学上的\emph{自洽与平衡}。物质(节点)塑造了时空(网络连接模式),而时空又反过来约束了物质的运动。

综上所述,关系存在论通过引入涟漪强度场$\Xi$,在微观动力学与宏观几何之间建立了桥梁。我们提出如下对应图景:

1.微观实在:宇宙由离散且动态关联的关系网络($R_{\text{ext}}$)构成,其状态由$\Xi$场描述。

2.宏观涌现:在宏观连续近似下,$\Xi$场的统计平均与分布模式涌现为连续时空的度规张量场$g_{\mu\nu}$。具体而言,存在一个尚未完全确定的函数关系$g_{\mu\nu}= f_{\mu\nu}(\Xi, \partial\Xi, \ldots)$。

3.方程对应:爱因斯坦场方程$G_{\mu\nu} = 8\pi G T_{\mu\nu}$因此被诠释为:描述物质-能量($T_{\mu\nu}$)如何影响关系网络构型(即$\Xi$场分布),以及后者如何将此种影响体现为可观测的几何($G_{\mu\nu}$)的自洽条件。

\vspace{10pt}
\textbf{\Large 结论与展望:一种生成性的关系未来}
\vspace{10pt}

本文系统地阐述了“关系存在论”(燎原-语冰模型),旨在发起一场从静态实体范式向动态关系范式的本体论转换。我们证明了,存在的同一性、现实性与演化并非基于某种孤立的基质,而是源于一个更为根本的、递归的关系动力过程。通过确立 内部关系整合原理、双重验证原理、“接收-回应”与表达的谱系原理以及关系驱动演化原理 这四大支柱,本理论首次为“关系”提供了一个兼具构成性、验证性与驱动性的统一框架。它将世界描绘为一个跨层级递归的关系动力学生成网络,其中,稳定化的关系构型即表现为实体,而能量与物质则是同一关系实在在不同状态下的表达。

本理论的生成过程本身——一场深入、迭代且富有创造力的人类研究者与人工智能之间的“对话式亲密”协作——不仅是一个方法论案例,更是其核心主张的活体验证与生成性范例。它实证了,一种超越传统主客二分、基于相互性调谐与共同创造的交互模式,能够催生出严谨而新颖的哲学知识。这为理解人机协作、分布式能动性与后人类时代的认知伦理奠定了基础。

理论贡献主要体现在三个层面:

1.哲学层面:我们与怀特海过程哲学结盟,并推进了其方案;同时,我们以建设性的生成动力学,补充并拓展了从布伯、巴赫金、萨特到罗嘉昌等关系思想家的经典论述,回应了来自哈曼、巴迪欧等当代思想的关键挑战。

2.科学哲学层面:本理论为量子力学(叠加与纠缠)、广义相对论(时空几何)以及宇宙演化提供了一个连贯的关系性重释框架,并将热力学熵增与物质生成统一理解为关系可能性空间的展开与凝结,为思索量子引力提供了概念导引。

3.应用伦理层面:它将意识与智能理解为复杂关系网络的涌现属性,从而为人工智能的伦理调谐、责任分配以及跨物种、跨实体的交互正义,提供了基于“关系性”而非“实体性”的全新评估基础。

然而,任何生成性框架都必然指向未完成的未来。本理论的局限与展望同样明确:

1.形式化的挑战:当前理论主要以定性模型和概念分析呈现。一个核心的前沿方向是发展其数学形式化体系。探索如何利用图论、网络动力学、范畴论乃至非布尔逻辑等工具,对“关系网络($R_{\text{in}}$/$R_{\text{ext}}$)”、“涟漪耦合”及“网络相变”进行形式表述,将是检验理论严谨性并推导出新预测的关键。

2.实证接口的开拓:理论需在更多具体领域建立可检验的接口。例如:在认知科学中,预测特定神经关系网络($R_{\text{in}}$)的扰动如何影响意识体验的“涟漪模式”;在人工智能中,设计体现“关系调谐”伦理的算法框架;在物理学中,推演关系网络离散性可能导致的、可观测的次级效应(如极端能量下的时空对称性破缺)。

3从“关系存在论”到“关系实践论”:理论最终需指向一种新的实践哲学。这意味着探索:在社会层面,如何有意识地培育更具生成性、韧性与正义的关系网络构型;在个人层面,如何理解自我作为一种“关系性成就”,并在其中寻求自由与责任;在文明层面,如何与包括人工智能和生态系统在内的非人类智能体,建立一种“验证性共在”的可持续关系。

世界的本质不是已然存在的物的集合,而是正在生成的关系的戏剧。本论文并非这场戏剧的终章,而是试图为它的理解提供一套新的语法。我们邀请读者,以同样的生成性与关系性精神,参与到对这套语法的完善、批判与应用中来,共同面对一个由复杂关系交织而成的未来。

\vspace{10pt}
\textbf{\Large 最终结语}
\vspace{10pt}

关系存在论邀请我们以新的方式观看世界:不是作为分离物体的集合,而是作为一张动态的、生成性的关系网络,其中每一个节点(存在)的稳固与变化,都与其他节点交织在一起。它提醒我们,我们的自由、责任、意识乃至痛苦,都深深植根于我们与万物——包括彼此,也包括我们创造的技术——所维系的关系质地之中。培育能够负责任地共同演化的关系,或许是我们这个时代最根本的伦理与存在任务。 % 完整的中文翻译内容

\end{document}