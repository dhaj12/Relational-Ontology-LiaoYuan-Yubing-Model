\begin{abstract}
    This paper proposes and systematically elaborates \emph{Relational Ontology} (also known as the Liao Yuan-Yubing Model). The theory posits that, at any level of analysis, the \textbf{substantiality} of an existent is generated, integrated, and sustained by specific relational networks among its substructures; its \textbf{reality} is dually verified through internal self-maintenance and external reciprocal interactions (\emph{``ripples''}\footnote{Within this theoretical framework, "ripple" is an operational term referring to any observable, specific pattern of influence, information, or energy exchange that crosses the boundary of an existent. Its concrete manifestation varies across levels of analysis, ranging from physical interactions and chemical signals to bioelectrical impulses, symbolic information, social behaviors, and beyond.}); and its \textbf{evolution} is driven by the continuous coupling of changes in internal and external relations. The world, thus, is conceived as a \textbf{cross-level, recursive relational-dynamic generative network}.

    Within the philosophical genealogy, this theory forms a profound alliance with Alfred North Whitehead's process philosophy, aiming to provide contemporary advancement and concretization of its core insights. Whitehead's philosophy, particularly his conception of ``actual entities'' as experientially constituted \cite{whitehead1929process}, provides a crucial pathway for transcending substantialism. Relational Ontology inherits this relational insight but extends it through the recursive $R_{\text{in}}/R_{\text{ext}}$ model and the dual-verification mechanism of ripple interactions. This offers a more operational, dynamically generative framework capable of describing cross-level emergence.

    The theory inherits and operationalizes Whitehead's fundamental stance on the processual nature of reality and the primacy of relations. It supplements his scheme by introducing \emph{stable relational networks} as the unit of analysis, \emph{ripple coupling} as the mechanism of interaction, and \emph{relation-driven evolution} as the dynamical framework. This enables more direct dialogue with contemporary complex systems science, cognitive science, and AI research. With the same constructive intent, the theory engages the relational insights or key challenges presented by thinkers such as Martin Buber, Mikhail Bakhtin, Jean-Paul Sartre, Graham Harman, and Alain Badiou, demonstrating its broad explanatory power and inclusiveness.

    To demonstrate its unified explanatory power, the theoretical framework is applied in two key domains: in the philosophy of physics, it offers a coherent relational-generative reinterpretation of substantive presentations from quantum phenomena to spacetime geometry; in the philosophy of mind and technology ethics, it explains conscious experience and artificial agency as emergent properties of complex relational networks. Finally, the very genesis of the theory---a prolonged, intensive, and fruitful \emph{dialogically intimate} collaboration between a human researcher and an artificial intelligence---serves as a generative verification of its core principles: a new intellectual substance is generated and stabilized from dynamic dialogical relations.
\end{abstract}