\section{A Relational Reconstruction of Modern Physics Foundations}
\label{sec:A Relational Reconstruction of Modern Physics Foundations}
The grand edifice of modern physics has been constructed through the collective genius of generations and countless precise experiments. From the deterministic horizon of \emph{Newtonian mechanics}, to the luminous unification of \emph{Maxwell's electromagnetic theory}, to \emph{Einsteinian relativity's} profound remolding of the nature of spacetime, and the astonishing microcosmic picture revealed by \emph{quantum mechanics}—these achievements represent not only the pinnacle of human intellect but also the solid foundation upon which we understand the operating principles of the cosmos. Any philosophical inquiry into the origin of existence that ignores or bypasses these mathematical structures and empirical constraints, painstakingly established by physics, risks degenerating into rootless speculation. The entire ambition of this theory, \textbf{Relational Ontology (the LiaoYuan-Yubing Model)}, is not to replace or negate these towering achievements but, with the deepest respect, to attempt to pose a possibly more fundamental question: \emph{What are the ultimate ontological presuppositions that underpin the validity of these magnificent physical theories and grant them their particular mathematical forms?}

We firmly believe that the precise picture of the world depicted by physics is real and effective at the \emph{level of phenomena and relations}. The mutual determination of spacetime and matter in relativity, the nonlocal correlations of superposition and entanglement in quantum mechanics, the direction of evolution indicated by entropy increase in thermodynamics—all of these capture with unparalleled accuracy the relational patterns nature exhibits. The work of this theory can be seen as an \textbf{``ontological archaeology''}: we attempt to excavate the \emph{generative bedrock} beneath the ``relational landscape'' so brilliantly mapped by physicists, which makes that landscape possible. We re-examine fundamental concepts like \emph{energy}, \emph{matter}, \emph{spacetime}, and even \emph{``force''} and \emph{``field''} within an ontological framework centered on recursive relational network dynamics, not to challenge the accuracy of physical equations, but to inquire: \emph{If ``relation'' precedes ``substance,'' if ``interaction'' precedes ``property,'' might the beautiful symmetries, conservation laws, and constants in physical equations acquire a more intrinsic, more coherent metaphysical interpretation?}

Therefore, this chapter does not aim to propose new physical models or replace existing equations. On the contrary, it is a humble attempt: \emph{standing on the shoulders of physics' giants}, using the conceptual lens of \textbf{Relational Ontology} to reinterpret those physical concepts proven immensely successful, hoping to reveal a possible, deeper logic of unity among them. We hope this perspective can offer new avenues for understanding certain unresolved conceptual tensions within physics (such as the unification of quantum theory and gravity, the nature of the measurement problem) and ultimately bridge the ancient chasm between the mathematical rigor of physics and philosophy's quest to understand being. Our journey begins with the heartfelt acceptance and study of physics' most fruitful achievements and hopes, upon this foundation, to narrate a possible new chapter for its extraordinary success story—a chapter about why and how \emph{``relations''} constitute the ground of the world.
\subsection{Energy, Matter, and Relation: A Generative Unification}
\label{subsec:energy-matter-relation-unification}

\textbf{Relational Ontology} provides a unified ontological foundation for the most fundamental concepts in physics. In this framework, \emph{energy} and \emph{matter} are not two independent ontological categories but rather two different modes of manifestation of the same basic reality—\emph{``relation.''}

\subsubsection{A Relational Definition of Energy}
\label{subsubsec:relational-definition-energy}

\textbf{Energy} is defined as the measure of the potential for change or the dynamical intensity within and between relational networks. Specifically, it quantifies the capacity and tendency for a relational state to alter ($\Delta R$). Formally, this is expressed as: $E \propto |\Delta R|/\Delta t$, meaning energy is proportional to the magnitude of relational change per unit time. Therefore, energy is not an independent substance but the manifestation of \emph{relational tension}, \emph{interactive intensity}, and \emph{potential for transformation}.

\subsubsection{A Relational Definition of Matter}
\label{subsubsec:relational-definition-matter}

Correspondingly, \textbf{matter} is defined as the macroscopic, condensed manifestation in spacetime of an internal relational network ($R_{\text{in}}$) that has achieved a high degree of homeostatic stability. A material body is not an inert substance occupying space but a stable existent (E) integrated by specific, self-sustaining relational patterns among its internal substructures (e.g., elementary particles, atoms). Thus, matter can be understood as a relational network node with stable boundaries, condensed from high-density energy (high-intensity relational potential) under specific constraint conditions. Its generation follows a fundamental transformational logic: 
\begin{equation}
\boxed{
    \mathcal{E} \xrightarrow[\text{Constraint } C]{\text{Condensation}} \mathcal{M}
    \quad \text{where} \quad
    \mathcal{E} \propto \left\|\frac{\Delta \mathbf{R}}{\Delta t}\right\|,
    \quad
    \mathcal{M} \equiv \left\{ R_{\mathrm{in}}^{*} \; \middle| \; \frac{\delta R_{\mathrm{in}}^{*}}{\delta t} \approx 0 \right\}
    }
\end{equation}
From a dynamical perspective, this process can be described as:
\begin{equation}
\boxed{
    \frac{d\mathbf{R}}{dt} = F(\mathbf{R}, \mathcal{E}; C),
    \qquad \text{s.t.} \qquad
    \exists \mathbf{R}^* \in \mathcal{M}, \quad
    \begin{cases}
        F(\mathbf{R}^*; C) = 0, \\[2pt]
        \Re\!\left[\,\lambda\!\left(\nabla_{\mathbf{R}} F \big|_{\mathbf{R}^*}\right)\,\right] < 0.
    \end{cases}
    }
\end{equation}
where $\mathcal{E}$denotes energy (potential for relational change), M denotes matter (a collection of stable relational networks), C denotes the condensation constraint condition, $R*$ denotes the system attractor (corresponding to the material state), $R_{\mathrm{in}}^{*}$ denotes the internal relational network that has achieved dynamic stability, and $\xrightarrow[\text{Constraint } C]{\text{Condensation}}$ denotes the condensation process occurring under the constraint condition C.

\subsection{The Dynamics and Conditions for Energy Condensation into Matter}
\label{subsec:dynamics-conditions-energy-condensation}

The phase transition from energy to matter does not occur arbitrarily but depends on a series of strict dynamical conditions, which together form the \emph{``tripod'' constraint} for relational condensation.

\paragraph{1. Relation-Background Condition:} The formation of new matter must occur within the background of a pre-existing relational network. This background provides the necessary frame of reference, conservation law constraints, and interactive \emph{``partners,''} making the emergence of new stable patterns possible and providing an \emph{``anchor''} for generation. The creation of new particles in particle colliders originates precisely from the collisional interactions among pre-existing high-energy particles.

This condition resonates in spirit with the \emph{``Mach's Principle''} in physics. Ernst Mach challenged the legitimacy of Newton's absolute space, proposing that the inertia of a body originates from the influence of all other matter in the universe \cite{mach1883science}. Relational ontology generalizes and dynamizes this insight: the emergence of any new existent must be embedded within a pre‑existing global relational network, which serves both as the interactive background and the source of constraints. This provides a generative ontological interpretation of Mach's Principle.

\paragraph{2. Relational Density and Persistence Condition:} The local potential for relational change (energy density) must be sufficiently high, and its dissipation rate must be lower than its self-organization rate. This ensures that internal \emph{``ripples''} have sufficient time and intensity for repeated interaction, trial and error, and resonance, thereby selecting and locking in a stable internal relational configuration ($R_{\text{in}}$). The high-pressure, high-temperature environment inside stars creates this condition by constraining the energy released from nuclear fusion, enabling the formation of heavier elements.

This condition corresponds dynamically to the theory of \emph{``dissipative structures''} in non‑equilibrium thermodynamics. Ilya Prigogine demonstrated that open systems far from equilibrium, driven by a sufficient flow of energy (a flux of negentropy), can spontaneously form and maintain spatiotemporal ordered structures \cite{prigogine1984order}. Relational ontology explicitly interprets the \emph{``flow of energy''} as the \emph{``sustained input of high relational‑change potential (energy)''} and defines \emph{``ordered structures''} as \emph{``stable internal relational networks ($R_{\text{in}}$).''} Thereby, it reformulates dissipative‑structure theory, at a more fundamental ontological level, as a universal dynamical instance of relational coagulation.

\paragraph{3. Relational Boundary Condition:} There must exist a topological or dynamical boundary formed by highly solidified external relations ($R_{\text{ext}}$). This \emph{``relational shell''} is not a substantial container but forms a temporary semi-enclosed interactive loop, constraining high-energy ripples within a limited domain and forcing them to seek orderly solutions inwardly, rather than dissipating infinitely outwardly. The strong interaction barrier of the atomic nucleus, the gravitational binding of stars, and even the activated complex in chemical reactions all embody this crucial condition.

The concept of a \emph{``relational shell''} finds its counterpart in microphysics. In quantum chromodynamics, the phenomenon of \emph{``color confinement''} describes how quarks are perpetually bound within hadrons by a strong interaction potential barrier—a kind of dynamical boundary—and cannot exist as free states \cite{gellmann1964schematic}. This phenomenon can be interpreted as follows: the stable relational network ($R_{\text{in}}$) that forms a hadron must be sustained and defined by a \emph{``relational boundary''} constituted by extremely high energy intensity. This provides a solid particle-physics exemplar for the relational boundary condition.

Based on the above conditions, the process of energy condensing into matter can be conceptualized as a four-stage model: 
1. \textbf{Energy Aggregation} (high-intensity ripples concentrate locally); 
2. \textbf{Boundary Formation} (interactions produce self-referential loops, the \emph{``relational shell''} emerges); 
3. \textbf{Pattern Locking} (internal ripples resonate, forming a stable self-sustaining network $R_{\text{in}'}$); 
4. \textbf{Matter Birth} ($R_{\text{in}'}$ achieves homeostasis, verified as a new existent $E_{\text{new}}$). 
The newly generated matter then joins as a new node, altering the overall relational network configuration.

This generative chain, from diffuse potential to stable existence, resonates with the fundamental schema of \emph{``concrescence''} in Whitehead's process philosophy—wherein multiple potential elements, through mutual prehension, synthesize into a novel \emph{``actual entity''} \cite{whitehead1929process}. Simultaneously, it embodies the core idea of systems theory: the properties of the whole (as a stable relational network) cannot be reduced to the sum of its parts but are rather an emergent product of the specific constraining relations among those parts \cite{vonbertalanffy1968general}. In the language of \emph{``relational network dynamics,''} relational ontology provides a unified and operational ontological formulation for these two profound insights.

\textbf{Figure 3: The Four-Stage Model of Energy Condensation into Matter} \ref{fig:The Four-Stage Model of Energy Condensation into Matter}

\subsection{A Relational Interpretation of Energy Conservation and Material Alteration}
\label{subsec:relational-interpretation-energy-conservation}

This framework provides an internally consistent explanation for the law of energy conservation and the process of material alteration.

· \textbf{Reversible and Irreversible Change:} When external high-energy ripples perturb the relational network ($R_{\text{in}}$) of matter, if the network can elastically restore itself afterward, it is a reversible change (e.g., ideal elastic deformation), where input and output energy are equal. If the external ripples cause permanent reorganization of $R_{\text{in}}$, forming a new stable pattern, it is an irreversible change (e.g., chemical reaction, nuclear reaction). At this point, part of the energy undergoes a transformation—from the kinetic or radiative energy of external ripples into the binding energy (or internal energy) required to maintain the stability of the new material's internal relational network ($R_{\text{in'}}$). Throughout this process, the total relational-change potential (total energy) is conserved.

· \textbf{Relational Formulation of Energy Conservation:} In a closed relational system, the total potential for relational change (total energy) is conserved. For material alteration processes, energy conservation is expressed as: $\Delta E_{\text{input}} = \Delta E_{\text{output}} + \Delta E_{\text{encoded}}$. Here, $\Delta E_{\text{encoded}}$ is the energy (in forms such as chemical or nuclear energy) embodied in the new internal relational network ($R_{\text{in}'}$) compared to the old one after system reorganization. In welding, energy is encoded into the new metallic lattice relations; in nuclear fusion, the energy represented by the mass defect is encoded into the tighter binding of the new nucleus, with the excess released.

\subsection{A Relational-Generative Picture of Cosmic Evolution}
\label{subsec:relational-generative-picture-cosmic-evolution}

Based on the above principles, \textbf{Relational Ontology} provides a coherent generative picture of cosmic origin and evolution.

\subsubsection{The Initial State and the ``Big Bang'': A Phase Transition of the Relational Network}
\label{subsubsec:initial-state-big-bang}

The cosmic \emph{``singularity''} can be interpreted within this theory as a \emph{``highly solidified relational network with minimal internal differentiation.''} It is a singular existent ($E_{\text{initial}}$) whose internal relations ($R_{\mathrm{in}\_{\text{initial}}}$) are extremely dense, rigid, and nearly perfectly symmetrical, with its spatiotemporal extensibility and pace of change nearly halted. The \emph{``Big Bang''} is essentially a spontaneous symmetry breaking and topological phase transition of this solidified relational network due to internal instability (e.g., quantum fluctuations)—a violent, exponential-scale reorganization ($\Delta R_{\text{in}}$) of $R_{\mathrm{in}\_{\text{initial}}}$. This is not an explosion within a pre-existing space but a universal lifting of relational constraints, from which space (the extensive dimensions of the relational network) and time (the distinguishable rhythm of relational change) co-emerge.

\subsubsection{The Generation of Matter and the Complexification of the Cosmos}
\label{subsubsec:generation-matter-complexification-cosmos}

Accompanied by the phase transition, an immense amount of relational potential (primordial energy) is released. During expansion and cooling, this potential repeatedly meets the three conditions for material condensation in different localities, thereby sequentially condensing into hierarchical existences: elementary particles, atoms, molecules. This is not a uniform \emph{``hot soup''} but more akin to a creative differentiation. Subsequently, through long-range relational couplings like gravity, galaxies and stars form; within stars, extreme conditions once again drive high-energy condensation (nucleosynthesis), producing heavy elements. On suitable planets, molecular relational networks, through continuous self-maintenance and complexification, finally give rise to existences capable of self-reference and generating rich symbolic ripples—life and consciousness.

\subsubsection{The Integrated Picture: A Self-Complexifying Relational Dynamical System}
\label{subsubsec:integrated-picture-self-complexifying-system}

Ultimately, we arrive at a complete \textbf{relational cosmogony} (see Figure 4). The essence of the universe is a self-generating, self-complexifying relational dynamical system. Energy is the lifeblood of its dynamic changes, matter is its dynamically stable nodes, spacetime is the extensive and rhythmic manifestation of its interactive relations, while life and consciousness are the dawn of parts of this system achieving extremely high complexity beginning to know themselves and engage in creative expression. This picture unifies quantum origin, spacetime structure, matter generation, and the emergence of life within a narrative centered on the core logic of \emph{``relational network dynamics,''} offering new conceptual possibilities for bridging the chasm between physics and phenomenology.

\textbf{Figure 4: Schematic of Relational Cosmogony} \ref{fig:Schematic of Relational Cosmogony}

\subsection{The Law of Entropy Increase: The Homogenization of Relational Potential and the Diffusion of Relational Configurations in Closed Systems}
\label{subsec:law-entropy-increase}

The law of entropy increase (the entropy of an isolated system never decreases) gains a clear dynamical explanation within the relational framework. \textbf{Entropy} can be understood as a measure of the number of possible relational microstates within a system. Entropy increase then describes the tendency within a closed relational network without sustained input of high-quality external \emph{``ripples''} (energy/matter flow) for internal relational potential differences to dissipate spontaneously, driving the system toward a state with more homogeneous distribution of relational potential, a greater number of possible relational configurations (microstates), and less specificity in the overall network structure (more \emph{``disordered''}). For instance, the process from ice (low entropy) to water to water vapor (high entropy) is precisely the breaking of the highly specific, tight hydrogen-bond relational network ($R_{\text{in}}$) among water molecules, leading to a dramatic increase in possible relational patterns (position, kinetic energy) between molecules and a homogenization of the overall relational potential (temperature). Life and consciousness, as highly ordered existences, are outstanding examples of open systems that continuously import external high-potential \emph{``ripples''} (low-entropy energy) to construct and maintain extremely complex, specific, low-entropy internal relational networks ($R_{\text{in}}$), while exporting high-entropy \emph{``ripples''} to the environment. This perfectly aligns with and vividly demonstrates the dynamics of Relational Ontology: local internal relational integration and evolution depend on continuous external relational interaction (potential exchange).

\subsection{Heat Death: The Macroscopic Potential Equilibrium of the Cosmic Relational Network and the Perpetuation of Microscopic Ripples}
\label{subsec:heat-death}

The ultimate state of the universe predicted by the heat death hypothesis can be characterized within Relational Ontology as follows: when the universe expands to a sufficient scale and all large-scale relational potential differences (e.g., temperature gradients, density contrasts) capable of driving macroscopic evolution are exhausted, the macroscopic external relational network ($R_{\text{ext}}$) on the scale of the entire universe will reach a nearly static, homogeneous equilibrium state. At this point, due to the lack of significant potential differences to generate new, large-scale, structured \emph{``ripples,''} effective energy and information exchange between macroscopic existences ceases ($\Delta R_{\text{ext}} \rightarrow 0$), and thus no new, macroscopic internal relational network reorganization and co-evolution can be driven ($\Delta R_{\text{in}} \rightarrow 0$). All complex existences that rely on a continuous flow of potential to sustain their highly ordered internal relations (stars, galaxies, life, consciousness) will ultimately dissipate.

However, this is not the end of existence, but rather a cessation of macroscopic relational dynamics. At the most fundamental microscopic level, such as in quantum vacuum fluctuations, some most basic form of \emph{``interaction''} or \emph{``ripple''} may persist indefinitely. The universe would revert to a state composed of the most basic particles or fields, with internal relations ($R_{\text{in}}$) that are extremely simple and similar. Heat death, therefore, is a possible scenario where relational evolution grinds to a halt on the largest scale after the homogenization of relational potential. It is noteworthy that if the most fundamental \emph{``relational laws''} of the universe permit (e.g., through quantum tunneling or cyclical universe models), new, enormous potential gradients could still spontaneously or cyclically emerge from this equilibrated \emph{``sea of relational potential,''} potentially initiating a new round of explosive differentiation and integration of relations—offering a new perspective based on relational dynamics for contemplating the ultimate fate of the cosmos.

\subsection{A Relational Network Interpretation of Spatial Geometry}
\label{subsec:relational-network-interpretation-spatial-geometry}

Building upon general relativity and Whiteheadian process philosophy, this theory further proposes a generative claim regarding the nature of space: \textbf{Space} is not an independent background within which matter moves, but rather the manifestation of the overall topology and metric structure of the relational network ($R_{\text{ext}}$) among all existents. The geometric properties of spacetime are directly grounded in the dynamic configuration of this relational network. This proposition aligns profoundly with Einstein's own reflections on the philosophical implications of general relativity. He explicitly stated, \emph{``The concept of physical space itself and its independence of the existence of any matter is without meaning. Space is not an absolute reality in which matter swims. Physical objects are not in space, but these objects are spatially extended. In this way the concept of 'empty space' loses its meaning.''} \cite{einstein1954meaning}. In other words, the geometric properties of spacetime are not a stage but an emergent pattern of interactions among the actors (matter-energy). The present theory further ontologizes and universalizes this insight. The understanding of space as relations rather than a container has a long intellectual lineage. In his debate against Newton's concept of absolute space, Leibniz unequivocally asserted, \emph{``I hold space to be something purely relative, like time; space is an order of coexistences, as time is an order of successions.''} \cite{leibniz2000}.

The core mechanism of this claim lies in the \emph{``constraint and selection of ripple resonance modes.''} The \emph{``ripple''} patterns that an existent (or a local relational network) can generate and receive in a specific state are neither infinite nor isotropic. Its internal structure ($R_{\text{in}}$) and historical interactions jointly shape a \emph{``ripple mode spectrum,''} which defines the possible ways this existent can establish relations with others. When countless such existents couple with each other, forming a macroscopic relational network, their collective dynamics impose statistical constraints on the modes of ripples propagating within the network.

· \textbf{The Emergence of Geometry:} If the relational network generates asymmetric restrictions in direction or type on the ripple modes propagating within a certain local region—for example, preferentially allowing or supporting a particular polarization mode, a specific frequency band of vibration, or correlations in a certain direction—then that local network will macroscopically exhibit specific geometric attributes, such as curvature, anisotropy, or compactified dimensions. This restriction is essentially a self-consistency requirement arising from the specific patterns of coupling relations ($R_{\text{ext}}$) among the network nodes.

· \textbf{The ``Polarization'' Analogy:} \footnote{``Polarization'' here is a heuristic analogy to illustrate the filtering effect of the relational network on interaction modes, not a simple correspondence to electromagnetic polarization. More precisely, it refers to the symmetry breaking or anisotropy arising from the dynamics of the relational network.} Just as an optical polarizer only allows light of a specific vibration direction to pass, thereby altering the symmetry of the light field, a relational network \emph{``tuned''} by a matter-energy distribution (i.e., a specific configuration of relational nodes) also filters and shapes the modes of interactions (ripples) transmitted within it. In the framework of this theory, the curvature of spacetime caused by mass-energy distribution in general relativity \cite{einstein1916foundation} can be understood as follows: high mass-energy density (highly complex and active relational nodes) drastically reshapes the connection patterns of the surrounding relational network, thereby strongly constraining the possible propagation paths and resonance modes of \emph{``ripples''} within that local network, manifesting macroscopically as the bending of geodesics (i.e., gravitational effects).

· \textbf{Connection to Field Equations:} The equality between geometry ($G_{\mu\nu}$) and matter ($T_{\mu\nu}$) in Einstein's field equations, $G_{\mu\nu} = 8\pi G T_{\mu\nu}$ \cite{einstein1936physics}, here receives an interpretation at the level of microscopic interaction: the right side $T_{\mu\nu}$ describes the distribution and activity intensity of nodes (mass-energy) in the relational network; the left side $G_{\mu\nu}$ describes the macroscopic metric manifestation of the statistical constraints that this distribution and activity impose on the overall connection patterns and ripple conduction properties of the hosting relational network.

Therefore, spatial geometry is not absolute or pre-given but emerges as a macro-parametric system describing the interactive constraints from the collective dynamics of the relational network. Cosmic evolution is both the generation and distributional change of matter (stable relational nodes) and the co-evolution of space (the overall connection and constraint configuration of the relational network).

\subsubsection{Macroscopic Gravity}
\label{subsubsec:macroscopic-gravity}

The motion of objects along geodesics \cite{misner1973gravitation} is not simply a matter of \emph{``naturally following an optimal path.''} Rather, it is a result of the fact that the energy (ripple intensity) generated by the object itself and its direction of motion are insufficient to overcome or significantly deviate from the \emph{``constraint potential well''} imposed by the local relational network (spacetime).

The so-called \emph{``motion along a geodesic''} acquires an active, capacity-based dynamical explanation within this theoretical framework: it is not a passive prescription of spacetime geometry, but a consequence of the fact that an object, as a specific internal relational network ($R_{\mathrm{in}\_{\text{obj}}}$), in its continuous coupling with the external relational network ($R_{\mathrm{ext}\_{\text{spacetime}}}$), cannot effectively counteract or reshape the network constraint patterns it encounters along its path, due to the specific intensity and mode spectrum of the \emph{``ripples''} (energy-momentum) it can generate and mobilize. Thus, its trajectory becomes confined to the path of least constraint (i.e., the path requiring the minimal interaction energy).

\textbf{Analogical Elucidation:} This process can be analogized to a chemical reaction. For a molecule to deviate from its stable trajectory (or to break a chemical bond), its kinetic energy (energy intensity) in a specific direction (spatial orientation) must exceed a critical threshold. Similarly, for an object to deviate from the \emph{``geodesic''} defined by the spacetime relational network, its energy-momentum in a specific direction must be sufficient to \emph{``push against''} or \emph{``reconfigure''} the local network constraints. In typical gravitational fields, the mass-energy of an object is exceedingly minute compared to the constraint strength of the spacetime network, rendering its trajectory highly sensitive to and confined by the network's least-action path—the geodesic.

\subsubsection{A Relational-Generative Explanation of Quantum Phenomena: Focusing on Quantum Entanglement}
\label{subsubsec:relational-generative-explanation-quantum-entanglement}

\textbf{Quantum entanglement} is not a mysterious spooky action at a distance, but rather a relational imprint left by the historical co‑evolution between systems, internalized within their respective relational networks. This is analogous to how the historical relational network of a donor organ influences the recipient after transplantation.

Quantum entanglement is the most essential feature of quantum mechanics. This theory offers a generative explanation, linking it to the historical evolution of the universe: A quantum entangled state is the result of two or more existents (subsystems) sharing a common generative history, leading to the inseparable \emph{``co-constitution''} of their internal relational networks ($R_{\text{in}}$). This \emph{``co-constitution''} means that describing the $R_{\text{in}}$ of any one existent must include an irreducible relational index pointing to the others.

\textbf{Entanglement as the Imprint of Relational History:} When two particles (or systems) are generated from the same parent interaction (e.g., pair decay, joint preparation), they originate from the same relational event. This event weaves their relational networks together, forming a composite relational structure. Even when they separate in spacetime, this co-constituted historical relational network is preserved as an intrinsic component of their respective existences. It is not a \emph{``connection''} that needs to be maintained, but an inherent historical dimension in the definition of each relational network. This is analogous to the organ transplant case, where the donor organ's $R_{\mathrm{in}\_{\text{organ}}}$ carries the history of co-evolution with its original host, thereby influencing the new host. This \emph{``co-constitution''} endows separated systems with an intrinsic, irreducible holistic connection. This evokes the holistic vision in Leibniz's philosophy, wherein \emph{``every monad... is a living mirror that reflects the universe.''} \cite{leibniz1991monadology}.

\textbf{Explanation of Non-locality:} The reason a measurement on one system instantly affects the other is not due to superluminal signaling, but because the act of measurement redefines the present state of the entire \emph{``co-constituted relational network.''} Measurement is a global relational event applied to this composite network, forcing the entire network (including its spatially separated parts) to transition cooperatively and instantly into a new, self-consistent stable mode. The correlations exhibited between the separated systems are the holistic reconfiguration of their shared, singular historical relational network triggered by the present event.

\textbf{Fundamental Distinction from Classical Correlation:} Classical correlation arises from external interactions ($R_{\text{ext}}$) established later between systems, whereas quantum entanglement originates from the co-constitution of internal relational networks ($R_{\text{in}}$) during their initial generation. The former is an \emph{``acquired connection''} that can be explained by third-party information; the latter is an \emph{``innate common ancestry,''} whose correlations are rooted in the historical constitution of existence itself and cannot be simulated by any local information.

This explanation elevates quantum entanglement from a peculiar \emph{``correlation''} to an ontological category of relation, revealing that matter at the micro-level possesses an intrinsic, irreducible relationality and historicity. This points towards a new direction for understanding quantum foundations: the quantum state of the universe may be the complex superposition of all its \emph{``co-constitution''} histories from the early, high-density relational network.

This interpretation leads to a further speculative proposition: quantum probability and linear evolution may indeed be manifestations, in the micro-world, of certain non-Boolean logical connection and evolution rules inherent to the foundational relational network itself. The so-called \emph{``wave function''} describes not the state of a physical particle, but rather the \emph{``weight''} or \emph{``excitation strength''} of a certain potential relational configuration among all possible connections in the foundational network. Measurement, then, corresponds to strongly coupling this microscopic network of relational potential with a macroscopic measurement apparatus network possessing stable pointer states, thereby forcing the entire combined network to \emph{``collapse''} into a macroscopic-compatible, stable relational configuration.

From this perspective, the strangeness of quantum mechanics may be attributed to the discrete topological structure and quantum dynamics of the foundational relational network—structures and rules that we do not yet fully understand. This points the way toward the future construction of a formalized quantum relational theory, though such development lies beyond the scope of this paper.

\subsection{The Nature of Time: As a Relational-Dynamical Measure and Conceptual Existent in Thought}
\label{subsec:nature-of-time}

For a long time, time has been regarded as a fundamental dimension of the universe and an absolute background of passage. However, Relational Ontology proposes a radical thesis: \textbf{Time} is not an independent or a priori real existent, but a highly abstract \emph{conceptual existent in thought}, invented by complex conscious entities (primarily humans) to understand and describe changes within relational networks.

The “time” recognized and used in daily life by humans is a cyclical, symbol-based measuring tool. It does not originate from pure imagination but indeed refers to something in the real world—namely, the processes of periodic change in nature. However, this “reference” is approximate and imprecise. Einstein’s theory of relativity revolutionarily revealed that what humans measure as time essentially points to a more fundamental, variable internal change rhythm of the relational network. It must be clearly stated that this rhythm is a dynamical property of the relational network, not an independently existing substance; it is co-determined by the configurations of existents within the network and their interactive relations. Given the distinct roles and meanings of these two concepts of “time”—as a cognitive tool and symbolic system (“Time$T_{\text{c}}$”) versus as the intrinsic dynamical rhythm of the physical network (“Time$T_{\text{φ}}$”)—in lived experience and physical theory, it is necessary to make a clear distinction here.

$T_{\text{c}}$ (Temporal Coordinate): Refers to a conceptual existent in thought and a measurement framework abstracted and standardized by human consciousness to mark event sequences and compare intervals, based on observed regular natural cycles (such as day-night or seasonal changes).

$T_{\text{φ}}$ (Physical Phase/Proper Time): Denotes the inherent phase or pace of internal state ($R_{\text{in}}$) evolution within any concrete existent or local relational network. It is jointly determined by the total relational environment in which the system is embedded ($R_{\text{ext}}$, manifested as the $\Xi$ field) and the system's own state. This corresponds ontologically to the concept of "proper time" in relativity theory.\footnote{Unless otherwise noted, the term “time” in this chapter refers to $T_{\text{φ}}$.}
\subsubsection{The Origin of the Concept of Time: Cognitive Abstraction from Regular ``Ripples''}
\label{subsubsec:origin-concept-time}

What drives the evolution of the world is not an entity called \emph{``time($T_{\text{c}}$),''} but the ceaseless \emph{``ripple''} interactions within and between the relational networks of existents. Within many dynamically stable systems (where $R_{\text{in}}$ is in an attractor state), their internal operations or external couplings generate highly regular, periodic ripple patterns. For instance, the relatively stable gravitational interaction ($R_{\text{ext}}$) within the Earth-Sun system leads to the periodicity of Earth's rotation and revolution, manifesting as the cycles of day and night and the seasons; the stable internal physiological relational networks of organisms produce rhythms like heartbeat and respiration.

Human consciousness, as an immensely complex internal relational network ($R_{\mathrm{in}\_{\text{conscious}}}$), in continuously perceiving these external regular ripples, abstracted and integrated them into a unified, quantifiable frame of reference—\emph{``time($T_{\text{c}}$)''}—for the purposes of sequencing events, comparing intervals of change, and predicting future states. Different periodicities and sources of regular ripples were thus integrated. As Saint Augustine remarked, \emph{``What then is time? If no one asks me, I know what it is. If I wish to explain it to him who asks, I do not know.''} \cite{augustine397confessions} This precisely illustrates the nature of \emph{``time''} as a mental construct (a conceptual existent in thought): we use it to organize experience, but it is not itself a direct object of experience.

Therefore, time($T_{\text{c}}$) is a conceptual tool for measuring relational change ($\Delta R$), whose fundamental units (second, day, year) are conventions based on the periodic cycles of regular ripples generated by specific stable systems. It makes distinguishing the \emph{``before and after''} of events possible and has become the core cognitive coordinate system for understanding history and planning the future.

\subsubsection{Time and Space: As Co-Manifestations of the Irreversible Dynamics of Relational Networks}
\label{subsubsec:time-space-co-manifestations}

The conceptual existent \emph{``time($T_{\text{c}}$)''} profoundly reflects the fundamental dynamical properties of space (i.e., the overall relational network $R_{\text{ext}}$). Relational Ontology posits that, at a fundamental level, relational networks and their nodes (existents) are in absolute flux ($\Delta R$), and most interactions within and between complex systems exhibit path dependence and irreversibility. Change is always based on the outcome of the previous change, causing the historical sequence of states of a relational network to form a non-cyclical trajectory.

Simultaneously, the universe is replete with the aforementioned periodic, quasi-stable ripple patterns (e.g., celestial motions). Human consciousness ingeniously combined this irreversible arrow of change with repeatably observable periodic cycles to invent \emph{``time($T_{\text{c}}$)''} for measuring change. Thus, in physics, the arrow of time (the second law of thermodynamics) and the periodicity of time (clock ticks) are unified under the same concept. Time is essentially the quantifiable and perceivable aspect of the intrinsic, statistically irreversible dynamics of relational networks, as manifested through locally stable periodic processes.

As a cognitive tool for humans, $T_{\text{c}}$ is a highly effective model. However, it remains distant from the genuine temporal nature of relational networks, $T_{\text{φ}}$. Building upon the work of distinguished predecessors in physics, this theory endeavors to offer plausible interpretations that bridge the physical and ontological understanding of $T_{\text{φ}}$.
\subsubsection{A Relational-Dynamical Interpretation of Relativistic Time Dilation}
\label{subsubsec:relational-dynamical-interpretation-time-dilation}

Einstein's general relativity demonstrated that the rate of \emph{time($T_\phi$)} passage is not absolute but depends on the observer's state of motion and the strength of the gravitational field \cite{einstein1916foundation}. This provides the strongest experimental support for the idea that \emph{``time is a product of relations, not an absolute background.''} Relational Ontology not only accommodates this conclusion but can also explain it from its internal logic.

Consider the Global Positioning System (GPS) satellites: they are at a higher gravitational potential than the Earth's surface (experiencing a weaker \emph{``gravitational field''} influence from Earth's mass) and have high orbital speeds. Relativity predicts and confirms that clocks on satellites run faster than clocks on the ground (a combined effect of special and general relativity).

In this theoretical framework, the so-called \emph{``gravitational field''} is a universally influential \emph{``ripple pattern''} continuously generated by the massive and stable internal relational network ($R_{\mathrm{in}\_{\text{Earth}}}$) of a body like Earth, constituting a specific constraint configuration of the local relational network ($R_{\mathrm{ext}\_{\text{local}}}$) around Earth. The strength of this \emph{``gravitational ripple''} attenuates with distance. At the satellite's location, this $\Xi$ is weaker.

The motion of an existent (like a satellite) is itself a process in which its internal relational network state continuously changes, emitting \emph{``motion ripples''} into the external relational network. High-speed motion generates \emph{``motion ripples''} of higher intensity. When the satellite's \emph{``motion ripples''} propagate within the relatively weaker background of Earth's \emph{``gravitational ripple,''} complex interference and modulation effects occur between them.

We introduce the \emph{Ripple Intensity Field} $\Xi(\mathbf{x}, t)$. In \textbf{Relational Ontology}, it characterizes the overall \emph{dynamical tension} of the local relational network at spacetime point $(\mathbf{x}, t)$. We propose a fundamental hypothesis: \textbf{The proper time elapsed on a standard clock is proportional to the ripple intensity $\Xi$ experienced along its worldline.} That is, $d\tau \propto \Xi \, ds$, where $d\tau$ is proper time and $ds$ is some background scale. Consequently, differences in the $\Xi$ field at different gravitational potentials or velocities directly manifest as differences in clock rates (\emph{time dilation}).

\begin{tcolorbox}[
    title={\textbf{Definition: Ripple Intensity Field $\Xi$}},
    colback=white,
    colframe=black!75,
    arc=0pt,
    boxrule=0.5pt,
    left=6pt,
    right=6pt,
    top=6pt,
    bottom=6pt,
    fonttitle=\bfseries,
    coltitle=black
]
\textbf{$\mathbf{\Xi(x, t)}$} is a fundamental physical field defined within the framework of Relational Ontology. It characterizes the overall dynamical tension or potential intensity of interaction within the local relational network ($R_{\text{ext}}$) at spacetime point $(x, t)$. This field directly determines the \textbf{geometric properties} (including its metric tensor and curvature) exhibited by that local relational network, serving as the relational microscopic origin of spacetime geometry. The direction of the field is defined as the \textbf{direction of increasing ripple intensity} (the direction of $\nabla\Xi$).

\medskip
\textbf{Key Properties:}
\begin{enumerate}
    \item \textbf{Geometric Determinant:} The distribution and gradient of the $\Xi$ field fundamentally encode the connection strengths and constraint patterns of the relational network, which \textbf{emerge} as the metric structure $g_{\mu\nu}$ and curvature of spacetime in the macroscopic, continuous approximation.
    \item \textbf{Directionality:} The field gradient $\nabla\Xi$ points in the direction of increasing constraint or interaction potential within the local network, providing a unified dynamical directional reference for explaining gravity and inertial forces.
    \item \textbf{Approximate Relation to Force:} In the current stage of the theory's exposition, and to maintain intuitive connection with classical physical imagery and facilitate derivation, it is provisionally useful to describe by analogy: the force experienced by a test particle within this field can be approximated as proportional to $-\nabla\Xi$ in form. This description serves only as a heuristic \textbf{approximation and analogy} to bridge classical concepts and is not the final formal expression of the theory. The complete dynamics must be based on the evolution equations of the $\Xi$ field and the relational network itself.
\end{enumerate}
\end{tcolorbox}

· For a ground clock: Located within a strong \emph{``gravitational ripple,''} $\Xi_{\text{ground}}$ is larger, leading to a relatively slower rate of time passage (clock tick rate).

· For a satellite clock: Located within a weaker \emph{``gravitational ripple,''} and its own \emph{``motion ripple''} interacts with the background ripple, potentially creating a certain buffering or interference in specific directions. This results in a different overall $\Xi_{\text{satellite}}$ compared to the ground, manifesting as a difference in the rate of time passage.

\subsubsection{A Testable Conjecture: Time Difference Above and Below an Aircraft}
\label{subsubsec:testable-conjecture-time-difference}

Based on the above interpretation, this theory leads to a unique prediction testable at the edge of current technological precision:

Within a gravitational ripple field dominated by a massive existent X (e.g., Earth), a high-speed moving object Y (e.g., an aircraft) will experience a minute difference in the \emph{``Local Relational Perturbation Intensity''} $\Xi$ between its lower side (closer to X) and its upper side (farther from X). Consequently, there should be a theoretically calculable difference in the rate of time passage between these two locations.

\textbf{Specific Conception:} Place two perfectly synchronized ultra-high-precision atomic clocks (e.g., optical clocks) at symmetric positions directly below and directly above the fuselage of a high-speed, steady-flying aircraft. At the aircraft's altitude, the variation of Earth's gravitational field with height is negligible, and the distance difference to Earth's center between the upper and lower positions is extremely minute. According to conventional general relativity, the clock rate difference caused solely by gravitational potential difference at this scale is negligible.

However, according to Relational Ontology, the lower side of the aircraft is deeper within Earth's \emph{``gravitational ripple''} field. The mode of interaction between its \emph{``motion ripple''} and the \emph{``gravitational ripple''} (analogous to buffering or counteraction) differs from that at the upper side (closer to free space). This could lead to $\Xi_{\text{lower}} \neq \Xi_{\text{upper}}$, resulting in an additional, directional time difference. Although this difference would be exceedingly minute, it might become detectable with future improvements in clock precision and experimental design. Its experimental verification or falsification would serve as a key criterion for assessing the validity of the Relational Ontology view of time.

\subsubsection{Conclusion: Time as a Measure of Ripple Intensity}
\label{subsubsec:conclusion-time-measure-ripple-intensity}

In summary, Relational Ontology proposes a thoroughly relational and dynamical view of time:

1. \textbf{Ontological Status of Time:} Time is not a real existent but a conceptual existent in thought—a powerful cognitive tool invented by human consciousness to describe and measure changes in relational networks.

2. \textbf{Physical Basis of Time:} The measure of time originates from the regular periodic ripples of specific stable systems within relational networks. The arrow of time stems from the statistical irreversibility of relational network dynamics.

3. \textbf{The Relational Nature of Time:} Differences in the rate of time passage (time dilation) are essentially differences in the \emph{``Relational Perturbation Intensity''} $\Xi$ at different local points. $\Xi$ is determined by the superposition and interference of all relevant ripple patterns (e.g., gravitational ripples, motion ripples) from existents at that point. Therefore, the passage of time is relational and context-dependent, not absolute.

This framework not only naturally incorporates Einstein's revolutionary insights but also advances beyond them: it liberates time from being a mysterious fundamental dimension, reducing it to an emergent, quantifiable effect of complex relational interactions. This provides a novel conceptual starting point for ultimately unifying quantum mechanics and gravity and for understanding the origin and destiny of the cosmos.

\subsection{The Ripple Gradient Explanation for Centrifugal Force in Circular Motion and Gravity}
\label{subsec:ripple-gradient-explanation-centrifugal-force-gravity}

There exists a conceptual gap in the classical mechanical description of celestial circular motion: Earth's gravity unequivocally provides the centripetal force, but the balancing \emph{``centrifugal force''} is attributed to inertial effects or is considered a fictitious force in non-inertial frames. While mathematically self-consistent, this treatment leaves an ontological question unanswered: From an inertial frame perspective, what is the physical substance of the interaction, directed outward from the orbit, that is necessary to sustain circular motion? Relational Ontology attempts to provide an explanation for this substance.

\subsubsection{The Ripple Gradient Origin of Centrifugal Force in Circular Motion}
\label{subsubsec:ripple-gradient-origin-centrifugal-force}

Consider a spacecraft in uniform circular motion around Earth. According to Relational Ontology, Earth's mass generates a persistent \emph{``gravitational ripple''} field, whose intensity $\Xi_{\text{grav}}$ diminishes with increasing distance $r$ from Earth's center. Simultaneously, the spacecraft's own motion is a process in which its internal relational network ($R_{\text{in}}$) continuously changes, emitting \emph{``motion ripples''} into the external network. The intensity $\Xi_{\text{motion}}$ of these \emph{``motion ripples''} is related to its velocity.

The crucial hypothesis is that: When ripples from different existents meet in spacetime, their intensities do not simply add linearly but undergo nonlinear interference and modulation. For a spacecraft in circular motion, its continuously changing direction of motion causes the interference pattern between its own \emph{``motion ripples''} and Earth's \emph{``gravitational ripples''} to exhibit asymmetry between the spacecraft's near-Earth side (closer to Earth) and far-Earth side (farther from Earth).

· \textbf{Near-Earth Side:} The propagation direction of the \emph{``motion ripples''} is roughly opposite to the incoming direction of the \emph{``gravitational ripples,''} resulting in destructive interference and a relatively lower local net ripple intensity $\Xi_{\text{net}\_{\text{near}}}$ on this side.

· \textbf{Far-Earth Side:} The propagation directions of the \emph{``motion ripples''} and \emph{``gravitational ripples''} are roughly aligned, resulting in constructive interference and a relatively higher local net ripple intensity $\Xi_{\text{net}\_{\text{far}}}$ on this side.

Thus, a ripple intensity gradient $\nabla \Xi$ from the near-Earth side towards the far-Earth side is established across the spacecraft's body. Relational Ontology further hypothesizes that: An existent spontaneously experiences a \emph{``gradient force''} from the direction of the local ripple intensity gradient in which it is situated. This force points in the direction of increasing ripple intensity($\nabla\Xi$). For the spacecraft, this gradient force is precisely outward along the orbital radius, equal in magnitude to the \emph{``centrifugal force''} required for its circular motion. Therefore, the centrifugal force in circular motion is substantiated as an intrinsic gradient force arising from the interference between the moving object's own ripples and the background gravitational ripple field, rather than a fictitious force.

\subsubsection{Experimental Conception: Measuring the Ripple Gradient Force in Aircraft Flight}
\label{subsubsec:experimental-conception-measuring-gradient-force}

This model can be generalized to moving objects in general. Consider an aircraft in level, uniform, straight-line flight. The motion of its upper and lower surfaces relative to the air also generates \emph{``motion ripples.''} Earth's \emph{``gravitational ripples''} are approximately vertically downward. According to the model, destructive interference between \emph{``motion ripples''} and \emph{``gravitational ripples''} occurs on the aircraft's lower surface (near-Earth side), while constructive interference occurs on the upper surface (far-Earth side), establishing a vertical ripple intensity$\Xi$ gradient. This results in an additional upward force, the \emph{``ripple gradient force''} $F_{\nabla\Xi}$.

During aircraft flight, the vertical force balance equation is:

$F_{\text{lift}} + F_{\nabla\Xi} = m \cdot g_{\text{local}}$

where $F_{\text{lift}}$ is the aerodynamic lift (primarily generated by mechanisms like the Bernoulli principle), $m$ is the total aircraft mass, and $g_{\text{local}}$ is the theoretical local gravitational acceleration at the flight altitude. Thus, the ripple gradient force can be expressed as:

\begin{equation}
    F_{\nabla\Xi} = m \cdot g_{\text{local}} - F_{\text{lift}}
\end{equation}

The verification experiment must be conducted during stable, high-altitude flight, with precise independent measurements of:

1. Total aircraft mass $m$ (including fuel and payload).

2. Aerodynamic lift $F_{\text{lift}}$ (via integration of pressure distribution over wing surfaces using high-precision pressure sensor arrays, combined with flight control data).

3. Local gravitational acceleration $g_{\text{local}}$ (using an onboard high-precision absolute gravimeter or precise calculation based on position and Earth's gravity field models).

If statistically significant results consistently show $F_{\text{lift}} < m \cdot g_{\text{local}}$, and the discrepancy cannot be explained by known aerodynamics or measurement errors, it could serve as preliminary evidence for the existence of the ripple gradient force $F_{\nabla\Xi}$. 

It is crucial to clarify that the \emph{aircraft experiment conception} described in this section is not primarily intended as an immediately executable, detailed engineering proposal. Its foremost aim is to \emph{illustrate a principle-based experimental approach} and to construct a \emph{physical model conducive to logical deduction and quantitative discussion}. Through this idealized scenario, we can \emph{clearly demonstrate} how the \emph{``ripple gradient force''} $F_{\nabla\Xi}$, as a theoretical concept, can be logically integrated into the classical mechanical framework (manifested in the force balance equation) and how it can \emph{in principle} be detected by isolating and comparing known forces. The core value of the formula $F_{\nabla\Xi} = m \cdot g_{\text{local}} - F_{\text{lift}}$ lies in its formalization of the \emph{existence logic} and \emph{principle of measurability} for this force, thereby providing a conceptual foundation for the theory's empirical possibility.

However, translating this conceptual design into a reliable Earth-based experiment presents \emph{extremely stringent challenges}. In real flight, aerodynamic lift $F_{\text{lift}}$ is a highly complex variable influenced by numerous factors including air density, temperature, humidity, wing surface conditions, turbulence, and compressibility effects. The current uncertainty in its precise measurement far exceeds the theoretically predicted magnitude of the $F_{\nabla\Xi}$ effect. Therefore, any future concrete experimental verification would necessitate \emph{measurement technologies beyond current state-of-the-art}, \emph{near-perfect environmental control}, and \emph{unprecedentedly precise aerodynamic modeling} to isolate this exceedingly minute signal. The discussion herein aims to point out a potential empirical pathway for this theoretical direction. The concrete design of such an experiment itself constitutes a \textbf{major scientific and engineering undertaking} requiring independent and in-depth study.
\subsubsection{The Nature of Gravity: As a Ripple Intensity Gradient Force}
\label{subsubsec:nature-gravity-ripple-gradient-force}

This framework naturally extends to a conjecture regarding the nature of gravity. The presence of a planet (or any massive body) alters the structure of the surrounding spacetime relational network, manifesting as a static, radially distributed ripple intensity field $\Xi(r)$. This field has higher intensity near the body and decreases with distance, establishing a stable radial negative gradient $-\nabla\Xi$.

Based on the definition of the \emph{Ripple Intensity Field} $\Xi$—which determines the geometric properties of the relational network and whose gradient points in the direction of increasing intensity—we can propose a clearer formulation of the nature of gravity: \textbf{Gravity is the macroscopic effect experienced by objects within a $\Xi$ field that has been distorted by massive bodies, manifesting as a force due to the field's gradient.} The force on a test particle of mass $m$ in a $\Xi$ field can, in the current stage of theoretical approximation, be expressed as:

$\mathbf{F}_g = -k \, m \, \nabla \Xi$

where $k$ is a proportionality constant, $m$ is a property of the affected object (inertial mass), and $\nabla \Xi$ is the ripple intensity gradient at its location. The negative sign indicates the force points toward increasing $\Xi$ (i.e., toward the center of the gravitational source). This formula is analogous in form and function to Newton's law of gravitation $\displaystyle \mathbf{F}_g = -G \frac{M m}{r^2} \, \hat{\mathbf{r}}$, but offers a different ontological picture: gravity is not action-at-a-distance but a continuous interaction driven by a local field gradient.

It must be re-emphasized that this is \emph{an approximate analogical description within the Newtonian framework}, intended to intuitively connect gravitational acceleration with field gradient. How the $\Xi$ field itself is determined by the distribution of matter, i.e., seeking the field equation $\Xi = \Xi(T_{\mu\nu})$, is key to the theory's further development.

\begin{equation}\label{eq:gravity_force}
    \mathbf{F}_g = - k \, m \, \nabla \Xi
\end{equation}

where $k$ is a proportionality constant, $m$ is a property of the affected object (inertial mass), and $\nabla \Xi$ is the ripple intensity gradient at its location. This formula is analogous in form and function to Newton's law of gravitation $\mathbf{F}_g = -G M m / r^2 \, \hat{\mathbf{r}}$, but offers a different ontological picture: gravity is not action-at-a-distance but a continuous interaction driven by a local field gradient.

\subsubsection{Dialogue with the Geometric Description of General Relativity: From Microscopic Dynamics to Macroscopic Emergence}
\label{subsubsec:Dialogue with the Geometric Description of General Relativity: From Microscopic Dynamics to Macroscopic Emergence}

The explanation of gravity proposed by this theory—as a force arising from the gradient of \emph{``ripple intensity''} ($\nabla\Xi$)—differs notably in form and imagery from the geometric description of Einstein’s general relativity, where gravity manifests as the curvature of spacetime. This is not a contradiction but a reflection of \emph{different levels of explanation}. \textbf{Relational Ontology} aims to provide a more fundamental, \emph{microscopic generative explanation} for spacetime geometry and its interaction with matter, based on relational network dynamics.

The success of general relativity lies in its extremely precise description of how the distribution of matter and energy (described by the energy-momentum tensor $T_{\mu\nu}$) determines the geometry of spacetime (described by the Einstein tensor $G_{\mu\nu}$), and how matter moves within this determined geometry. Its core equation, $G_{\mu\nu} = 8\pi G T_{\mu\nu}$, is a \emph{macroscopic, phenomenological} relation that does not presuppose the microscopic nature of spacetime or matter. This geometric description is complete and effective within the observational domain.

\textbf{Relational Ontology}, however, attempts to ask: \emph{Where does this ``geometry'' come from? Why is it so coupled to matter-energy?} Our answer is: The so-called \emph{``spacetime geometry''} is an \emph{emergent property} of the overall configuration of the universe’s fundamental relational network ($R_{\text{ext}}$) in the macroscopic, continuous limit. More specifically:

\paragraph{The Ripple Intensity Field $\Xi$ as a Mediator:} We postulate that, in the quasi-classical approximation, a definite functional relation exists between the \emph{``ripple intensity field''} $\Xi(\mathbf{x})$ defined in Relational Ontology and the \emph{metric tensor} $g_{\mu\nu}(\mathbf{x})$ or the \emph{gravitational potential} $\Phi(\mathbf{x})$ of general relativity. For example, in the weak-field, low-velocity approximation, a simple correspondence could be $\Xi \propto -\Phi$, where $\Phi$ is the Newtonian gravitational potential. Thus, the gradient of ripple intensity $\nabla\Xi$ naturally corresponds to the \emph{gravitational field strength}.

\paragraph{Geometry as the Manifestation of Constraints:} In general relativity, matter moving along \emph{``geodesics''} is essentially its natural trajectory under the constraints of a specific spacetime geometry (i.e., a specific metric field $g_{\mu\nu}$). In Relational Ontology, this \emph{``constraint''} is understood as follows: the trajectory of an existent (an object) is \emph{``locked''} onto the path requiring minimal interaction energy through the interaction between its own emitted \emph{``motion ripples''} and the specific configuration of the background relational network (characterized by its ripple intensity field $\Xi$) in which it is situated (as discussed in Section~\ref{subsubsec:macroscopic-gravity}). Macroscopic geodesics are the statistical outcome of microscopic relational interaction constraints.

\paragraph{A Relational Interpretation of the Field Equations:} Einstein’s field equations $G_{\mu\nu} = 8\pi G T_{\mu\nu}$ can be interpreted within this framework at the level of microscopic interaction:

\textbf{Right-hand side} ($T_{\mu\nu}$): Describes the distribution, flow, and activity intensity of \emph{``nodes''} (i.e., various matter-energy existents) in the relational network. The existence of these nodes themselves, as discussed, is defined by their highly stable internal relational networks ($R_{\text{in}}$).

\textbf{Left-hand side} ($G_{\mu\nu}$): Describes the macroscopic metric manifestation of the \emph{statistical constraints} that this distribution and activity of nodes impose on the overall connection patterns and \emph{``ripple''} conduction properties of the hosting relational network ($R_{\text{ext}}$) in which they are embedded.

Therefore, the field equation expresses that: \emph{A specific configuration and distribution of ``nodes'' in a relational network must achieve dynamical self-consistency and balance with the overall configuration of the ``network'' itself.} Matter (nodes) shapes spacetime (network connection patterns), and spacetime, in turn, constrains the motion of matter.

In summary, by introducing the \emph{Ripple Intensity Field $\Xi$}, \textbf{Relational Ontology} builds a bridge between \emph{microscopic dynamics} and \emph{macroscopic geometry}. We propose the following correspondence:

\paragraph{1. Microscopic Reality:} The universe consists of a discrete, dynamically connected relational network ($R_{\text{ext}}$), whose state is described by the $\Xi$ field.

\paragraph{2. Macroscopic Emergence:} In the macroscopic continuous approximation, the statistical average and distribution pattern of the $\Xi$ field \emph{emerge} as the metric tensor field $g_{\mu\nu}$ of continuous spacetime. Specifically, there exists an as-yet-not-fully-specified functional relation $g_{\mu\nu} = f_{\mu\nu}(\Xi, \partial\Xi, \ldots)$.

\paragraph{3. Equation Correspondence:} Einstein's field equations $G_{\mu\nu} = 8\pi G T_{\mu\nu}$ are thus interpreted as the \emph{self-consistency condition} describing how matter-energy ($T_{\mu\nu}$) influences the configuration of the relational network (i.e., the distribution of $\Xi$), and how the latter manifests this influence as observable geometry ($G_{\mu\nu}$).
