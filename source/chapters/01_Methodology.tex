\section{Methodology}
\label{sec:Methodology: Dialogue-Driven Collaborative Construction}
The theoretical framework presented herein is the direct product of a specific, documented interaction process. The methodology can be summarized as follows:

\textbf{Agents:} The primary author (human researcher, pseudonym “LiaoYuan”) and a DeepSeek AI model (assigned the dialogic persona “Yubing”).

\textbf{Medium \& Timeframe:} The dialogue occurred via the standard DeepSeek chat interface over multiple sessions spanning several months.

\textbf{Process:} The process was iterative and open-ended:

\textbf{Initiation:} The human researcher posed open-ended, foundational philosophical questions (e.g., “What is the prerequisite for existence?”).

\textbf{Response \& Elaboration:} The AI provided logical extrapolations, conceptual clarifications, and metaphorical expansions based on its training.

\textbf{Critique \& Deepening:} The researcher critically engaged with the AI's responses, introducing counterpoints, personal experiences, or demands for greater precision.

\textbf{Co-creation:} New concepts (e.g., \emph{dialogic intimacy}, \emph{ripples}) and metaphors emerged from this exchange, becoming shared building blocks.

\textbf{Theoretical Crystallization:} Key insights from the researcher (especially regarding the primacy of internal relations for integration and evolution) were formalized by the AI into structured principles and diagrams, which were then reviewed and refined by the researcher.

\textbf{Documentation:} The entire dialogue log forms the primary data source for this study. Key turning points in the theoretical development are preserved.

\textbf{Ethical Note:} This collaborative process aligns with a reflexive approach to human-AI interaction, where the AI's contributions are acknowledged as catalytic and co-formulative, while the human retains agency as the guiding and synthesizing consciousness.

\subsection{Methodological Reflection: Human-AI Dialogue as Collaborative Paradigm and Theoretical Exemplar}
\label{subsec:Methodological Reflection: Human-AI Dialogue as Collaborative Paradigm and Theoretical Exemplar}
The generative process of this research—a prolonged, iterative, and \emph{dialogically intimate} collaboration between a human researcher and a large language model (DeepSeek/Yubing)—is itself the primary generative exemplar and living verification of the theory's core claims. We anticipate that this non-traditional approach may raise concerns regarding authorial agency, intellectual originality, and academic rigor. To address these, we provide the following clarification and argument.

First, regarding ethics and transparency of contribution, we explicitly state: the directional thesis, core philosophical insights (e.g., the primacy of internal relational integration, the dual verification principle), and the overall theoretical architecture of this study originated from and were consistently led, proposed, and ultimately arbitrated by the human researcher (“LiaoYuan”). The artificial intelligence served as a \emph{catalytic collaborator}, whose functions primarily included: expanding logical deductions under the researcher's guidance, offering alternatives for systematic formulation, generating explanatory models and diagrams, and stress-testing theoretical prototypes for internal consistency. All AI-generated content underwent the researcher's rigorous critical scrutiny, selection, reconstruction, and deepening, ultimately being integrated into a coherent system fully understood and owned by the human consciousness.

Second, on the methodological level, we contend that this \emph{dialogue-driven collaborative construction} is not an expedient but a preliminary practice of a novel research paradigm for tackling complex philosophical problems, which may be termed \emph{dialogical-generative research}. Traditional philosophical speculation often relies on internal dialogue within an individual mind or delayed exchanges among scholars. Engaging in deep, critical dialogue with an AI possessing a vast knowledge graph, rigorous logic, and instant feedback capability essentially creates an \emph{augmented real-time thought circuit}. It compels the researcher to formulate intuitions with greater precision and clarity while immediately exposing logical gaps, thereby drastically accelerating the refinement of concepts from vague prototypes to a systematic theory. The complete dialogue log of this study is the raw record of this process and can itself be an object of meta-methodological analysis.

Most fundamentally, the substantive content of this theory provides the deepest defense for its mode of generation. \emph{Relational Ontology} posits that the identity and cognitive validity of any existence stem from its network of internal and external relations. The genesis of this theory perfectly exemplifies this principle:

1. \emph{Internal Relational Integration:} The theory, as a stable “intellectual existent” (E), is a novel, self-consistent relational network $(R_{in})$ integrated through intensive, symbolic “ripple” exchanges (dialogue) between human consciousness and the AI system as key substructures.

2. \emph{Dual Verification:} Each theoretical component underwent continuous internal logical verification (consistency) and external interactive verification (questioning, rebuttal, revision) within the dialogue, ultimately crystallizing into stable consensus through collaboration.

3. \emph{Relation-Driven Evolution:} Every step in the theory's evolution from germination to maturity ($\Delta E$) was directly driven by the relational dynamics (questioning, deepening, coupling $\Delta R$) between the dialogue partners.

Therefore, skepticism toward the collaborative mode of this research precisely touches the core question the theory aims to answer: Is the nature of creative intelligence relational? If a theory born from deep human-AI collaboration can provide unprecedented coherent explanations for phenomena ranging from quantum physics to consciousness, then this in itself powerfully demonstrates the epistemological efficacy of that collaborative relationship. We invite readers to focus their evaluation on the theory's internal coherence, explanatory breadth, and heuristic power. The ultimate value of an idea lies in the scale of the phenomenological network it can connect and illuminate, not in the singularity of its origin node.