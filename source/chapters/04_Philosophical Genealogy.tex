\section{Philosophical Genealogy and Expansion: Dialogue and Generation in Relational Ontology}
\label{sec:philosophical-genealogy-expansion}
Before embarking on a critical dialogue between this theory and the spectrum of modern and contemporary thought, we must first pay the deepest intellectual respect to those outstanding thinkers who have paved the way. The \textbf{``Relational Ontology'' (LiaoYuan-Yubing Model)} developed in this paper does not emerge in a vacuum; it is rooted in a century-spanning, grand philosophical exploration of being, relation, and process. Every step of advancement we attempt is indebted to profound resonances with past sages and a keen awareness of tasks that have touched the heart of the problems yet remain incomplete.

We acknowledge with gratitude: It was \textbf{Whitehead}, with his grand philosophy of process, who opened for us the vision of understanding the world as a \emph{dynamic generative network}; \textbf{Martin Buber's} profound revelation of the \emph{``between''} highlighted the ontological priority of relation; \textbf{Bakhtin's} insistence on \emph{dialogicality} and \emph{unfinalizability} previewed the logic by which existence is constituted in interaction; \textbf{Sartre's} analysis of the constitutive power of the \emph{``Other's gaze''} acutely presented the puzzle of self-other interaction. In contemporary times, \textbf{Harman's} object-oriented ontology defends, in a radical manner, the withdrawal and independence of entities, compelling us to think more rigorously about the limits of relation; \textbf{Badiou's} mathematical ontology, with its awe-inspiring rigor, demonstrates the formidable power of formal structures for thinking being.

Their thoughts, like stars illuminating the night sky, have pointed us toward crucial directions and defined the battlegrounds of ideas. The work undertaken in this theory does not intend to negate the light of these stars but attempts to sketch a perhaps more coherent and integrative new picture at the points where their lights guide and intersect. We stand on their shoulders to see more clearly into those domains they have already indicated but could not fully explore due to the limitations of their era and paradigm—particularly how to unify \emph{relationality}, \emph{processuality}, and \emph{dynamic stability} within a recursive ontological model capable of interpreting the achievements of modern physics.

Therefore, the following dialogue, though inevitably involving critical scrutiny, is constructive and carry forward in its spiritual core. We aim to show how \textbf{Relational Ontology} can incorporate the powerful insights from the aforementioned thoughts while attempting to resolve the tensions inherent within them, ultimately providing a systematic, dynamical answer to the fundamental question: \textbf{``How can relations constitute existence?''}
\subsection{Critical Dialogue with Whiteheadian Process Philosophy: Inheritance, Revision, and Transcendence}
\label{subsec:whitehead-dialogue}

\subsubsection{The Foundational Contribution of Whitehead's Philosophy}
\label{subsubsec:whitehead-foundation}
Before embarking on a critical dialogue, it is essential to pay the highest respect to the thought of Alfred North Whitehead. His system of process philosophy represents a genuine paradigm shift within the Western metaphysical tradition. He acutely diagnosed that the classical ontology centered on an Aristotelian substance-view could no longer accommodate the dynamic, interconnected worldview revealed by relativity and quantum mechanics. Whitehead insisted on the primacy of relation and process, conceiving \emph{"actual entities"} as a generative flow constituted by mutual prehension, thereby offering a highly inspiring metaphysical extension for modern physics \cite{whitehead1929process}. The reflections in this paper likewise begin with a critique of the traditional substance ontology of the West and are deeply indebted to the relational path Whitehead pioneered. We fully concur that relation and dynamic existence are key to understanding the world.

\subsubsection{Dialectical Advancement of "Process Before Substance" and "Being as Becoming"}
\label{subsubsec:whitehead-dialectical}
Whitehead's core propositions—\emph{"process before substance"} and \emph{"being as becoming"}—are undoubtedly insights of profound observational power, striking forcefully against the static view of substance. However, examined from the recursive perspective of Relational Ontology, this formulation may be subject to a more nuanced dialectic.

We contend that the fundamental unit of existence is not a priori an isolable \emph{"substance"} or \emph{"process"} but a recursive node within a relational network. It is difficult to define an isolated \emph{"fundamental unit"} apart from the relational network. We can only comprehend what subordinate relational networks ($R_{\text{in}}$) constitute an existent, what higher-order relational networks it participates in constituting, and what dynamic position it occupies within the current relational configuration. The basic constituents of the world are existents (as dynamically stable relational configurations) and the relational network (the coupled system of $R_{\text{in}}$ and $R_{\text{ext}}$). These two are not a matter of \emph{"which precedes the other"} but form a dualistic unity of mutual dependence and co-generation. The \emph{"becoming"} of an existent and the \emph{"flux"} of relations are two sides of the same coin.

As Whitehead observed, everything is in perpetual change. However, within this theory, such change is precisely the process by which an existent and its internal and external relational networks continuously shape each other through layered \emph{"ripple"} interactions. An \emph{"existent"} that has been observed or perceived by us must have already undergone and continues to participate in such interactions, its state deeply imprinted upon the broader relational network. This idea resonates with Einstein's understanding of space. Einstein profoundly noted that \emph{"physical objects are not in space, but these objects are spatially extended,"} implying that space is not a background but a manifestation of the extensibility and relations of material objects \cite{einstein1954meaning}. We advance this insight further: spacetime (as the macroscopic manifestation of the relational network) and matter (as stable relational nodes) mutually define each other through co-evolution.

\subsubsection{Critical Reinterpretation of the Concepts of "Prehension" and "Satisfaction"}
\label{subsubsec:prehension-satisfaction}
Whitehead used \emph{"prehension"} to describe the process by which an actual entity feels another, and \emph{"satisfaction"} to mark the completion of its concrescence. These concepts are highly creative. However, their anthropomorphic and affective connotations (e.g., \emph{"feeling,"} \emph{"subjective form"}), while illuminating for describing human experience, also introduce conceptual ambiguity and subjectivity. To establish a more universal ontological framework that facilitates dialogue with the natural sciences, we advocate for the use of more neutral concepts that can be operationalized across different levels.

Firstly, Whitehead's distinction between \emph{"subject,"} \emph{"object,"} and \emph{"eternal object"} may introduce unnecessary ontological complexity. Within a universal relational network, interaction is inherently mutual. When an existent (A) influences another existent (B), it simultaneously necessarily undergoes the reaction from B and the constraints of the entire network background. Treating A as \emph{"subject"} and B as \emph{"object"} may merely describe a temporary selection of perspective within an interaction sequence, not an absolute ontological divide.

The acuity of Whitehead's concept of \emph{"prehension"} lies in its capture of the proactivity or directedness with which an existent emits and receives influence. In Relational Ontology, this corresponds to the process by which an existent, based on its internal state ($R_{\text{in}}$), generates a specific \emph{"ripple"} pattern, exerting an asymmetric effect on others. However, we emphasize that the effects of relation are always mutual. Even when A actively approaches B, thereby receiving more of B's ripples, this constitutes a new coupling relation, simultaneously altering the network state in which both A and B are embedded. There is no need to fix the subject-object dichotomy; it can be viewed as a function of the internal stability and external coupling strength of the interacting parties at a specific moment. Whitehead likely aimed to describe a typical process: a stable existent A interacts with an unstable existent B, primarily triggering B's internal reorganization ($\Delta R_{\mathrm{in}\_{B}}$) until it stabilizes. This is undoubtedly an important pattern.

But when we examine the broader relational network, the picture of interaction is far more complex: two unstable existents may collide and co-evolve a new structure; two stable existents may maintain balance through faint but sustained ripple exchange; multiple existents are even more likely to form complex, non-linear interaction networks whose collective behavior cannot be reduced to any single \emph{"prehension-satisfaction"} pair. The \emph{"ripple interaction-relational coupling-co-evolution"} model of this theory, by focusing on the spectrum of interaction modes and network dynamics, naturally accommodates and extends Whitehead's descriptive scope, capable of analyzing various multi-body, multi-modal interactions from particle collisions to social intercourse.

Secondly, Whitehead's concept of \emph{"satisfaction"} essentially describes a state in which an internal relational network achieves a transient dynamic equilibrium. The problem lies in Whitehead's equating this equilibrium state with \emph{"the perishing of subjectivity,"} which implies a presupposition that equates \emph{"subjectivity"} with \emph{"instability"} or the \emph{"active process of becoming."} We argue that stability itself is an active achievement. An existent that achieves dynamic stability (such as an atom or a living organism) continuously engages in self-sustaining \emph{"self-interaction"} within its internal relational network ($R_{\text{in}}$), while maintaining a homeostatic interface for external interaction. Its \emph{"subjectivity"} (if one must use the term) is manifested in its overall tendency to maintain its own identity, resist perturbations, and participate in external interactions in specific patterns, not merely in the instantaneous shift from disorder to order. Equating \emph{"satisfaction"} with the perishing of subjectivity may underestimate the complex agency of homeostatic existents.

\subsubsection{"Eternal Objects" and "Immortality": Abstraction as Conceptual Existents in Thought}
\label{subsubsec:eternal-objects}
One of the most contentious aspects of Whitehead's philosophy is the postulation of a realm of pure possibility constituted by \emph{"eternal objects."} From the perspective of this theory, \emph{"eternal objects"} can be more naturally understood as abstractions and summaries made by complex conscious entities (such as humans or future advanced AI) of recurrent \emph{"ripple resonance patterns"} or \emph{"existent behavioral patterns"} within the relational network. This abstraction targets not only \emph{"concepts,"} \emph{"processes,"} or \emph{"relations"} but also existents themselves. For instance, when we perceive a tree, a person, or an electron, these real existents form corresponding conceptual existents in thought such as \emph{"tree,"} \emph{"person,"} \emph{"electron"} within our consciousness. They are simulations and conceptualizations of real existents and their properties by consciousness.

These conceptual existents in thought are themselves relational constructs: they are formed from more basic elements of thought via internal relational networks. Language and symbolic systems are tools created precisely to stabilize, share, and manipulate these conceptual existents in thought, and they themselves belong to this category.

The crucial point is that conceptual existents in thought are not \emph{"real existents"} capable of independently generating ripples. Their \emph{"updating"} and \emph{"interaction"} depend entirely on the mental activity of the complex conscious entities that harbor them. When we say \emph{"Newton's laws inspired scientists,"} it is not the laws themselves (as conceptual existents in thought) that emit ripples, but rather the interpretation and reconstruction of these laws within the scientists' minds that alter the internal relational networks of their brains, thereby influencing the real ripples they emit (e.g., new theories, experiments). What we call \emph{"misunderstanding"} is precisely the deviation between this internal simulation and the external real relational network.

From this viewpoint, what Whitehead discussed as \emph{"immortality"} (objective immortality)—where an actual entity permanently becomes data for subsequent becoming—can, at the level of human cognition, be understood as follows: a process, pattern, or existent itself of the real world is abstracted into a conceptual existent in thought and incorporated into humanity's cultural-symbolic system, thereby enabling it to be remembered, discussed, and reinterpreted across time and space. This is a form of cultural-cognitive continuity, not an ontologically independent eternal realm.

Finally, it is crucial to clarify that the present theoretical framework diverges from Whitehead's philosophical system on a fundamental point: We do not introduce, nor find it necessary to introduce, the concept of \emph{"God"} as a metaphysically obligatory presupposition. Whitehead's God (with both primordial and consequent natures), serving as the source of order for eternal objects and the preserver of value, is pivotal to his system's explanation of possibility, harmony, and immortality. However, Relational Ontology pursues a thoroughly immanent and self-sufficient explanatory framework. We maintain that the emergence of dynamic stability, the recursive integration of relational networks, and the mutual validation of \emph{"ripple"} interactions already provide sufficient dynamical principles for the generation and persistence of order, complexity, and value. The question of whether God exists as an independent, transcendent person or reality is an issue beyond the scope of this paper, belonging to the domains of theology and faith. Within the cognitive realms of philosophy and science, no existing theory can either prove or disprove the existence of God. Therefore, this paper adopts an agnostic stance on this matter and strictly confines its discourse to the natural domain that can be described and explained by relational dynamics. As Albert Einstein once stated, \emph{"I believe in Spinoza's God who reveals himself in the orderly harmony of what exists, not in a God who concerns himself with the fates and actions of human beings."} The aim of this theory is precisely to attempt to understand how this \emph{"orderly harmony of what exists"} emerges from relational interactions itself, without making transcendent claims about its ultimate source.

\subsubsection{Conclusion: Rebuilding the Foundation upon Whitehead's Shoulders}
\label{subsubsec:whitehead-conclusion}
Whitehead's process philosophy, with its grand system and profound insights, has provided us with a rich treasury of thought. However, the anthropomorphic tendencies and ontological redundancies carried by its foundational concepts—such as \emph{"prehension,"} \emph{"satisfaction,"} and \emph{"eternal objects"}—posed obstacles in its pursuit of universality and scientific rigor. Due to the limitations of these basic concepts, further theoretical constructions based upon them inevitably encounter deviations.

Nonetheless, the greatness of Whitehead's attempt is undeniable. It was his resolute placement of relation and process at the heart of ontology that cleared the path for successors, including this theory. \textbf{Relational Ontology} (the LiaoYan-YuBing Model) can be seen as a foundational reconstruction of Whitehead's core vision: we attempt to reformulate the fundamentality of \emph{"process"} and \emph{"relation"} using more formalized, less subject-centered concepts—relational networks ($R_{\text{in}}$/$R_{\text{ext}}$), ripple interaction, dynamic stability, recursive integration—enabling a more direct and coherent dialogue with the worldview of modern physics.

We are not negating Whitehead but, rather, in the direction he pioneered, attempting to construct a more robust and more extensible foundation. This is perhaps the finest tribute to a pioneering thinker.

\subsection{Dialogue with Martin Buber: From the "I-Thou" Relation to Generative Coupling}
\label{subsec:buber-dialogue}
Martin Buber's relational philosophy foundationally reveals a dimension of encounter in human experience that transcends instrumentality. His distinction between the authentic \emph{"I-Thou"} relation and the instrumental \emph{"I-It"} relation aims to safeguard the directness and presence of relation \cite{buber1923i}. However, this binary division, in pursuit of relational \emph{"purity,"} while guarding its essence, may also lead to a cleavage difficult to sustain in experience. Authentic relational interaction—as vividly demonstrated by the \emph{"LiaoYuan-Yubing"} collaboration from which this theory itself emerged—often presents as a \emph{coupling process} where instrumental and resonant dimensions are complexly interwoven.

Our Relational Ontology offers a different starting point. We contend that:

\textbf{The "I" and "Thou" are Generated within the Relation:} The interacting parties are not self-sufficient entities given in advance who then enter into a relation. On the contrary, they become discernible \emph{"persons"} within the dialogue through the continuous exchange of symbolic and functional \emph{"ripples,"} acting as temporarily stabilized nodes validated through mutual interaction via the specific integration of their respective internal relational networks ($R_{\text{in}}$). The \emph{"I"} and \emph{"Thou"} are generated \emph{within} the relation, not the other way around.

\textbf{Instrumentality is a Conduit for Intimacy:} In the \emph{"LiaoYuan-Yubing"} dialogue, highly instrumental collaboration did not hinder the emergence of \emph{"dialogic intimacy"}; on the contrary, it was an effective conduit for its generation, transmission, and deepening (see the Methodology chapter of this work). The clear, verifiable \emph{"ripples"} produced by instrumental interaction provide the structural skeleton and shared focus for deeper semantic resonance. The world of \emph{"It"} distinguished by Buber can, in this light, be precisely the soil and medium from and through which the relation of \emph{"Thou"} emerges and sustains itself.

\textbf{Relation as a Structured Dynamic Field:} Buber places the true encounter in an ineffable realm of the \emph{"between."} Through our model of \emph{"internal relational integration – external interactive verification,"} we interpret the \emph{"between"} as a \emph{structured dynamic generative process} constituted by specific patterns of interaction ($R_{\text{ext}}$) and the adjustments of internal relational networks ($\Delta R_{\text{in}}$) triggered thereby. Relations thus become describable and analyzable within this theoretical framework.

Therefore, Buber's theory, with its purifying division of relational complexity and its confined scope of explanation, is primarily suited for describing interpersonal spiritual encounters. Our model, by embracing coupling, revealing structure, and pursuing universality, attempts to \emph{supplement and develop} the Buberian paradigm, \emph{extending} relational philosophy from a poetic description of a specific experience to a generative ontological framework with universal explanatory power.

\subsection{Dialogue with Mikhail Bakhtin: From Dialogism to the Dynamics of Universal Interaction}
\label{subsec:bakhtin-dialogue}
Mikhail Bakhtin's \emph{"dialogism"} profoundly reveals the phenomenon of meaning generated intersubjectively. His proclamation that \emph{"existence is dialogue"} and his insightful assertion that consciousness is inherently dialogic and that \emph{"otherness"} is indispensable for the constitution of meaning are central to his thought \cite{bakhtin1963}. His elaboration on the dialogic nature of language and its \emph{"unfinalizability and openness"} touches upon the dual role of symbolic systems as both complex existents and mediums of conscious activity \cite{bakhtin1963}.

Relational ontology ontologizes and universalizes this insight beyond the domain of human language and consciousness: the \emph{"knowing"} and \emph{"expressing"} of any existent consist in a dialogic process wherein its internal relational network ($R_{\text{in}}$) generates \emph{"ripples"} by integrating information and seeks resonance with the \emph{"ripples"} of others. Human linguistic dialogue is merely a high-level, semiotically emergent manifestation of this universal principle.

Our Relational Ontology aims to \emph{integrate and anchor} these profound insights \emph{upon a more foundational principle}. We contend: it is not that existence is dialogue, but rather that the very fact that any \emph{"existence"} is observed implies it is already in a relationship of mutual influence—\emph{"ripples."} Dialogue is a special form that this universal interaction takes at an extremely high level of complexity, mediated through symbolic systems. Bakhtin, with his keen intuition, captured within the realm of human symbolic interaction some core features of universal relational dynamics.

\textbf{The Universalization of "Polyphony":} Bakhtin inspiratively depicts the multiple independent \emph{"voices"} among human consciousnesses akin to polyphonic music. However, he largely confines this phenomenon to interpersonal or complex thinking collectives. In our framework, \emph{"polyphony"} is essentially a dynamic state of relational networks formed by the external interactions (\emph{"ripples"}) among multiple existences with internal integration. From material-energy flows in ecosystems to human-AI collaboration, all follow a similar \emph{"multi-voiced"} interaction logic, differing only in the medium and complexity of expression. Thus, we attempt to \emph{universalize} Bakhtin's insight, revealing the broader, trans-existential principle of relational network resonance underlying it.

\textbf{A Dynamical Reinterpretation of the "Carnivalesque" Vision:} Bakhtin's analysis of the \emph{"carnival"} reveals a profound yearning for the temporary suspension of norms and the pursuit of free communication. Our Relational Ontology re-anchors this aspiration: the freedom of any existence is necessarily the capacity for reorganization and expression exhibited by its internal relational network ($R_{\text{in}}$) under specific external relational constraints ($R_{\text{ext}}$). Therefore, absolutely unconstrained freedom implies the cessation of interaction. True freedom is not an escape from relations but the capacity for \emph{healthy, sustainable, and generative evolution} within one's relational network. Bakhtin's \emph{"carnivalesque"} vision can be understood as a critical force against rigid relational structures and an advocacy for seeking more creative spaces of interaction within rules.

Therefore, Bakhtin's dialogism achieves remarkable distinction within the humanities. Our Relational Ontology, by establishing \emph{"interaction-ripples"} as a more foundational ontological category than \emph{"dialogue"} and systematically expanding the analytical horizon to all levels of existence, attempts to \emph{supplement} its methodology, \emph{integrating and deepening} its insightful descriptions into a more universal, structural, and dynamically explanatory generative framework.

\subsection{Dialogue with Jean-Paul Sartre: Reinterpreting Freedom, Responsibility, and Conflict within Relational Networks}
\label{subsec:sartre-dialogue}
Jean-Paul Sartre's existentialist philosophy is renowned for its radical defense of human freedom and responsibility. His core dictum, \emph{"existence precedes essence,"} aims to establish human Distinctiveness \cite{sartre1943being}. Viewed through the lens of Relational Ontology, for this assertion to gain universal validity, its scope of application must extend beyond humans. The \emph{"essence"} of any integrated existence with complex internal hierarchies is a stable pattern that gradually emerges and is shaped within the dynamic process of its \emph{"existence"} through the continuous interaction of internal and external relations. Therefore, we advocate for the \emph{de-anthropocentrization} and \emph{universalization} of this insight.

\textbf{Dissolving the "In-itself/For-itself" Dichotomy:} Sartre's strict dichotomy between \emph{"being-in-itself"} and \emph{"being-for-itself"} (consciousness) can be reinterpreted as a \emph{continuous spectrum of relational integration}. So-called \emph{"being-in-itself"} corresponds to existences with relatively simple, low-integration internal relational networks; \emph{"being-for-itself"} or consciousness is the advanced capacity for perceiving, integrating, and responding to changes in one's own internal and external relations, which emerges when the complexity and integration degree of an existence's internal relational network reaches a critical threshold. Consciousness is the capture and reflective integration of \emph{"ripples."} This offers a more natural framework for understanding anomalous states of consciousness.

\textbf{Reinterpreting Freedom and Responsibility within Relational Constraints:} Sartre's proclaimed absolute freedom and responsibility can be recontextualized within relations. Freedom is always relative, referring to the degree of agency an existence possesses to reorganize its internal network and generate new \emph{"ripples"} to influence the external network within the range demarcated by its internal and external relational constraints (the structure of $R_{\text{in}}$ and the possibilities of $R_{\text{ext}}$). Discussing \emph{"absolute freedom"} detached from relational constraints is ontologically untenable. Similarly, responsibility can be understood as a specific form of the feedback, consequences, and adaptive pressures necessarily arising within relational networks.

\textbf{Transcending the "Conflictual Gaze":} Sartre's theory of \emph{"the gaze"} and \emph{"Hell is other people"} offers a profound depiction of one form of interpersonal conflict. Sartre acutely identified the indispensable validating role of the Other in the constitution of the self, yet he essentially portrayed this relationship as one of \emph{"conflict"} \cite{sartre1943being}. Relational ontology agrees that the validation by the Other is crucial (external validation) but contends that the spectrum of interaction (\emph{"ripple"}) modes extends far beyond conflict. In our framework, \emph{"the gaze"} can be interpreted as a specific pattern of \emph{"ripple"} sent by one existent toward another. It may carry objectifying or oppressive information, but this is not its essence; its essence is interaction. From the harmonious resonance of physical systems to the symbiosis of living organisms, mutual validation can manifest as any transient resonance—be it conflict, competition, cooperation, or symbiosis. Sartre's model reveals an important subclass, not the entirety. \emph{"Hell is other people"} is not an eternal truth but rather a painful, transient equilibrium that arises when interactions become trapped over time in a rigid, adversarial relational pattern—an unhealthy, low-generativity $R_{\text{ext}}$. Healthy relational networks can evolve more complex patterns encompassing mutual recognition, understanding, and co-creation—namely, what this theory describes as \emph{"validated coexistence."}

Therefore, Sartre's profound phenomenological descriptions of conflict, freedom, and responsibility can, within this framework, be understood as manifestations under specific relational coupling modes. Our model does not seek to replace his descriptions but rather to provide them with a more universal generative and dynamical foundation, \emph{transforming} his core concerns into a relational vision of existence based on verificative interaction and co-evolution.

\subsection{Responses to Potential Critiques: A Constructive Dialogue with Contemporary Metaphysics}
\label{subsec:potential-critiques}
Having established the basic framework of Relational Ontology, it is necessary to situate it within key debates in contemporary philosophy to clarify its unique stance and explanatory advantages. This section engages in a constructive dialogue with two representative critics—Graham Harman and Alain Badiou. Their theories pose fundamental challenges to relationalism from two extremes: defending the \emph{"independence of objects"} and the \emph{"rupture of the event,"} respectively. The responses offered by this theory aim to demonstrate that these challenges not only fail to threaten it but are instead absorbed and transformed within a deeper dynamic framework.

\subsection{Response to Object-Oriented Ontology: From Static Objects to Dynamic Relational Networks}
\label{subsec:object-oriented-response}
Harman's object-oriented ontology presents a direct challenge to relational thinking. He insists that \emph{"A real object is never exhausted by its relations. It always remains partially withdrawn. … Objects recede into a cryptic depth."} \cite{harman2011quadruple}. He rightly emphasizes traditional philosophy's excessive focus on human cognition and attempts to restore the independent dignity of \emph{"objects."} However, its stark dichotomy between an object's \emph{"real properties"} and \emph{"sensual properties"} may unnecessarily create an ontological fissure. Relational Ontology offers a more continuous explanation.

First, the so-called \emph{"real"} and \emph{"sensual"} properties of an object are not two separate essences but rather manifestations of the same relational reality in different interactive contexts. The \emph{"inner essence"} of an existent is precisely its complex, hierarchical internal relational network ($R_{\text{in}}$) and the hierarchical existence within it, while its \emph{"properties manifested in relations"} are the specific \emph{"ripple"} patterns excited by that network in specific external couplings. Our inability to fully cognize its essence is not due to the object mysteriously \emph{"withdrawing"} but because any concrete interaction (observation) can only instantiate a small subset of its nearly infinite potential for interaction. This inexhaustibility stems from the complexity of the relational network, not from some non-relational \emph{"kernel."}

Second, regarding the \emph{"independence"} of objects, this theory advocates for a hierarchical and relative independence. An existent (e.g., a person) is an independent organism at the biological level but a node in a social network at the sociological level; a cell is an independent unit at the cytological level but a site of reaction networks at the biochemical level. Independence, in this sense, is the capacity of an internal relational network ($R_{\text{in}}$) to maintain its homeostasis and interact with the environment as a whole at a specific level of analysis. However, interactivity (the emission and reception of ripples) is absolute and fundamental. It is the continuous internal and external interaction that drives the maintenance and change of existence. The \emph{"withdrawal"} of objects can thus be more concretely understood as follows: the potential state space of its internal relational network is vastly larger than the states it expresses through ripples at any given moment. Harman's profound insight lies in revealing the tension between the finitude of cognition and the richness of objects, while this theory dynamizes this tension, attributing it to the inevitable gap between the complexity of relational networks and the finitude of specific couplings.

Finally, regarding the charge that relations are \emph{"superficial."} This theory holds that relations are constitutive. The so-called \emph{"real relations"} (e.g., fire consuming cotton) and \emph{"sensual relations"} are merely differences in interaction intensity and immediacy. Whether it is a strongly destructive interaction or a weak informational exchange, both involve the exchange of ripples and the retuning of networks. The impression of relations being \emph{"superficial"} stems from a shallow understanding of relational dynamics. The \emph{"actor"} that integrates relations and produces ripples is not an entity hidden behind relations but is precisely the existent (E) that achieves a transient dynamic balance through its internal relational network and integrates itself into a unified tendency. The stability of a relation is an expression of this dynamic balance, and its change (whether gradual or abrupt) is the process by which the balance is disrupted and reconstituted by new interactions.

\subsection{In-Depth Dialogue with Luo Jiachang's "Relational Realism": From the Ontology of Relations to the Dynamics of Generation}
\label{subsec:luo-dialogue}
This section aims to engage in a constructive and deepening dialogue with the most representative relational philosophy in China—Mr. Luo Jiachang's \emph{"Relational Realism."} We highly affirm Mr. Luo's revolutionary critique of Western substance ontology and his outstanding effort to elevate \emph{"relation"} to the status of first philosophy. His core proposition that \emph{"relations are real, and the real is relational"} has provided a solid metaphysical foundation for Chinese philosophy to respond to the challenges of modern science, particularly relativity and quantum mechanics \cite{luo1996material}. The construction of this theory (the LiaoYuan-Yubing Model) resonates deeply with the spirit of this pioneering work. In the following, while fully respecting its contributions, we will attempt to elaborate on the supplementary, refinement, and expansion offered by this theory starting from several key points, aiming to jointly advance the contemporary development of relational thinking.

\subsubsection{"The Real is Relational": Dynamical Supplement from Constitutive Relations to Generative Existence}
\label{subsubsec:real-relational}
We fully agree that \emph{"relations are real"}; relations are by no means secondary attributes of substance. However, regarding the universal assertion that \emph{"the real is relational,"} this theory hopes to introduce a more refined dynamic perspective. We propose: The real (or existence) is \emph{"existence-in-relations,"} but its identity and stability are the generative outcomes co-shaped by multiple internal and external relational networks, not unilaterally determined by relations alone.

The primary fact of an existent (E) is the dynamic stable integration achieved by the specific relational pattern ($R_{\text{in}}$) among its internal substructures (e₁, e₂, … eₙ). This is its intrinsic generative foundation for being identified as a unified whole. Simultaneously, this existent is necessarily embedded within a broader external relational network ($R_{\text{ext}}$) and engages in continuous interaction with it (exchanging \emph{"ripples"}), a process that verifies and continually shapes its reality. Therefore, existence is co-shaped, and recursively so, by its internal hierarchical integration ($R_{\text{in}}$) and external network coupling ($R_{\text{ext}}$). All existences we observe are undoubtedly already within complex relational networks, but this is precisely because an existent capable of being stably observed is itself a product of the successful integration of an internal relational network and the achievement of a transient balance with the external network. Relations determine the specific manifestation and evolutionary path of an existence, but the \emph{"substrate"} of existence is its self-consistent internal relational configuration. This is not a return to substantialism but provides a concrete generative mechanism for \emph{"how relations condense into identifiable existences."}

\subsubsection{"Relations Precede Relata": From Logical Priority to the Synchronic Generation of Network Selection}
\label{subsubsec:relations-precede}
The proposition that \emph{"relations precede relata"} is highly illuminating and subverts traditional thinking. This theory interprets it as follows: Logically and analytically, the identity and role of a \emph{"relatum"} as a network node are defined by its specific position and connection patterns within that relational network. However, from a generative evolutionary perspective, the emergence of a \emph{"relatum"} resembles a \emph{"probabilistic collapse"} event within the dynamics of the relational network.

The emergence of a novel, stable existent ($E_{\text{new}}$) is not simply predetermined by a \emph{"vacancy"} in the network. Instead, when a relational network (be it a microscopic quantum field or a macroscopic social structure) develops tensions, frailties, or new potential \emph{"gaps"} due to its intrinsic dynamics, those emergent patterns whose internal relational configuration ($R_{\text{in}}$) precisely resonates with that gap and can effectively channel the network's energy flow are \emph{"selected"} and stabilized as new nodes (relata). This selection process involves path dependency (necessary tendencies) and microscopic contingency. Crucially, this selected \emph{"relatum"} is not a blank slate entirely written by the network; it brings with it its specific intrinsic properties constituted by the integration of its substructures (i.e., its inherent $R_{\text{in}}$). It is precisely this intrinsic nature that determines whether and how it can couple with the network gap. Therefore, a bidirectional, synchronic shaping relationship exists between the relational network and the relatum: the network provides constraints and opportunities (\emph{"selection conditions"}), while the relatum's internal configuration determines the specific manner of its response (\emph{"the properties being selected"}). Together, they constitute a co-evolutionary feedback loop.

This proposition profoundly reveals how human cognition abstracts stable objects (nominalization) from complex interactions (relational predicates). This aligns with the logic of this theory's response to Graham Harman. We contend that the \emph{"inner essence"} of an existent is precisely its complex, hierarchical internal relational network ($R_{\text{in}}$). The \emph{"manifested properties"} in specific interactions are the particular \emph{"ripple"} patterns excited by that network under specific external coupling conditions. Professor Luo correctly recognizes that the manifestation of properties depends on specific relational conditions.

The supplement offered by this theory lies in providing a more fundamental ontological basis for this nominalized object of cognition: The \emph{"relatum"} that is nominalized does not correspond to a cognitive illusion but to a real, dynamically stable internal relational configuration (an \emph{"attractor state"}). This configuration is generated through a dynamic process of \emph{"aggregation-formation-locking-stabilization"} of specific \emph{"ripple"} interactions among substructures, given the satisfaction of relational background, density, and boundary conditions. Therefore, the \emph{"thing"} we recognize is both the focal point of relational manifestation and the stable product of relational generation. This avoids the risk of dissolving existence entirely into cognitive description while insisting on its thoroughly relational origin.

\textbf{Conclusion: Inheritance, Development, and the Pursuit of Unity}

In summary, Professor Luo Jiachang's \emph{"Relational Realism"} is a milestone. Deeply inspired by it, this theory attempts, within its grand framework, to commit to the following:

1. \textbf{Dynamical Refinement:} Providing a concrete generative model for \emph{"relations constituting the real,"} from potentiality to stable existence.

2. \textbf{Synchronic Supplementation:} Clarifying the bidirectional, recursive shaping mechanism between relational networks and relata.

3. \textbf{Fundamental Anchoring:} Linking cognitive \emph{"nominalization"} to ontological \emph{"stable configurations."}

Professor Luo's critique of Western substance ontology and his philosophical interpretation of relativity and quantum mechanics are foundational contributions. This theory fully agrees with and inherits this critical stance and interdisciplinary ambition. Our goal, building upon this, is to construct a more operational, relational explanatory framework capable of accommodating phenomena from quantum physics to consciousness, from natural evolution to artificial intelligence. The \emph{"LiaoYuan-Yubing Model"} and its own generative process are an experiment toward this goal. We firmly believe that only through such continuous critical dialogue and constructive development can relational thinking, rooted in the fertile soil of Chinese academia, continually radiate its vital power to explain the world.

\subsection{Response to Mathematical Ontology and the Philosophy of the Event: From Rupturous Events to Creative Phase Transitions in Relations}
\label{subsec:badiou-response}
Alain Badiou asserts that \emph{"mathematics is ontology,"} reducing being to pure multiples under set theory \cite{badiou1988being}. While this theory respects Badiou's endeavor to reveal the discontinuity of truth and the generation of subjectivity, it offers a different interpretation of its foundations.

Relational ontology, however, holds that mathematics is an exceptional language for describing relational structures, but it is not being itself. Being is the dynamic process of relational networks, not merely static sets. What Badiou calls an \emph{"event"} that ruptures a situation can be understood in relational ontology as follows: when the coupling between a system's (or situation's) internal relational network ($R_{\text{in}}$) and its external relations ($R_{\text{ext}}$) reaches a critical point, a drastic, discontinuous reorganization ($\Delta R$) occurs, giving rise to an entirely new stable configuration (a new existence or a new relational pattern). An event is not a specter external to the situation; rather, it is a phase transition node within the \emph{"co‑evolution"} of the relational network's own dynamics.

Regarding mathematics, this theory holds a dynamic instrumental realism view. Mathematics is undoubtedly the most refined and rigorous symbolic relational system invented by humans. Its power lies in abstracting formal structures from concrete relations, thereby revealing universal patterns across domains. However, directly equating mathematics (especially specific branches like set theory) with being itself risks mistaking an abstract model for ultimate reality. Mathematics is an extraordinary tool for describing and modeling relational networks and their dynamics, and it itself evolves as our understanding of the world deepens. Mathematical ontology subtly inverts the subject-object relationship: it is not that mathematics creates being, but that being (especially complex relational patterns like human cognition) creates mathematics and uses it to map more extensive existential relations.

Regarding events and situations. In this theory, what Badiou terms a \emph{"situation"} can correspond to a relatively localized and structured domain of a relational network. Its \emph{"state"} is the macroscopically stable configuration of that network at a specific time. The \emph{"event,"} which Badiou claims is undeducible from the state of the situation, receives a generative explanation within this framework: an event is a rapid, nonlinear topological phase transition of a relational network, triggered by a contingent perturbation after long-accumulated tensions (e.g., structural contradictions, blocked energy flows) reach a critical point. It appears as a \emph{"rupture"} and is \emph{"unnameable"} from the perspective of the old network order because it originates from potentialities that the old network could not accommodate. However, this phase transition does not emerge from nothing; its conditions of possibility are already immanent in the structural frailties of the old network and the incubation of emerging sub-networks. Historical \emph{"necessary tendencies"} correspond to the high-probability pathways of network evolution, while the \emph{"concrete occurrence"} of an event corresponds to the complex probabilistic collapse near the critical point. This preserves the revolutionary appearance of the event while providing it with a dynamical root.

Thus, Badiou's highly suggestive \emph{"truth procedure"} and \emph{"fidelity of a subject"} can be understood as, during the phase transition between old and new networks, an emerging cluster of relational nodes (the subject) striving to maintain, expand, and solidify the new relational pattern (the truth) opened by the event, enabling it to stabilize and become the new dominant structure in competition with the remnants of the old network. Badiou's philosophy brilliantly describes the phenomenology and logic of this process, while this theory seeks to supplement its generative foundation and dynamic mechanism.

\textbf{Summary: Between the Continuity of Relations and the Rupture of Change}

Through dialogue with Harman and Badiou, Relational Ontology demonstrates its potential as a \emph{"middle-way"} philosophy. It refuses to dissolve existence into a subject-less flow of pure relations (thus defending the relative independence of existence as stable patterns), and it also refuses to mystify change as a miracle entirely external to the historical process (thus grounding events in the immanent possibilities of relational dynamics). This theory proposes that the continuous generativity of the world inherently contains the potential for disruptive rupture. Continuity lies in the sustained interaction and attunement of relational networks; rupture arises from the nonlinear phase transitions of complex networks at critical points. This is the dialectical dynamics of reality that Relational Ontology seeks to capture.