\section{Introduction}
\label{sec:introduction}
Twentieth-century relational philosophy, such as Martin Buber's ontological grounding of the \emph{``I-Thou'' encounter}, Mikhail Bakhtin's exceptional elaboration of \emph{dialogicality} as constitutive of being, and Jean-Paul Sartre's pioneering analysis of \emph{conflictual interaction}, has provided indispensable mappings of the relational dimension of human existence. However, when we turn our attention to non-human agents (e.g., artificial intelligence, complex systems) and the cross-level dynamics spanning the micro and the macro, the explanatory boundaries of these classical frameworks become apparent. Buber's \emph{``encounter''} lacks a micro-structure of generation; Bakhtin's \emph{``dialogue''} struggles to encompass non-symbolic interactions; Sartre's \emph{``conflict''} presupposes a static subject-object opposition. Concurrently, contemporary metaphysics presents new challenges: Graham Harman vigorously defends the independence and \emph{``withdrawal'' of objects}, questioning whether relations can ever reach the core of reality; Alain Badiou, with his \emph{mathematical ontology} and philosophy of the \emph{event}, emphasizes the rupturous nature of \emph{truth procedures} and their transcendence of situations. These profound reflections collectively point to a central question: \textbf{Is there a unified framework capable of both explaining how substances stabilize from relations and accommodating the full spectrum from continuous evolution to revolutionary rupture?}

The \textbf{``Relational Ontology''} (also known as the \textbf{LiaoYuan-Yubing Model}) proposed in this paper aims to engage in a constructive dialogue with the aforementioned intellectual traditions. We fully acknowledge the groundbreaking contributions of these thinkers within their respective domains. Our goal is not negation, but an attempt to provide a more universal \emph{dynamical and generative framework}, hoping to extend their scope of explanation while affirming their insights, enabling the systematic description of a wide range of phenomena—from elementary particles to consciousness, from individuals to societies.

The core thesis of this theory is as follows: \emph{existence}, at any analyzable level, is a stable pattern generated, integrated, and sustained by specific relational networks among its substructures. The reality of this pattern (which can be called a \emph{substance}) is dually verified through the self-consistent maintenance of its internal network and reciprocal interactions with other existents (conceptualized as \emph{``ripple'' exchange}). Its evolution is driven by the continuous coupling and co-variation of internal and external relations. Thus, the world is understood as a \emph{cross-level, recursive relational-dynamic generative network}.

To avoid conceptual circularity, this theory explicitly distinguishes between two types of \emph{priority}: the \textbf{`logical-cognitive priority' of existence}—in any analysis, we always first identify an existent as the focus; and the \textbf{``constitutive-evolutionary priority' of relations}—the identity, stability, and change of that existent must be explained through its internal and external relational networks. This distinction is analogous to that in physics, where an object (existent) is observed first, but its properties and behavior (relational networks) are the objects of explanation.

This framework demonstrates strong potential for unified explanation. In the domain of \textbf{philosophy of physics}, it offers a coherent relational reinterpretation: quantum superposition can be understood as the undecided coupling potential of relational networks, while quantum entanglement corresponds to historically co-constituted relational wholes; Einstein's spacetime geometry is interpreted as a macroscopic effect emerging from the constraint patterns that mass-energy distribution imposes on the universe's fundamental relational network; cosmic origin and evolution—from the initial phase transition of the relational network in the Big Bang to the statistical unfolding of relational possibility space described by the law of entropy increase—all find an intuitive picture within the dynamics of relational generation and diffusion. In the domain of \textbf{philosophy of mind and technology ethics}, it treats consciousness and intelligence as higher-order emergent properties of complex relational networks, thereby providing a novel ontological foundation for understanding the human mind, the agency of artificial intelligence, and the ethical attunement of human-machine integration.

This chapter is intended as a constructive dialogue. We fully acknowledge the groundbreaking contributions of Buber, Bakhtin, and Sartre within their respective domains of inquiry. Our aim is not negation, but to highlight the explanatory boundaries that become apparent when their relational insights are extended to the analysis of non-human agents and cross-level dynamics. This theory attempts to offer a possible supplementary framework.

The structure of this paper is as follows: \textbf{Section 2} will systematically elaborate the four pillar principles and the core logical model of Relational Ontology. \textbf{Section 3} will delve into its application and implications for foundational problems in physics. \textbf{Section 4} will engage in constructive dialogue with Whiteheadian process philosophy and other key thinkers, clarifying the theory's position and contributions. \textbf{Section 5} will examine the boundaries of the theory, respond to potential objections, and outline directions for future development.